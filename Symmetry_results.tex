%%%%%%%%%%%%%%%%%%%%%%%%%%%%%
\section{Symmetry results}
\label{Sec:SymmetryResults}
%%%%%%%%%%%%%%%%%%%%%%%%%%%%%




This section is devoted to prove the following two symmetry results. The first one is a result for positive solutions in the whole space

\begin{theorem}
	\label{Th:SymmetryWholeSpace}
	Let $L$ be an integro-differential operator with kernel $K$ satisfying 99. Let $u$ be a bounded solution to
	\begin{equation}
	\label{Eq:PositiveWholeSpace}
	\beqc{\PDEsystem}
	L u &=& f(u) & \textrm{ in }\R^n\,,\\
	u &\geq& 0 & \textrm{ in } \R^n\,,
	\eeqc
	\end{equation}
	with the nonlinearity $f\in C^1$ satisfying
	\begin{itemize}
		\item $f(0) = f(1) = 0$,
		\item $f'(0)>0$,
		\item $f>0$ in $(0,1)$, and
		\item $f<0$ in $(1,+\infty)$.
	\end{itemize}
	Then, $u\equiv 0$ or $u \equiv 1$.
\end{theorem}

The second one is a symmetry result for equations in a half-space.

\begin{theorem}
	\label{Th:SymmHalfSpace}
	Let $L$ be an integro-differential operator with kernel $K$ satisfying 99. Let $u$ be a bounded solution of one of these two problems
	\begin{equation}
	\leqnomode
	\tag{P3}
	\label{Eq:P3}
	\beqc{\PDEsystem}
	Lu &=& f(u)   &\textrm{ in } \,\R^n_+,\\
	u &>& 0   &\textrm{ in } \,\R^n_+,\\
	u &=& 0   &\textrm{ in } \,\overline{\R^n_-},
	\eeqc
	\end{equation}
	
	\begin{equation}
	\leqnomode
	\tag{P4}
	\label{Eq:P4}
	\beqc{\PDEsystem}
	Lu &=& f(u)   &\textrm{ in } \,\R^n_+,\\
	u &>& 0   &\textrm{ in } \,\R^n_+,\\
	u(x',x_n) &=& -u(x',-x_n)   &\textrm{ in } \,\R^n.
	\eeqc
	\end{equation}
	
	\reqnomode
	
	Assume that the kernel $K$ of the integral operator $L$ satisfies
	$$
	K(x-y) \geq K(x-y^*) \,\,\,\,\text{for all } \,\, x,y\in \R^n_+,
	$$
	where $y^*$ is the reflection of $y$ with respect to $\{x_n = 0\}$. Suppose that the nonlinearity $f$ is Lipsitchz and
	\begin{itemize}
		\item $f(0) = f(1) = 0$,
		\item $f'(0)>0$, and $f'(t)\leq 0$ for all $t\in[1-\delta,1]$ for some $\delta>0$,
		\item $f>0$ in $(0,1)$, and
		\item $f$ is odd in the case of \eqref{Eq:P4}.
	\end{itemize}
	Then, $u$ depends only on $x_n$ and it is increasing in that direction.
\end{theorem}


The result in the case of problem \eqref{Eq:P3} will not be used in this paper. However, since the proof is exactly the same as for \eqref{Eq:P4} we include it here for completeness and further reference.

\subsection{Preliminary: Parabolic Maximum principle}





\begin{theorem}
\label{Th:ParabolicmaxPrpBdd}
\todo{Acotada donde? Creo que no hace falta que sea en todo $\R^n$}
Assume that $v$ is bounded and $\cp{2\s+\epsilon}(B_R\times(0,T])$ such that
\begin{equation*}
\beqc{\PDEsystem}
\partial_t v + L v &\leq& 0 & \textrm{ in }B_R\times(0,T]\,,\\
v_0:=v(x,0) &\leq& 0 & \textrm{ in } B_R\,,\\
v &\leq& 0 & \textrm{ in } \left( \R^n\setminus B_R\right) \times (0,T] \,.
\eeqc
\end{equation*}
Then, $$ v\leq 0 \,\,\,\, \text{ in } \,\,\, \R^n\times (0,T].  $$
\end{theorem}

\begin{proof}
Let us proceed by contradiction. Assume $v$ attains a positive maximum $M$. That is, $v(x_0,t_0) = M>0$. For the exterior conditions $x_0$ must be in $B_R$. Now we distinguish two cases:
\begin{itemize}
\item If $t_0\in(0,T)$, then $(x_0,t_0)$ is an interior absolute maximum and it must satisfy $v_t(x_0,t_0)=0$ and $Lv(x_0,t_0)>0$, which is a contradiction with the equation.\\
\item If $t_0 = T$, then it must satisfy $v_t(x_0,t_o)\geq 0$ and $Lv(x_0,t_0)>0$, which is also a contradiction with the equation.
\end{itemize}
\end{proof}

\begin{lemma}
\label{Lemma:NoBddSolL=1}
There is no bounded solution of $$Lv=1 \,\,\, \text{ in } \,\,\, \R^n.$$
\end{lemma}

\begin{proof}
Assume by contradiction that such solution exists. Then, by interior regularity (see Section~\ref{Sec:Preliminaries}) $v\in\cp{1}(\R^n)$ and $|\nabla v|\leq C$ in $\R^n$. By differentiating the equation with respect to $x_i$ we obtain
\begin{equation*}
\beqc{\PDEsystem}
L v_{x_i} &=& 0 & \textrm{ in } \R^n\,,\\
|v_{x_i}| &\leq& C & \textrm{ in } \R^n\,.
\eeqc
\end{equation*}
By Liouville Theorem, $v_{x_i}$ is constant. \todo[inline]{Quizás está bien escribir en algún sitio o como minimo referenciarlo} Hence, since this can be done for each partial derivative we obtain that $\nabla v$ is constant, and thus $v$ is affine. But since $u$ is bounded, $v$ must be constant too, and we arrive to a contradiction with $Lv=1$.
\end{proof}

\begin{lemma}
\label{Lemma:SolBall}
Let $L$ be an integral operator with kernel $K$ satisfying 99-99 and let $R>0$ be given. Then, there exists $\phi_R$ a solution of
\begin{equation*}
\beqc{\PDEsystem}
L \phi_R &=& 1 & \textrm{ in } B_R\,,\\
\phi_R &=& 0 & \textrm{ in } \R^n\setminus B_R\,,
\eeqc
\end{equation*}
and satisfying
$$
M_R:= \sup_{B_R} \phi_R \to \infty \quad \text{as } R\to \infty\,.
$$
\end{lemma}
\begin{proof}
The existence of a weak solution is given by Riesz representation theorem. Moreover, by regularity results (see section \ref{Subsec:Regularity}) it is in fact a classical solution and by the maximum principle, $\phi_R>0$ in $B_R$. Now consider the new function
$$ \varphi_R := \frac{\phi_R}{M_R}, $$
which satisfies
\begin{equation}
\beqc{\PDEsystem}
L \varphi_R &=& 1/M_R & \textrm{ in } B_R\,,\\
\varphi_R &=& 0 & \textrm{ in } \R^n\setminus B_R\,, \label{Eq: varphi}\\
||\varphi_R||_{\lp{\infty}} &=& 1.
\eeqc
\end{equation}
Let us assume by contradiction that $M_R$ does not tend to infinity. Then, since $M_R$ is increasing (use the maximum principle to compare $\phi_R$ and $\phi_{R'}$ with $R>R'$), it must have a limit $M<+\infty$.

Therefore, applying Lemma \ref{Lemma:CompactnessLemma} we deduce that $\varphi_R$ converges (up to a subsequence) in $C^{2\s+\epsilon}$-norm to a function $\varphi$ that is solution of
$$ L \varphi = \frac{1}{M}  \textrm{ in } \,\R^n\,, $$
and moreover $||\varphi||_{\lp{\infty}}=1$ for being the uniform limit of functions with this norm.

But such $\varphi$ cannot exists by Lemma \ref{Lemma:NoBddSolL=1}. Thus, we arrive to a contradiction and $M_R$ goes to infinity. 
\end{proof}

\begin{lemma}
\label{Lemma:SolBallToZero}
Let $M_R$ be as in the previous Lemma. Then, there exists a function $\psi_R\geq 0$ solution of
\begin{equation*}
\beqc{\PDEsystem}
L \psi_R &=& -\frac{1}{M_R} & \textrm{ in } B_R\,,\\
\psi_R &=& 1 & \textrm{ in } \R^n\setminus B_R\,,
\eeqc
\end{equation*}
such that
$$ \psi_R \xrightarrow{\text{as } R\to \infty}{} 0. $$
\end{lemma}

\begin{proof}
Let us define
$$ \psi_R := 1-\frac{\phi_R}{M_R} = 1-\varphi_R. $$
By Lemma \ref{Lemma:SolBall}, it is clear that $\psi_R$ defined previously solves the problem and is nonnegative. Then we only need to show the limit condition. Note that this is equivalent to show that $\varphi_R \to 1$ as $R\to\infty$. Recall that $\varphi_R$ solves problem \eqref{Eq: varphi}. Therefore, letting $R$ to infinity and knowing that $M_R\to \infty$ we can apply Lemma \ref{Lemma:CompactnessLemma} to deduce that $\varphi_R$ converges in  $C^{2\s+\epsilon}$-norm (up to a subsequence) to a function $\varphi\geq 0$ that solves
$$ L\varphi = 0 \,\,\,\,\, \text{ in } \,\,\, \R^n. $$
By Liouville Theorem, $\varphi$ must be constant, and since its $\lp{\infty}$-norm is one and it is nonnegative, then $\varphi\equiv 1$, and the result is proved.
\end{proof}

\begin{theorem}
\label{Th:ParaMaxPrp}
Assume $v$ is a bounded function such that
\begin{equation*}
\beqc{\PDEsystem}
\partial_t v + L v + c\,v &\leq& 0 & \textrm{ in }\R^n\times(0,+\infty)\,,\\
v_0:=v(x,0) &\leq& 0 & \textrm{ in } \R^n\,,
\eeqc
\end{equation*}
with $c=c(x)$ a bounded function. Then,
$$ v(x,t) \leq 0 \,\,\,\,\,\text{ in } \,\,\, \R^n\times[0,+\infty). $$
\end{theorem}

\begin{proof}
First of all, note that with the change of function $\tilde{v}(x,t) = \e^{-\alpha\,t} v(x,t)$ we can reduce the initial problem to
\begin{equation*}
\beqc{\PDEsystem}
\partial_t \tilde{v} + L \tilde{v} &\leq& 0 & \textrm{ in } \Omega \subseteq\R^n\times(0,+\infty)\,,\\
\tilde{v} &\leq& 0 & \textrm{ in }  \left(\R^n\times(0,+\infty)\right) \setminus  \Omega\,,\\
\tilde{v}_0 &\leq& 0 & \textrm{ in } \R^n\,,
\eeqc
\end{equation*}
if we take $\alpha > ||c||_{\lp{\infty}}$ and $\Omega = \{(x,t)\in \R^n\times(0,+\infty) \ \textrm{such that } \ v(x,t) > 0\}$.

Now, consider the function
$$ w_R(x,t) = \normLinf{v} \left(  \psi_R + \frac{t}{M_R} \right), $$
which satisfies
\begin{equation*}
\beqc{\PDEsystem}
\partial_t w_R + L w_R &=& 0 & \textrm{ in }B_R\times(0,T]\,,\\
w_R(x,0) &\geq& 0 & \textrm{ in } B_R\,,\\
w_R(x,t) &\geq& \normLinf{v}  & \textrm{ in } \left( \R^n\setminus B_R\right) \times (0,T] \,.
\eeqc
\end{equation*}
By the maximum principle in $(B_R\times[0,T])\cap \Omega$, Lemma \ref{Th:ParabolicmaxPrpBdd}, we conclude that $ w_R\geq \tilde{v} $ in $B_R\times(0,T]$.

Now, given an arbitrary point $(x_0,t_0)$, take $R_0>0$ and $T>0$ such that $(x_0,t_0)\in B_{R_0}\times [0,T]$. Then
$$ \tilde{v}(x_0,t_0) \leq w_R(x_o,t_0) = \normLinf{v} \left(  \psi_R(x_0) + \frac{t_0}{M_R} \right), \,\,\,\,\,\text{ for any }\,\,\, R\geq R_0. $$
Finally, letting $R \to \infty$ and using that $\psi_R(x_0) \to 0$ and $M_R \to \infty$ by Lemmas \ref{Lemma:SolBall} and \ref{Lemma:SolBallToZero}, we conclude
$$ \tilde{v}(x_0,t_0) \leq 0, $$
and therefore
$$ v(x_0,t_0) = \e^{\alpha\,t_0}\,\tilde{v}(x_0,t_0) \leq 0, $$
\end{proof}

%%%%%%%%%%%%%%%%%%%%%%%%%%%%%%%%%%%%%%%%%%%%%%%%%%%%%%%%%%%%%%%%%%%%%%%%%%%%%%%%%%%%%%%%%%%%%%%%%%%%%%%%%%%%%%%%%%%%%%%%%%%%%%%%%%%%%%%%%%%%%%%%%%%

\subsection{A symmetry result for positive solutions in the whole space}



\begin{proof}[Proof of Theorem~\ref{Th:SymmetryWholeSpace}]
The proof follows the ideas of Berestycki, Hamel and Nadirashvili from Theorem 2.2. in \cite{BerestyckiHamelNadi} but adapted to the whole space and with a nonlocal operator.

Assume $u\not\equiv 0$. Then, by the strong maximum principle $u>0$. Our goal is to show that $u \equiv 1$, and this will be accomplished in two steps.

\textbf{Step 1:} We show that $m:=\inf_{\R^n} u >0$.

By contradiction, we will assume $m=0$. Then, there exists a sequence $\{x_k\}$ such that $u(x_k)\rightarrow 0$ as $k \rightarrow +\infty$. By the Harnack Inequality  from Di Casto, Kuusi and Palatucci in \cite{DiCastoKuusiPalatucci}, given any $R>0$ we have
\todo[inline, color=red]{MIRAR LO DE MATTEO!!!!}
\begin{equation}
\label{Eq:Harnack}
\sup_{B_R(x_k)}u \leq C_R \inf_{B_R(x_k)}u \leq C \, u(x_k) \rightarrow 0 \,\,\text{as}\,\, k\rightarrow +\infty.
\end{equation}


Since $f(0) = 0 $ and $f'(0)>0$, it is easy to show that $f(t)\geq f'(0)t/2$ if $t$ is small enough. Indeed, since $f(0)=0$,
\begin{align*}
f'(0) = \lim_{t\to 0} \frac{f(t)}{t}
\end{align*}
and by the definition of limit we get that there exists $t_0>0$ such that
\begin{align}
\label{Eq:TaylorSimplification}
f(t)\geq \frac{f'(0)}{2}t \ \ \textrm{ for all} \ \ 0\leq t\leq t_0.
\end{align}
\todo[inline]{Igual esto se puede quitar que es fácil}
Therefore, with \eqref{Eq:Harnack} and \eqref{Eq:TaylorSimplification}, we deduce that there exists $M(R)\in\N$ such that
\begin{align}
\label{Eq:WholeSpace2}
L u - \frac{f'(0)}{2}u \geq 0 \,\,\textrm{ in }\ B_R(x_{M(R)})\,. 
\end{align}
On the other hand, let us define
$$ \lambda_R^{x_0} = \inf_{\substack{\varphi\in\cp{1}_{0}(B_R(x_0))\\ \varphi\not\equiv 0}} \frac{\ds \int_{\R^n}\int_{\R^n}|\varphi(x)-\varphi(y)|^2\,K(x-y) \d x \d y}{\ds \int_{\R^n}\varphi^2 \d x}, $$
which decreases to zero uniformly in $x_0$ as $R$ goes to infinity from being $L\in\mathcal{L}_0$ (see the proof of Lemma~\ref{Lemma:FirstOddEigenfunction} and also Proposition~9 of \cite{ServadeiValdinoci}). Therefore, there exists $R_0>0$ such that
$$ \lambda_R^x < \frac{f'(0)}{2} $$
for all $x\in \R^n$ and $R\geq R_0$. In particular, by choosing $x=x_{M(R_0)}$ there exists $w\in\cp{1}_0(B_{R_0}(x_{M(R_0)}))$ such that $w\not\equiv 0$ and
\begin{equation}
\label{Eq:Eigenfunction}
\int_{\R^n}\int_{\R^n}|\varphi(x)-\varphi(y)|^2\,K(x-y) \d x \d y < \frac{f'(0)}{2}\int_{\R^n}w^2 \d x.
\end{equation}

If we multiply \eqref{Eq:WholeSpace2} by $w^2/u\geq 0$ and integrate
\begin{align*}
0 &\leq \int_{\R^n} Lu\,\frac{w^2}{u} \d x - \frac{f'(0)}{2}\int_{\R^n} w^2 \d x \\
&= \int_{\R^n}\int_{\R^n}\left\{ u(x)-u(y) \right\} \left( \frac{w^2(x)}{u(x)}-\frac{w^2(y)}{u(y)} \right) K(x-y) \d x \d y - \frac{f'(0)}{2}\int_{\R^n} w^2 \d x \\
&\leq \int_{\R^n}\int_{\R^n} \left\{ w(x)-w(y)\right\}^2 K(x-y) - \frac{f'(0)}{2}\int_{\R^n} w^2 \d x ,
\end{align*}
which contradicts \eqref{Eq:Eigenfunction}. Then $\inf_{\R^n} u >0$.\\

\textbf{Step 2:} We show that $u\equiv 1$.

Now, choose $0<\xi_0<\min\{1,m\}$, which is well define by Step 1, and let $\xi(t)$ be the solution of the ODE
$$
\beqc{\PDEsystem}
\dot{\xi}(t) &=& f(\xi(t)) & \textrm{ in }(0,\infty)\,,\\
\xi(0) &=& \xi_0.
\eeqc
$$
Since $f>0$ in $(0,1)$ and $f(1) = 0$ we have that $\dot{\xi}(t)>0$ for all $t\geq 0$ and $\ds \lim_{t\to 0} \xi(t) = 1$.

Now, note that both $u(x)$ and $\xi(t)$ solve the parabolic equation
$$ \partial_t w + Lw = f(w) \,\,\, \textrm{ in }\R^n\times (0,\infty)\,, $$
and satisfy
$$ u(x) \geq m \geq \xi_0 = \xi(0). $$
Thus, by the parabolic maximum principle, Theorem \ref{Th:ParaMaxPrp}, $u(x)\geq \xi(t)$ for all $x\in\R^n$ and $t\in(0,\infty)$. By letting $t \to \infty$ we obtain
$$ u(x) \geq 1 \,\, \textrm{ in }\R^n\,.  $$
In a similar way, taking $\tilde{\xi}_0>\norm{u}_{L^\infty} \geq 1$, using $f<0$ in $(1,\infty)$, $f(1)=0$ and the parabolic maximum principle, we obtain the upper bound $u\leq 1$.

\end{proof}


%%%%%%%%%%%%%%%%%%%%%%%%%%%%%%%%%%%%%%%%%%%%%%%%%%%%%%%%%%%%%%%%%%%%%%%%%%%%%%%%%%%%%%%%%%%%%%%%%%%%%%%%%%%%%%%%%%%%%%%%%%%%%%%%%%%%%%%%%%%%%%%%%%%%%%%%%

\subsection{A one-dimensional symmetry result for positive solutions ``in a half-space''}


First we show that the solution is monotone. We do it using a moving planes argument, and for this reason we need the following maximum principle in ``narrow'' domains. Recall that for a domain $\Omega \subset \R^n$, we define the quantity $R(\Omega)$ as the smallest positive $R$ for which
$$
\dfrac{|B_R(x)\setminus \Omega|}{|B_R(x)|}\geq \dfrac{1}{2} \quad \text{ for every } x \in \Omega.
$$
If no such radius exists, we define $R(\Omega) = +\infty$.

\begin{proposition}
	\label{Prop:MaxPrpNarrowOdd}
	Let $H$ be a half-space in $\R^n$, and denote by $x^*$ the reflection of any point $x$ with respect to the hyperplane $\partial H$. Let $L$ be an integro-differential operator with a positive kernel $K$ satisfying
	\begin{equation}
	\label{Eq:KernelSymmetry}
	K(x-y) \geq K(x-y^*), \,\,\,\,\text{for all } \,\, x,y\in H.
	\end{equation}
	Assume that $v $ 99 satisfies 
	\begin{equation}
	\beqc{\PDEsystem}
	L v &\geq& c(x)\,v  &\textrm{ in } \Omega\subseteq H,\\
	v &\geq& 0 &\textrm{ in } H\setminus\Omega,\\
	v(x) &=& -v(x^*) &\textrm{ in } \R^n.
	\eeqc
	\end{equation}
	Then, $v \geq 0$ whenever $\Omega$ is narrow in the sense that\todo[inline]{Hay que poner In which sense??}.
\end{proposition}

\begin{proof}
	Let us begin by defining $\Omega_- = \{x\in \Omega \,:\,\, v<0\}$. We shall prove that $\Omega_-$ is empty. Assume by contradiction that it is not empty. Then, we split 
	$$ v = v_1+v_2\,, $$
	where
	\begin{equation*}
	v_1(x) = 
	\begin{cases}
	v(x)  &\textrm{ in } \Omega_-,\\
	0 &\textrm{ in } \R^n\setminus\Omega_-,
	\end{cases}
	\quad \text{ and } \quad
	v_2(x) = 
	\begin{cases}
	0  &\textrm{ in } \Omega_-,\\
	v(x) &\textrm{ in } \R^n\setminus\Omega_-.
	\end{cases}
	\end{equation*}
	Let us first show that $Lv_2\leq 0$ in $\Omega_-$. To see this, let us take $x\in\Omega_-$ and thus
	$$ 
	Lv_2(x) = \int_{\R^n\setminus\Omega_-} -v_2(y)K(x-y) \d y = -\int_{\R^n\setminus\Omega_-} v(y)K(x-y) \d y.  
	$$
	Now, we split $\R^n\setminus\Omega_-$ into
	$$ 
	A_1 = \Omega_-^*,\,\,\,\,\,\,\,\text{ and }\,\,\,\,\,\,\, A_2 = \left(H\setminus\Omega_-\right)\cup\left(H\setminus\Omega_-\right)^*,
	$$
	and we compute
	\begin{align*}
	-\int_{A_1} v(y)K(x-y) \d y = -\int_{\Omega_-} v(y^*)K(x-y^*) \d y  = \int_{\Omega_-} v(y)K(x-y^*) \d y \leq 0,
	\end{align*}
	where the last inequality comes from being $v$ negative in $\Omega_-$ and the kernel positive in all $\R^n$.
	On the other hand
	\begin{align*}
	-\int_{A_2} v(y)K(x-y) \d y = -\int_{H\setminus\Omega_-} v(y)K(x-y) \d y  -\int_{H\setminus\Omega_-} v(y^*)K(x-y^*) \d y \\ 
	= -\int_{H\setminus\Omega_-} v(y)\left\{K(x-y)-K(x-y^*)\right\} \d y \leq 0,
	\end{align*}
	where we have use the kernel condition \eqref{Eq:KernelSymmetry} and the odd symmetry of $v$. Thus, we get $Lv_2 \leq 0$ in $\Omega_-$, which means
	$$ Lv_1 = Lv-Lv_2 \geq Lu \geq c(x)\,v = c(x)\,v_1 \,\,\,\,\text{ in }\,\,\Omega_-. $$
	Therefore $v_1$ solves
	\begin{equation*}
	\beqc{\PDEsystem}
	Lv_1 &\geq& c(x)\,v_1   &\textrm{ in } \,\Omega_-,\\
	v_1 &=& 0 &\textrm{ in }\,\R^n\setminus\Omega_-,
	\eeqc
	\end{equation*}
	and we can apply the usual maximum principle for narrow domains (see Theorem~2.4 in \cite{QuaasXia}) to $v_1$ in $\Omega_-$ in order to deduce that $v_1\geq 0$ in all $\R^n$. But this is a contradiction with the definition of $v_1$ and the fact that the set $\Omega_-$ is not empty.
\end{proof}

This maximum principle in narrow domains for odd functions with respect to a hyperplane is the principal tool to use the moving plane argument (in the setting of odd functions\todo{ Esto esta bien dicho?}). This allows us to show the following result.


\begin{proposition}
	\label{Prop:MonotonyHalfSpace}
	Let $u$ be a bounded solution of one of the problems \eqref{Eq:P3} or \eqref{Eq:P4}, with $L$ an integro-differential operator with a positive kernel $K$ satisfying \eqref{Eq:KernelSymmetry} and $f$ a Lipschitz nonlinearity such that $f>0$ in $(0,||u||_{\lp{\infty}(\R^n_+)})$. Then,
	$$ 
	\frac{\partial u}{\partial x_n} > 0 \,\,\,\, \text{ in } \,\,\R^n_+.
	$$
\end{proposition}


\begin{proof}
	The proof is based on the moving planes method, exactly the analogue proof of Quaas and Xia in \cite{QuaasXia}, where they establish an equivalent result for the fractional Laplacian. The only ingredient that was missing to extend the proof to more general integro-differential operators of the form $L$ is the maximum principle of Proposition~\ref{Prop:MaxPrpNarrowOdd}. 
\end{proof}












\begin{proposition}
\label{Prop:HalfSpaceLimUnif}
Let $u$ be a bounded solution of one of the following problems

\begin{equation}
\leqnomode
\tag{P1}
\label{Eq:P1}
\beqc{\PDEsystem}
L u &=& f(u)  &\textrm{ in }\R^n\,,\\
\ds \lim_{x_n \to \pm \infty} u(x',x_n) &=& \pm 1 \,\,\, &\textrm{ uniformly}.
\eeqc
\end{equation}

\begin{equation}
\leqnomode
\tag{P2}
\label{Eq:P2}
\beqc{\PDEsystem}
L u &=& f(u)  &\textrm{ in }\R^n_+ = \{ x_n>0\} \,,\\ 
u &=& 0  &\textrm{ in }\overline{\R^n_-} = \{ x_n\leq 0\}\,,\\
\ds \lim_{x_n \to + \infty} u(x',x_n) &=& 1 \,\,\, &\textrm{ uniformly}.
\eeqc
\end{equation}

\reqnomode

Assume that there exists $\delta > 0$ such that
$$ f'(t) \leq 0 \,\, \text{ in } \,\,\, [-1,-1+\delta]\cup[1-\delta,1], $$
for problem \eqref{Eq:P1} and
$$ f'(t) \leq 0 \,\, \text{ in } \,\,\, [-1,-1+\delta] $$
for problem \eqref{Eq:P2}.

Then, $u$ only depends on $x_n$ and is increasing in that direction.
\end{proposition}

\begin{proof}
The proof follows the ideas, sliding method, of Berestycki, Hamel and Monneau from Theorem~1 in \cite{BerestyckiHamelMonneau} but adapted to a nonlocal operator.

Let us define $ u^t(x) := u(x+\nu t) $ for any $\nu\in\R^n$ with $|\nu|=1$ and $\nu_n>0$, and call $v^t(x):=u^t(x)-u(x)$. The aim is to show that $v^t(x)\geq 0$ for all $t\geq 0$.

\begin{enumerate}
\item[Step 1:] For $t>0$ big enough $v^t\geq 0$.\\
Since the limits of $u$ at infinity are uniform, let $A>0$ be such that for all $|x_n|\geq A$ and $x'\in \R^{n-1}$
\begin{equation}
\label{Eq:DefinitionA}
u(x',x_n)\not\in (-1+\delta,1-\delta).
\end{equation}
Let $H=\{v^t\geq 0\} = \{u^t(x) \geq u(x)\}$. Therefore, $v^t$ solves
$$
\beqc{\PDEsystem}
L v^t &=& d^t(x) v^t  &\textrm{ in }\R^n\setminus H,\\
v^t &\geq& 0  &\textrm{ in } H\,,
\eeqc
$$
with $d^t(x)\leq 0$. To show this, note that
\begin{equation}
\label{Eq:Definition_dt}
Lv^t = Lu^t - Lu = f(u^t) - f(u) = \frac{f(u^t) - f(u)}{u^t-u} v^t = d^t(x) v^t.
\end{equation}
Then we have to show that $ \frac{f(u^t) - f(u)}{u^t-u} \leq 0 $ out of $H$ for $t$ large enough. In order to prove it we distinguish the two different problems, \eqref{Eq:P1} and \eqref{Eq:P2}.

For \eqref{Eq:P1}: Take $x\not\in H$, that is $u^t(x)<u(x)$, and $t\geq 2A/\nu_n$, then
\begin{itemize}
\item If $x_n\geq -A$, then $1-\delta \leq u^t(x) < u(x)$ and $f$ is nonincreasing, which means
$$ d^t(x) = \frac{f(u^t) - f(u)}{u^t-u} \leq 0.  $$
\item If $x_n\leq -A$, then $u^t(x) \leq u(x) \leq -1+\delta$ and $f$ is nonincreasing, which also means
$$ d^t(x) = \frac{f(u^t) - f(u)}{u^t-u} \leq 0.  $$
\end{itemize}

For \eqref{Eq:P2}: Note that in this case $\R^n\setminus H$ is contained in $\R^n_+$. Take $x\not\in H$, that is $u^t(x)<u(x)$, and $t\geq A/\nu_n$, then $1-\delta \leq u^t(x) \leq u(x)$ and $f$ is nonincreasing, which means
$$ d^t(x) = \frac{f(u^t) - f(u)}{u^t-u} \leq 0.  $$

Therefore, by the maximum principle, $v^t\geq 0$ in $\R^n$ for $t$ large enough. Note that we have quantify how large must be $t$ in terms of the nonlinearity $f$ and the vector $\nu$.

\item[Step 2:] $v^t\geq 0$ for all $t\geq 0$.\\
For this, define 
$$\tau = \inf \{t \,\,\text{ such that }\, u^t\geq u\}, $$ 
and we claim that $\tau=0$. Assume, in order to get a contradiction, that $\tau>0$. Let us define
$$ m = \inf_{\R^{n-1}\times [-A,A]} (u^\tau - u) $$
in the case of \eqref{Eq:P1} and
$$ m = \inf_{\R^{n-1}\times [0,A]} (u^\tau - u) $$
in the case of \eqref{Eq:P2}. Now we can distinguish two cases depending if $m>0$ or $m=0$.

\item[Step 2.1:] $v^t\geq 0$ for all $t\geq 0$ if $m>0$.\\
By interior estimates for $L$, $u$ is globally $\cp{s}$-H\"older continuous. Then, there exists $\eta_0$ small so that for any $t\in(\tau-\eta_0,\tau)$ 
$$ u^t\geq u \,\,\,\, \textrm{in }\,\,\, \R^{n-1}\times [-A,A] \,\, \left( \textrm{in }\,\,\, \R^{n-1}\times [0,A] \right). $$
That is,
\begin{align*}
u^t(x)-u(x) &= u(x+t\nu)-u(x)-u(x+\tau\nu)+u(x+\tau\nu) \\
& \geq u^\tau(x)-u(x) - |u(x+t\nu)-u(x+\tau\nu)| \\
& \geq m - C|t-\tau|^s \geq 0,
\end{align*}
if $|t-\tau|^s\leq m/C$.

On the other hand, if we define $d^{\tau-\eta}(x)$ as in equation \eqref{Eq:Definition_dt}, we have
$$ L v^{\tau-\eta} = d^{\tau-\eta} (x) \,v^{\tau-\eta}. $$
Let us show now that $d^{\tau-\eta}\leq 0$ in the sets
$$ \hat{H} = \left\{|x_n|\geq A\right\} \cap \left\{u^{\tau-\eta}<u\right\}  \,\,\,\, \textrm{for \eqref{Eq:P1}}, $$
and
$$ \hat{H} = \{x_n\geq A\} \cap \{u^{\tau-\eta}<u\}  \,\,\,\, \textrm{for \eqref{Eq:P2}}. $$

To see this fact, we have to proceed as before. Let $x\in \tilde{H}$. Then,
\begin{itemize}
\item If $x_n\leq -A$, then $u^{\tau-\eta}(x) < u(x) \leq -1+\delta$ and $f$ is nonincreasing, which means
$$ d^{\tau-\eta}(x) = \frac{f(u^t) - f(u)}{u^t-u} \leq 0.  $$
\item If $x_n\geq A$, then $1-\delta \leq u^{\tau-\eta}(x) < u(x) $ and $f$ is nonincreasing, which also means
$$ d^t(x) = \frac{f(u^t) - f(u)}{u^t-u} \leq 0.  $$
\end{itemize}
Therefore, for $0<\eta \leq \eta_0$ we can apply the maximum principle in $\tilde{H}$ to obtain
$$ u^{\tau-\eta} \geq u \,\,\, \text{ in } \,\, \R^n, $$
which contradicts the minimality of $\tau$. This means that $\tau = 0$.

\item[Step 2.2:] $v^t\geq 0$ for all $t\geq 0$ if $m=0$.\\
By definition of $m$ we can take a sequence $x_k= (x_k',x_k^n) \in \R^{n-1}\times[-A,A]$ such that
\begin{equation}
\label{Eq:Limit}
u^\tau(x_k) - u(x_k) \rightarrow 0.
\end{equation}
Since $x_k^n$ is bounded, we can choose a subsequence that converges to $x_\infty^n \in [-A,A]$.
Call $u_k(x) = u(x'+x_k',x_k^n)$, which satisfies problem \eqref{Eq:P1} or \eqref{Eq:P2} for each $k$ since the problems are invariant under translations. Therefore, by elliptic estimates, $u_k$ converges to a function $u_\infty$ which satisfies
$$ L u_\infty = f(u_\infty) \,\,\, \text{ in } \,\,\, \R^n. $$
Note that since $u$ satisfies \eqref{Eq:DefinitionA} and the translations are only in the $n-1$ first variables, $u_k$ and its limit $u_\infty$ also satisfy this property. 

Then, if we define $w= u_\infty^\tau - u_\infty$, it solves
$$
\beqc{\PDEsystem}
L w &=& d(x) w  &\textrm{ in }\R^n,\\
w &\geq& 0  &\textrm{ in }\R^n,\\
w(0, x_\infty^n ) &=& 0,
\eeqc
$$
with 
$$ d(x) = \begin{cases}
\frac{f(u_\infty^\tau(x))-f(u_\infty(x))}{u_\infty^\tau(x)-u_\infty(x)} \,\,\, &\text{if } \,\, u_\infty^\tau(x)\not=u_\infty(x)\\
0 \,\, & \text{otherwise}.
\end{cases} $$


That is, we have that $w\geq 0$ in $\R^n$ since $u^\tau \geq u$. And on the other hand, from \eqref{Eq:Limit} we get
\begin{align*}
w(0,x_\infty^n) &= \lim_{k\to\infty} u_k^\tau(0,x_k^n)- u_k(0,x_k^n) \\ 
&= \lim_{k\to\infty} u^\tau(x_k',x_k^n)- u_k(x_k',x_k^n) \\ 
&= \lim_{k\to\infty}  u^\tau(x_k)- u_k(x_k) = 0
\end{align*}

By applying the strong maximum principle we get $w\equiv 0$, which means that $u_\infty(x) = u_\infty(x+\tau\nu)$. But this is a contradiction with the fact that $u_\infty(x',x_n) \geq 1-\delta$ for $x_n>A$, $u_\infty(x',x_n) \leq -1+\delta$ for $x_n<-A$ since $\delta<<1$ and $\tau_n>0$. That is,
$$ 1-\delta \leq u_\infty(x',A) = u_\infty \left( x' - \left\lceil \frac{2A}{\tau\nu_n} \right\rceil \tau \nu', A - \left\lceil \frac{2A}{\tau\nu_n} \right\rceil \tau \nu_n\right) \leq -1+\delta. $$

\end{enumerate}

Once we have shown that $v^t\geq 0$ for all $t>0$, then it is clear that
\begin{equation}
\label{Eq:nuDerivative}
\nabla u(x) \cdot \nu = \frac{\partial u}{\partial \nu} (x) = \lim_{t \to 0}  \frac{u(x+t\nu)-u(x)}{t} \geq 0
\end{equation}
for any $x\in \R^n$ and any $\nu\in \R^n$ such that $\nu_n>0$.
By making the limit $\nu_n$ to zero in the previous expression we get
$$ \nabla u \cdot (\nu',0) \geq 0 \,\,\, \text{for all } \,\, \nu'\in\R^{n-1}. $$
And choosing now the opposite vector $-\nu'$ we finally get
$$ \nabla u \cdot (\nu',0) = 0 \,\,\, \text{for all } \,\, \nu'\in\R^{n-1}, $$
which is equivalent to say that $u$ only depends on $x_n$. Moreover, if we choose $\nu = e_n$, then from \eqref{Eq:nuDerivative} we obtain
$$ \frac{\partial u}{\partial x_n} \geq 0. $$
\end{proof}






With all these ingredients we can now proce Theorem~\ref{Th:SymmHalfSpace}

\begin{proof}[Proof of Theorem~\ref{Th:SymmHalfSpace}]
Note that by Proposition~\ref{Prop:HalfSpaceLimUnif} we only need to prove that 
$$
\ds \lim_{x_n\to \infty} u(x',x_n) = 1
$$
uniformly. Therefore we divide the proof in two steps, first we prove that the limit exists and is one and then we prove that it is uniform.
\begin{enumerate}
\item[Step 1:] Given $x'\in \R^{n-1}$, then  $\ds \lim_{x_n\to \infty} u(x',x_n) = 1$.\\

By Proposition \ref{Prop:MonotonyHalfSpace} we know that $u$ is strictly increasing in the direction $x_n$. Since $u$ is also bounded by hypothesis, we know that that given $x'\in\R^{n-1}$, the one variable function $u(x',\cdot)$ has a limit, that is in fact positive for being $u(x',0) = 0$ and $u_{x_n}>0$. Let us define $\overline{u}(x')$ as this limit.

Let $x_n^k$ be any increasing sequence tending to infinity. Then we define the following sequence of functions
$$ u_k(x',x_n) = u(x',x_n+x_n^k), $$
which solves the problem
\begin{equation}
\beqc{\PDEsystem}
Lu_k &=& f(u_k)   &\textrm{ in } \,\{x_n>-x_n^k\},\\
u_k &>& 0   &\textrm{ in } \,\{x_n>-x_n^k\},\\
u &=& 0 \,\, \text{or odd}  &\textrm{ in } \,\{x_n\leq-x_n^k\},,
\eeqc
\end{equation}
and are uniformly bounded.

By letting $k\to\infty$ and using compactness \todo{!!!!} we have that $u_k$ converges (up to a subsequence) to a function $u_\infty$ that is solution of
\begin{equation}
\beqc{\PDEsystem}
Lu_\infty &=& f(u_\infty)   &\textrm{ in } \,\R^n,\\
u_\infty &\geq& 0   &\textrm{ in } \,\R^n.
\eeqc
\end{equation}
By Theorem \ref{Th:SymmetryWholeSpace} it is clear that $u_\infty\equiv 0$ or $u_\infty \equiv 1$. But, by construction
$$ u_\infty(x',0) = \lim_{k\to \infty} u_k(x',0) = \lim_{k\to \infty} u(x',x_n^k) = \overline{u}(x') > 0, $$
and therefore the only possibility is
$$ \lim_{x_n\to \infty} u(x',x_n) = 1 \,\,\,\,\, \text{ for all} \, x'\in\R^{n-1}. $$

\item[Step 2:] The limit is uniform in $x'$.

Let us proceed by contradiction. Suppose that the limit is not uniform. This means that given any $\epsilon>0$ small enough, there exists a sequence of points $(x_k',x_n^k)$ with $x_n^k\to \infty$ such that $u(x_k',x_n^k) = 1-\epsilon$. Then we define the following sequence of functions
$$ u_k(x',x_n) = u(x'+x_k',x_n+x_n^k), $$
which solves the problem
\begin{equation}
\beqc{\PDEsystem}
Lu_k &=& f(u_k)   &\textrm{ in } \,\{x_n>-x_n^k\},\\
u_k &>& 0   &\textrm{ in } \,\{x_n>-x_n^k\},\\
u &=& 0 \,\, \text{or odd}  &\textrm{ in } \,\{x_n\leq-x_n^k\},,
\eeqc
\end{equation}
and are uniformly bounded.

By letting $k\to\infty$ and using compactness \todo{!!!!} we have that $u_k$ converges (up to a subsequence) to a function $u_\infty$ that is solution of
\begin{equation}
\beqc{\PDEsystem}
Lu_\infty &=& f(u_\infty)   &\textrm{ in } \,\R^n,\\
u_\infty &\geq& 0   &\textrm{ in } \,\R^n.
\eeqc
\end{equation}
By Theorem \ref{Th:SymmetryWholeSpace} it is clear that $u_\infty\equiv 0$ or $u_\infty \equiv 1$. But, by construction
$$ u_\infty(0,0) = \lim_{k\to \infty} u_k(0,0) = \lim_{k\to \infty} u(x'_k,x_n^k) = 1-\epsilon, $$
which is a contradiction for $\epsilon>0$ small enough. Thus, the limit is uniform and applying Proposition \ref{Prop:HalfSpaceLimUnif} we get that $u$ depends only on $x_n$ and is increasing in that direction.
\end{enumerate}
\end{proof}

