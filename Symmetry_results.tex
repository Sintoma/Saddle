%%%%%%%%%%%%%%%%%%%%%%%%%%%%%
\section{Symmetry results}
\label{Sec:SymmetryResults}
%%%%%%%%%%%%%%%%%%%%%%%%%%%%%




This section is devoted to prove the following two symmetry results. The first one is a result for positive solutions in the whole space

\begin{theorem}
	\label{Th:SymmetryWholeSpace}
	Let $L$ be an integro-differential operator with kernel $K$ satisfying 99. Let $u$ be a bounded solution to
	\begin{equation}
	\label{Eq:PositiveWholeSpace}
	\beqc{\PDEsystem}
	L u &=& f(u) & \textrm{ in }\R^n\,,\\
	u &\geq& 0 & \textrm{ in } \R^n\,,
	\eeqc
	\end{equation}
	with the nonlinearity $f\in C^1$ satisfying
	\begin{itemize}
		\item $f(0) = f(1) = 0$,
		\item $f'(0)>0$,
		\item $f>0$ in $(0,1)$, and
		\item $f<0$ in $(1,+\infty)$.
	\end{itemize}
	Then, $u\equiv 0$ or $u \equiv 1$.
\end{theorem}

The second one is a symmetry result for equations in a half-space.

\begin{theorem}
	\label{Th:SymmHalfSpace}
	Let $L$ be an integro-differential operator with kernel $K$ satisfying 99. Let $u$ be a bounded solution to one of these two problems
	\begin{equation}
	\leqnomode
	\tag{P3}
	\label{Eq:P3}
	\beqc{\PDEsystem}
	Lu &=& f(v)   &\textrm{ in } \,\R^n_+,\\
	v &>& 0   &\textrm{ in } \,\R^n_+,\\
	v &=& 0   &\textrm{ in } \,\overline{\R^n_-},
	\eeqc
	\end{equation}
	
	\begin{equation}
	\leqnomode
	\tag{P4}
	\label{Eq:P4}
	\beqc{\PDEsystem}
	Lv &=& f(v)   &\textrm{ in } \,\R^n_+,\\
	v &>& 0   &\textrm{ in } \,\R^n_+,\\
	v(x',x_n) &=& -v(x',-x_n)   &\textrm{ in } \,\R^n.
	\eeqc
	\end{equation}
	
	\reqnomode
	
	Assume that the kernel $K$ of the integral operator $L$ satisfies
	$$
	K(x-y) \geq K(x-y^*) \,\,\,\,\text{for all } \,\, x,y\in \R^n_+,
	$$
	where $y^*$ is the reflection of $y$ with respect to $\{x_n = 0\}$. Suppose that the nonlinearity $f$ is Lipsitchz and
	\begin{itemize}
		\item $f(0) = f(1) = 0$,
		\item $f'(0)>0$, and $f'(t)\leq 0$ for all $t\in[1-\delta,1]$ for some $\delta>0$,
		\item $f>0$ in $(0,1)$, and
		\item $f$ is odd in the case of \eqref{Eq:P4}.
	\end{itemize}
	Then, $v$ depends only on $x_n$ and it is increasing in that direction.
\end{theorem}


The result in the case of problem \eqref{Eq:P3} will not be used in this paper. However, since the proof is exactly the same as for \eqref{Eq:P4} we include it here for completeness and further reference.

\subsection{Preliminary: Parabolic Maximum principle}





\begin{theorem}
\label{Th:ParabolicmaxPrpBdd}
\todo{Acotada donde? Creo que no hace falta que sea en todo $\R^n$}
Assume that $v$ is bounded and $\cp{2\s+\epsilon}(B_R\times(0,T])$ such that
\begin{equation*}
\beqc{\PDEsystem}
\partial_t v + L v &\leq& 0 & \textrm{ in }B_R\times(0,T]\,,\\
v_0:=v(x,0) &\leq& 0 & \textrm{ in } B_R\,,\\
v &\leq& 0 & \textrm{ in } \left( \R^n\setminus B_R\right) \times (0,T] \,.
\eeqc
\end{equation*}
Then, $$ v\leq 0 \,\,\,\, \text{ in } \,\,\, \R^n\times (0,T].  $$
\end{theorem}

\begin{proof}
Let us proceed by contradiction. Assume $v$ attains a positive maximum $M$. That is, $v(x_0,t_0) = M>0$. For the exterior conditions $x_0$ must be in $B_R$. Now we distinguish two cases:
\begin{itemize}
\item If $t_0\in(0,T)$, then $(x_0,t_0)$ is an interior absolute maximum and it must satisfy $v_t(x_0,t_0)=0$ and $Lv(x_0,t_0)>0$, which is a contradiction with the equation.\\
\item If $t_0 = T$, then it must satisfy $v_t(x_0,t_o)\geq 0$ and $Lv(x_0,t_0)>0$, which is also a contradiction with the equation.
\end{itemize}
\end{proof}

\begin{lemma}
\label{Lemma:NoBddSolL=1}
There is no bounded solution of $$Lv=1 \,\,\, \text{ in } \,\,\, \R^n.$$
\end{lemma}

\begin{proof}
Assume by contradiction that such solution exists. Then, by interior regularity (see Section~\ref{Sec:Preliminaries}) $v\in\cp{1}(\R^n)$ and $|\nabla v|\leq C$ in $\R^n$. By differentiating the equation with respect to $x_i$ we obtain
\begin{equation*}
\beqc{\PDEsystem}
L v_{x_i} &=& 0 & \textrm{ in } \R^n\,,\\
|v_{x_i}| &\leq& C & \textrm{ in } \R^n\,.
\eeqc
\end{equation*}
By Liouville Theorem, $v_{x_i}$ is constant. \todo[inline]{Quizás está bien escribir en algún sitio o como minimo referenciarlo} Hence, since this can be done for each partial derivative we obtain that $\nabla v$ is constant, and thus $v$ is affine. But since $u$ is bounded, $v$ must be constant too, and we arrive to a contradiction with $Lv=1$.
\end{proof}

\begin{lemma}
\label{Lemma:SolBall}
Let $L$ be an integral operator with kernel $K$ satisfying 99-99 and let $R>0$ be given. Then, there exists $\phi_R$ a solution of
\begin{equation*}
\beqc{\PDEsystem}
L \phi_R &=& 1 & \textrm{ in } B_R\,,\\
\phi_R &=& 0 & \textrm{ in } \R^n\setminus B_R\,,
\eeqc
\end{equation*}
and satisfying
$$
M_R:= \sup_{B_R} \phi_R \to \infty \quad \text{as } R\to \infty\,.
$$
\end{lemma}
\begin{proof}
The existence of a weak solution is given by Riesz representation theorem. Moreover, by regularity results (see section \ref{Subsec:Regularity}) it is in fact a classical solution and by the maximum principle, $\phi_R>0$ in $B_R$. Now consider the new function
$$ \varphi_R := \frac{\phi_R}{M_R}, $$
which satisfies
\begin{equation}
\beqc{\PDEsystem}
L \varphi_R &=& 1/M_R & \textrm{ in } B_R\,,\\
\varphi_R &=& 0 & \textrm{ in } \R^n\setminus B_R\,, \label{Eq: varphi}\\
||\varphi_R||_{\lp{\infty}} &=& 1.
\eeqc
\end{equation}
Let us assume by contradiction that $M_R$ does not tend to infinity. Then, since $M_R$ is increasing (use the maximum principle to compare $\phi_R$ and $\phi_{R'}$ with $R>R'$), it must have a limit $M<+\infty$.

Therefore, applying Lemma \ref{Lemma:CompactnessLemma} we deduce that $\varphi_R$ converges (up to a subsequence) in $C^{2\s+\epsilon}$-norm to a function $\varphi$ that is solution of
$$ L \varphi = \frac{1}{M}  \textrm{ in } \,\R^n\,, $$
and moreover $||\varphi||_{\lp{\infty}}=1$ for being the uniform limit of functions with this norm.

But such $\varphi$ cannot exists by Lemma \ref{Lemma:NoBddSolL=1}. Thus, we arrive to a contradiction and $M_R$ goes to infinity. 
\end{proof}

\begin{lemma}
\label{Lemma:SolBallToZero}
Let $M_R$ be as in the previous Lemma. Then, there exists a function $\psi_R\geq 0$ solution of
\begin{equation*}
\beqc{\PDEsystem}
L \psi_R &=& -\frac{1}{M_R} & \textrm{ in } B_R\,,\\
\psi_R &=& 1 & \textrm{ in } \R^n\setminus B_R\,,
\eeqc
\end{equation*}
such that
$$ \psi_R \xrightarrow{\text{as } R\to \infty}{} 0. $$
\end{lemma}

\begin{proof}
Let us define
$$ \psi_R := 1-\frac{\phi_R}{M_R} = 1-\varphi_R. $$
By Lemma \ref{Lemma:SolBall}, it is clear that $\psi_R$ defined previously solves the problem and is nonnegative. Then we only need to show the limit condition. Note that this is equivalent to show that $\varphi_R \to 1$ as $R\to\infty$. Recall that $\varphi_R$ solves problem \eqref{Eq: varphi}. Therefore, letting $R$ to infinity and knowing that $M_R\to \infty$ we can apply Lemma \ref{Lemma:CompactnessLemma} to deduce that $\varphi_R$ converges in  $C^{2\s+\epsilon}$-norm (up to a subsequence) to a function $\varphi\geq 0$ that solves
$$ L\varphi = 0 \,\,\,\,\, \text{ in } \,\,\, \R^n. $$
By Liouville Theorem, $\varphi$ must be constant, and since its $\lp{\infty}$-norm is one and it is nonnegative, then $\varphi\equiv 1$, and the result is proved.
\end{proof}

\begin{theorem}
\label{Th:ParaMaxPrp}
Assume $v$ is a bounded function such that
\begin{equation*}
\beqc{\PDEsystem}
\partial_t v + L v + c\,v &\leq& 0 & \textrm{ in }\R^n\times(0,+\infty)\,,\\
v_0:=v(x,0) &\leq& 0 & \textrm{ in } \R^n\,,
\eeqc
\end{equation*}
with $c=c(x)$ a bounded function. Then,
$$ v(x,t) \leq 0 \,\,\,\,\,\text{ in } \,\,\, \R^n\times[0,+\infty). $$
\end{theorem}

\begin{proof}
First of all, note that with the change of function $\tilde{v}(x,t) = \e^{-\alpha\,t} v(x,t)$ we can reduce the initial problem to
\begin{equation*}
\beqc{\PDEsystem}
\partial_t \tilde{v} + L \tilde{v} &\leq& 0 & \textrm{ in } \Omega \subseteq\R^n\times(0,+\infty)\,,\\
\tilde{v} &\leq& 0 & \textrm{ in }  \left(\R^n\times(0,+\infty)\right) \setminus  \Omega\,,\\
\tilde{v}_0 &\leq& 0 & \textrm{ in } \R^n\,,
\eeqc
\end{equation*}
if we take $\alpha > ||c||_{\lp{\infty}}$ and $\Omega = \{(x,t)\in \R^n\times(0,+\infty) \ \textrm{such that } \ v(x,t) > 0\}$.

Now, consider the function
$$ w_R(x,t) = \normLinf{v} \left(  \psi_R + \frac{t}{M_R} \right), $$
which satisfies
\begin{equation*}
\beqc{\PDEsystem}
\partial_t w_R + L w_R &=& 0 & \textrm{ in }B_R\times(0,T]\,,\\
w_R(x,0) &\geq& 0 & \textrm{ in } B_R\,,\\
w_R(x,t) &\geq& \normLinf{v}  & \textrm{ in } \left( \R^n\setminus B_R\right) \times (0,T] \,.
\eeqc
\end{equation*}
By the maximum principle in $(B_R\times[0,T])\cap \Omega$, Lemma \ref{Th:ParabolicmaxPrpBdd}, we conclude that $ w_R\geq \tilde{v} $ in $B_R\times(0,T]$.

Now, given an arbitrary point $(x_0,t_0)$, take $R_0>0$ and $T>0$ such that $(x_0,t_0)\in B_{R_0}\times [0,T]$. Then
$$ \tilde{v}(x_0,t_0) \leq w_R(x_o,t_0) = \normLinf{v} \left(  \psi_R(x_0) + \frac{t_0}{M_R} \right), \,\,\,\,\,\text{ for any }\,\,\, R\geq R_0. $$
Finally, letting $R \to \infty$ and using that $\psi_R(x_0) \to 0$ and $M_R \to \infty$ by Lemmas \ref{Lemma:SolBall} and \ref{Lemma:SolBallToZero}, we conclude
$$ \tilde{v}(x_0,t_0) \leq 0, $$
and therefore
$$ v(x_0,t_0) = \e^{\alpha\,t_0}\,\tilde{v}(x_0,t_0) \leq 0, $$
\end{proof}

%%%%%%%%%%%%%%%%%%%%%%%%%%%%%%%%%%%%%%%%%%%%%%%%%%%%%%%%%%%%%%%%%%%%%%%%%%%%%%%%%%%%%%%%%%%%%%%%%%%%%%%%%%%%%%%%%%%%%%%%%%%%%%%%%%%%%%%%%%%%%%%%%%%

\subsection{A symmetry result for positive solutions in the whole space}



\begin{proof}[Proof of Theorem~\ref{Th:SymmetryWholeSpace}]
The proof follows the ideas of Berestycki, Hamel and Nadirashvili from Theorem 2.2. in \cite{BerestyckiHamelNadi} but adapted to the whole space and with a nonlocal operator.

Assume $u\not\equiv 0$. Then, by the strong maximum principle $u>0$. Our goal is to show that $u \equiv 1$, and this will be accomplished in two steps.

\textbf{Step 1:} We show that $m:=\inf_{\R^n} u >0$.

By contradiction, we will assume $m=0$. Then, there exists a sequence $\{x_k\}$ such that $u(x_k)\rightarrow 0$ as $k \rightarrow +\infty$. By the Harnack Inequality  from Di Casto, Kuusi and Palatucci in \cite{DiCastoKuusiPalatucci}, given any $R>0$ we have
\todo[inline, color=red]{MIRAR LO DE MATTEO!!!!}
\begin{equation}
\label{Eq:Harnack}
\sup_{B_R(x_k)}u \leq C_R \inf_{B_R(x_k)}u \leq C \, u(x_k) \rightarrow 0 \,\,\text{as}\,\, k\rightarrow +\infty.
\end{equation}


Since $f(0) = 0 $ and $f'(0)>0$, it is easy to show that $f(t)\geq f'(0)t/2$ if $t$ is small enough. Indeed, since $f(0)=0$,
\begin{align*}
f'(0) = \lim_{t\to 0} \frac{f(t)}{t}
\end{align*}
and by the definition of limit we get that there exists $t_0>0$ such that
\begin{align}
\label{Eq:TaylorSimplification}
f(t)\geq \frac{f'(0)}{2}t \ \ \textrm{ for all} \ \ 0\leq t\leq t_0.
\end{align}
\todo[inline]{Igual esto se puede quitar que es fácil}
Therefore, with \eqref{Eq:Harnack} and \eqref{Eq:TaylorSimplification}, we deduce that there exists $M(R)\in\N$ such that
\begin{align}
\label{Eq:WholeSpace2}
L u - \frac{f'(0)}{2}u \geq 0 \,\,\textrm{ in }\ B_R(x_{M(R)})\,. 
\end{align}
On the other hand, let us define
$$ \lambda_R^{x_0} = \inf_{\substack{\varphi\in\cp{1}_{0}(B_R(x_0))\\ \varphi\not\equiv 0}} \frac{\ds \int_{\R^n}\int_{\R^n}|\varphi(x)-\varphi(y)|^2\,K(x-y) \d x \d y}{\ds \int_{\R^n}\varphi^2 \d x}, $$
which decreases to zero uniformly in $x_0$ as $R$ goes to infinity from being $L\in\mathcal{L}_0$ (see the proof of Lemma~\ref{Lemma:FirstOddEigenfunction} and also Proposition~9 of \cite{ServadeiValdinoci}). Therefore, there exists $R_0>0$ such that
$$ \lambda_R^x < \frac{f'(0)}{2} $$
for all $x\in \R^n$ and $R\geq R_0$. In particular, by choosing $x=x_{M(R_0)}$ there exists $w\in\cp{1}_0(B_{R_0}(x_{M(R_0)}))$ such that $w\not\equiv 0$ and
\begin{equation}
\label{Eq:Eigenfunction}
\int_{\R^n}\int_{\R^n}|\varphi(x)-\varphi(y)|^2\,K(x-y) \d x \d y < \frac{f'(0)}{2}\int_{\R^n}w^2 \d x.
\end{equation}

If we multiply \eqref{Eq:WholeSpace2} by $w^2/u\geq 0$ and integrate
\begin{align*}
0 &\leq \int_{\R^n} Lu\,\frac{w^2}{u} \d x - \frac{f'(0)}{2}\int_{\R^n} w^2 \d x \\
&= \int_{\R^n}\int_{\R^n}\left\{ u(x)-u(y) \right\} \left( \frac{w^2(x)}{u(x)}-\frac{w^2(y)}{u(y)} \right) K(x-y) \d x \d y - \frac{f'(0)}{2}\int_{\R^n} w^2 \d x \\
&\leq \int_{\R^n}\int_{\R^n} \left\{ w(x)-w(y)\right\}^2 K(x-y) - \frac{f'(0)}{2}\int_{\R^n} w^2 \d x ,
\end{align*}
which contradicts \eqref{Eq:Eigenfunction}. Then $\inf_{\R^n} u >0$.\\

\textbf{Step 2:} We show that $u\equiv 1$.

Now, choose $0<\xi_0<\min\{1,m\}$, which is well define by Step 1, and let $\xi(t)$ be the solution of the ODE
$$
\beqc{\PDEsystem}
\dot{\xi}(t) &=& f(\xi(t)) & \textrm{ in }(0,\infty)\,,\\
\xi(0) &=& \xi_0.
\eeqc
$$
Since $f>0$ in $(0,1)$ and $f(1) = 0$ we have that $\dot{\xi}(t)>0$ for all $t\geq 0$ and $\ds \lim_{t\to 0} \xi(t) = 1$.

Now, note that both $u(x)$ and $\xi(t)$ solve the parabolic equation
$$ \partial_t w + Lw = f(w) \,\,\, \textrm{ in }\R^n\times (0,\infty)\,, $$
and satisfy
$$ u(x) \geq m \geq \xi_0 = \xi(0). $$
Thus, by the parabolic maximum principle, Theorem \ref{Th:ParaMaxPrp}, $u(x)\geq \xi(t)$ for all $x\in\R^n$ and $t\in(0,\infty)$. By letting $t \to \infty$ we obtain
$$ u(x) \geq 1 \,\, \textrm{ in }\R^n\,.  $$
In a similar way, taking $\tilde{\xi}_0>\norm{u}_{L^\infty} \geq 1$, using $f<0$ in $(1,\infty)$, $f(1)=0$ and the parabolic maximum principle, we obtain the upper bound $u\leq 1$.

\end{proof}


%%%%%%%%%%%%%%%%%%%%%%%%%%%%%%%%%%%%%%%%%%%%%%%%%%%%%%%%%%%%%%%%%%%%%%%%%%%%%%%%%%%%%%%%%%%%%%%%%%%%%%%%%%%%%%%%%%%%%%%%%%%%%%%%%%%%%%%%%%%%%%%%%%%%%%%%%

\subsection{A one-dimensional symmetry result for positive solutions ``in a half-space''} In this subsection...



\todo[inline]{Introduction...}

First we show that the solution is monotone.\todo{Escribir bien} We do it using a moving planes argument, and for this reason we need the following maximum principle in ``narrow'' domains. Recall that for a domain $\Omega \subset \R^n$, we define the quantity $R(\Omega)$ as the smallest positive $R$ for which
$$
\dfrac{|B_R(x)\setminus \Omega|}{|B_R(x)|}\geq \dfrac{1}{2} \quad \text{ for every } x \in \Omega.
$$
If no such radius exists, we define $R(\Omega) = +\infty$. Thus, we say that a domain $\Omega$ is ``narrow'' ir $R(\Omega)$ is small.


An important result that we need is the following ABP-type estimate. 
\todo[inline]{Esto aparece en Quaas y Xia para el fractional Laplacian, la demo es parecida y es como la de Cabre. Para L no he encontrado que la hicieran}
\begin{theorem}
	\label{Th:ABPEstimate}
	Let $\Omega \subset \R^n$ with $R(\Omega) < +\infty$. Let $L \in \lcal_0(\s)$\todo{Esta es buena notacion?} and let $v\in C^{\beta}(\Omega)$, with $\beta > 2\s$, such that $\sup_{\Omega} v < +\infty$ and satisfying
	$$
	\beqc{\PDEsystem}
	Lv - c(x)v &\leq & h & \text{ in } \Omega\,, \\
	v & \leq & 0 & \text{ in } \R^n\setminus \Omega\,,
	\eeqc
	$$
	with $c(x)\leq 0$ in $\Omega$ and $h\in L^\infty(\Omega)$.
	
	Then,
	$$
	\sup_\Omega v \leq C R(\Omega)^{2\s} \norm{h}_{L^{\infty}(\Omega)}\,,
	$$
	where $C$ is a constant depending on \todo{Poner}.
\end{theorem}

As a consequence of this result, one can deduce easily a maximum principle in ``narrow'' domains.

\begin{corollary}
	\label{Cor:MaxPpleNarrowDomains}
	Let $\Omega \subset \R^n$ with $R(\Omega) < +\infty$ and let $v\in C^{\beta}(\Omega)$, with $\beta > 2\s$, satisfy
	$$
	\beqc{\PDEsystem}
	Lv + c(x)v &\leq & 0 & \text{ in } \Omega\,, \\
	v & \leq & 0 & \text{ in } \R^n\setminus \Omega\,,
	\eeqc
	$$
	with $c(x)$ bounded below.
	
	Then, there exists a number $\bar{R} > 0$ such that $v \leq 0$ in $\Omega$ whenever $R(\Omega)< \bar{R}$.
	
\end{corollary}

\begin{proof}
	We write $c= c^+ - c^-$, and therefore $Lv -(-c^+)v \leq c^- v^+	$. By Theorem~\ref{Th:ABPEstimate} we get
	$$
	\sup_\Omega v \leq C R(\Omega)^{2\s} \norm{c^- v^+}_{L^\infty(\Omega)} \leq C R(\Omega)^{2\s} \norm{c^-}_{L^\infty(\Omega)} \sup_\Omega v 
	$$
	and if $C R(\Omega)^{2\s} \norm{c^-}_{L^\infty(\Omega)}  <1 $, we deduce that $v\leq 0$ in $\Omega$.
\end{proof}

The only ingredient needed to show Theorem~\ref{Th:ABPEstimate} is the following weak Harnack inequality, which follows easily from \cite{DiCastoKuusiPalatucci}

\todo[inline]{De hecho seguro que está en lo de Matteo, CHECK!!
	De momento no la pongo}

\begin{proof}[Proof of Theorem~\ref{Th:ABPEstimate}]
	We are going to show it for $v$ satisfying,
	$$
	\beqc{\PDEsystem}
	Lv &\leq & h & \text{ in } \Omega\,, \\
	v & \leq & 0 & \text{ in } \R^n\setminus \Omega\,,
	\eeqc
	$$
	since the case $c\neq 0$ can be reduced to this one. Indeed, if we consider $\Omega_0 = \{x \in \Omega \ : \ v > 0\}$, then $Lv \leq Lv - c(x)v \leq h$ in $\Omega_0$.
	
	Assume first that $\Omega$ is bounded. Then the supremum of $v$ must be achieved at an interior point $x_0\in \Omega$. Define	$M:= v(x_0) = \sup_\Omega v$ and consider the function $w := M - v^+$. Note that $0 \leq v \leq M$, $v(x_0) = 0$ and $v \equiv M$ in $\R^n \setminus \Omega$. If we extend $h$ to be $0$ outside $\Omega$, we see that $Lv \geq -h$ in $B_R(x_0)$. 
	
	Now, by choosing $R= 2R(\Omega)$, and using the weak Harnack inequality \todo{citar o poner bien} we get
	\begin{align*}
		M \dfrac{1}{2^\varepsilon} & \leq \bpar{M^{\varepsilon}\dfrac{|B_{R/2}(x_0)\setminus \Omega|}{|B_{R/2}(x_0)|}}^{1/\varepsilon}= \bpar{\dfrac{1}{|B_{R/2}(x_0)|} \int_{B_{R/2}(x_0)\setminus \Omega} v^\varepsilon}^{1/\varepsilon} \\ 
		& \leq \bpar{ \fint_{B_{R/2}(x_0)} v^\varepsilon}^{1/\varepsilon} \leq C \bpar{\inf_{B_{R}(x_0)} v + R^{2\s} \norm{h}_{L^{\infty}(\Omega)} }\,.
	\end{align*}
	The conclusion follows from the fact that $v(x_0)= \inf_{B_{R}(x_0)} v = 0$.
	
	In the case of $\Omega$ being unbounded, the proof is the same with minor changes. We define $M$ as before and we consider, for every $\delta > 0$, a point $x_0$ such that $M-\delta \leq v(x_0)$. We proceed as before and the desired estimate follows by letting $\delta \to 0$. 
\end{proof}
 

\todo[inline]{Ellos lo demuestran dentro de la demo de los moving planes. En realidad sirve para los dos problemas. De todas formas, como el argumento es interesante y corto, lo ponemos aqui por completitud y para futura referencia. Hay que poner que es de ellos justo en el enunciado¿¿¿???}

In order to apply the moving planes method, the previous maximum principle in narrow domains is not powerful enough. The reason for this is that we need a prescribed constant sign of a function outside the domain, but in the application of the moving planes argument, since our functions are odd with respect to a hyperplane, they cannot have a constant sign in the exterior of a narrow band. For this reason, we need another version of a maximum principle in ``narrow'' domains that applies to odd functions and only assumes a constant sign of the function at one side of a hyperplane.

\begin{proposition}
	\label{Prop:MaxPrpNarrowOdd}
	Let $H$ be a half-space in $\R^n$, and denote by $x^*$ the reflection of any point $x$ with respect to the hyperplane $\partial H$. Let $L$ be an integro-differential operator with a positive kernel $K$ satisfying
	\begin{equation}
	\label{Eq:KernelSymmetry}
	K(x-y) \geq K(x-y^*), \,\,\,\,\text{for all } \,\, x,y\in H.
	\end{equation}
	Assume that $v $ 99 satisfies 
	\begin{equation}
	\beqc{\PDEsystem}
	L v &\geq& c(x)\,v  &\textrm{ in } \Omega\subseteq H,\\
	v &\geq& 0 &\textrm{ in } H\setminus\Omega,\\
	v(x) &=& -v(x^*) &\textrm{ in } \R^n.
	\eeqc
	\end{equation}
	Then, there exist a number $\overline{R}$ such that $v \geq 0$ whenever $R(\Omega) \leq \overline{R}$.
\end{proposition}

\begin{proof}
	Let us begin by defining $\Omega_- = \{x\in \Omega \,:\,\, v<0\}$. We shall prove that $\Omega_-$ is empty. Assume by contradiction that it is not empty. Then, we split 
	$$ v = v_1+v_2\,, $$
	where
	\begin{equation*}
	v_1(x) = 
	\begin{cases}
	v(x)  &\textrm{ in } \Omega_-,\\
	0 &\textrm{ in } \R^n\setminus\Omega_-,
	\end{cases}
	\quad \text{ and } \quad
	v_2(x) = 
	\begin{cases}
	0  &\textrm{ in } \Omega_-,\\
	v(x) &\textrm{ in } \R^n\setminus\Omega_-.
	\end{cases}
	\end{equation*}
	Let us first show that $Lv_2\leq 0$ in $\Omega_-$. To see this, let us take $x\in\Omega_-$ and thus
	$$ 
	Lv_2(x) = \int_{\R^n\setminus\Omega_-} -v_2(y)K(x-y) \d y = -\int_{\R^n\setminus\Omega_-} v(y)K(x-y) \d y.  
	$$
	Now, we split $\R^n\setminus\Omega_-$ into
	$$ 
	A_1 = \Omega_-^*,\,\,\,\,\,\,\,\text{ and }\,\,\,\,\,\,\, A_2 = \left(H\setminus\Omega_-\right)\cup\left(H\setminus\Omega_-\right)^*,
	$$
	and we compute
	\begin{align*}
	-\int_{A_1} v(y)K(x-y) \d y = -\int_{\Omega_-} v(y^*)K(x-y^*) \d y  = \int_{\Omega_-} v(y)K(x-y^*) \d y \leq 0,
	\end{align*}
	where the last inequality comes from being $v$ negative in $\Omega_-$ and the kernel positive in all $\R^n$.
	On the other hand
	\begin{align*}
	-\int_{A_2} v(y)K(x-y) \d y = -\int_{H\setminus\Omega_-} v(y)K(x-y) \d y  -\int_{H\setminus\Omega_-} v(y^*)K(x-y^*) \d y \\ 
	= -\int_{H\setminus\Omega_-} v(y)\left\{K(x-y)-K(x-y^*)\right\} \d y \leq 0,
	\end{align*}
	where we have use the kernel condition \eqref{Eq:KernelSymmetry} and the odd symmetry of $v$. Thus, we get $Lv_2 \leq 0$ in $\Omega_-$, which means
	$$ Lv_1 = Lv-Lv_2 \geq Lv \geq c(x)\,v = c(x)\,v_1 \,\,\,\,\text{ in }\,\,\Omega_-. $$
	Therefore $v_1$ solves
	\begin{equation*}
	\beqc{\PDEsystem}
	Lv_1 &\geq& c(x)\,v_1   &\textrm{ in } \,\Omega_-,\\
	v_1 &=& 0 &\textrm{ in }\,\R^n\setminus\Omega_-,
	\eeqc
	\end{equation*}
	and we can apply the usual maximum principle for narrow domains (Corollary~\ref{Cor:MaxPpleNarrowDomains}) to $v_1$ in $\Omega_-$ in order to deduce that $v_1\geq 0$ in all $\R^n$. But this is a contradiction with the definition of $v_1$ and the fact that the set $\Omega_-$ is not empty.
\end{proof}

This maximum principle in narrow domains for odd functions with respect to a hyperplane is the principal tool to use the moving plane argument. This allows us to show the following result.

\todo[inline]{Mirar lo que hacen Barrios, Quaas, etc en MR3624935 (2017) por si hay que citarlo}
\begin{proposition}
	\label{Prop:MonotonyHalfSpace}
	Let $v$ be a bounded solution of one of the problems \eqref{Eq:P3} or \eqref{Eq:P4}, with $L$ an integro-differential operator with a positive kernel $K$ satisfying \eqref{Eq:KernelSymmetry} and $f$ a Lipschitz nonlinearity such that $f>0$ in $(0,||v||_{\lp{\infty}(\R^n_+)})$. Then,
	$$ 
	\frac{\partial v}{\partial x_n} > 0 \,\,\,\, \text{ in } \,\,\R^n_+.
	$$
\end{proposition}


\begin{proof}
	The proof is based on the moving planes method, and is exactly the same as the analogue proof of Theorem~3.1 in \cite{QuaasXia}, where Quaas and Xia establish an equivalent result for the fractional Laplacian. For this reason, we give here just a sketch. As usual, for $\lambda > 0$ one defines $w_\lambda (x) = v(x',2\lambda - x_n)-v(x',x_n)$ and since the nonlinearity is Lipschitz, $w_\lambda$ solves, in any of the two cases ---\eqref{Eq:P3} or \eqref{Eq:P4}---, the following problem:
	$$
	\beqc{\PDEsystem} 
	L w_\lambda &=& c_\lambda(x)\,w_\lambda  &\textrm{ in } \Sigma_\lambda\subseteq H_\lambda,\\ 
	w_\lambda &\geq& 0 &\textrm{ in } H_\lambda\setminus\Sigma_\lambda,\\ 
	w_\lambda(x) &=& w_\lambda(x_\lambda) &\textrm{ in } \R^n, 
	\eeqc 
	$$
	where $\Sigma_\lambda := \left\{ x = (x',x_n) \ : \ 0<x_n<\lambda \right\}$ and $H_\lambda := \left\{ x = (x',x_n) \ : \ x_n<\lambda \right\}$ and $c_\lambda$ is a bounded function. Note that $w_\lambda$ is odd with respect to $\partial H_\lambda$. Then, using the maximum principle in narrow domains  Proposition~\ref{Prop:MaxPrpNarrowOdd} one shows that, if $\lambda$ is small enough, $w_\lambda>0$ in $\Sigma_\lambda$. To conclude the proof, we define 
	$$ 
	\lambda^* := \sup\{\lambda \ : \ w_\eta>0 \,\, \text{ in } \,\, \Sigma_\lambda \,\, \text{ for all } \,\, \eta<\lambda\}. 
	$$
	Note that $\lambda^*$ is well defined by the previous argument. Then, to conclude the proof one has to show that $\lambda^*=\infty$. This is done by showing that, if $\lambda^*$ is finite, then there exists a small $\delta_0 > 0$ such that for every $\delta \in (0,\delta_0]$ we have
	$$
	w_{\lambda^* +  \delta} (x) > 0 \quad \text{ in } \Sigma_{\lambda^*-\varepsilon}\setminus \Sigma_{\varepsilon}
	$$
	for some small $\varepsilon$.
	This is established using a compactness argument exactly as in Lemma~3.1 \cite{QuaasXia}. Finally, by the maximum principle in narrow domains we can deduce that $w_{\lambda^* +  \delta} (x) > 0 $ in $\Sigma_{\lambda^*+\delta}$, contradicting the definition of $\lambda^*$.
\end{proof}

\todo[inline]{habría que citar la harnack q se usa en Lemma 3.1, seguramente la de Matteo funciona}










\begin{proposition}
\label{Prop:HalfSpaceLimUnif}
Let $u$ be a bounded solution of one of the following problems

\begin{equation}
\leqnomode
\tag{P1}
\label{Eq:P1}
\beqc{\PDEsystem}
L u &=& f(u)  &\textrm{ in }\R^n\,,\\
\ds \lim_{x_n \to \pm \infty} u(x',x_n) &=& \pm 1 \,\,\, &\textrm{ uniformly}.
\eeqc
\end{equation}

\begin{equation}
\leqnomode
\tag{P2}
\label{Eq:P2}
\beqc{\PDEsystem}
L u &=& f(u)  &\textrm{ in }\R^n_+ = \{ x_n>0\} \,,\\ 
u &=& 0  &\textrm{ in }\overline{\R^n_-} = \{ x_n\leq 0\}\,,\\
\ds \lim_{x_n \to + \infty} u(x',x_n) &=& 1 \,\,\, &\textrm{ uniformly}.
\eeqc
\end{equation}

\reqnomode

Assume that there exists $\delta > 0$ such that
$$ f'(t) \leq 0 \,\, \text{ in } \,\,\, [-1,-1+\delta]\cup[1-\delta,1], $$
for problem \eqref{Eq:P1} and
$$ f'(t) \leq 0 \,\, \text{ in } \,\,\, [-1,-1+\delta] $$
for problem \eqref{Eq:P2}.

Then, $u$ only depends on $x_n$ and is increasing in that direction.
\end{proposition}

\begin{proof}
The proof follows the ideas, sliding method, of Berestycki, Hamel and Monneau from Theorem~1 in \cite{BerestyckiHamelMonneau} but adapted to a the integro-differential operator $L$.

Let us define $ v^t(x) := v(x+\nu t) $ for any $\nu\in\R^n$ with $|\nu|=1$ and $\nu_n>0$, and call $w^t(x):=v^t(x)-v(x)$. The aim is to show that $w^t(x)\geq 0$ for all $t\geq 0$.


\textbf{Step 1:} We show that $v^t\geq 0$ if $t>0$ is big enough.

Since the limits of $u$ at infinity are uniform, let $A>0$ be such that for all $|x_n|\geq A$ and $x'\in \R^{n-1}$
\begin{equation}
\label{Eq:DefinitionA}
v(x',x_n)\not\in (-1+\delta,1-\delta).
\end{equation}
Let $H=\{w^t\geq 0\} = \{v^t(x) \geq v(x)\}$. Therefore, $w^t$ solves
$$
\beqc{\PDEsystem}
L w^t &=& d^t(x) w^t  &\textrm{ in }\R^n\setminus H,\\
w^t &\geq& 0  &\textrm{ in } H\,,
\eeqc
$$
with 
\begin{equation}
\label{Eq:Definition_dt}
d^t(x) = \frac{f(v^t) - f(v)}{v^t-v} \,.
\end{equation} 

We claim that $d^t \leq 0$ in $\R^n\setminus H$ for $t$ big enough. If we prove this, Step~1 follows from the maximum principle. To see $d^t \leq 0$, we distinguish between the two different problems, \eqref{Eq:P1} and \eqref{Eq:P2}. In both cases, let $x\not\in H$ (thus $v^t(x) < v(x)$).

\begin{itemize}
	\item \eqref{Eq:P1}: Take $t\geq 2A/\nu_n$.
	On the one hand, if $x_n\geq -A$, then $1-\delta \leq v^t(x) < v(x)$ ---here we used that $x\not\in H$ and \eqref{Eq:DefinitionA}. Since $f$ is nonincreasing in $[1-\delta, 1]$, we get $d^t(x) \leq 0$. On the other hand, if $x_n\leq -A$, then $v^t(x) < v(x) \leq -1+\delta$ and the conclusion is the same using that $f$ is nonincreasing in $[-1, -1+\delta]$. 
	
	\item \eqref{Eq:P2}: Note that in this case $\R^n\setminus H$ is contained in $\R^n_+$. Take $t\geq A/\nu_n$.  Then, $1-\delta \leq v^t(x) < v(x)$ and, since $f$ is nonincreasing $[1-\delta, 1]$, we get $d^t(x) \leq 0$.	
\end{itemize}



\textbf{Step 2:} We show that $w^t\geq 0$ for all $t\geq 0$.

Define $\tau := \inf \{t \ : \ v^t\geq v\}$, and we claim that $\tau=0$. Assume, in order to get a contradiction, that $\tau>0$. Let us define
$$ 
m := \inf_{\R^{n-1}\times [-A,A]} (v^\tau - v)\, \quad \text{ or } \quad
m := \inf_{\R^{n-1}\times [0,A]} (v^\tau - v) \,,
$$
in the case of \eqref{Eq:P1} or \eqref{Eq:P2} respectively. We distinguish two cases depending whether $m>0$ or $m=0$.
\begin{itemize}
\item Case $m=0$:
On the one hand, the regularity theory for the operator $L$ yields that $u$ is globally $C^\s$-H\"older continuous. Thus, it is easy to see that there exists $\eta_0$ small such that, for every  $t\in(\tau-\eta_0,\tau)$,
$$ 
v^t\geq v \quad \textrm{ in } \quad  \R^{n-1}\times [-A,A] 
\quad \text{ or } \quad  
v^t\geq v \quad \textrm{ in } \quad  \R^{n-1}\times [0,A], 
$$
in the case of \eqref{Eq:P1} or \eqref{Eq:P2} respectively. Indeed,
\begin{align*}
v^t(x)-v(x) &= v(x+t\nu)-v(x)-v(x+\tau\nu)+v(x+\tau\nu) \\
& \geq v^\tau(x)-u(x) - |v(x+t\nu)-v(x+\tau\nu)| \\
& \geq m - C|t-\tau|^s \geq 0,
\end{align*}
if $0 < |t-\tau|^\s\leq m/C$.

On the other hand, if we define $d^{\tau-\eta}(x)$ as in equation \eqref{Eq:Definition_dt}, we have
$$ L w^{\tau-\eta} = d^{\tau-\eta} (x) \,w^{\tau-\eta}. $$
Note that $d^{\tau-\eta}\leq 0$ in the sets
$$ 
\hat{H} = \left\{|x_n|\geq A\right\} \cap \left\{v^{\tau-\eta}<v\right\},   \quad  \textrm{ or }  \quad  
\hat{H} = \{x_n\geq A\} \cap \{v^{\tau-\eta}<v\},
$$
in the case of \eqref{Eq:P1} or \eqref{Eq:P2} respectively. This can be seen exactly as in Step~1. Therefore, for $0<\eta \leq \eta_0$ we can apply the maximum principle in $\tilde{H}$ to obtain $u^{\tau-\eta} \geq u$ in $\R^n$, which contradicts the minimality of $\tau$. This yields that $\tau = 0$.

\item Case $m=0$: By definition of $m$, we can take a sequence $x_k= (x_k',x_k^n) \in \R^{n-1}\times[-A,A]$ ---in $\R^{n-1}\times[0,A]$ for problem \eqref{Eq:P2}---, such that
\begin{equation}
\label{Eq:Limit}
v^\tau(x_k) - v(x_k) \rightarrow 0.
\end{equation}
Since $x_k^n$ is bounded, we can choose a subsequence that converges to $x_\infty^n \in [-A,A]$ ---in $[0,A]$ for problem \eqref{Eq:P2}.
Call $v_k(x) = v(x'+x_k',x_k^n)$, which satisfies the same problem as $v$ for each $k$ (both \eqref{Eq:P1} and \eqref{Eq:P2} are invariant under translations in the first $n-1$ coordinates). Therefore, by elliptic estimates\todo{Poner bien}, $u_k$ converges to a function $u_\infty$ which satisfies the same problem as $v$ \todo{EL DE ABAJO NO!}
$$ L u_\infty = f(u_\infty) \,\,\, \text{ in } \,\,\, \R^n. $$
Note that since the translations are only in the $n-1$ first variables, $v_k$ and its limit $v_\infty$ also satisfy \eqref{Eq:DefinitionA}. 

Then, if we define $w= v_\infty^\tau - v_\infty$, it solves
$$
\beqc{\PDEsystem}
L w &=& d(x) w  &\textrm{ in }\R^n,\\
w &\geq& 0  &\textrm{ in }\R^n,\\
w(0, x_\infty^n ) &=& 0,
\eeqc
$$
with 
$$ d(x) = \begin{cases}
\dfrac{f(u_\infty^\tau(x))-f(u_\infty(x))}{u_\infty^\tau(x)-u_\infty(x)} \,\,\, &\text{if } \,\, u_\infty^\tau(x)\not=u_\infty(x)\\
0 \,\, & \text{otherwise}.
\end{cases} $$


That is, we have that $w\geq 0$ in $\R^n$, since $v^\tau \geq v$. And on the other hand, from \eqref{Eq:Limit} we get
\begin{align*}
w(0,x_\infty^n) &= \lim_{k\to\infty} v_k^\tau(0,x_k^n)- v_k(0,x_k^n) \\ 
&= \lim_{k\to\infty} v^\tau(x_k',x_k^n)- v_k(x_k',x_k^n) \\ 
&= \lim_{k\to\infty}  v^\tau(x_k)- v_k(x_k) = 0
\end{align*}

By applying the strong maximum principle we get $w\equiv 0$, which means that $v_\infty(x) = v_\infty(x+\tau\nu)$. But this is a contradiction with the fact that $v_\infty(x',x_n) \geq 1-\delta$ for $x_n>A$, $v_\infty(x',x_n) \leq -1+\delta$ for $x_n<-A$ since $\delta<<1$ and $\tau_n>0$. That is,
$$ 1-\delta \leq v_\infty(x',A) = v_\infty \left( x' - \left\lceil \frac{2A}{\tau\nu_n} \right\rceil \tau \nu', A - \left\lceil \frac{2A}{\tau\nu_n} \right\rceil \tau \nu_n\right) \leq -1+\delta. $$

\end{itemize}

\textbf{Step 3: }Conclusion.

Once we have shown that $w^t\geq 0$ for all $t>0$, then it is clear that
\begin{equation}
\label{Eq:nuDerivative}
\nabla v(x) \cdot \nu = \frac{\partial v}{\partial \nu} (x) = \lim_{t \to 0}  \dfrac{v(x+t\nu)-v(x)}{t} \geq 0
\end{equation}
for any $x\in \R^n$ and any $\nu\in \R^n$ such that $\nu_n>0$.
By making the limit $\nu_n$ to zero in the previous expression we get
$$ \nabla v \cdot (\nu',0) \geq 0 \,\,\, \text{for all } \,\, \nu'\in\R^{n-1}. $$
Choosing now the opposite vector $-\nu'$ we finally get
$$ \nabla v \cdot (\nu',0) = 0 \,\,\, \text{for all } \,\, \nu'\in\R^{n-1}, $$
which is equivalent to say that $u$ only depends on $x_n$. Moreover, if we choose $\nu = e_n$, then from \eqref{Eq:nuDerivative} we obtain
$$ \frac{\partial v}{\partial x_n} \geq 0. $$
\end{proof}






With all these ingredients we can now proce Theorem~\ref{Th:SymmHalfSpace}

\begin{proof}[Proof of Theorem~\ref{Th:SymmHalfSpace}]
Note that by Proposition~\ref{Prop:HalfSpaceLimUnif} we only need to prove that 
$$
\ds \lim_{x_n\to \infty} u(x',x_n) = 1
$$
uniformly. Therefore we divide the proof in two steps, first we prove that the limit exists and is one and then we prove that it is uniform.
\begin{enumerate}
\item[Step 1:] Given $x'\in \R^{n-1}$, then  $\ds \lim_{x_n\to \infty} u(x',x_n) = 1$.\\

By Proposition \ref{Prop:MonotonyHalfSpace} we know that $u$ is strictly increasing in the direction $x_n$. Since $u$ is also bounded by hypothesis, we know that that given $x'\in\R^{n-1}$, the one variable function $u(x',\cdot)$ has a limit, that is in fact positive for being $u(x',0) = 0$ and $u_{x_n}>0$. Let us define $\overline{u}(x')$ as this limit.

Let $x_n^k$ be any increasing sequence tending to infinity. Then we define the following sequence of functions
$$ u_k(x',x_n) = u(x',x_n+x_n^k), $$
which solves the problem
\begin{equation}
\beqc{\PDEsystem}
Lu_k &=& f(u_k)   &\textrm{ in } \,\{x_n>-x_n^k\},\\
u_k &>& 0   &\textrm{ in } \,\{x_n>-x_n^k\},\\
u &=& 0 \,\, \text{or odd}  &\textrm{ in } \,\{x_n\leq-x_n^k\},,
\eeqc
\end{equation}
and are uniformly bounded.

By letting $k\to\infty$ and using compactness \todo{!!!!} we have that $u_k$ converges (up to a subsequence) to a function $u_\infty$ that is solution of
\begin{equation}
\beqc{\PDEsystem}
Lu_\infty &=& f(u_\infty)   &\textrm{ in } \,\R^n,\\
u_\infty &\geq& 0   &\textrm{ in } \,\R^n.
\eeqc
\end{equation}
By Theorem \ref{Th:SymmetryWholeSpace} it is clear that $u_\infty\equiv 0$ or $u_\infty \equiv 1$. But, by construction
$$ u_\infty(x',0) = \lim_{k\to \infty} u_k(x',0) = \lim_{k\to \infty} u(x',x_n^k) = \overline{u}(x') > 0, $$
and therefore the only possibility is
$$ \lim_{x_n\to \infty} u(x',x_n) = 1 \,\,\,\,\, \text{ for all} \, x'\in\R^{n-1}. $$

\item[Step 2:] The limit is uniform in $x'$.

Let us proceed by contradiction. Suppose that the limit is not uniform. This means that given any $\epsilon>0$ small enough, there exists a sequence of points $(x_k',x_n^k)$ with $x_n^k\to \infty$ such that $u(x_k',x_n^k) = 1-\epsilon$. Then we define the following sequence of functions
$$ u_k(x',x_n) = u(x'+x_k',x_n+x_n^k), $$
which solves the problem
\begin{equation}
\beqc{\PDEsystem}
Lu_k &=& f(u_k)   &\textrm{ in } \,\{x_n>-x_n^k\},\\
u_k &>& 0   &\textrm{ in } \,\{x_n>-x_n^k\},\\
u &=& 0 \,\, \text{or odd}  &\textrm{ in } \,\{x_n\leq-x_n^k\},,
\eeqc
\end{equation}
and are uniformly bounded.

By letting $k\to\infty$ and using compactness \todo{!!!!} we have that $u_k$ converges (up to a subsequence) to a function $u_\infty$ that is solution of
\begin{equation}
\beqc{\PDEsystem}
Lu_\infty &=& f(u_\infty)   &\textrm{ in } \,\R^n,\\
u_\infty &\geq& 0   &\textrm{ in } \,\R^n.
\eeqc
\end{equation}
By Theorem \ref{Th:SymmetryWholeSpace} it is clear that $u_\infty\equiv 0$ or $u_\infty \equiv 1$. But, by construction
$$ u_\infty(0,0) = \lim_{k\to \infty} u_k(0,0) = \lim_{k\to \infty} u(x'_k,x_n^k) = 1-\epsilon, $$
which is a contradiction for $\epsilon>0$ small enough. Thus, the limit is uniform and applying Proposition \ref{Prop:HalfSpaceLimUnif} we get that $u$ depends only on $x_n$ and is increasing in that direction.
\end{enumerate}
\end{proof}

