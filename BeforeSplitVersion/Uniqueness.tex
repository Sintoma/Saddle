%%%%%%%%%%%%%%%%%%%%%%%%%%%%%%%%%%%%%%%%%%%%%%%%%%%
\section{Uniqueness of the saddle-shaped solution}
%%%%%%%%%%%%%%%%%%%%%%%%%%%%%%%%%%%%%%%%%%%%%%%%%%%
\label{Sec:Uniqueness}

In this section we show the uniqueness of the saddle-shaped solution.





% \begin{lemma}
% 	\label{Lemma:SaddleUnderSolutions}
% 	Assume that $u_1$ and $u_2$ are two saddle-shaped solutions of \eqref{Eq:NonlocalAllenCahn}. Then, there exists $u$ a saddle-shaped solution of \eqref{Eq:NonlocalAllenCahn} such that
% 	\begin{equation}
% 	\label{Eq:SaddleUnderSolutions}
% 	u\leq u_1 \quad \textrm{ and } u \leq u_2  \quad \textrm{ in } \ocal\,.
% 	\end{equation}
% \end{lemma}

% \begin{proof}
% 	First, let $u_R$ be a solution of
% 	$$
% 	\beqc{\PDEsystem}
% 	Lu_R & = & f(u_R) & \textrm{ in } B_R\,, \\
% 	u_R &=& \varphi &  \textrm{ in } \R^{2m} \setminus B_R\,, 
% 	\eeqc
% 	$$
% 	where 
% 	$$
% 	\varphi := 
% 	\begin{cases}
% 	\min \{u_1, u_2\} & \textrm{ in } \ocal\,, \\
% 	\max \{u_1, u_2\} & \textrm{ in } \R^{2m} \setminus \ocal\,.
% 	\end{cases}
% 	$$
% 	The existence of such $u_R$ is given by Proposition~\ref{Prop:MonotoneIterationOdd}. Indeed, it is not difficult to verify that $\usub = 0$ and $\usup = \varphi$ satisfy the hypotheses of such result. \todo{CHECK!!} 
% 	Moreover, $u_R > 0$ in $\ocal \cap B_R$, by the strong maximum principle ---Proposition~\eqref{Prop:StrongMaximumPrincipleForOddFunctions}. 
	
% 	Now, let $R \to +\infty$ and by compactness, up to a subsequence, there exists $u$ a solution to \eqref{Eq:NonlocalAllenCahn} such that $u$ depends only on $s$ and $t$, is odd with respect to $\ccal$ and $0 \leq u\leq u_1, \ u_2$. To see that $u$ is a saddle solution we only need to see that $0 < u < 1$ in $\ocal$. Clearly, $u < 1 $ in $\ocal$, since $u \leq u_1 < 1$ in $\ocal$. To see that $u>0$ in $\ocal$, we use the strong maximum principle (Proposition~\eqref{Prop:StrongMaximumPrincipleForOddFunctions}), so we just need to verify that $u\not \equiv 0$.
	
% 	By contradiction, assume that $u\equiv 0$. Then, note that any $u_R$ is a positive supersolution of the linearized operator $L - f'(u_R)$ in $\ocal$ ---recall the argument in \eqref{Eq:uSupersolLinearized}--- and therefore $Q_{u_R} (\xi) \geq 0$ for every smooth function with compact support in $\ocal \cap B_R$ (see Lemma~\ref{Lemma:EquivalenceStability}). Hence, letting $R \to +\infty$  we are led to $Q_u(\xi) \geq 0$ for every smooth $\xi$ with compact support in $\ocal$. This would be a contradiction with $u\equiv 0$, since in such case we would have  $f'(u) = f'(0)>0$ and hence the linearized operator $L - f'(0)$ is negative in balls of $\ocal$ with sufficiently large radius. Hence, $u \not \equiv 0$.	
% \end{proof}




