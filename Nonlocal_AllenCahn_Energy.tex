%%%%%%%%%%%%%%%%%%%%%%%%
\section{Non-local Allen-Cahn Energy}
%%%%%%%%%%%%%%%%%%%%%%%%
\label{Sec:Nonlocal_AllenCahn_Energy}

We follow the same notation as in \cite{CozziPassalacqua}.


\begin{definition}
\label{Def:FunctionalSpaceHK}
Given a set $\Omega \subseteq \R^n$ and a kernel $K \in \lcal_0$, we define the space
$$
\H^K(\Omega) := \setcond{w \in L^2(\Omega)}{[w]^2_{\H^K(\Omega)} < + \infty},
$$
where
$$
[w]^2_{\H^K(\Omega)} := \dfrac{1}{2}\int\int_{\R^{2n} \setminus (\R^n\setminus\Omega)^2} |w(x) - w(y)|^2 K(x-y) \d x \d y\,.
$$
We also define
\begin{align*}
	\H^K_0(\Omega) &:= \setcond{w \in \H^K(\Omega)}{ w = 0 \quad \textrm{a.e. in } \R^n \setminus \Omega} \\
	&\ = \setcond{w \in \H^K(\R^n)}{ w = 0 \quad \textrm{a.e. in } \R^n \setminus \Omega}.
\end{align*}
Assume that $\Omega \subseteq \R^{2m}$ is of double revolution. Then, we define
$$
\widetilde{\H}^K(\Omega) := \setcond{w \in \H^K(\Omega)}{w \textrm{ is doubly radial a.e.}}.
$$
and
$$
\widetilde{\H}^K_0(\Omega) := \setcond{w \in \H^K_0(\Omega)}{w \textrm{ is doubly radial a.e.}}.
$$
We will add the subscript `odd' and `even' to these spaces to consider only functions that are odd (respectively even) with respect to the Simons cone.
\end{definition}


Note that when $K$ satisfies \eqref{Eq:Ellipticity}, then $\H^K_0 (\Omega) = \H^\s_0 (\Omega)$, which is the space associated to the kernel of the fractional Laplacian, $K(y) = |y|^{-n-2\s}$. Furthermore, $\H^\s(\Omega) \subset H^\s(\Omega)$, the usual fractional Sobolev space (see \cite{HitchhikerGuide}).  For more comments on this, see~\cite{CozziPassalacqua}.

\begin{definition}
\label{Def:Energy}
Given a kernel $K \in \lcal_0$, a potential $G$ and a function $w\in \H^K(\Omega)$, with $\Omega\subseteq \R^{n}$, we define the energy
$$
\ecal(w, \Omega) := \ecal_\mathrm{K}(w,\Omega) + \ecal_\mathrm{P}(w,\Omega)\,, 
$$
where
$$
\ecal_\mathrm{K}(w, \Omega) := \dfrac{1}{2} [w]^2_{\H^K(\Omega)} \quad \text{ and } \quad  \ecal_\mathrm{P}(w, \Omega) := \int_{\Omega} G(w) 
\,.
$$
\end{definition}
%\int_{\R^{2m}} \int_{\R^{2m}} |w(x) - w(y)|^2 K(x-y) \d x \d y + \int_{\Omega} F(w) \d x
For short, we will denote $\ecal(w, \R^n) =: \ecal(w)$. Note that, for functions $w\in \H^K_0(\Omega)$, $\ecal_\mathrm{K}(w,\Omega) = \ecal_\mathrm{K}(w)$. Moreover, if $G\geq 0$, the energy satisfies
$$
\ecal(w, \Omega) \leq \ecal(w, \Omega') \quad \text{ whenever } \quad \Omega \subseteq \Omega'\,.
$$





Let us introduce a notation that will make the expression of the kinetic energy more simple. For $A$, $B\subseteq \R^n$, we define formally
$$
I_w(A,B) := \int_A  \int_B  \ |w(x)-w(y)|^2 K(|x-y|)  \d x \d y\,.
$$
Then, if $w \in \H^K(\Omega)$,
\begin{equation}
\label{Eq:EnergyWithInteractions}
2 \ecal_\mathrm{K}(w,\Omega) = \dfrac{1}{2}I_w(\Omega, \Omega) + I_w(\Omega,\R^n\setminus\Omega)\,.
\end{equation}
When working with spaces of even dimension, is it also convenient to consider the following interaction. If $A$, $B\subseteq \R^{2m}$ we denote
$$
I^\star_w(A,B) := \int_A \int_B |w(x)-w(y^\star)|^2 K(|x-y^\star|) \d x \d y\,.
$$
Note that if $w(x^\star) = - w(x)$, then at least at a formal level we have 
$$
I^\star_w(A,B) = I_w(A,B^\star) = I_w(A^\star,B)\,.
$$

From now on, we always assume that $n=2m$. The first task is to write the energy using the kernel $\overline{K}$, and writing a different expression for the Energy of doubly radial and odd functions. This is done in the following lemma.  \todo{Mejorar}


\begin{lemma}
\label{Lemma:InteractionsWithOverlineK}
Let $A$, $B\subseteq \R^{2m}$ be two domains of double revolution and let $w$ a doubly radial function for which $I_w(A,B)$ is well defined. Then, the following statements hold:
\begin{enumerate}
\item $I_w(A,B)$ can be written as
$$ 
I_w(A,B) = \int_A  \int_B  \ |w(x)-w(y)|^2 \overline{K}(x,y)  \d x \d y\,.
$$
\item If $w$ is odd with respect to the isometry $(\cdot)^\star$, then
$$
I^\star_w(A,B) = \int_A \int_B |w(x)+w(y)|^2 \overline{K}(x,y^\star) \d x \d y\,.
$$
\item If $\Omega$ is a set of double revolution such that $\Omega^\star = \Omega$, then
$$ 
\ecal_\mathrm{K}(w, \Omega) = \dfrac{1}{2} \big \{ I_w(\Omega\cap \ocal, \Omega\cap \ocal) + I^\star_w(\Omega\cap \ocal, \Omega\cap \ocal) \big \} + I_w(\Omega\cap \ocal,\ocal\setminus\Omega)  + I^\star_w(\Omega\cap \ocal,\ocal\setminus\Omega)\,. 
$$
\end{enumerate}
\end{lemma}


\begin{proof}
Given any $R\in SO(m)^2$ we have
\begin{align*}
	I_w(A,B) &= \int_A  \int_B  \ |w(x)-w(y)|^2 K(|x-y|)  \d x \d y\\
    &=  \int_A  \int_B  \ |w(R\overline{x})-w(y)|^2 K(|R\overline{x}-y|)  \d x \d y\\
    &=  \int_A  \int_B  \ |w(x)-w(y)|^2 K(|R\overline{x}-y|)  \d x \d y\,,
\end{align*}
where the first equality comes from the change of variables $x = R\tilde{x}$, which is an isometry, and the second one comes from the double radial symmetry of the function $w$. Now, if we integrate over all the rotations in $SO(m)^2$ we get the desired result. That is,
\begin{align*}
	I_w(A,B) &=\average_{SO(m)^2} I_w(A,B) \d R \\
    &= \fint_{SO(m)^2} \int_A  \int_B  \ |w(x)-w(y)|^2 K(|R\overline{x}-y|)  \d x \d y \d R \\
    &= \int_A \int_B |w(x)-w(y)|^2 \overline{K}(x,y) \d x \d y\,.
\end{align*}

The second statement follows from the relation $I^\star_w(A,B) = I_w(A,B^\star)$ and the previous computation just using the change of variables $\bar{y} = y^\star$.

Finally, the last statement follows from the expression \eqref{Eq:EnergyWithInteractions} and the relation $I^\star_w(A,B) = I_w(A,B^\star) = I_w(A^\star,B)$. We compute
\begin{align*}
I_w(\Omega, \Omega) &= I_w(\Omega, \Omega\cap \ocal) + I_w(\Omega, \Omega\cap \ical) \\
&= I_w(\Omega\cap \ocal, \Omega\cap \ocal) + I_w(\Omega\cap \ical, \Omega\cap \ocal) \\
&  \qquad \qquad + I_w(\Omega\cap \ocal, \Omega\cap \ical) + I_w(\Omega\cap \ical, \Omega\cap \ical) \\
&= I_w(\Omega\cap \ocal, \Omega\cap \ocal) + 2 I_w(\Omega\cap \ical, \Omega\cap \ocal)  + I_w(\Omega\cap \ical, \Omega\cap \ical) \\
&= I_w(\Omega\cap \ocal, \Omega\cap \ocal) + 2I_w((\Omega\cap \ocal)^\star, \Omega\cap \ocal) + I_w((\Omega\cap \ocal)^\star, (\Omega\cap \ocal)^\star) \\
&= I_w(\Omega\cap \ocal, \Omega\cap \ocal) + 2I_w^\star(\Omega\cap \ocal, \Omega\cap \ocal) + I_w(\Omega\cap \ocal, \Omega\cap \ocal) \\
&=  2 I_w(\Omega\cap \ocal, \Omega\cap \ocal) + 2I_w^\star(\Omega\cap \ocal, \Omega\cap \ocal)\,.
\end{align*}
Similarly,
\begin{align*}
I_w(\Omega, \R^n\setminus \Omega) &= I_w(\Omega,  \ocal \setminus \Omega) + I_w(\Omega,\ical \setminus \Omega) \\
&= I_w(\Omega\cap \ocal, \ocal \setminus \Omega) + I_w(\Omega\cap \ical, \ocal \setminus \Omega) \\
&  \qquad \qquad + I_w(\Omega\cap \ocal, \ical \setminus \Omega) + I_w(\Omega\cap \ical, \ical \setminus \Omega) \\
&= I_w(\Omega\cap \ocal, \ocal \setminus \Omega) + I_w((\Omega\cap \ocal)^\star, \ocal \setminus \Omega) \\
&  \quad \quad + I_w(\Omega\cap \ocal, (\ocal \setminus \Omega)^\star) + I_w((\Omega\cap \ocal)^\star, (\ocal \setminus \Omega)^\star) \\
&= I_w(\Omega\cap \ocal, \ocal \setminus \Omega) + I_w^\star(\Omega\cap \ocal, \ocal \setminus \Omega) \\
&  \qquad \qquad + I_w^\star(\Omega\cap \ocal, \ocal \setminus \Omega) + I_w(\Omega\cap \ocal, \ocal \setminus \Omega) \\
&= 2 I_w(\Omega\cap \ocal, \ocal \setminus \Omega) + 2I_w^\star(\Omega\cap \ocal, \ocal \setminus \Omega) \,.
\end{align*}
\end{proof}

As a consequence of this result, if $w\in \widetilde{\H}^K_{\mathrm{odd}}(\R^{2m})$, then
\begin{equation}
\label{Eq:EnergyOddInOcal}
\ecal_\mathrm{K}(w) = \frac{1}{2}\int_{\ocal} \int_\ocal|w(x) - w(y)|^2 \overline{K}(x,y) + |w(x) + w(y)|^2 \overline{K}(x,y^\star) \d x \d y. 
\end{equation}



The following are two lemmas regarding the decrease of the energy under some operations. They are stated for $\ecal(w)$, but the proofs are exactly the same for $\ecal(w, \Omega)$ when $\Omega^\star = \Omega$.
\begin{lemma}
\label{Lemma:TruncationOfFunctions1DecreaseEnergy}
Given $u\in \widetilde{\H}^K_{\mathrm{odd}}(\R^{2m})$, we define
\begin{equation*}
v(x) = \begin{cases}
\hspace{3.2mm}|u(x)| \,\,\, &\text{if } \,\,\, x\in\ocal,\\
-|u(x)| \,\,\, &\text{if } \,\,\, x\in\ical.
\end{cases}
\end{equation*}
Then
$$ \ecal(v) \leq \ecal(u)  $$
\end{lemma}

\begin{proof}
First, it is clear by definition that $v\in \widetilde{\H}^K_{\mathrm{odd}}(\R^{2m})$. By defining
$$ \ocal_0^+ = \setcond{ x\in \ocal}{u(x)\geq 0} \quad \textrm{ and } \quad \ocal_0^- = \setcond{ x\in \ocal}{u(x) < 0}, $$
and using the expression \eqref{Eq:EnergyOddInOcal}, we get
\begin{align*}
\ecal_\mathrm{K}(v) &= \frac{1}{2}\int_{\ocal} \int_\ocal|v(x) - v(y)|^2 \overline{K}(x,y) + |v(x) + v(y)|^2 \overline{K}(x^\star,y) \d x \d y \\
&= \frac{1}{2}\int_{\ocal_0^+} \int_{\ocal_0^+}|u(x) - u(y)|^2 \overline{K}(x,y) + |u(x) + u(y)|^2 \overline{K}(x^\star,y) \d x \d y \\
&\quad \quad+ \frac{1}{2}\int_{\ocal_0^+} \int_{\ocal_0^-}|u(x) + u(y)|^2 \overline{K}(x,y) + |u(x) - u(y)|^2 \overline{K}(x,y^\star) \d x \d y \\
&\quad \quad+ \frac{1}{2}\int_{\ocal_0^-} \int_{\ocal_0^+}|-u(x) - u(y)|^2 \overline{K}(x,y) + |-u(x) + u(y)|^2 \overline{K}(x,y^\star) \d x \d y \\
&\quad \quad+ \frac{1}{2}\int_{\ocal_0^-} \int_{\ocal_0^-}|-u(x) + u(y)|^2 \overline{K}(x,y) + |-u(x) - u(y)|^2 \overline{K}(x,y^\star) \d x \d y \\
&= \frac{1}{2}\int\!\!\int_{(\ocal_0^+\times\ocal_0^+) \cup (\ocal_0^-\times\ocal_0^-)}|u(x) - u(y)|^2 \overline{K}(x,y) + |u(x) + u(y)|^2 \overline{K}(x,y^\star) \d x \d y \\
&\quad \quad+ \frac{1}{2}\int\!\!\int_{(\ocal_0^+\times\ocal_0^-) \cup (\ocal_0^-\times\ocal_0^+)}|u(x) + u(y)|^2 \overline{K}(x,y) + |u(x) - u(y)|^2 \overline{K}(x,y^\star) \d x \d y,
\end{align*}
and
\begin{align*}
\ecal_\mathrm{K}(u) &= \frac{1}{2}\int\!\!\int_{(\ocal_0^+\times\ocal_0^+) \cup (\ocal_0^-\times\ocal_0^-)}|u(x) - u(y)|^2 \overline{K}(x,y) + |u(x) + u(y)|^2 \overline{K}(x,y^\star) \d x \d y \\
&\quad \quad+ \frac{1}{2}\int\!\!\int_{(\ocal_0^+\times\ocal_0^-) \cup (\ocal_0^-\times\ocal_0^+)}|u(x) - u(y)|^2 \overline{K}(x,y) + |u(x) + u(y)|^2 \overline{K}(x,y^\star) \d x \d y.
\end{align*}
Hence
\begin{align*}
\ecal_\mathrm{K}(v) - \ecal_\mathrm{K}(u) &= \frac{1}{2}\int\!\!\int_{(\ocal_0^+\times\ocal_0^-) \cup (\ocal_0^-\times\ocal_0^+)} \left\{ |u(x) + u(y)|^2 - |u(x) - u(y)|^2 \right\} \overline{K}(x,y) \d x \d y \\
&\quad \quad+ \frac{1}{2}\int\!\!\int_{(\ocal_0^+\times\ocal_0^-) \cup (\ocal_0^-\times\ocal_0^+)} \left\{ |u(x) - u(y)|^2 - |u(x) + u(y)|^2 \right\} \overline{K}(x,y^\star) \d x \d y \\
&= \frac{1}{2}\int\!\!\int_{(\ocal_0^+\times\ocal_0^-) \cup (\ocal_0^-\times\ocal_0^+)} 4u(x)u(y) \left\{ \overline{K}(x,y)-\overline{K}(x,y^\star) \right\} \d x \d y \leq 0
\end{align*}
since $u(x)u(y)<0$ in $(\ocal_0^+\times\ocal_0^-) \cup (\ocal_0^-\times\ocal_0^+)$ and $\overline{K}(x,y)\geq \overline{K}(x,y^\star)$  in $\ocal \times \ocal$.

Concerning the potential energy, since $G$ is an even function we have that $\ecal_\mathrm{P}(v) = \ecal_\mathrm{P}(u)$, and therefore we get the desired result by adding the kinetic and potential energies.
\end{proof}

\begin{lemma}
\label{Lemma:TruncationOfFunctions2DecreaseEnergy}
Given $u\in \widetilde{\H}^K_{\mathrm{odd}}(\R^{2m})$, we define
\begin{equation*}
v(x) = \begin{cases}
\hspace{3.6mm}\min\{1,u(x)\} \,\,\, &\text{if } \,\,\, x\in\ocal,\\
\,\,\,\max\{-1,u(x)\} \,\,\, &\text{if } \,\,\, x\in\ical.
\end{cases}
\end{equation*}
Then,
$$ \ecal(v) \leq \ecal(u)\,.  $$
\end{lemma}

\begin{proof}
Note that without loss of generality, by Lemma \ref{Lemma:TruncationOfFunctions1DecreaseEnergy} we can assume that $u$ is nonnegative in $\ocal$. First, it is clear by definition that $v\in \widetilde{\H}^K_{\mathrm{odd}}(\R^{2m})$. By defining
$$ \ocal_1^+ = \setcond{x\in \ocal}{ u(x)\geq 1} \quad \textrm{ and } \quad  \ocal_1^- = \setcond{x\in \ocal}{u(x)< 1}\,, $$
we get
\begin{align*}
\ecal_\mathrm{K}(v) &= \frac{1}{2}\int_{\ocal} \int_\ocal|v(x) - v(y)|^2 \overline{K}(x,y) + |v(x) + v(y)|^2 \overline{K}(x,y^\star) \d x \d y \\
&= \frac{1}{2}\int_{\ocal_1^+} \int_{\ocal_1^+} |1 - 1|^2 \overline{K}(x,y) + |1 + 1|^2 \overline{K}(x,y^\star) \d x \d y \\
&\quad \quad+ \frac{1}{2}\int_{\ocal_1^+} \int_{\ocal_1^-}|1 - u(y)|^2 \overline{K}(x,y) + |1 + u(y)|^2 \overline{K}(x,y^\star)  \d x \d y \\
&\quad \quad+ \frac{1}{2}\int_{\ocal_1^-} \int_{\ocal_1^+}|u(x) - 1|^2 \overline{K}(x,y) + |u(x) + 1|^2 \overline{K}(x,y^\star)\d x \d y \\
&\quad \quad+ \frac{1}{2}\int_{\ocal_1^-} \int_{\ocal_1^-} |u(x) - u(y)|^2 \overline{K}(x,y) + |u(x) + u(y)|^2 \overline{K}(x,y^\star)\d x \d y\\
&= \frac{1}{2}\int_{\ocal_1^+} \int_{\ocal_1^+} 4 \overline{K}(x,y^\star) \d x \d y \\
&\quad \quad+ \int_{\ocal_1^-} \int_{\ocal_1^+}|u(x) - 1|^2 \overline{K}(x,y) + |u(x) + 1|^2 \overline{K}(x,y^\star)\d x \d y \\
&\quad \quad+ \frac{1}{2}\int_{\ocal_1^-} \int_{\ocal_1^-} |u(x) - u(y)|^2 \overline{K}(x,y) + |u(x) + u(y)|^2 \overline{K}(x,y^\star)\d x \d y\,.
\end{align*}
In the last equality we have used the symmetry of $\overline{K}$ and the property $\overline{K}(x,y^\star) =\overline{K}(x^\star,y)$.

Hence, by using the expression \eqref{Eq:EnergyOddInOcal}, we see that
\begin{align*}
\ecal_\mathrm{K}(v) - \ecal_\mathrm{K}(u) &= \int_{\ocal_1^-} \int_{\ocal_1^+} \left\{|u(x) - 1|^2 - |u(x) - u(y)|^2 \right\} \overline{K}(x,y) \d x \d y \\
&\quad \quad+ \int_{\ocal_1^-} \int_{\ocal_1^+} \left\{|u(x) + 1|^2 - |u(x) + u(y)|^2 \right\} \overline{K}(x,y^\star) \d x \d y \\
&\quad \quad+ \frac{1}{2}\int_{\ocal_1^+} \int_{\ocal_1^+} \left\{4 - |u(x) + u(y)|^2 \right\} \overline{K}(x,y^\star) \d x \d y \\
&\quad \quad- \frac{1}{2}\int_{\ocal_1^+} \int_{\ocal_1^+} |u(x) - u(y)|^2 \overline{K}(x,y) \d x \d y \\
&\leq \int_{\ocal_1^-} \int_{\ocal_1^+} \left\{ 1+u(y)-2u(x)\right \} \left\{ 1-u(y) \right \} \overline{K}(x,y) \d x \d y \\
&\quad \quad+ \int_{\ocal_1^-} \int_{\ocal_1^+} \left \{1+u(y)+2u(x)\right \} \left\{ 1-u(y)\right \} \overline{K}(x,y^\star) \d x \d y \\
&\quad \quad+ \frac{1}{2}\int_{\ocal_1^+} \int_{\ocal_1^+} \left \{ 2+u(x)+u(y)\right \} \left\{ 2-u(x)-u(y) \right \} \overline{K}(x,y^\star) \d x \d y. \\
\end{align*}
On the one hand, since $1-u(y)<0$ and $1+u(y)-2u(x) = 2 (1-u(x))+(u(y)-1) \geq 0$ in $\ocal_1^-\times\ocal_1^+$ and $\overline{K}(x,y)\geq \overline{K}(x,y^\star)$  in $\ocal \times \ocal$ we have
$$ 
\{1+u(y)-2u(x)\}\{1-u(y)\} \overline{K}(x,y) \leq  \{ 1+u(y)-2u(x) \}\{ 1-u(y) \} \overline{K}(x,y^\star) \,\,\,\, \text{in} \,\,\, \ocal_1^-\times\ocal_1^+. $$

On the other hand, since $2+u(x)+u(y)\geq 0$ in $\ocal \times \ocal$, $2-u(x)-u(y) = (1-u(x))+(1-u(y)) \leq 0$ in $\ocal_1^+\times\ocal_1^+$ and $\overline{K}\geq 0$ in $\R^{2m}\times\R^{2m}$ we have
$$
\{2+u(x)+u(y)\}\{2-u(x)-u(y)\} \overline{K}(x,y^\star) \leq 0 \,\,\,\, \text{in} \,\,\, \ocal_1^+\times\ocal_1^+. $$

Thus,
\begin{align*}
\ecal_\mathrm{K}(v) - \ecal_\mathrm{K}(u) &\leq \int_{\ocal_1^-} \int_{\ocal_1^+} \{ 1+u(y)-2u(x) \} \{ 1-u(y) \} \overline{K}(x,y^\star) \d x \d y \\
&\quad \quad + \int_{\ocal_1^-} \int_{\ocal_1^+}  \{ 1+u(y)+2u(x) \}\{ 1-u(y) \} \overline{K}(x,y^\star) \d x \d y \\
&= \int_{\ocal_1^-} \int_{\ocal_1^+} 2 \{ 1+u(y) \} \{ 1-u(y) \} \overline{K}(x,y^\star) \d x \d y \leq 0.
\end{align*}

Concerning the potential energy, since $G$ is such that $G(x)\geq G(1) = G(-1) = 0$ if $|x|\geq 1$, then clearly $\ecal_\mathrm{P}(v) \leq \ecal_\mathrm{P}(u)$, and therefore we get the desired result by adding the kinetic and potential energies.
\end{proof}

We will also need the following decomposition lemma
\begin{lemma}
	\label{Lemma:DecompositionHK}
	Let $\Omega \subseteq \R^{2m}$ be a domain of double revolution such that $\Omega = \Omega^\star$. Then, the space $\widetilde{\H}^K_0(\Omega)$ can be decomposed as the following direct sum:
$$
\widetilde{\H}^K_0(\Omega) = \widetilde{\H}^K_{0,\, \mathrm{even}}(\Omega) \oplus \widetilde{\H}^K_{0,\,\mathrm{odd}}(\Omega) \,.
$$
Moreover,
$$
\widetilde{\H}^K_{0,\, \mathrm{even}}(\Omega) \perp \widetilde{\H}^K_{0,\,\mathrm{odd}}(\Omega)\,,
$$
where we are considering the spaces equipped with the scalar product
$$
\langle v,w \rangle_{\widetilde{\H}^K_0} := \dfrac{1}{2}\int_{\R^{2m}} \int_{\R^{2m}}  \big(v(x) - v(y)\big)\big(w(x) - w(y)\big) \overline{K} (x,y) \d x \d y\,.
$$
\end{lemma}


\begin{proof}
	Obviously, every function in $w \in \widetilde{\H}^K_0(\Omega)$ can be written as
	$$
	w(x) = \dfrac{w(x) + w(x^\star)}{2} + \dfrac{w(x) - w(x^\star)}{2}\,,
	$$
	and such representation is unique if we show that 
	$$
	\widetilde{\H}^K_{0,\, \mathrm{even}}(\Omega) \cap \widetilde{\H}^K_{0,\,\mathrm{odd}}(\Omega)  = 0\,.
	$$
	This will follow from the orthogonality of $\widetilde{\H}^K_{0,\, \mathrm{even}}(\Omega) $ and $\widetilde{\H}^K_{0,\,\mathrm{odd}}(\Omega)$. Therefore, it only remains to prove that if $v \in \widetilde{\H}^K_{0,\,\mathrm{odd}}(\Omega)$ and $w \in \widetilde{\H}^K_{0,\, \mathrm{even}}(\Omega)$, then
	$$
	\langle v,w \rangle_{\widetilde{\H}^K_0} = 0\,.
	$$
	
	Let us show this by direct computation. Using the change given by $(\cdot)^\star$ in the integrals in $\ical$, the symmetries of $v$ and $w$, and the properties of Lemma~\ref{Lemma:PropertiesStar}, we see that
    \begin{align*}
    2\langle v,w \rangle_{\widetilde{\H}^K_0} &=  
    \int_{\ocal} \int_{\ocal}  \big(v(x) - v(y)\big)\big(w(x) - w(y)\big) \overline{K} (x,y) \d x \d y \\
    &\quad \quad + \int_{\ocal} \int_{\ical}  \big(v(x) - v(y)\big)\big(w(x) - w(y)\big) \overline{K} (x,y) \d x \d y \\
    & \quad \quad + \int_{\ical} \int_{\ocal}   \big(v(x) - v(y)\big)\big(w(x) - w(y)\big) \overline{K} (x,y) \d x \d y \\
    &\quad \quad  + \int_{\ical} \int_{\ical}  \big(v(x) - v(y)\big)\big(w(x) - w(y)\big) \overline{K} (x,y) \d x \d y \\
    &= \int_{\ocal} \int_{\ocal}  \big(v(x) - v(y)\big)\big(w(x) - w(y)\big) \overline{K} (x,y) \d x \d y \\
    &\quad \quad + \int_{\ocal} \int_{\ocal}  \big(v(x) - v(y^\star)\big)\big(w(x) - w(y^\star)\big) \overline{K} (x,y^\star) \d x \d y \\
    & \quad \quad + \int_{\ocal} \int_{\ocal}   \big(v(x^\star) - v(y)\big)\big(w(x^\star) - w(y)\big) \overline{K} (x^\star,y) \d x \d y \\
    &\quad \quad  + \int_{\ocal} \int_{\ocal}  \big(v(x^\star) - v(y^\star)\big)\big(w(x^\star) - w(y^\star)\big) \overline{K} (x^\star,y^\star) \d x \d y \\
   &= \int_{\ocal} \int_{\ocal}  \big(v(x) - v(y)\big)\big(w(x) - w(y)\big) \overline{K} (x,y) \d x \d y \\
    &\quad \quad + \int_{\ocal} \int_{\ocal}  \big(v(x) + v(y)\big)\big(w(x) - w(y)\big) \overline{K} (x,y^\star) \d x \d y \\
    & \quad \quad - \int_{\ocal} \int_{\ocal}   \big(v(x) + v(y)\big)\big(w(x) - w(y)\big) \overline{K} (x,y^\star) \d x \d y \\
    &\quad \quad  - \int_{\ocal} \int_{\ocal}  \big(v(x) - v(y)\big)\big(w(x) - w(y)\big) \overline{K} (x,y) \d x \d y \\
    &= 0\,.
    \end{align*}
\end{proof}


The last result we will use is the following.

\begin{proposition}
	\label{Prop:WeakSolutionForAllTestFunctions}
	Let $\Omega \subset \R^{2m}$ be a bounded set of double revolution. Let $u\in \widetilde{\H}^K_{0}(\Omega)$ such that 
	$$
	\int_{\R^{2m}}\int_{\R^{2m}} \{u(x)-u(y)\}\{\xi(x)-\xi(y)\} K(|x-y|) \d x \d y = \int_{\R^{2m}} f(u(x)) \xi(x) \d x
	$$
	for every $\xi \in C^\infty_0(\Omega)$ that is doubly radial. Then, $u$ is a weak solution of
	$$
	\beqc{\PDEsystem}
	Lu &=& f(u) & \text{in } \Omega\,,\\
	u &=& 0 & \text{in } \R^{2m}\setminus \Omega\,,
	\eeqc
	$$
	i.e., 
	$$
	\int_{\R^{2m}}\int_{\R^{2m}} \{u(x)-u(y)\}\{\eta(x)-\eta(y)\} K(|x-y|) \d x \d y = \int_{\R^{2m}} f(u(x)) \eta(x) \d x
	$$
	for every $\eta \in C^\infty_0(\Omega)$ (not necessarily symmetric).
\end{proposition}

\begin{proof}
	Let $\eta \in C^\infty_0(\Omega)$. Then, given $R\in SO(m)^2$, 
	\begin{align*}
	&\int_{\R^{2m}}\int_{\R^{2m}} \{u(x)-u(y)\}\{\eta(x)-\eta(y)\} K(|x-y|) \d x \d y = \\
	&\quad \quad \quad = \int_{\R^{2m}}\int_{\R^{2m}} \{u(R^{-1}x)-u(R^{-1}y)\}\{\eta(x)-\eta(y)\} K(|x-y|) \d x \d y \\
	&\quad \quad \quad = \int_{\R^{2m}}\int_{\R^{2m}} \{u(x)-u(y)\}\{\eta(R x)-\eta(R y)\} K(|x-y|) \d x \d y\,,
	\end{align*}
	where we have used the change $x = R\tilde{x}$, $y = R \tilde{y}$. Integrating the previous expression with respect to $R$ and taking the average, we get
	\begin{align*}
	& \int_{\R^{2m}}\int_{\R^{2m}} \{u(x)-u(y)\}\{\eta(x)-\eta(y)\} K(|x-y|) \d x \d y = \\
	&\quad \quad \quad =\average_{SO(m)^2} \int_{\R^{2m}}\int_{\R^{2m}} \{u(x)-u(y)\}\{\eta(R x)-\eta(R y)\} K(|x-y|) \d x \d y \d R \\
	&\quad \quad \quad= \int_{\R^{2m}}\int_{\R^{2m}} \{u(x)-u(y)\}\left \{\average_{SO(m)^2}\eta(R x)\d R-\average_{SO(m)^2} \eta(Ry) \d R \right \} K(|x-y|) \d x \d y \\
	&\quad \quad \quad= \int_{\R^{2m}}\int_{\R^{2m}} \{u(x)-u(y)\}\left \{\overline{\eta}(x) -\overline{\eta}(y)  \right \} K(|x-y|) \d x \d y \,.
	\end{align*}
	Here we have used the notation
	$$
	\overline{\eta}(x) := \average_{SO(m)^2}\eta(R x)\d R\,.
	$$
	On the other hand, using the change $x = R\tilde{x}$, we have
	$$
	\int_{\Omega} f(u(x)) \eta(x) \d x = \int_{\Omega} f(u(R^{-1}x)) \eta(x) \d x = \int_{\Omega} f(u(x)) \eta(Rx) \d x\,,
	$$
	and integrating this expression with respect to $R$ and taking the average, we get
	$$
	\int_{\Omega} f(u(x)) \eta(x) \d x = \average_{SO(m)^2} \int_{\Omega} f(u(x)) \eta(Rx) \d x \d R = \int_{\Omega} f(u(x))\overline{\eta}(x) \d x\,.  
	$$
	
	Hence, since $\overline{\eta} \in C^\infty_0(\Omega)$ is double radially symmetric, we have
	\begin{align*}
		&\int_{\R^{2m}}\int_{\R^{2m}} \{u(x)-u(y)\}\{\eta(x)-\eta(y)\} K(|x-y|) \d x \d y - \int_{\Omega} f(u(x)) \eta(x) \d x \\
		&\quad \quad= \int_{\R^{2m}}\int_{\R^{2m}} \{u(x)-u(y)\}\left \{\overline{\eta}(x) -\overline{\eta}(y)  \right \} K(|x-y|) \d x \d y - \int_{\Omega} f(u(x))\overline{\eta}(x) \d x \\
		&\quad \quad= 0\,,
	\end{align*}
	and thus the result is proved.
\end{proof}

Note that in the previous result we do not need to use the kernel $\overline{K}$. 


%%%%%%%%%%%%%%%%%%%%%%%%%%%%%%%%%%%%%%%%%%%%%%%%%%%%%%%%%%%%%%%%%%%%%%
%%%%%%%%%%%%%%%%%%%%%%%%%%%%%%%%%%%%%%%%%%%%%%%%%%%%%%%%%%%%%%%%%%%%%%
\subsection{Energy estimate}

In this section we present a sharp estimate for the energy in $B_S$ of minimizers in the space $\widetilde{\H}^K_{0, \textrm{odd}}(B_R)$ of the energy in $B_R$.

In order to prove this result we need to define some auxiliary functions and sets. That is,

$$ \Psi_S(x) := \max\left\{-1+2\min\{(|x|-S-1)_+,1\},-\dist(x,\ccal) \right\},  $$
if $x\in \ocal$ and odd reflected in $\ical$,
$$ d_S(x) := \max\left\{1,\min\{S+1-|x|,\dist(x,\ccal)\} \right\},  $$
and
$$ \Omega_S := \left( B_{S+2}\setminus \overline{B_s} \right) \cup \left( B_{S+2} \cap \{\dist(x,\ccal)< 1\}\right). $$

First note that both $\Psi_S$ and $d_s$ are Lipschitz functions, with Lipschitz norm independent of $S$. Moreover $\Psi_S$ is odd and $d_s$ even with respect to the Simon's cone. On the other hand, we can see $\overline{\Omega_S}$ as the preimage of $1$ through $d_S$ inside $\overline{B_{S+2}}$.

Now we show some auxiliary results concerning the previous definitions.

\begin{lemma}
\label{Lemma: AdaptedLipschitzConditionWith_dFunction}
Given $S>0$, if $(x,y) \in \left(\Omega_S\cap \ocal\right) \times \ical$ or $(x,y)\in \left(B_{S+2}\cap \ocal\right) \times \ocal$, then
$$ |\Psi_S(x) - \Psi_S(y)| \leq C \frac{|x-y|}{d_S(x)} \ \ \ \ \ \textrm{whenever} \ \ |x-y|\leq d_S(x), $$
with $C>0$ independent of $S$.
\end{lemma}

\begin{proof}
On the one hand, if $x\in \Omega_S \cap \ocal$, then $d_S(x)=1$ and the result is trivial by the Lipschitz continuity of $\Psi_S$. 

Hence, it only rests to show the result for the case $x\in B_S\cap \{\dist(x,\ccal)\geq 1\} \cap \ocal$ and $y\in \ocal$. For this case we have $\Psi_S(x)=-1$. Moreover, since $x\in B_S$ and $\dist(x,\ccal)\geq 1$ we get
$$ d_S(x) = \min\left\{S+1-|x|,\dist(x,\ccal)\right\} \leq S+1-|x|,$$
and therefore
$$ |y|\leq |x-y| + |x| \leq d_S(x)+|x| \leq S+1. $$
In addition, if $y\in B_{S+1} \cap \{\dist(x,\ccal)\geq 1\}\cap \ocal$ the $\Psi_S(y)=-1$, and the result is trivial from being also $\Psi_S(x)=-1$.

Therefore, we have proven the result for all the cases with the exception of
$$\begin{cases}
x\in B_S \cap \{\dist(\cdot,\ccal)\geq 1\}\cap \ocal \ \ \ &\Rightarrow \ \ \ \Psi(x)=-1 \\
y\in B_{S+1} \cap \{\dist(\cdot,\ccal)\leq 1\}\cap \ocal \ \ \ &\Rightarrow \ \ \ \Psi(y)=-\dist(y,\ccal). \\
\end{cases}$$

Given $x,y \in \R^{2m}$ it is easy to prove by using the triangular inequality and the definition of distance to the cone that
\begin{equation} \label{eq: tirangularCone}
\dist(x,\ccal) \leq |x-y| + \dist(y,\ccal).
\end{equation}
Therefore we have
\begin{equation} \label{eq: tirangularCone2}
1-|x-y|-\dist(y,\ccal) \leq 1-\dist(x,\ccal) \leq 0
\end{equation}
Now, multiplying $|1-\dist(y,\ccal)|$ by equation \eqref{eq: tirangularCone} and using \eqref{eq: tirangularCone2} we obtain
\begin{align*}
|1-\dist(y,\ccal)|\,\dist(x,\ccal) &\leq |1-\dist(y,\ccal)| \left(|x-y| + \dist(y,\ccal)\right) \\
&= \left(1-\dist(y,\ccal)\right) \left(|x-y| + \dist(y,\ccal)\right) \\
&= |x-y|+\dist(y,\ccal) \left\{ -|x-y|+ 1- \dist(y,\ccal) \right\} \\
&\leq |x-y|.
\end{align*}

Hence,
$$ |\Psi_S(x)-\Psi_S(y)| = |1-\dist(y,\ccal)| \leq \frac{|x-y|}{\dist(x,\ccal)} \leq  \frac{|x-y|}{d_S(x)},$$
completing the proof.
\end{proof}

\begin{lemma}
\label{Lemma: Integrability_dFunction}
Given $S>0$ we have
$$ \int_{B_{S+2}} d_S(x)^{-2\s} \d x \leq \begin{cases}
C \ S^{2m-2\s}\ \ \ \ &\textrm{if } \ \ \s\in(0,1/2),\\
C\ \log(S)\,S^{2m-2\s}\ \ \ \ &\textrm{if } \ \ \s=1/2,\\
C \ S^{2m-1}\ \ \ \ &\textrm{if } \ \ \s\in(1/2,1),\\
\end{cases} $$
with $C>0$ independent of $S$ and only depending on $m$ and $\s$.
\end{lemma}

\begin{proof}
In order to prove this result we first note that $d_S(x)=1$ in $\Omega_S$. Thus, the contribution to the integral of this part is just its measure, which is well known to be of order $2m-1$ (see the proof of the energy estimate in \cite{CabreTerraI}). That is,
$$\int_{\Omega_S} d_S(x)^{-2\s} \d x \leq C\,S^{2m-1}.$$

For the other part of the integral we can write
\begin{align*}
\int_{B_{S+2}\setminus \Omega} d_S(x)^{-2\s} \d x &= \int_{B_{S}\cap \dist\{x,\ccal\}>1} d_S(x)^{-2\s} \d x \\
& \leq \int_{B_{S}\cap \dist\{x,\ccal\}>1} \left( S+1-|x| \right)^{-2\s} \d x + \int_{B_{S}\cap \dist\{x,\ccal\}>1} \dist(x,\ccal)^{-2\s} \d x.
\end{align*}
Note that from the computations of Savin and Valdinocci in \cite{SavinValdinoci-EnergyEstimate}, in order to complete the proof it only remains to estimate the second integral.

This integral can be estimated by writing it in $(y,z)$ variables, since $z$ is the distance to the cone. That is,
\begin{align*}
\int_{B_{S}\cap \dist\{x,\ccal\}>1} \dist(x,\ccal)^{-2\s} \d x &\leq C \int \int_{B_{S}\cap \{y\geq|z|>1\}} |z|^{-2\s} \, (y^2-z^2)^{m-1} \d y\d z \\
& \leq C \int \int_{B_{S}\cap \{y\geq|z|>1\}} |z|^{-2\s} \, y^{2m-2} \d y\d z \\
& \leq C\, \int_1^S \d z \int_0^S \d y\ z^{-2\s} \, y^{2m-2} \\
& \leq C\, \left(\int_1^S z^{-2\s} \d z \right)  \left(  \int_0^S \d y \, y^{2m-2} \right) \\
& \leq \begin{cases}
C \ S^{2m-2\s}\ \ \ \ &\textrm{if } \ \ \s\in(0,1/2),\\
C\ \log(S)\,S^{2m-2\s}\ \ \ \ &\textrm{if } \ \ \s=1/2,\\
C \ S^{2m-1}\ \ \ \ &\textrm{if } \ \ \s\in(1/2,1),\\
\end{cases}
\end{align*}
\end{proof}

\begin{lemma}
\label{Lemma: InteractionInequalityMinimumFunction}
\todo{Escribir bien las condiciones de las funciones}
Let $A\subset\R^{2m}$ be a set of double revolution such that $A^\star = A$ and let be $\omega, \phi, \varphi \in $ such that
$$\begin{cases}
\omega = \phi \leq \varphi \ \ \ \ \textrm{in } \ \ \ \ocal \setminus A\,,\\
\omega = \varphi \leq \phi \ \ \ \ \textrm{in } \ \ \ \ocal \cap A\,.
\end{cases}$$
Then,
\begin{align*}
I_\omega(\ocal\cap A, \ocal \setminus A) + I_\omega^\star(\ocal\cap A, \ocal \setminus A) &\leq I_\phi(\ocal\cap A, \ocal \setminus A) + I_\phi^\star(\ocal\cap A, \ocal \setminus A)\\
&\hspace{5mm} + I_\varphi(\ocal\cap A, \ocal \setminus A) + I_\varphi^\star(\ocal\cap A, \ocal \setminus A)
\end{align*}
\end{lemma}

\begin{proof}
Let us begin by proving that given $x\in \ocal \cap A$ and $y\in \ocal \setminus A$ we have that
$$ |\phi(x)-\phi(y)|^2+|\varphi(x)-\varphi(y)|^2\geq |\omega(x)-\omega(y)|^2. $$
That is,
\begin{align*}
|\phi(x)-\phi(y)|^2+|\varphi(x)&-\varphi(y)|^2 - |\omega(x)-\omega(y)|^2 \\
&= |\phi(x)-\phi(y)|^2+|\varphi(x)-\varphi(y)|^2 - |\varphi(x)-\phi(y)|^2 \\
&= \phi^2(x)-2\phi(x)\phi(y)+\varphi^2(y)-2\varphi(x)\varphi(y)+2\varphi(x)\phi(y) \\
&= \left( \phi(x) - \varphi(y)\right) ^2+2\left( \phi(x)-\varphi(x) \right) \left( \varphi(y)-\phi(y) \right) \\
&\geq 0.
\end{align*}
Therefore, by using this inequality and the kernel's reflexion property (Proposition \ref{Prop:KernelInequalityReflexion}) we obtain
\todo{Hay que alinear bien las ecuaciones}
\begin{align*}
I_\phi(\ocal\cap A, \ocal \setminus A) &+ I_\phi^\star(\ocal\cap A, \ocal \setminus A) + I_\varphi(\ocal\cap A, \ocal \setminus A) + I_\varphi^\star(\ocal\cap A, \ocal \setminus A) \\
&\hspace{6mm}- I_\omega(\ocal\cap A, \ocal \setminus A) - I_\omega^\star(\ocal\cap A, \ocal \setminus A)\\
&\hspace{-26mm}= \int_{\ocal\cap A} \d x \int_{\ocal\setminus A} \d y \left[\left\{|\phi(x)-\phi(y)|^2+|\varphi(x)-\varphi(y)|^2-|\omega(x)-\omega(y)|^2 \right\} \overline{K}(x,y) \right. \\
&\hspace{6mm}+ \left.\left\{|\phi(x)+\phi(y)|^2+|\varphi(x)+\varphi(y)|^2-|\omega(x)+\omega(y)|^2 \right\} \overline{K}(x,y^\star) \right] \\
&\hspace{-26mm}\geq \int_{\ocal\cap A} \d x \int_{\ocal\setminus A} \d y \left\{|\phi(x)-\phi(y)|^2+|\varphi(x)-\varphi(y)|^2-|\omega(x)-\omega(y)|^2 \right.\\ &\hspace{6mm}+\left.|\phi(x)+\phi(y)|^2+|\varphi(x)+\varphi(y)|^2-|\omega(x)+\omega(y)|^2 \right\} \overline{K}(x,y^\star)  \\
&\hspace{-26mm}= \int_{\ocal\cap A} \d x \int_{\ocal\setminus A} \d y \left\{ 2\phi^2(x)+2\varphi^2(y)\right\} \overline{K}(x,y^\star) \geq 0.
\end{align*}
Here we have used that $\omega(x) = \varphi(x)$ if $x\in \ocal\cap A$ and that $\omega(y) = \phi(y)$ if $y\in \ocal \setminus A$.
\end{proof}

\begin{theorem}
\label{Th:EnergyEstimate}
Let $u$ be a minimizer of the non-local Allen-Cahn energy in $B_{R}$, with $R>S+2$, among functions that are doubly radial, odd with respect to the Simon's cone and zero outside $B_R$. Then
%$$ \lim_{R\to +\infty} \frac{1}{S^n} \ecal (u,B_S) = 0. $$
%More precisely,
$$ \ecal (u,B_S) \leq \begin{cases}
C \ S^{2m-2\s}\ \ \ \ &\textrm{if } \ \ \s\in(0,1/2),\\
C\ \log(S)\,S^{2m-2\s}\ \ \ \ &\textrm{if } \ \ \s=1/2,\\
C \ S^{2m-1}\ \ \ \ &\textrm{if } \ \ \s\in(1/2,1),\\
\end{cases} $$
with $C$ a positive constant depending only on $m$, $\s$, $\Lambda$ and $G$.
\end{theorem}

\begin{proof}

Note that, by Lemmas~\ref{Lemma:TruncationOfFunctions1DecreaseEnergy} and \ref{Lemma:TruncationOfFunctions2DecreaseEnergy} we can assume without loss of generality that $-1 \leq u \leq 1$ and that $u \geq 0$ in $\ocal$ and $u \leq 0$ in $\ical$. In fact, it also true that $0\leq u < 1$ in $\ocal$. In order to prove it we first need to show that $u$ is a weak solution of
\begin{equation}
\label{Eq:ProofExistenceProblemBR}
	\beqc{\PDEsystem}
	L u &=& f(u) & \textrm{ in } B_R\,,\\
	u &=& 0 & \textrm{ in }\R^{2m} \setminus B_R.
	\eeqc
\end{equation}
To see this, we consider on the one hand perturbations $u +  \varepsilon \xi$, with $\xi \in \widetilde{\H}^K_{0, \,\mathrm{odd}}(B_R)$ and such that $\xi$ has compact support in $B_R$. Then, 
$$
0 = \dfrac{\d}{\d \varepsilon}\evalat{\varepsilon = 0} \ecal(u +  \varepsilon \xi, B_R) = \langle u,\xi \rangle_{\widetilde{\H}^K_0(B_R)} - \langle f(u),\xi \rangle_{L^2(B_R)}\,.
$$
On the other hand, take $\xi \in \widetilde{\H}^K_{0, \,\mathrm{even}}(B_R)$. Since $u$ is odd with respect to the Simons cone, so is $f(u)$. Then, by Lemma~\ref{Lemma:DecompositionHK} and the same decomposition in $L^2(B_R)$, we find that
$$
\langle v_R,\xi \rangle_{\widetilde{\H}^K_0(B_R)} = 0 \quad \textrm{ and } \quad  \langle f(v_R),\xi \rangle_{L^2(B_R)} = 0\,.
$$
Therefore, we have that
$$
\langle u,\xi \rangle_{\widetilde{\H}^K_0(B_R)} = \langle f(u),\xi \rangle_{L^2(B_R)}
$$
for every $\xi \in\widetilde{\H}^K_0(B_R)$ with compact support in  $B_R$. Therefore,
$$
\int_{\R^{2m}}\int_{\R^{2m}} \{u(x)-u(y)\}\{\xi(x)-\xi(y)\} K(|x-y|) \d x \d y = \int_{\R^{2m}} f(u(x)) \xi(x) \d x
$$
for every $\xi \in C^\infty_0(\Omega)$ that is double radially symmetric.

By Proposition~\ref{Prop:WeakSolutionForAllTestFunctions}, $u$ is a weak solution of \eqref{Eq:ProofExistenceProblemBR}, and by the regularity result of Corollary \ref{Cor:C2regularity}, since $u$ is bounded, it is a classical solution.

From being $u$ a classical solution it is easy to show that it cannot be $1$ or $-1$ and therefore that it satisfies $0\leq u < 1$ in $\ocal$. That is, let us suppose that there exists $x_0\in\R^{2m}$ such that $|u(x_0)|=1$. It is clear that we can take $x_0\in\ocal\cap B_R$. Then, from equation \eqref{Eq:ProofExistenceProblemBR} and the fact of being $x_0$ an absolute maximum we can arrive at a contradiction:
\begin{align*}
0 &= f(1) = f(u(x_0)) = Lu(x_0) = \int_\ocal (1-u(y)) \overline{K}(x,y) + (1+u(y)) \overline{K}(x,y^\star)  \d y \\
&\geq \int_\ocal (1-u(y)) \overline{K}(x,y^\star) + (1+u(y)) \overline{K}(x,y^\star)  \d y = 2\int_\ocal \overline{K}(x,y^\star) \d y\\
&>0.
\end{align*}


Now we introduce the function
$$ v(x) := \min\{u(x),\Psi_S(x)\}, $$
if $x\in \ocal$ and odd reflected in $\ical$. Since both $u$ and $v$ are equal outside $B_{S+2} \subset B_R$, $v$ is going to be the competitor that will give us the energy estimate. Let us also define
$$ A = \{v=\Psi_S\}. $$ 
Then it is easy to check that we have the inclusions
$$ B_{S+1} \subseteq A \subseteq B_{S+2}. $$
Since $A$ is symmetric with respect to the Simon's cone we only need to prove it inside $\ocal$. That is, on the one hand
$$ x\in B_{S+1}\cap \ocal \Rightarrow \Psi_S(x) = \max\{-1,-\dist(x,\ccal)\} \leq 0 \leq u(x) \Rightarrow v(x) = \Psi_S(x)  \Rightarrow x\in A\cap \ocal,  $$
and on the other hand
$$ x\in A\cap \ocal \Rightarrow \Psi_S(x) \leq u(x) < 1 \Rightarrow x\in B_{S+2}.  $$

Let us decompose the energy of $u$ in $B_R$ in terms of interactions between sets that involve $A$. That is,
\begin{align*}
\ecal(u,B_R) &= \frac{1}{2}I_u^\star(\ocal \cap A, \ocal \cap A) + \frac{1}{2}I_u(\ocal \cap A, \ocal \cap A)\\
& \hspace{5mm} + I_u^\star(\ocal \cap A, \ocal \setminus A) + I_u(\ocal \cap A, \ocal \setminus A) \\
& \hspace{5mm} + \frac{1}{2}I_u^\star(\ocal \setminus A \cap B_R, \ocal \setminus A \cap B_R) + \frac{1}{2}I_u(\ocal \setminus A \cap B_R, \ocal \setminus A \cap B_R)\\
& \hspace{5mm} + I_u^\star(\ocal \setminus A \cap B_R, \ocal \setminus B_R) + I_u(\ocal \setminus A \cap B_R, \ocal \setminus B_R) \\
& \hspace{5mm} + \int_A G(u) + \int_{B_R\setminus A} G(u)
\end{align*}
Since $u$ is a minimizer, $v=\Psi_S$ in $A$ and $u=v$ out of $A$, we obtain from the previous expression
\begin{align*}
0 &\leq \ecal(v,B_R)-\ecal(u,B_R)\\
&= \frac{1}{2}I_{\Psi_S}^+(\ocal \cap A, \ocal \cap A) + \frac{1}{2}I_{\Psi_S}^-(\ocal \cap A, \ocal \cap A)\\
& \hspace{5mm} -\frac{1}{2}I_u^+(\ocal \cap A, \ocal \cap A) - \frac{1}{2}I_u^-(\ocal \cap A, \ocal \cap A)\\
& \hspace{5mm} + I_v^+(\ocal \cap A, \ocal \setminus A) + I_v^-(\ocal \cap A, \ocal \setminus A) \\
& \hspace{5mm} - I_u^+(\ocal \cap A, \ocal \setminus A) - I_u^-(\ocal \cap A, \ocal \setminus A) \\
& \hspace{5mm} + \int_A G(\Psi_S) - \int_{A} G(u) 
\end{align*}
Since $v = \min\{u,\Psi_S\}$ in $\ocal$ we can apply Lemma \ref{Lemma: InteractionInequalityMinimumFunction} to obtain 
\begin{align*}
\frac{1}{2}I_u^\star(\ocal \cap A, \ocal \cap A) + \frac{1}{2}I_u(\ocal \cap A, \ocal \cap A) + \int_{A} G(u)\\ 
&\hspace{-65mm}\leq \frac{1}{2}I_{\Psi_S}^\star(\ocal \cap A, \ocal \cap A) + \frac{1}{2}I_{\Psi_S}(\ocal \cap A, \ocal \cap A)\\
&\hspace{-60mm} + I_{\Psi_S}^\star(\ocal \cap A, \ocal \setminus A) + I_{\Psi_S}(\ocal \cap A, \ocal \setminus A) + \int_A G(\Psi_S)  \\
&\hspace{-65mm} = \ecal(\Psi_S, A) \leq \ecal(\Psi_S,B_{S+2})
\end{align*}
Therefore we get an estimate of the energy of $u$ in $B_S$. That is,
\begin{align*}
\ecal(u,B_S) &\leq \frac{1}{2}I_u^\star(\ocal \cap A, \ocal \cap A) + \frac{1}{2}I_u(\ocal \cap A, \ocal \cap A) + \int_{A} G(u)\\
& \hspace{5mm} +I_u^\star(\ocal \setminus B_{S+1}, \ocal \cap B_S) + I_u(\ocal \setminus B_{S+1}, \ocal \cap B_S) \\
& \leq I_u^\star(\ocal \setminus B_{S+1}, \ocal \cap B_S) + I_u(\ocal \setminus B_{S+1}, \ocal \cap B_S) \\
& \hspace{5mm} + \ecal(\Psi_S,B_{S+2})
\end{align*}
Once we are at this point we only have to bound this three terms in order to obtain the desired energy estimate.

\begin{itemize}
\item Estimate for $\ecal(\Psi_S,B_{S+2})$\\
In order to make this estimate we use the definition of the energy that involves the original kernel $K$ and not the adapted one $\overline{K}$. That is,
\begin{align*}
\ecal(\Psi_S,B_{S+2}) &= \frac{1}{4} \int_{B_{S+2}} \int_{B_{S+2}} |\Psi_S(x)-\Psi(y)|^2K(|x-y|) \d x\d y \\
&\hspace{5mm} +\frac{1}{2} \int_{B_{S+2}} \int_{\R^{2m} \setminus B_{S+2}} |\Psi_S(x)-\Psi(y)|^2K(|x-y|) \d x\d y + \int_{B_{S+2}} G(\Psi_S) \\
&\leq \frac{1}{2} \int_{B_{S+2}} \int_{\R^{2m}} |\Psi_S(x)-\Psi(y)|^2K(|x-y|) \d x\d y + \int_{B_{S+2}} G(\Psi_S) \\
&= \int_{\ocal \cap B_{S+2}} \int_{\R^{2m}} |\Psi_S(x)-\Psi(y)|^2K(|x-y|) \d x\d y + \int_{B_{S+2}} G(\Psi_S),
\end{align*}
where last inequality comes from making the change of variables $x'=x^\star$ and $y'=y^\star$ and the fact that $|x-y|=|x^\star-y^\star|$. Now, by using the ellipticity condition:
\begin{align*}
\ecal(\Psi_S,B_{S+2}) &\leq \Lambda \int_{\ocal \cap B_{S+2}} \int_{\R^{2m}} \frac{|\Psi_S(x)-\Psi(y)|^2}{|x-y|^{n+2\s}} \d x\d y + \int_{B_{S+2}} G(\Psi_S)\\
&= \Lambda \int_{\ocal \cap B_{S+2}} \int_{\ocal} \frac{|\Psi_S(x)-\Psi(y)|^2}{|x-y|^{n+2\s}} \d x\d y \\
&\hspace{5mm} + \Lambda \int_{\Omega \cap \ocal} \int_{\ical} \frac{|\Psi_S(x)-\Psi(y)|^2}{|x-y|^{n+2\s}} \d x\d y \\
&\hspace{5mm} + \Lambda \int_{(B_{S+2}\setminus \Omega) \cap \ocal} \int_{\ical} \frac{|\Psi_S(x)-\Psi(y)|^2}{|x-y|^{n+2\s}} \d x\d y + \int_{B_{S+2}} G(\Psi_S) \\
&=:I_1+I_2+I_3+I_4.
\end{align*}
Let us compute this four integrals:
\begin{align*}
I_1 &= \Lambda \int_{\ocal \cap B_{S+2}} \int_{\ocal} \frac{|\Psi_S(x)-\Psi(y)|^2}{|x-y|^{n+2\s}} \d x\d y\\
&= \Lambda \int_{\ocal \cap B_{S+2}} \int_{\ocal\cap\{|x-y|\leq d_S(x)\}} \frac{|\Psi_S(x)-\Psi(y)|^2}{|x-y|^{n+2\s}} \d x\d y\\
&\hspace{5mm} + \Lambda \int_{\ocal \cap B_{S+2}} \int_{\ocal\cap\{|x-y|\geq d_S(x)\}} \frac{|\Psi_S(x)-\Psi(y)|^2}{|x-y|^{n+2\s}} \d x\d y\\
&\leq C \int_{\ocal \cap B_{S+2}} d_S(x)^{-2}\int_{\ocal\cap\{|x-y|\leq d_S(x)\}} |x-y|^{2-n-2\s} \d y\d x\\
&\hspace{5mm} + C \int_{\ocal \cap B_{S+2}} \int_{\ocal\cap\{|x-y|\geq d_S(x)\}} |x-y|^{-n-2\s} \d x\d y\\
&\leq C \int_{\ocal \cap B_{S+2}} d_S(x)^{-2}\int_0^{d_S(x)} \rho^{1-2\s} \d \rho\d x + C \int_{\ocal \cap B_{S+2}} \d x \int_{d_S(x)}^\infty \rho^{-1-2\s} \d\rho\\
&\leq C \int_{\ocal \cap B_{S+2}} d_S(x)^{-2\s} \d x,
\end{align*}
where in the first inequality we have used Lemma \ref{Lemma: AdaptedLipschitzConditionWith_dFunction} and the uniform bound of $\Psi_S$. The bound of $I_2$ is essentially the same. That is,
\begin{align*}
I_2 &= \Lambda \int_{\Omega \cap \ocal} \int_{\ical} \frac{|\Psi_S(x)-\Psi(y)|^2}{|x-y|^{n+2\s}} \d x\d y\\
&= \Lambda \int_{\Omega \cap \ocal} \int_{\ical\cap\{|x-y|\leq d_S(x)\}} \frac{|\Psi_S(x)-\Psi(y)|^2}{|x-y|^{n+2\s}} \d x\d y\\
&\hspace{5mm} + \Lambda \int_{\Omega \cap \ocal} \int_{\ical\cap\{|x-y|\geq d_S(x)\}} \frac{|\Psi_S(x)-\Psi(y)|^2}{|x-y|^{n+2\s}} \d x\d y\\
&\leq C \int_{\Omega \cap \ocal} d_S(x)^{-2}\int_0^{d_S(x)} \rho^{1-2\s} \d \rho\d x + C \int_{\Omega \cap \ocal} \d x \int_{d_S(x)}^\infty \rho^{-1-2\s} \d\rho\\
&\leq C \int_{\Omega \cap \ocal} d_S(x)^{-2\s} \d x \leq C \int_{\ocal \cap B_{S+2}} d_S(x)^{-2\s} \d x.
\end{align*}
For the case of $I_3$ we use the fact that given $x\in (B_{S+2}\setminus \Omega)\cap \ocal$ then $\dist(x,\ccal)\geq d_S(x)$ and therefore $\ical \subset \R^{2m}\setminus B_{d_S(x)}(x)$.
\begin{align*}
I_3 &= \Lambda \int_{(B_{S+2}\setminus \Omega) \cap \ocal} \int_{\ical} \frac{|\Psi_S(x)-\Psi(y)|^2}{|x-y|^{n+2\s}} \d x\d y\\
&\leq C \int_{(B_{S+2}\setminus \Omega) \cap \ocal} \int_{\R^{2m}\setminus B_{d_S(x)}(x)} |x-y|^{-n-2\s} \\
&\leq C \int_{\ocal \cap B_{S+2}} \int_{d_S(x)}^\infty \rho^{-1-2\s} \d \rho\d x \\
&\leq C \int_{\ocal \cap B_{S+2}} d_S(x)^{-2\s} \d x.
\end{align*}
Now, for the case of $I_4$, since $\Psi_S\equiv -1$ in $\Omega_S$ we have
\begin{align*}
I_4 = \int_{B_{S+2}} G(\Psi_S) = \int_{\Omega_S} G(\Psi_S) + \int_{B_{S+2}\setminus \Omega_S} G(\Psi_S) \leq C |B_{S+2}\setminus \Omega_S| \leq C\,S^{2m-1}
\end{align*}
Then, we obtain
\begin{align*}
\ecal(\Psi_S,B_{S+2}) &\leq C \left(\int_{\ocal \cap B_{S+2}} d_S(x)^{-2\s} \d x + S^{2m-1} \right)  \\
&\leq C\left(\int_{\ocal \cap B_{S}} d_S(x)^{-2\s} \d x + S^{2m-1} \right)
\end{align*}
\item Estimate for $I_u(\ocal \setminus B_{S+1}, \ocal \cap B_S) + I_u^\star(\ocal \setminus B_{S+1}, \ocal \cap B_S)$
First we prove that if $x\in B_S\cap \ocal$ and $y\in \R^{2m}\setminus B_{S+1}$, then $|x-y|\geq d_S(x)$. It is clear that being $x\in B_S$ then $d_S(x) \leq S+1-|x|$ and therefore we have $|x-y|\geq |y|-|x|\geq |y|+d_S(x)-S-1 \geq  d_S(x)$. Thus we have
\begin{align*}
I_u(\ocal \setminus B_{S+1}, \ocal \cap B_S) &+ I_u^\star(\ocal \setminus B_{S+1}, \ocal \cap B_S) \\
&= \int_{\ocal\cap B_S} \d x \int_{\R^{2m}\setminus B_{S+1}} \d y \ |u(x)-u(y)|^2 \, K(|x-y|) \\
&\leq \int_{\ocal\cap B_S} \d x \int_{\R^{2m}\setminus B_{S+1}} \d y \ \frac{|u(x)-u(y)|^2}{|x-y|^{2m+2\s}} \\
&\leq \int_{\ocal\cap B_S} \d x \int_{|x-y|\geq d_S(x)} \d y \ |x-y|^{-2m-2\s} \\
&\leq C \int_{\ocal \cap B_{S}} d_S(x)^{-2\s} \d x.
\end{align*}
\end{itemize}
Finally, by adding up this estimates and applying Lemma \ref{Lemma: Integrability_dFunction} we finally obtain the desired result. That is,
\begin{align*}
\ecal(u,B_S) &\leq \ecal(\Psi_S,B_{S+2}) + I_u(\ocal \setminus B_{S+1}, \ocal \cap B_S) + I_u^\star(\ocal \setminus B_{S+1}, \ocal \cap B_S) \\
&\leq C\left(\int_{\ocal \cap B_{S}} d_S(x)^{-2\s} \d x + S^{2m-1} \right)\\
&\leq \begin{cases}
C \ S^{2m-2\s}\ \ \ &\textrm{if } \ \ \s\in(0,1/2),\\
C\ \log(S)\,S^{2m-2\s}\ \ \ \ &\textrm{if } \ \ \s=1/2,\\
C \ S^{2m-1}\ \ \ \ &\textrm{if } \ \ \s\in(1/2,1).\\
\end{cases}
\end{align*}

\end{proof}