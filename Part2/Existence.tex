%%%%%%%%%%%%%%%%%%%%%%%%%%%%%%%%%%%%%%%%%%%%%%%%%%
\section{Existence of the saddle-shaped solution}
%%%%%%%%%%%%%%%%%%%%%%%%%%%%%%%%%%%%%%%%%%%%%%%%%%
\label{Sec:Existence}


In this section we give a proof of Theorem~\ref{Th:Existence} based on the maximum principle and the existence of a positive subsolution. To do this, we need a version of the monotone iteration procedure for doubly radial functions which are odd with respect to the Simons cone $\ccal$. Along this section we will call odd sub/supersolutions to problem \eqref{Eq:SemilinearSolutionInBall} the functions that are doubly radial, odd with respect to the Simons cone and satisfy the corresponding problem in \eqref{Eq:SemilinearSubSuperSolutionInBall}. In view of Remark~\ref{Remark:MaxPrincipleSingularity}, we do not need the operator to be finite in the whole set when applied to a subsolution (respectively supersolution), it can be $-\infty$ (respectively $+\infty$) at some points.

\begin{proposition}
	\label{Prop:MonotoneIterationOdd}
	Let $L_K\in \lcal_\star(2m,\s)$ for some $\s\in (0,1)$. Assume that $\vsub \leq \vsup$ are two bounded functions which are doubly radial and odd with respect to the Simons cone. Furthermore, assume that $\vsub\in C^\s(\R^{2m})$ and that $\vsub$ and $\vsup$ satisfy respectively   
	\begin{equation}
	\label{Eq:SemilinearSubSuperSolutionInBall}
	\beqc{\PDEsystem}
	L_K\vsub & \leq & f(\vsub) & \textrm{ in } B_R \cap \ocal\,, \\
	\vsub & \leq & \varphi & \textrm{ in } \ocal \setminus B_R\,, 
	\eeqc
	\quad \textrm{ and } \quad 
	\beqc{\PDEsystem}
	L_K\vsup & \geq & f(\vsup) & \textrm{ in } B_R \cap \ocal\,, \\
	\vsup & \geq & \varphi & \textrm{ in } \ocal \setminus B_R\,, 
	\eeqc
	\end{equation}
	with $f$ a $C^1$ odd function and $\varphi$ a doubly radial odd function.
	
	Then, there exists $v\in C^{2\s+\varepsilon}(B_R)$ for some $\varepsilon>0$, a solution to
	\begin{equation}
	\label{Eq:SemilinearSolutionInBall}
	\beqc{\PDEsystem}
	L_K v & = & f(v) & \textrm{ in } B_R\,, \\
	v &=& \varphi &  \textrm{ in } \R^{2m} \setminus B_R\,, 
	\eeqc
	\end{equation}
	such that $v$ is doubly radial, odd with respect to the Simons cone and  $\vsub \leq v \leq \vsup$ in $\ocal$.
\end{proposition}


\begin{proof}
	The proof follows the classical monotone iteration method for elliptic equations (see for instance \cite{Evans}). We just give here a sketch of the proof. 
	First, let $M \geq 0$ be such that $-M \leq \vsub \leq \vsup \leq M$ and set
	$$
	b := \max \left \{{0, - \min_{[-M,M]}f'}\right \}\geq 0\,.
	$$
	Then one defines 
	$$
	\tilde{L}_K w := L_Kw + b w 	\quad \text{ and } \quad 	g(\tau) := f(\tau) + b \tau\,.
	$$
	Therefore, our problem is equivalent to find a solution to
	$$
	\beqc{\PDEsystem}
	\tilde{L}_Kv & = & g(v) & \textrm{ in } B_R\,, \\
	v &=& \varphi &  \textrm{ in } \R^{2m} \setminus B_R\,, 
	\eeqc
	$$
	such that $v$ is doubly radial, odd with respect to the Simons cone and  $\vsub \leq v \leq \vsup$ in $\ocal$. Here the main point is that $g$ is also odd but satisfies $g'(\tau) \geq 0$ for $\tau \in [-M,M]$. Moreover, since $b \geq 0$, $\tilde{L}_K$ satisfies the maximum principle for odd functions in $\ocal$ (as in Proposition~\ref{Prop:WeakMaximumPrincipleForOddFunctions}).
	
	We define $v_0 = \vsub$ and, for $k\geq 1$, let $v_k$ be the solution to the linear problem
	$$
	\beqc{\PDEsystem}
	\tilde{L}_K v_k & = & g(v_{k-1}) & \textrm{ in } B_R\,, \\
	v_k &=& \varphi &  \textrm{ in } \R^{2m} \setminus B_R\,. 
	\eeqc
	$$
	It is easy to see by induction and the regularity results from Proposition~\ref{Prop:InteriorRegularity} that $v_k\in L^\infty(B_R) \cap C^{2\s+2\varepsilon}(B_R)$ for some $\varepsilon>0$. Moreover, given $\Omega\subset B_R$ a compact set, then $||v_k||_{C^{2\s+2\varepsilon}(\Omega)}$ is uniformly bounded in $k$.
	
	Then, using the maximum principle it is not difficult to show by induction that 
	$$
	\vsub = v_0 \leq v_1 \leq \ldots \leq v_k \leq v_{k+1} \leq \ldots \vsup \quad \text{ in }\ocal\,,
	$$
	and that each function $v
	_k$ is doubly radial and odd with respect to $\ccal$. Finally, by Arzelà-Ascoli theorem and the compact embedding of H\"older spaces we see that, up to a subsequence, $v_k$ converges uniformly on compacts in $C^{2\s+\varepsilon}$ norm to the desired solution.
\end{proof}

In order to construct a positive subsolution, we also need a characterization and some properties of the first odd eigenfunction and eigenvalue for the operator $L_K$, which are presented next.

\begin{lemma}
	\label{Lemma:FirstOddEigenfunction}
	Let $\Omega\subset \R^{2m} $ be a bounded set of double revolution and let $L_K\in \lcal_\star(2m,\s,\lambda, \Lambda)$. Define 
	\begin{equation}
	\label{Eq:DefLambda1}
	\lambda_{1, \, \mathrm{odd}}(\Omega, L_K) := \inf_{w \in \widetilde{\H}^K_{0, \, \mathrm{odd}}(\Omega)} \dfrac{\dfrac{1}{2}  \ds\int_{\R^{2m}} \int_{\R^{2m}} |w(x) - w(y)|^2 \overline{K}(x,y) \d x \d y}{ \ds \int_\Omega w(x)^2 \d x}\,.
	\end{equation}
	
	Then, such infimum is attained at a function $\phi_1\in \widetilde{\H}^K_{0, \, \mathrm{odd}}(\Omega)\cap L^\infty(\Omega)$ which solves
	$$
	\beqc{\PDEsystem}
	L_K \phi_1 &=& \lambda_{1, \, \mathrm{odd}}(\Omega, L_K) \phi_1 & \textrm{ in } \Omega\,,\\
	\phi_1 & = & 0 & \textrm{ in } \R^{2m}\setminus \Omega\,,
	\eeqc
	$$
	and satisfies that $\phi_1 > 0$ in $\Omega \cap \ocal$.
	We call such function the \emph{first odd eigenfunction of $L_K$ in $\Omega$} and $\lambda_{1, \, \mathrm{odd}}(\Omega, L_K) $ the \emph{first odd eigenvalue}. 
	
	Moreover, in the case $\Omega = B_R$, there exists a constant $C$ depending only on $n$, $\s$ and $\Lambda$ such that
	$$
	\lambda_{1, \, \mathrm{odd}}(B_R, L_K) \leq C R^{-2\s}\,. 
	$$ 
\end{lemma}


\begin{proof}
	The first two statements are deduced exactly as in Proposition~9 of \cite{ServadeiValdinoci}, using the same arguments as in  Lemma~3.4. of \cite{FelipeSanz-Perela:IntegroDifferentialI} to guarantee that $\phi_1$ is nonnegative in $\ocal$. The fact that $\phi_1 > 0$ in $\Omega \cap \ocal$ follows from the strong maximum principle (see Proposition~\ref{Prop:StrongMaximumPrincipleForOddFunctions}).
	
	We show the third statement. Let $\widetilde{w} (x):= w(Rx)$ for every $w\in \widetilde{\H}^K_{0, \, \mathrm{odd}}(B_R)$. Then,
	\begin{align*}
	& \min_{w \in \widetilde{\H}^K_{0, \, \mathrm{odd}}(B_R)} \dfrac{\dfrac{1}{2}  \ds\int_{\R^{2m}} \int_{\R^{2m}} |w(x) - w(y)|^2 \overline{K}(x,y) \d x \d y}{ \ds \int_{B_R} w(x)^2 \d x} \quad \quad \quad \quad \quad \quad \quad \quad \quad \quad \quad \quad\\
	&   \quad \quad \quad \quad \quad \quad \leq \min_{\widetilde{w} \in \widetilde{\H}^K_{0, \, \mathrm{odd}}(B_1)} \dfrac{\dfrac{c_{n, \s}\Lambda}{2}  \ds\int_{\R^{2m}} \int_{\R^{2m}} |\widetilde{w}(x/R) - \widetilde{w}(y/R)|^2 |x - y|^{-n-2 \s}\d x \d y}{ \ds \int_{B_R} \widetilde{w}(x/R)^2 \d x}
	\\
	& \quad \quad \quad \quad \quad \quad = R^{-2 \s }\min_{\widetilde{w} \in \widetilde{\H}^s_{0, \, \mathrm{odd}}(B_1)} \dfrac{\dfrac{c_{n, \s}\Lambda}{2}  \ds\int_{\R^{2m}} \int_{\R^{2m}} |\widetilde{w}(x) - \widetilde{w}(y)|^2 |x - y|^{-n-2 \s}\d x \d y}{ \ds \int_{B_1} \widetilde{w}(x)^2 \d x}
	\\
	& \quad \quad \quad \quad \quad \quad = \lambda_{1, \, \mathrm{odd}}(B_1, \fraclaplacian) \Lambda R^{-2 \s } \,.
	\end{align*}
\end{proof}

\begin{remark}
	\label{Remark:CsRegularityFirstEigenfunction}
	Note that, by standard regularity results for $L_K$, we have that $\phi_1 \in C^\s(\overline{\Omega})\cap C^\infty(\Omega)$, and the regularity up to the boundary is optimal (see \cite{RosOton-Survey} and the references therein for the details). Due to this and the fact that $\phi_1 >0$ in $\Omega\cap \ocal$ while $\phi_1=0$ in $\R^{2m}\setminus \Omega$, it is easy to check by using \eqref{Eq:OperatorOddF} that $-\infty <L_K \phi_1 < 0$ in $\ocal\setminus \overline{\Omega}$ and $L_K \phi_1 = -\infty$ in $\partial \Omega \cap \ocal$.
\end{remark}

With these ingredients, we can proceed with the proof of Theorem~\ref{Th:Existence}.

\begin{proof}[Proof of Theorem~\ref{Th:Existence}]
	The strategy is to build a suitable solution $u_R$ of 
	\begin{equation}
	\label{Eq:ProofExistenceProblemBR}
	\beqc{\PDEsystem}
	L_K u_R &=& f(u_R) & \textrm{ in } B_R\,,\\
	u_R &=& 0 & \textrm{ in }\R^{2m} \setminus B_R\,,
	\eeqc
	\end{equation}
	and then let $R\to+ \infty$ to get a saddle-shaped solution.
	
	Let $\phi_1^{R_0}$ be the first odd eigenfunction of $L_K$ in $B_{R_0} \subset \R^{2m}$, given by Lemma~\ref{Lemma:FirstOddEigenfunction}, and let  $\lambda_1^{R_0} := \lambda_{1, \, \mathrm{odd}}(B_{R_0}, L_K)$. Then, we claim that for $R_0$ big enough and $\varepsilon$ small enough, $\usub_R = \varepsilon\phi_1^{R_0} $ is an odd subsolution of \eqref{Eq:ProofExistenceProblemBR} for every $R\geq R_0$. To see this, first note that, without loss of generality, we can assume that $\norm{\phi_1^{R_0}}_{L^\infty(B_R)}=1$. Then, since $\varepsilon \phi_1^{R_0}>0$ in $B_{R_0}\cap \ocal$ and using \eqref{Eq:PropertyConcavityf}, we see that for every $x\in B_{R_0}\cap \ocal$,
	$$
	\dfrac{f(\varepsilon \phi_1^{R_0}(x))}{\varepsilon \phi_1^{R_0}(x)} > f'(\varepsilon \phi_1^{R_0}(x)) \geq f'(0)/2
	$$
	if $\varepsilon$ is small enough, independently of $x$. Therefore, since $f'(0)>0$, taking $R_0$ big enough so that $\lambda_1^{R_0} < f'(0)/2$ (see the last statement of Lemma~\ref{Lemma:FirstOddEigenfunction}), we have that for every $x\in B_{R_0}\cap \ocal$,  $f(\varepsilon \phi_1^{R_0}(x)) > \lambda_1 \varepsilon \phi_1^R(x)$. Thus,
	$$
	L_K \usub_R = \lambda_1^{R_0} \varepsilon \phi_1^{R_0} < f(\varepsilon\phi_1^{R_0}) = f(\usub_R) \quad \textrm{ in } B_{R_0}\cap \ocal\,.
	$$
	In addition, if $x\in (B_R\setminus B_{R_0})\cap\ocal$, by Remark~\ref{Remark:CsRegularityFirstEigenfunction} we have that
	$$
	L_K \usub_R < 0 = f(0) =  f(\usub_R) \quad \textrm{ in } (B_R\setminus B_{R_0})\cap \ocal\,.
	$$
	Hence, the claim is proved.
	
	Now, if we define $\usup_R := \chi_{\ocal \cap B_R} - \chi_{\ical \cap B_R}$, a simple computation shows that it is an odd supersolution to \eqref{Eq:ProofExistenceProblemBR}. Therefore, using the monotone iteration procedure given in Proposition~\ref{Prop:MonotoneIterationOdd} (taking into account Remarks~\ref{Remark:MaxPrincipleSingularity} and \ref{Remark:CsRegularityFirstEigenfunction} when using the maximum principle), we obtain a solution $u_R$ to \eqref{Eq:ProofExistenceProblemBR} such that it is doubly radial, odd with respect to the Simons cone and $\varepsilon \phi_1^{R_0} = \usub_R \leq u_R \leq \usup_R$ in $\ocal$. Note that, since $\usub_R > 0$ in $\ocal \cap B_{R_0}$, the same holds for $u_R$.
	
	Using a standard compactness argument as in \cite{FelipeSanz-Perela:IntegroDifferentialI}, we let $R\to +\infty$ to obtain a sequence $u_{R_j}$ converging on compacts in  $C^{2\s + \eta}(\R^{2m})$ norm, for some $\eta > 0$, to a solution $u \in C^{2\s + \eta}(\R^{2m})$ of $L_K u = f(u)$ in $\R^{2m}$. Note that $u$ is doubly radial, odd with respect to the Simons cone and $0\leq u \leq 1$ in $\ocal$. Let us show that $0 < u < 1$ in $\ocal$ and hence $u$ is a saddle-shaped solution. Indeed, the usual strong maximum principle yields $u<1$ in $\ocal$. Moreover, since $u_R\geq\varepsilon \phi_1^{R_0}>0$ in  $\ocal \cap B_{R_0}$ for $R>R_0$, also the limit $u\geq\varepsilon \phi_1^{R_0}>0$ in  $\ocal \cap B_{R_0}$. Therefore, by applying the strong maximum principle for odd functions (see Proposition~\ref{Prop:StrongMaximumPrincipleForOddFunctions}) we obtain that $0 < u < 1$ in $\ocal$.
\end{proof}


The fact of being $u$ positive in $\ocal$ yields that $u$ is stable in this set, as explained in the following remark. 


\begin{remark}
	\label{Remark:Stability}
	Note that if $w$ is a bounded positive solution to $L_K w = f(w)$ in a domain $\Omega\subset \R^n$, then $w$ is stable, that is, \eqref{Eq:StablityCondition} holds. The proof of this is rather simple and we present it next. It is a consequence of the fact that, under these assumptions, $w$ is a positive supersolution of the linearized operator $L_K - f'(w)$ (a more detailed discussion can be found in \cite{HamelRosOtonSireValdinoci}). 
	
	On the one hand, note that by \eqref{Eq:PropertyConcavityf}, we have that $f'(w)w<f(w)$ in $\Omega$, since $w$ is positive in $\Omega$. On the other hand, the following holds for all functions $\varphi$ and $\xi$, with $\varphi>0$:
	\begin{equation}
	\label{Eq:IdentityStability}
	\big (\varphi(x) - \varphi(y) \big) \bpar{\dfrac{\xi^2(x)}{\varphi(x)} - \dfrac{\xi^2(y)}{\varphi(y)} } \leq |\xi (x) - \xi(y)|^2\,.
	\end{equation}
	Indeed, developing the squares and the products, this last inequality is equivalent to
	$$
	2 \xi(x) \xi(y) \leq \dfrac{\varphi(x)}{\varphi(y)} \xi^2(y) +  \dfrac{\varphi(y)}{\varphi(x)} \xi^2 (x)\,,
	$$
	which in turn is equivalent to
	$$
	\bpar{\xi (x)\sqrt{\dfrac{\varphi(y)}{\varphi(x)}} - \xi(y) \sqrt{\dfrac{\varphi(x)}{\varphi(y)} } }^2 \geq 0\,.
	$$
	Using these two facts, for every $\xi\in C^\infty_0(\Omega)$ we have
	\begin{align*}
	\int_\Omega f'(w) \xi^2 \d x & \leq \int_\Omega  \dfrac{\xi^2}{w} f(w) \d x = \int_\Omega  \dfrac{\xi^2}{w} L_Kw \d x \\ 
	&= \dfrac{1}{2} \int_{\R^{2m}} \int_{\R^{2m}} \big ( w(x) - w(y) \big) \bpar{\dfrac{\xi^2(x)}{w(x)} - \dfrac{\xi^2(y)}{w(y)} } K(x - y) \d x \d y
	\\ 
	&\leq \dfrac{1}{2} \int_{\R^{2m}} \int_{\R^{2m}} |\xi (x) - \xi(y)|^2 K(x - y) \d x \d y\,.
	\end{align*}
	Thus, $w$ is stable in $\Omega$.	
\end{remark}


