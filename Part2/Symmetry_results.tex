%%%%%%%%%%%%%%%%%%%%%%%%%%%%%
\section{Liouville type and symmetry results}
\label{Sec:SymmetryResults}
%%%%%%%%%%%%%%%%%%%%%%%%%%%%%




This section is devoted to prove the two symmetry results \ref{Th:LiouvilleSemilinearWholeSpace} and \ref{Th:SymmHalfSpace}. Both results will be needed in the following section to establish the asymptotic behavior of the saddle-shaped solution. 

\subsection{A symmetry result for positive solutions in the whole space}

In the proof of Theorem~\ref{Th:LiouvilleSemilinearWholeSpace} we will need two main ingredients, that we present next. The first one is a Harnack inequality for a solution to the semilinear problem \eqref{Eq:PositiveWholeSpace}. This inequality follows readily from the results of Cozzi in \cite{Cozzi-DeGiorgiClassesLong}, although the precise result that we need is not stated there. For the reader's convenience and for future reference, we present the result here and indicate how to deduce it from the results in \cite{Cozzi-DeGiorgiClassesLong}.

\begin{proposition}
	\label{Prop:HarnackSemilinear}
	Let $L_K\in\mathcal{L}_0(n,\s,\lambda,\Lambda)$ and let $w$ be a solution to \eqref{Eq:PositiveWholeSpace} with $f$ a Lipschitz nonlinearity such that $f(0) = 0$. Then, for every $x_0 \in \R^n$ and every $R>0$, it holds
	$$
	\sup_{B_R(x_0)} w \leq C  \inf_{B_R(x_0)} w, 
	$$
	with $C>0$  depending only on $n,\s,\lambda,\Lambda$ and $R$.
\end{proposition}

\begin{proof}
	Following the notation of \cite{Cozzi-DeGiorgiClassesLong}, since $f$ is Lipschitz and $f(0) = 0$, we have
	$$
	|f(u)|\leq d_1 + d_2 |u|^{q-1} \quad \text{ in } \R^n\,,
	$$ 
	with $d_1=0$, $d_2 =||f||_{\mathrm{Lip}}$ and $q=2$. With this choice of the parameters, we only need to repeat the proof of Proposition~8.5 from \cite{Cozzi-DeGiorgiClassesLong} (with $p=2$ and $\Omega = \R^n$) in order to obtain that $u$ belongs to the fractional De Giorgi class $\mathrm{DG}^{\s,2} ( \R^n , 0, H, -\infty,2\s/n,2\s,+\infty)$	for some constant $H>0$ (see \cite{Cozzi-DeGiorgiClassesLong} for the precise definition of these classes). Therefore, the Harnack inequality follows from Theorem~6.9 in \cite{Cozzi-DeGiorgiClassesLong}.
\end{proof}


The second ingredient that we need in the proof of Theorem~\ref{Th:LiouvilleSemilinearWholeSpace} is the following parabolic maximum principle in the unbounded set $\R^n \times (0,+\infty)$. 

\begin{proposition}
	\label{Prop:ParaMaxPrp}
	Let $L_K \in \lcal_0(n,\s)$ and let $v$ be a bounded function, $C^\alpha$ with $\alpha > 2\s$ in space and $C^1$ in time, such that
	\begin{equation*}
	\beqc{\PDEsystem}
	\partial_t v + L_K  v + c(x)\,v &\leq& 0 & \textrm{ in }\R^n\times(0,+\infty)\,,\\
	v_0:=v(x,0) &\leq& 0 & \textrm{ in } \R^n\,,
	\eeqc
	\end{equation*}
	with $c(x)$ a bounded function. Then,
	$$ v(x,t) \leq 0 \,\,\,\,\,\text{ in } \,\,\, \R^n\times[0,+\infty). $$
\end{proposition}

This result can be deduced from the usual parabolic maximum principle in a bounded (in space and time) set with a rather simple argument. Since we have not found a specific reference where such result is stated, let us present its proof with full detail for the sake of clarity. First of all, we present the usual parabolic maximum principle in a bounded set. The proof for cylindrical sets can be found for instance in \cite{BarriosPeralSoriaValdinoci}. Although the argument is essentially the same, since it is short we include here the complete proof for the sake of completeness.

\begin{lemma}
\label{Lemma:ParabolicmaxPrpBdd}
Let $L_K$ be an integro-differential operator of the form \eqref{Eq:DefOfLu} with kernel satisfying \eqref{Eq:Symmetry&IntegrabilityOfK}, and let $v$ be a bounded function, $C^\alpha$ with $\alpha > 2\s$ in space and $C^1$ in time, satisfying
\begin{equation*}
\beqc{\PDEsystem}
\partial_t v + L_K v &\leq& 0 & \textrm{ in } \Omega \subset B_R\times(0,T)\,,\\
v_0:=v(x,0) &\leq& 0 & \textrm{ in } \overline{\Omega} \cap \{t=0\} \subset B_R\,,\\
v &\leq& 0 & \textrm{ in } ( \R^n \times (0,T))\setminus \Omega \,.
\eeqc
\end{equation*}
Then, $v\leq 0$ in $\R^n\times [0,T]$.
\end{lemma}

\begin{proof}
By contradiction, for every $\varepsilon > 0$ assume that 
$$
M:=\sup_{\R^n \times (0,T-\varepsilon)}v > 0
$$
By the initial conditions and since $v \leq 0 $ in $(\R^n \times (0,T))\setminus \Omega$, $v$  attains this positive value $M$ at a point $(x_0,t_0) \in \Omega$ with $t_0\leq T-\varepsilon$. If $t_0\in(0,T-\varepsilon)$, then $(x_0,t_0)$ is an interior global maximum (in $\R^n \times (0,T-\varepsilon)$) and it must satisfy $v_t(x_0,t_0)=0$ and $L_K v(x_0,t_0)>0$, which contradicts the equation. If $t_0 = T-\varepsilon$, then $v_t(x_0,t_0)\geq 0$ and $L_K v(x_0,t_0)>0$, which is also a contradiction with the equation. Thus, $v\leq 0$ in $\R^n\times [0,T-\varepsilon)$ and since this holds for $\varepsilon>0$ arbitrarily small, we deduce $v\leq 0$ in $\R^n\times [0,T)$, and by continuity in $\R^n\times [0,T]$.
\end{proof}

To establish Proposition~\ref{Prop:ParaMaxPrp} from Lemma~\ref{Lemma:ParabolicmaxPrpBdd}, we need to introduce an auxiliary function enjoying certain properties (see Lemma~\ref{Lemma:SolBallToZero} below). Before presenting such function, we need the following result.

\begin{lemma}
\label{Lemma:NoBddSolL=1}
There is no bounded solution to $L_K v=1$ in $\R^n$ for any $L_K \in \lcal_0(n,\s)$.
\end{lemma}

\begin{proof}
Assume by contradiction that such solution exists. Then, by interior regularity (see Section~\ref{Sec:Preliminaries}) $v\in C^1(\R^n)$ and $|\nabla v|\leq C$ in $\R^n$. For every $i = 1,\ldots, n$, we differentiate the equation with respect to $x_i$ to obtain
\begin{equation*}
\beqc{\PDEsystem}
L_K  v_{x_i} &=& 0 & \textrm{ in } \R^n\,,\\
|v_{x_i}| &\leq& C & \textrm{ in } \R^n\,.
\eeqc
\end{equation*}
By the Liouville Theorem for the operator $L_K $ (it is proved exactly as in \cite{RosOtonSerra-Stable}, see also \cite{SerraC2s+alphaRegularity}), $v_{x_i}$ is constant. Hence, $\nabla v$ is constant, and thus $v$ is affine. But since $u$ is bounded, $v$ must be constant too, and we arrive to a contradiction with $L_K v=1$.
\end{proof}

With this result we can introduce the auxiliary function that we will use to prove the parabolic maximum principle of Proposition~\ref{Prop:ParaMaxPrp}.

\begin{lemma}
	\label{Lemma:SolBallToZero}
	Let $L_K \in \lcal_0(n,\s)$. Then, for every $R>0$ there exists a constant $M_R>0$ and a function $\psi_R\geq 0$ solution to
	\begin{equation}
	\label{Eq:psiRProblem}
	\beqc{\PDEsystem}
	L_K  \psi_R &=& -1/M_R & \textrm{ in } B_R\,,\\
	\psi_R &=& 1 & \textrm{ in } \R^n\setminus B_R\,,
	\eeqc
	\end{equation}
	satisfying 
	$$
	 \psi_R \to  0 \quad \text{ and } \quad M_R  \to +\infty \quad \text{as } R\to +\infty\,.
	$$
\end{lemma}

\begin{proof}
	First, consider $\phi_R$ the solution to
	\begin{equation*}
	%\label{Eq:phiRProblem}
	\beqc{\PDEsystem}
	L_K  \phi_R &=& 1 & \textrm{ in } B_R\,,\\
	\phi_R &=& 0 & \textrm{ in } \R^n\setminus B_R\,.
	\eeqc
	\end{equation*}
	Note that the existence of a weak solution to the previous problem is given by the Riesz representation theorem. Moreover, by standard regularity results (see Section \ref{Subsec:Regularity}), $\phi_R$ is in fact a classical solution and by the maximum principle, $\phi_R>0$ in $B_R$.
	
	Define $M_R := \sup_{B_R} \phi_R$. Since $M_R$ is increasing (use the maximum principle to compare $\phi_R$ and $\phi_{R'}$ with $R>R'$), it must have a limit $M$. Assume by contradiction that $M<+\infty$. Now, consider the new function $ \varphi_R := \phi_R/M_R$, which satisfies
	\begin{equation}
	\label{Eq:varphiRProblem}
	\beqc{\PDEsystem}
	L_K  \varphi_R &=& 1/M_R & \textrm{ in } B_R\,,\\
	\varphi_R &=& 0 & \textrm{ in } \R^n\setminus B_R\,, \\
	\varphi_R &\leq & 1\,.
	\eeqc
	\end{equation}
	Therefore, by a standard compactness argument, we deduce that $\varphi_R$ converges (up to a subsequence) as $R\to +\infty$ to a function $\varphi$ that is solution to $L_K  \varphi = 1/M$ in $\R^n$ and satisfies  $|\varphi| \leq 1$. This contradicts Lemma~\ref{Lemma:NoBddSolL=1} and therefore, $M_R \to +\infty$ as $R\to +\infty$. 
	
	Define now $\psi_R := 1-\phi_R/M_R = 1-\varphi_R$, which solves trivially \eqref{Eq:psiRProblem}. Thus, we only need to show that $\phi \to 0$ as $R\to +\infty$. We will see that $\varphi_R \to	1$ as $R\to +\infty$. Recall that $\varphi_R$ solves problem \eqref{Eq:varphiRProblem}, and by the previous 	arguments, letting $R\to +\infty$ we deduce that $\varphi_R$ converges to a bounded function $\varphi\geq 0$ that solves $ L_K \varphi = 0 $ in $\R^n$. By the Liouville Theorem, $\varphi$ must be constant, and since its $L^\infty$ norm is $1$ and $\varphi\geq 0$, we conclude $\varphi\equiv 1$.	
\end{proof}

With these ingredients, we establish now the parabolic maximum principle in $\R^n \times (0,+\infty)$. 

\begin{proof}[Proof of Proposition~\ref{Prop:ParaMaxPrp}]
First of all, note that with the change of function $\tilde{v}(x,t) = \e^{-\alpha\,t} v(x,t)$ we can reduce the initial problem to
\begin{equation*}
\beqc{\PDEsystem}
\partial_t \tilde{v} + L_K  \tilde{v} &\leq& 0 & \textrm{ in } \Omega \subset\R^n\times(0,+\infty)\,,\\
\tilde{v} &\leq& 0 & \textrm{ in }  \left(\R^n\times(0,+\infty)\right) \setminus  \Omega\,,\\
\tilde{v}_0 &\leq& 0 & \textrm{ in } \R^n\,,
\eeqc
\end{equation*}
if we take $\alpha > ||c||_{L^\infty}$ and $\Omega = \{(x,t)\in \R^n\times(0,+\infty) \ : \ v(x,t) > 0\}$.

Now, consider the function 
$$ 
w_R(x,t) := \norm{\tilde{v} }_{L^\infty(\R^n \times (0,+\infty))} \left( \psi_R + \dfrac{t}{M_R} \right)\,,
$$
where $\psi_R$ and $M_R$ are defined in Lemma~\ref{Lemma:SolBallToZero}. Then, it is easy to check that $w_R$ satisfies
\begin{equation*}
\beqc{\PDEsystem}
\partial_t w_R + L_K  w_R &=& 0 & \textrm{ in }B_R\times(0,T)\,,\\
w_R(x,0) &\geq& 0 & \textrm{ in } B_R\,,\\
w_R(x,t) &\geq& \norm{\tilde{v}}_{L^\infty(\R^n \times (0,+\infty))}  & \textrm{ in } \left( \R^n\setminus B_R\right) \times (0,T) \,,
\eeqc
\end{equation*}
for every $T>0$ and $R>0$. Since $w_R \geq 0$, by the maximum principle in $(B_R\times (0,T))\cap \Omega$ (see Lemma \ref{Lemma:ParabolicmaxPrpBdd}) we can easily deduce that $ w_R\geq \tilde{v} $ in $B_R\times(0,T)$.

Finally, given an arbitrary point $(x_0,t_0) \in \Omega$, take $R_0>0$ and $T>0$ such that $(x_0,t_0)\in B_{R_0}\times (0,T)$. Thus,
$$ 
\tilde{v}(x_0,t_0) \leq w_R(x_0,t_0) =  \norm{\tilde{v} }_{L^\infty(\R^n \times (0,+\infty))} \left( \psi_R (x_0) + \dfrac{t_0}{M_R} \right), \,\,\,\,\,\text{ for every }\,\,\, R\geq R_0.
$$
Letting $R \to +\infty$ and using that $\psi_R(x_0) \to 0$ and $M_R \to +\infty$ (see Lemma~\ref{Lemma:SolBallToZero}), we conclude $ \tilde{v}(x_0,t_0) \leq  0$, and therefore $ v(x_0,t_0) = \e^{\alpha\,t_0}\,\tilde{v}(x_0,t_0) \leq 0$.
\end{proof}

%%%%%%%%%%%%%%%%%%%%%%%%%%%%%%%%%%%%%%%%%%%%%%%%%%%%%%%%%%%%%%%%%%%%%%%%%%%%%%%%%%%%%%%%%%%%%%%%%%%%%%%%%%%%%%%%%%%%%%%%%%%%%%%%%%%%%%%%%%%%%%%%%%%


By using the Harnack inequality and the parabolic maximum principle we can now establish Theorem~\ref{Th:LiouvilleSemilinearWholeSpace}. The proof follows the ideas of Berestycki, Hamel, and Nadirashvili from Theorem 2.2. in \cite{BerestyckiHamelNadi} but adapted to the whole space and with an integro-differential operator.

\begin{proof}[Proof of Theorem~\ref{Th:LiouvilleSemilinearWholeSpace}]


Assume $v\not\equiv 0$. Then, by the strong maximum principle $v>0$. Our goal is to show that $v\equiv 1$, and this will be accomplished in two steps.

\textbf{Step 1:} We show that $m:=\inf_{\R^n} v >0$.

By contradiction, we will assume $m=0$. Then, there exists a sequence $\{x_k\}_{k\in\N}$ such that
$v(x_k)\rightarrow 0$ as $k \rightarrow +\infty$. By the Harnack Inequality from Proposition~\ref{Prop:HarnackSemilinear}, given any $R>0$ we have 
\begin{equation}
\label{Eq:Harnack}
\sup_{B_R(x_k)}v \leq C_R \inf_{B_R(x_k)}v \leq C_R \, v(x_k) \rightarrow 0 \,\,\text{as}\,\, k\rightarrow +\infty.
\end{equation}


Since $f(0) = 0 $ and $f'(0)>0$, it is easy to show that $f(t)\geq f'(0)t/2$ if $t$ is small enough. Therefore, from this and \eqref{Eq:Harnack}  we deduce that there exists $M(R)\in\N$ such that
\begin{align}
\label{Eq:WholeSpace2}
L_K  v - \frac{f'(0)}{2}v \geq 0 \,\,\textrm{ in }\ B_R(x_{M(R)})\,.
\end{align}
On the other hand, let us define
$$  \lambda_R^{x_0} = \inf_{\substack{\varphi\in C^1_0(B_R(x_0))\\ \varphi\not\equiv 0}} \frac{\ds \int_{\R^n}\int_{\R^n}|\varphi(x)-\varphi(y)|^2\,K(x-y) \d x \d y}{\ds \int_{\R^n}\varphi(x)^2 \d x}, 
$$
which decreases to zero uniformly in $x_0$ as $R\to +\infty$ from being $L_K \in\mathcal{L}_0$ (see the proof of Lemma~\ref{Lemma:FirstOddEigenfunction} and also Proposition~9 of \cite{ServadeiValdinoci}). Therefore, there exists $R_0>0$ such that $ \lambda_R^x < f'(0)/2$ for all $x\in \R^n$ and $R\geq R_0$. In particular, by choosing $x=x_{M(R_0)}$ there exists $w\in C^1_0(B_{R_0}(x_{M(R_0)}))$ such that $w\not\equiv 0$ and
\begin{equation}
\label{Eq:Eigenfunction}
\int_{\R^n}\int_{\R^n}|w(x)-w(y)|^2\,K(x-y) \d x \d y < \frac{f'(0)}{2}\int_{\R^n}w^2 \d x.
\end{equation}

Now, multiply \eqref{Eq:WholeSpace2} by $w^2/v\geq 0$ and integrate in $\R^n$, we get
\begin{align*}
0 &\leq \int_{\R^n} \frac{w^2}{v}  L_K v \d x - \frac{f'(0)}{2}\int_{\R^n} w^2 \d x \\
&= \int_{\R^n}\int_{\R^n}\left\{ v(x)-v(y) \right\} \left( \frac{w^2(x)}{v(x)}-\frac{w^2(y)}{v(y)} \right) K(x-y) \d x \d y - \frac{f'(0)}{2}\int_{\R^n} w^2 \d x \\
&\leq \int_{\R^n}\int_{\R^n} |w(x)-w(y)|^2 K(x-y) - \frac{f'(0)}{2}\int_{\R^n} w^2 \d x ,
\end{align*}
which contradicts \eqref{Eq:Eigenfunction}. Here we have used that the kernel is positive and the inequality \eqref{Eq:IdentityStability}. Therefore, $\inf_{\R^n} v >0$.

\textbf{Step 2:} We show that $v\equiv 1$.

Now, choose $0<\xi_0<\min\{1,m\}$, which is well defined by Step~1, and let $\xi(t)$ be the solution of the ODE
$$
\beqc{\PDEsystem}
\dot{\xi}(t) &=& f(\xi(t)) & \textrm{ in }(0,\infty)\,,\\
\xi(0) &=& \xi_0\,.
\eeqc
$$
Since $f>0$ in $(0,1)$ and $f(1) = 0$ we have that $\dot{\xi}(t)>0$ for all $t\geq 0$ and $\ds \lim_{t\to 0} \xi(t) = 1$.

Now, note that both $v(x)$ and $\xi(t)$ solve the parabolic equation
$$ \partial_t w + L_K w = f(w) \,\,\, \textrm{ in }\R^n\times (0,\infty)\,, $$
and satisfy
$$ v(x) \geq m \geq \xi_0 = \xi(0). $$
Thus, by the parabolic maximum principle (Proposition~\ref{Prop:ParaMaxPrp}) applied to $v-\xi$ and taking $c(x) = -\{f(v)-f(\xi)\}/(v-\xi)$, we deduce that $v(x)\geq \xi(t)$ for all $x\in\R^n$ and $t\in(0,\infty)$. By letting $t \to +\infty$ we obtain
$$
 v(x) \geq 1 \,\, \textrm{ in }\R^n\,.  
$$
In a similar way, taking $\tilde{\xi}_0>\norm{v}_{L^\infty} \geq 1$, using $f<0$ in $(1,+\infty)$, $f(1)=0$ and the parabolic maximum principle, we obtain the upper bound $v\leq 1$.

\end{proof}


%%%%%%%%%%%%%%%%%%%%%%%%%%%%%%%%%%%%%%%%%%%%%%%%%%%%%%%%%%%%%%%%%%%%%%%%%%%%%%%%%%%%%%%%%%%%%%%%%%%%%%%%%%%%%%%%%%%%%%%%%%%%%%%%%%%%%%%%%%%%%%%%%%%%%%%%%

\subsection{A one-dimensional symmetry result for positive solutions in a half-space}

In this subsection we establish Theorem~\ref{Th:SymmHalfSpace}. To do it, we proceed in three steps. First, we show that the solution is monotone in the $x_n$ direction by using a moving planes argument (see Proposition~\ref{Prop:MonotonyHalfSpace} below). Once this is shown, we can deduce that the solution $v$ has uniform limits as $x_n\pm\to \infty$. Finally, by using the sliding method (see Proposition~\ref{Prop:HalfSpaceLimUnif} below), we deduce the one-dimensional symmetry of the solution.

We proceed now with the details of the arguments. As we have said, the first step is to show that the solution is monotone. We establish the following result.

\begin{proposition}
	\label{Prop:MonotonyHalfSpace}
	Let $v$ be a bounded solution to one of the problems \eqref{Eq:P1} or \eqref{Eq:P2}, with $L_K \in \lcal_0(n,\s)$ such that the kernel $K$ satisfies \eqref{Eq:KernelSymmetry}. Let $f$ be a Lipschitz nonlinearity such that $f>0$ in $(0,||v||_{L^\infty(\R^n_+)})$. Then,
	$$
	\frac{\partial v}{\partial x_n} > 0 \,\,\,\, \text{ in } \,\,\R^n_+.
	$$
\end{proposition}

To prove it, we use a moving planes argument, and for this reason we need a maximum principle in ``narrow'' sets for odd functions with respect to a hyperplane. Recall that for a set $\Omega \subset \R^n$, we define the quantity $R(\Omega)$ as the smallest positive $R$ for which

\begin{equation}
	\label{Eq:DefNarrow}
	\dfrac{|B_R(x)\setminus \Omega|}{|B_R(x)|}\geq \dfrac{1}{2} \quad \text{ for every } x \in \Omega.
\end{equation}

If no such radius exists, we define $R(\Omega) = +\infty$. Thus, we say that a set $\Omega$ is ``narrow'' ir $R(\Omega)$ is small depending on certain quantities.

An important result that we need is the following ABP-type estimate. It is proved in \cite{QuaasXia} for the fractional Laplacian, following the arguments in \cite{Cabre-ABP} (see also \cite{Cabre-Topics}). The proof for a general operator $L_K$ do not differ significantly from the one for the fractional Laplacian. Nevertheless, we include it here for the sake of completeness.

\begin{theorem}
	\label{Th:ABPEstimate}
	Let $\Omega \subset \R^n$ with $R(\Omega) < +\infty$. Let $L_K \in \lcal_0(n,\s,\lambda, \Lambda)$ and let $v\in L^1_\s(\R^n)\cap C^{\alpha}(\Omega)$, with $\alpha > 2\s$, such that $\sup_{\Omega} v < +\infty$ and satisfying
	$$
	\beqc{\PDEsystem}
	L_K v - c(x)v &\leq & h & \text{ in } \Omega\,, \\
	v & \leq & 0 & \text{ in } \R^n\setminus \Omega\,,
	\eeqc
	$$
	with $c(x)\leq 0$ in $\Omega$ and $h\in L^\infty(\Omega)$.
	
	Then,
	$$
	\sup_\Omega v \leq C R(\Omega)^{2\s} \norm{h}_{L^{\infty}(\Omega)}\,,
	$$
	where $C$ is a constant depending on $n$, $\s$ and $\Lambda$.
\end{theorem}


The only ingredient needed to show Theorem~\ref{Th:ABPEstimate} is the following weak Harnack inequality proved in  \cite{Cozzi-DeGiorgiClassesShort}.

\begin{proposition}[see Corollary 4.4 of \cite{Cozzi-DeGiorgiClassesShort}]
	
	\label{Prop:WeakHarnack}
		
	Let $\Omega \subset \R^n$ and $L_K\in (n,\s,\lambda, \Lambda)$. Let $w \in L^1_\s(\R^n)\cap C^{\alpha}(\Omega)$, with $\alpha > 2\s$, such that $w\geq 0$ in $\R^n$. Assume that $w$ satisfies weakly $L_K w \geq h$ in $\Omega$, for some $h\in L^\infty (\Omega)$. Then, there exists an exponent $\varepsilon > 0 $  and a constant $C > 1$, both depending on $n$, $\s$ and $\Lambda$, such that
	$$
	 \bpar{ \fint_{B_{R/2}(x_0)} w^\varepsilon}^{1/\varepsilon} \leq C \bpar{\inf_{B_{R}(x_0)} w + R^{2\s} \norm{h}_{L^{\infty}(\Omega)} }
	$$
	for every $x_0\in \Omega$ and $0<R<\dist(x_0, \partial \Omega)$.
\end{proposition}

With the previous weak Harnack inequality we can now establish the ABP estimate.

\begin{proof}[Proof of Theorem~\ref{Th:ABPEstimate}]
	First, note that it is enough to show it for $v > 0$ in $\Omega$ satisfying
	$$
	\beqc{\PDEsystem}
	L_K v &\leq & h & \text{ in } \Omega\,, \\
	v & \leq & 0 & \text{ in } \R^n\setminus \Omega\,.
	\eeqc
	$$
	Indeed, if we consider $\Omega_0 = \{x \in \Omega \ : \ v > 0\}$, then since $c \leq 0$ we have $L_K v \leq L_K v - c(x)v \leq h$ in $\Omega_0$.
	
	Assume first that $\Omega$ is bounded. Then the supremum of $v$ must be achieved at an interior point $x_0\in \Omega$. Define	$M:= v(x_0) = \sup_\Omega v$ and consider the function $w := M - v^+$. Note that $0 \leq w \leq M$, $w(x_0) = 0$ and $w \equiv M$ in $\R^n \setminus \Omega$. If we extend $h$ to be $0$ outside $\Omega$, we can easily see that $L_K w \geq -h$ in $B_R(x_0)$.
	
	Now, by choosing $R= 2R(\Omega)$, and using the weak Harnack inequality of Proposition~\ref{Prop:WeakHarnack}, we get
	\begin{align*}
	M \dfrac{1}{2^\varepsilon} & \leq \bpar{M^{\varepsilon}\dfrac{|B_{R/2}(x_0)\setminus \Omega|}{|B_{R/2}(x_0)|}}^{1/\varepsilon}= \bpar{\dfrac{1}{|B_{R/2}(x_0)|} \int_{B_{R/2}(x_0)\setminus \Omega} w^\varepsilon}^{1/\varepsilon} \\
	& \leq \bpar{ \fint_{B_{R/2}(x_0)} w^\varepsilon}^{1/\varepsilon} \leq C \bpar{\inf_{B_{R}(x_0)} w + R^{2\s} \norm{h}_{L^{\infty}(\Omega)} }\,.
	\end{align*}
	The conclusion follows from the fact that $w(x_0)= \inf_{B_{R}(x_0)} w = 0$.
	
	In the case of $\Omega$ being unbounded, the proof is the same with minor changes. We define $M$ as before and we consider, for every $\delta > 0$, a point $x_0$ such that $M-\delta \leq v(x_0)$. We proceed as before and the desired estimate follows by letting $\delta \to 0$.
\end{proof}

As a consequence of this result, one can deduce easily a general maximum principle in ``narrow'' sets.

\begin{corollary}
	\label{Cor:MaxPpleNarrowDomains}
	Let $\Omega \subset \R^n$ with $R(\Omega) < +\infty$. Let $L_K \in \lcal_0(n,\s,\lambda, \Lambda)$ and let $v\in L^1_\s(\R^n)\cap C^{\alpha}(\Omega)$, with $\alpha > 2\s$, such that $\sup_{\Omega} v < +\infty$ and satisfying
	$$
	\beqc{\PDEsystem}
	L_K v + c(x)v &\leq & 0 & \text{ in } \Omega\,, \\
	v & \leq & 0 & \text{ in } \R^n\setminus \Omega\,,
	\eeqc
	$$
	with $c(x)$ bounded below.
	
	Then, there exists a number $\bar{R} > 0$ such that $v \leq 0$ in $\Omega$ whenever $R(\Omega)< \bar{R}$.	
\end{corollary}

\begin{proof}
	We write $c= c^+ - c^-$, and therefore $L_K v -(-c^+)v \leq c^- v^+	$. By Theorem~\ref{Th:ABPEstimate} we get
	$$
	\sup_\Omega v \leq C R(\Omega)^{2\s} \norm{c^- v^+}_{L^\infty(\Omega)} \leq C R(\Omega)^{2\s} \norm{c^-}_{L^\infty(\Omega)} \sup_\Omega v\,.
	$$
	Hence, if $C R(\Omega)^{2\s} \norm{c^-}_{L^\infty(\Omega)}  <1 $, we deduce that $v\leq 0$ in $\Omega$.
\end{proof}


The previous maximum principle in ``narrow'' sets is not powerful enough to apply the moving planes method. The reason for this is that in the hypotheses of Corollary~\ref{Cor:MaxPpleNarrowDomains} there is a prescribed constant sign of a function outside the set. Nevertheless, in the application of the moving planes argument, since our functions are odd with respect to a hyperplane, they cannot have a constant sign in the exterior of a ``narrow'' band. Thus, we need another version of a maximum principle in ``narrow'' sets that applies to odd functions and only requires a constant sign of the function at one side of a hyperplane.

\begin{proposition}
	\label{Prop:MaxPrpNarrowOdd}
	Let $H$ be a half-space in $\R^n$, and denote by $x^\#$ the reflection of any point $x$ with respect to the hyperplane $\partial H$. Let $L_K \in \lcal_0(n,\s)$ with a positive kernel $K$ satisfying
	\begin{equation}
	\label{Eq:KernelSymmetry}
	K(x-y) \geq K(x-y^\#), \,\,\,\,\text{for all } \,\, x,y\in H.
	\end{equation}
	Assume that $v\in C^{\beta}(\Omega)$, with $\beta > 2\s$, satisfies
	\begin{equation*}
	\beqc{\PDEsystem}
	L_K  v &\geq& c(x)\,v  &\textrm{ in } \Omega\subset H,\\
	v &\geq& 0 &\textrm{ in } H\setminus\Omega,\\
	v(x) &=& -v(x^\#) &\textrm{ in } \R^n.
	\eeqc
	\end{equation*}
	
	Then, there exist a number $\overline{R}$ such that $v \geq 0$ whenever $R(\Omega) \leq \overline{R}$.
\end{proposition}

\begin{proof}
	Let us begin by defining $\Omega_- = \{x\in \Omega \,:\,\, v<0\}$. We shall prove that $\Omega_-$ is empty. Assume by contradiction that it is not empty. Then, we split $ v = v_1+v_2$, where
	\begin{equation*}
	v_1(x) =
	\begin{cases}
	v(x)  &\textrm{ in } \Omega_-,\\
	0 &\textrm{ in } \R^n\setminus\Omega_-,
	\end{cases}
	\quad \text{ and } \quad
	v_2(x) =
	\begin{cases}
	0  &\textrm{ in } \Omega_-,\\
	v(x) &\textrm{ in } \R^n\setminus\Omega_-.
	\end{cases}
	\end{equation*}
	Let us first show that $Lv_2\leq 0$ in $\Omega_-$. To see this, let us take $x\in\Omega_-$ and thus
	$$
	L_K v_2(x) = \int_{\R^n\setminus\Omega_-} -v_2(y)K(x-y) \d y = -\int_{\R^n\setminus\Omega_-} v(y)K(x-y) \d y.
	$$
	Now, we split $\R^n\setminus\Omega_-$ into
	$$
	A_1 = \Omega_-^\#,\,\,\,\,\,\,\,\text{ and }\,\,\,\,\,\,\, A_2 = \left(H\setminus\Omega_-\right)\cup\left(H\setminus\Omega_-\right)^\#,
	$$
	and we compute the previous integral in these two sets separately. On the one hand,
	\begin{align*}
	-\int_{A_1} v(y)K(x-y) \d y = -\int_{\Omega_-} v(y^\#)K(x-y^\#) \d y  = \int_{\Omega_-} v(y)K(x-y^\#) \d y \leq 0,
	\end{align*}
	where the last inequality follows from being $v$ negative in $\Omega_-$ and the kernel positive in all $\R^n$.
	On the other hand,
	\begin{align*}
	-\int_{A_2} v(y)K(x-y) \d y = -\int_{H\setminus\Omega_-} v(y)K(x-y) \d y  -\int_{H\setminus\Omega_-} v(y^\#)K(x-y^\#) \d y \\
	= -\int_{H\setminus\Omega_-} v(y)\left\{K(x-y)-K(x-y^\#)\right\} \d y \leq 0,
	\end{align*}
	where we have use the kernel condition \eqref{Eq:KernelSymmetry} and the odd symmetry of $v$. Thus, we get $L_K v_2 \leq 0$ in $\Omega_-$, which means
	$$ L_K v_1 = L_K v-L_K v_2 \geq L_K v \geq c(x)\,v = c(x)\,v_1 \,\,\,\,\text{ in }\,\,\Omega_-. $$
	Therefore $v_1$ solves
	\begin{equation*}
	\beqc{\PDEsystem}
	L_K v_1 &\geq& c(x)\,v_1   &\textrm{ in } \,\Omega_-,\\
	v_1 &=& 0 &\textrm{ in }\,\R^n\setminus\Omega_-,
	\eeqc
	\end{equation*}
	and we can apply the usual maximum principle for ``narrow'' sets (Corollary~\ref{Cor:MaxPpleNarrowDomains}) to $v_1$ in $\Omega_-$ in order to deduce that $v_1\geq 0$ in all $\R^n$. But this is a contradiction with the definition of $v_1$ and the fact that the set $\Omega_-$ is not empty.
\end{proof}


\begin{remark}
	A maximum principle such as Proposition~\ref{Prop:MaxPrpNarrowOdd} was already proved for the fractional Laplacian in \cite{ChenLiLi}, but with the additional hypothesis that either $\Omega$ is bounded or $\liminf_{x\in  \Omega,\ |x|\to \infty} v(x) \geq 0$. In the proof of Theorem~3.1 in \cite{QuaasXia}, Quaas and Xia use a suitable argument (the truncation used in the previous proof, previously used by Felmer and Wang in \cite{FelmerWang}) to avoid the requirement of such additional hypotheses on $\Omega$ or $v$.
\end{remark}

With the maximum principle in ``narrow'' sets for odd functions with respect to a hyperplane we can use the moving plane argument. Now we can show Proposition~\ref{Prop:MonotonyHalfSpace}.

\begin{proof}[Proof of Proposition~\ref{Prop:MonotonyHalfSpace}]
	The proof is based on the moving planes method, and is exactly the same as the analogue proof of Theorem~3.1 in \cite{QuaasXia}, where Quaas and Xia establish an equivalent result for the fractional Laplacian. For this reason, we give here just a sketch. As usual, for $\lambda > 0$ we defines $w_\lambda (x) = v(x',2\lambda - x_n)-v(x',x_n)$ and since the nonlinearity is Lipschitz, $w_\lambda$ solves, in both cases ---\eqref{Eq:P1} or \eqref{Eq:P2}---, the following problem:
	$$
	\beqc{\PDEsystem} 
	L_K  w_\lambda &=& c_\lambda(x)\,w_\lambda  &\textrm{ in } \Sigma_\lambda\subset H_\lambda,\\ 
	w_\lambda &\geq& 0 &\textrm{ in } H_\lambda\setminus\Sigma_\lambda,\\ 
	w_\lambda(x',2\lambda - x_n) &=& - w_\lambda(x',x_n)  &\textrm{ in } \R^n, 
	\eeqc 
	$$
	where $\Sigma_\lambda := \left\{ x = (x',x_n) \ : \ 0<x_n<\lambda \right\}$ and $H_\lambda := \left\{ x = (x',x_n) \ : \ x_n<\lambda \right\}$ and $c_\lambda$ is a bounded function. Note that $w_\lambda$ is odd with respect to $\partial H_\lambda$. Then, using the maximum principle in ``narrow'' sets  Proposition~\ref{Prop:MaxPrpNarrowOdd} one shows that, if $\lambda$ is small enough, $w_\lambda>0$ in $\Sigma_\lambda$. To conclude the proof, we define
	$$
	\lambda^* := \sup\{\lambda \ : \ w_\eta>0 \,\, \text{ in } \,\, \Sigma_\lambda \,\, \text{ for all } \,\, \eta<\lambda\}.
	$$
	Note that $\lambda^*$ is well defined (but may be infinite) by the previous argument. To conclude the proof, one has to show that $\lambda^*=\infty$. This is done by proving that, if $\lambda^*$ is finite, then there exists a small $\delta_0 > 0$ such that for every $\delta \in (0,\delta_0]$ we have
	$$
	w_{\lambda^* +  \delta} (x) > 0 \quad \text{ in } \Sigma_{\lambda^*-\varepsilon}\setminus \Sigma_{\varepsilon}
	$$
	for some small $\varepsilon$. This is established using a compactness argument exactly as in Lemma~3.1 of \cite{QuaasXia} (in the argument a Harnack inequality is needed, one can use for instance Proposition~\ref{Prop:HarnackSemilinear}). Finally, by the maximum principle in ``narrow'' sets we can deduce that $w_{\lambda^* +  \delta} (x) > 0 $ in $\Sigma_{\lambda^*+\delta}$ if $\delta$ is small enough, contradicting the definition of $\lambda^*$.
\end{proof}


Now, we present the other important ingredient needed in the proof of Theorem~\ref{Th:SymmHalfSpace}. It is the following symmetry result.

\begin{proposition}
\label{Prop:HalfSpaceLimUnif}
Let $L_K \in \lcal_0(n,\s)$ and let $v$ be a bounded solution to one of the following problems:
\begin{equation}
\leqnomode
\tag{P3}
\label{Eq:P3}
\beqc{\PDEsystem}
L_K  v &=& f(v)  &\textrm{ in }\R^n\,,\\
\ds \lim_{x_n \to \pm \infty} v(x_H,x_n) &=& \pm 1 \,\,\, &\textrm{ uniformly}.
\eeqc
\end{equation}

\begin{equation}
\leqnomode
\tag{P4}
\label{Eq:P4}
\beqc{\PDEsystem}
L_K  v &=& f(v)  &\textrm{ in }\R^n_+ = \{ x_n>0\} \,,\\
v &=& 0  &\textrm{ in } \R^n \setminus \R^n_+ = \{ x_n\leq 0\}\,,\\
\ds \lim_{x_n \to + \infty} v(x_H,x_n) &=& 1 \,\,\, &\textrm{ uniformly}.
\eeqc
\end{equation}

\reqnomode

Assume that there exists a $\delta > 0$ such that
$$ f'(t) \leq 0 \quad \text{ in } \quad [-1,-1+\delta]\cup[1-\delta,1], $$
for problem \eqref{Eq:P3} and
$$ f'(t) \leq 0 \quad \text{ in } \quad [1-\delta,1] $$
for problem \eqref{Eq:P4}.

Then, $v$ depends only on $x_n$ and is increasing in that direction.
\end{proposition}

\begin{proof}
It is based on the sliding method, exactly as in the proof of Theorem~1 in \cite{BerestyckiHamelMonneau}. The idea is, as usual,   to define $ v^t(x) := v(x+\nu t) $ for every $\nu\in\R^n$ with $|\nu|=1$ and $\nu_n>0$, and the aim is to show that $v^t(x)-v(x)\geq 0$ for all $t\geq 0$. Despite the fact that $L_K$ is a nonlocal operator, the proof is exactly the same as the one in \cite{BerestyckiHamelMonneau} ---it only relies on the maximum principle, the translation invariance of the operator and the symmetry result of Theorem~\ref{Th:LiouvilleSemilinearWholeSpace}. Therefore, we do not include here the details.
\end{proof}






Finally, we can proceed with the proof of Theorem~\ref{Th:SymmHalfSpace}.

\begin{proof}[Proof of Theorem~\ref{Th:SymmHalfSpace}]
Note that by Proposition~\ref{Prop:HalfSpaceLimUnif} we only need to prove that
$$
\ds \lim_{x_n\to+ \infty} v(x_H,x_n) = 1
$$
uniformly. Therefore we divide the proof in two steps: first, we prove that the limit exists and is $1$, and then we prove that it is uniform.


\textbf{Step 1:} Given $x_H\in \R^{n-1}$, then  $\ds \lim_{x_n\to +\infty} v(x_H,x_n) = 1$.

By Proposition \ref{Prop:MonotonyHalfSpace} we know that $v$ is strictly increasing in the direction $x_n$. Since $v$ is also bounded by hypothesis, we know that, given $x_H\in\R^{n-1}$, the one variable function $v(x_H,\cdot)$ has a limit, that we call $\overline{v}(x_H)$. Note that, since $v(x_H,0) = 0$ and $v_{x_n}>0$, we deduce that $\overline{v}(x_H) > 0$.

Let $x_n^k$ be any increasing sequence tending to infinity. Define $v_k(x_H,x_n) := v(x_H,x_n+x_n^k)$. By the regularity theory of the operator $L_K $ (see Section~\ref{Sec:Preliminaries}) and a standard compactness argument, we see that, up to a subsequence, $v_k$ converge uniformly on compact sets to a function $v_\infty$ that is a classical solution to
\begin{equation}
\label{Eq:ProofSymmHalf-SemilinearEqWholeSpace}
\beqc{\PDEsystem}
L_K v_\infty &=& f(v_\infty)   &\textrm{ in } \,\R^n,\\
v_\infty &\geq& 0   &\textrm{ in } \,\R^n.
\eeqc
\end{equation}
By Theorem \ref{Th:LiouvilleSemilinearWholeSpace}, either $v_\infty\equiv 0$ or $v_\infty \equiv 1$. But, by construction,
$$ v_\infty(x_H,0) = \lim_{k\to + \infty} v_k(x_H,0) = \lim_{k\to + \infty} v(x_H,x_n^k) = \overline{v}(x_H) > 0, $$
and therefore the only possibility is
$$ \lim_{x_n\to \infty} v(x_H,x_n) = 1 \quad \text{ for all } \ x_H\in\R^{n-1}. $$

\textbf{Step 2:} The limit is uniform in $x_H$.

Let us proceed by contradiction. Suppose that the limit is not uniform. This means that given any $\varepsilon>0$ small enough, there exists a sequence of points $(x_H^k,x_n^k)$ with $x_n^k\to +\infty$ such that $v(x_H^k,x_n^k) = 1-\varepsilon$. Similarly as before, the sequence of functions $\tilde{v}_k(x_H,x_n) = v(x_H+x_H^k,x_n+x_n^k)$ converge uniformly on compact sets to a function $\tilde{v}_\infty$ that solves also \eqref{Eq:ProofSymmHalf-SemilinearEqWholeSpace}. By Theorem \ref{Th:LiouvilleSemilinearWholeSpace}, either $\tilde{v}_\infty\equiv 0$ or $\tilde{v}_\infty \equiv 1$. But, by construction
$$ 
\tilde{v}_\infty(0,0) = \lim_{k\to +\infty} \tilde{v}_k(0,0) = \lim_{k\to +\infty} v(x_H^k,x_n^k) = 1-\varepsilon, 
$$
which is a contradiction for $\varepsilon>0$ small enough. Thus, the limit is uniform.

Finally, by applying Proposition~\ref{Prop:HalfSpaceLimUnif}, we get that $v$ depends only on $x_n$ and is increasing in that direction.
\end{proof}
