%%%%%%%%%%%%%%%%%%%%%%%%%%%%%
\section{Asymptotic result for the saddle-shaped solution}
\label{Sec:Asymptotic}
%%%%%%%%%%%%%%%%%%%%%%%%%%%%%

In this section, we establish Theorem~\ref{Th:AsymptoticBehaviourSaddleSolution}, concerning the asymptotic behavior of the saddle-shaped solution. 



To study the asymptotic behavior of the saddle-shaped solution it is important to relate the Allen-Cahn equation in $\R^{2m}$ with the same equation in $\R$. In the local case, this is very easy, since if $v$ is a solution to $-\ddot{v} = f(v)$ in $\R$,  then $w(x) = v(x\cdot e)$ solves $-\Delta w = f(w)$ in $\R^n$ for every unitary vector $e\in \R^n$. This happens also for the fractional Laplacian

Nevertheless, for a general operator $L_K$ this is not true anymore (note that in dimension $n$ there can be more kernels that in dimension $1$, in particular the class of kernels in dimension $n$ include all the one-dimensional ones). Therefore, we need a way to associate a solution of a one-dimensional problem with a one-dimensional solution to a $n$-dimensional problem. This is given in the next result.



\begin{proposition}
	\label{Prop:KernelsDimension}
	Let $L_K $ and $L_{K_1}$ be two symmetric and translation invariant integral operators with kernels $K$ and $K_1$ in dimensions $n$ and $1$ respectively. Let also be $u:\R^n\to\R$ and $v:\R\to\R$ such that $u(x) = v(x_n)$.
	
	If,
	$$ K_1(t) = |t|^{n-1} \int_{\R^{n-1}} K\left((t\sigma,t)\right) \d \sigma, $$
	then
	\begin{enumerate}[label=(\roman{*})]
		\item $L_K u(x) = L_{K_1} v(x_n)$.
		\item If $L_K \in \mathcal{L}_0 (n,\s,\lambda,\Lambda)$, then   $L_{K_1} \in \mathcal{L}_0 (1,\s,\lambda,\Lambda)$.
	\end{enumerate}
\end{proposition}

Note that if $L_K $ is the fractional Laplacian in dimension $n$, then $L_{K_1}$ is also the fractional Laplacian, but in dimension $1$.

\begin{proof}
	We start proving point $(i)$. We write $y=(y',y_n)$, with $y'\in \R^{n-1}$.
	\begin{align*}
	L_K u(x) &= \int_{\R^n} \left\{ u(x)-u(y) \right\} K(x-y) \d y \\
	&=\int_{\R^n} \left\{ v(x_n)-v(y_n) \right\} K\left((x'-y',x_n-y_n)\right) \d y' \d y_n.
	\end{align*}
	Now we make the change of variables $y' = x'-(x_n-y_n)\sigma$. That is,
	\begin{align*}
	L_K u(x) &= \int_{\R} \left\{ v(x_n)-v(y_n) \right\} \int_{\R^{n-1}} K\left((\sigma(x_n-y_n),x_n-y_n\right)) |x_n-y_n|^{n-1} \d \sigma \d y_n \\
	&= \int_{\R} \left\{ v(x_n)-v(y_n) \right\} |x_n-y_n|^{n-1} \int_{\R^{n-1}} K\left((x_n-y_n)(\sigma,1)\right) \d\sigma \d y_n \\
	&= \int_{\R} \left\{ v(x_n)-v(y_n) \right\} K_1(x_n-y_n ) \d y_n = L_{K_1}v(x_n). \\
	\end{align*}
	
	We establish now point $(ii)$. To do it, we bound the kernel $K_1$. 
	\begin{align*}
	K_1(t) &= |t|^{n-1} \int_{\R^{n-1}} K\left(t(\sigma,1)\right) \d\sigma \geq |t|^{n-1} \int_{\R^n} c_{n,\s} \frac{\lambda}{|t|^{n+2\s}(|\sigma|^2+1)^{\frac{n+2s}{2}}} \d\sigma \\
	&= c_{n,\s} \frac{\lambda}{|t|^{1+2\s}} \int_{\R^{n-1}} \frac{\d\sigma}{(|\sigma|^2+1)^{\frac{n+2\s}{2}}} = c_{n,\s} \frac{\lambda}{|t|^{1+2\s}} \frac{c_{1,\s}}{c_{n,\s}} \\
	&= c_{1,\s} \frac{\lambda}{|t|^{1+2\s}},
	\end{align*}
	where the only inequality that appears comes from the lower bound of the kernel $K$ for being $L_K $ in $\mathcal{L}_0$. The upper bound for $K_1$ is obtained in the same way.
\end{proof}





\todo[inline]{Esto va a ir en la intro}

\begin{theorem}
\label{Th:AsymptoticBehaviourSaddleSolution}
Let $f\in C^{2,\alpha}(\R)$ of the Allen-Cahn type. Let $u$ be a saddle-shaped solution to \eqref{Eq:NonlocalAllenCahn} with $L_K \in \lcal_\star$. Let $U$ be the function defined by 99.

Then,
$$
\norm{u-U}_{L^\infty(\R^n\setminus B_R)}
+\norm{\nabla u-\nabla U}_{L^\infty(\R^n\setminus B_R)}
+\norm{D^2u-D^2U}_{L^\infty(\R^n\setminus B_R)} \to 0
$$
as $ R\to +\infty$.
\end{theorem}

\begin{proof}
By contradiction, assume that the result does not hold. Then, there exists an $\varepsilon>0$ and a sequence $\{x_k\}$, such that
\begin{equation}
\label{Eq:ContradictionAsymptotic}
|u(x_k)-U(x_k)|+|\nabla u(x_k)-\nabla U(x_k)|+|D^2u(x_k)-D^2U(x_k)| > \varepsilon.
\end{equation}
By the symmetry of $u$, we may assume without loss of generality that $x_k \in \overline{\ocal}$, and by continuity we can further assume $ x_k \notin \ccal$. 

Let $d_k:=\dist(x_k,\ccal)$. We distinguish two cases:

\textbf{Case 1: $\{d_k\}$ is an unbounded sequence.} In this situation, we may assume that $d_k \geq 2k$. Define
$$
w_k(x) := u(x+x_k), 
$$
which satisfies $0<w_k<1$ in $\overline{B_k}$ and
$$
L_K w_k = f(w_k) \ \textrm{ in } B_k.
$$
By letting $k\to +\infty$, by the uniform estimates for the operators of the class $\lcal_0$ and the Arzelà-Ascoli theorem, we have that, up to a subsequence, $w_k$ converges on compact sets to a function $w$ which is a pointwise solution to
$$
\beqc{\PDEsystem}
L_K  w &=& f(w) & \textrm{ in }\R^n\,,\\
w &\geq& 0 & \textrm{ in } \R^n\,.
\eeqc
$$

By Theorem~\ref{Th:SymmetryWholeSpace}, either $w\equiv 0$ or $w\equiv 1$. First, note that $w$ cannot be zero. Indeed, since $w_k$ are stable\todo{Decir en algun sitio! Remark dp de existencia} with respect to perturbations supported in $B_k$, $w$ is stable in $\R^n$, which means that the linearized operator $L_K-f'(w)$ is a positive operator. Nevertheless, if $w\equiv 0$, then the linearized operator $L_K-f'(w) = L_K-f'(0)$ is negative for sufficiently large balls, since $f'(0)>0$ and the first eigenvalue of $L_K$ is of order $R^{-2\s}$ in balls of radius $R$. Therefore $w\equiv 1$. 

On the other hand, since $d_k\rightarrow +\infty$ and $U(x_k) =  u_0(d_k)$, we get by the properties of the layer solution \todo{poner en la intro de la seccion} that $U(x_k) \rightarrow 1$, $\nabla U(x_k) \rightarrow 0$ and $D^2U(x_k) \rightarrow 0$. From this and condition \eqref{Eq:ContradictionAsymptotic} we get
$$
|u(x_k)-1|+|\nabla u(x_k)|+|D^2u(x_k)| > \varepsilon/2,
$$
which means that $w \not\equiv 1$. Therefore we arrive to a contradiction with Theorem~\ref{Th:SymmetryWholeSpace}.

\textbf{Case 2: $\{d_k\}$ is a bounded sequence.}
In this situation, at least for a subsequence, we have that $d_k \rightarrow d$. Now, for each $x_k$ we define $x_k^0$ as its projection on $\ccal$. Therefore, we have that $ \nu_k^0 := (x_k-x_k^0)/d_k$ is the unit normal to $\ccal$. Through a subsequence, $ \nu_k^0 \rightarrow \nu$ with $|\nu|=1$.

We define
$$ w_k (x) := u(x+x_k^0), $$
which solves
$$ L_K  w_k = f(w_k) \ \text{in } \R^n. $$
Similarly as before, by letting $k\to +\infty$, up to a subsequence $w_k$ converges on compact sets to a function $w$ which is a pointwise solution to
$$
\beqc{\PDEsystem}
%L w &=& f(w)  \textrm{ in } &H:=\{x\cdot \nu >0\}\,,\\
%w &\geq& 0  \textrm{ in } &H\,,\\
%w &\text{ odd}& \hspace{-2mm} \text{with respect} \text{to} &H \,.
L_K  w &=& f(w)  &\textrm{ in } H:=\{x\cdot \nu >0\}\,,\\
w &\geq& 0  &\textrm{ in } H\,,\\
\,\,w \text{ odd with respect to } H. \span\span\span \,
\eeqc
$$
For the details about the fact that $\ocal \rightarrow H$, see \cite{CabreTerraI}.

As in the previous case, by stability $w$ cannot be zero, and then, by Theorem \ref{Th:SymmHalfSpace}, $w$ only depends on $x\cdot \nu$ and is increasing. Therefore, by the uniqueness of the layer solution, $w(x) = u_0(x\cdot \nu)$ and
\begin{align*}
u(x_k) &= w_k(x_k-x_k^0) = w(x_k-x_k^0) + o(1) \\
&= u_0((x_k-x_k^0)\cdot \nu) + o(1) \\
&= u_0((x_k-x_k^0)\cdot \nu_k^0) + o(1) \\
&= u_0(d_k |\nu_k^0|^2) + o(1) \\
&= u_0(d_k) + o(1) = \ucal (x_k) + o(1),
\end{align*}
contradicting \eqref{Eq:ContradictionAsymptotic}. The same is done for $\nabla u$ and $D^2 u$.
\end{proof}

\begin{remark}
	\label{Remark:u>delta}
	The previous result yields that, for small $\delta$, the level set $\{0<u<\delta\}$ of the saddle-shaped solution is contained in a $\varepsilon$-neighborhood of the cone. Indeed, consider the set $\ocal_\varepsilon := \{(x',x'')\in \R^m\times\R^m \ : \ |x''|+\varepsilon <|x'| \}$. 
	
	Since $u_0$ is monotone, for every small $\delta$ we can take $\varepsilon$ small enough and $R$ big enough so that $U(x)\geq 2\delta$ in $\ocal_\varepsilon \setminus B_R$. Thus, by taking $R$ big enough so that $|u-U|< \delta$ in $\ocal \setminus B_R$, we have that $u(x) \geq U(x)-\delta \geq \delta$ whenever $x\in \ocal_\varepsilon \setminus B_R$. Finally, since $\overline{B_R \cap \ocal}$ is a compact and $u$ is positive there, we can take $\varepsilon$ small enough so that $u\geq \delta$ in $B_R\cap \ocal_\varepsilon$.
\end{remark}