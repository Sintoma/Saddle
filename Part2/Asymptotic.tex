%%%%%%%%%%%%%%%%%%%%%%%%%%%%%
\section{Asymptotic behavior of a saddle-shaped solution}
\label{Sec:Asymptotic}
%%%%%%%%%%%%%%%%%%%%%%%%%%%%%

In this section, we establish Theorem~\ref{Th:AsymptoticBehaviorSaddleSolution}, concerning the asymptotic behavior of the saddle-shaped solution. 



In order to study this behavior, it is important to relate the Allen-Cahn equation in $\R^{2m}$ with the same equation in $\R$. In the local case, this is very easy, since if $v$ is a solution to $-\ddot{v} = f(v)$ in $\R$,  then $w(x) = v(x\cdot e)$ solves $-\Delta w = f(w)$ in $\R^n$ for every unitary vector $e\in \R^n$. The same fact also happens for the fractional Laplacian, that is, if $v$ is a solution to $\fraclaplacian v = f(v)$ in $\R$, then $w(x)= v(x\cdot e)$ solves the same equation in $\R^n$. We can easily see this relation via the local extension problem.

Nevertheless, for a general operator $L_K$ this is not true anymore and we need a way to relate a solution to a one-dimensional problem with a one-dimensional solution to a $n$-dimensional problem. This is given in the next result. Some of its points appear in \cite{CozziPassalacqua} with a different notation but we state and prove them here for completeness.

\begin{proposition}
	\label{Prop:KernelsDimension}
	Let $L_K \in \mathcal{L}_0 (n,\s,\lambda,\Lambda)$ be a symmetric and translation invariant integro-differential operator of the form \eqref{Eq:DefOfLu} with kernel $K:\R^n\setminus \{0\} \to (0,+\infty) $. Define the one dimensional kernel $K_1 : \R \setminus \{0\} \to (0,+\infty) $ by
	\begin{equation}
	\label{Eq:OneDimKernel}
	K_1(t) := \int_{\R^{n-1}} K\left(\theta,t\right) \d \theta = |t|^{n-1} \int_{\R^{n-1}} K\left(t\sigma,t\right) \d \sigma.
	\end{equation}
	\begin{enumerate}[label=(\roman{*})]
		\item Let $v:\R\to\R$ and consider $w:\R^n\to\R$ defined by $w(x) = v(x_n)$. Then, $L_K w(x) = L_{K_1} v(x_n)$. If we assume moreover that $K$ is radially symmetric, then the same happens with $w(x) = v(x\cdot e)$ for every unitary vector $e\in \Sph^{n-1}$. That is, $L_K w(x) = L_{K_1} v(x \cdot e)$.
		\item  If $K$ is nonincreasing/decreasing in  the $x_n$-direction in $\{x_n>0\}$, then $K_1(t)$ is nonincreasing/decreasing in $(0,+\infty)$.
		\item $L_{K_1} \in \mathcal{L}_0 (1,\s,\lambda,\Lambda)$, and moreover, if $L_K $ is the fractional Laplacian in dimension $n$, then $L_{K_1}$ is the fractional Laplacian in dimension $1$.
		
	\end{enumerate}
\end{proposition}

\begin{proof}
	We start proving point $(i)$. We write $y=(y_H,y_n)$, with $y_H\in \R^{n-1}$.
	\begin{align*}
	L_K w(x) &= \int_{\R^n} \left\{ w(x)-w(y) \right\} K(x-y) \d y \\
	&=\int_{\R^n} \left\{ v(x_n)-v(y_n) \right\} K\left(x_H-y_H,x_n-y_n\right) \d y_H \d y_n.
	\end{align*}
	Now we make the change of variables $\theta = x_H-y_H$. That is,
	\begin{align*}
	L_K w(x) 	&= \int_{\R} \left\{ v(x_n)-v(y_n) \right\} \int_{\R^{n-1}} K\left(\theta,x_n-y_n\right) \d \theta \d y_n \\
	&= \int_{\R} \left\{ v(x_n)-v(y_n) \right\} K_1(x_n-y_n ) \d y_n = L_{K_1}v(x_n).
	\end{align*}
	This shows the first equality in \eqref{Eq:OneDimKernel}. The alternative expression of the kernel $K_1$, that is useful in some cases, can be obtained from the change of variables $\theta = t\sigma$. Furthermore, in the case of $K$ radially symmetric, the result is valid for $u(x) = v(x\cdot e)$ for every unitary vector $e\in \Sph^{n-1}$ after a change of variables in the previous computations.
	
	The proof of point $(ii)$ follows directly from the first expression of the unidimensional kernel $K_1$. That is,
	$$ 
	K_1(t_2)-K_1(t_1) = \int_{\R^{n-1}} \left\{ K(\theta,t_2) - K(\theta,t_1)\right\} \d \theta \geq 0 \quad \text{ for any } \quad t_2>t_1>0. 
	$$
	
	We establish now point $(iii)$. To do it, we bound the kernel $K_1$ using the ellipticity condition on $K$:
	\begin{align*}
	K_1(t) &= |t|^{n-1} \int_{\R^{n-1}} K\left(t(\sigma,1)\right) \d\sigma \geq |t|^{n-1} \int_{\R^n} c_{n,\s} \frac{\lambda}{|t|^{n+2\s}(|\sigma|^2+1)^{\frac{n+2s}{2}}} \d\sigma \\
	&= c_{n,\s} \frac{\lambda}{|t|^{1+2\s}} \int_{\R^{n-1}} \frac{\d\sigma}{(|\sigma|^2+1)^{\frac{n+2\s}{2}}} = c_{n,\s} \frac{\lambda}{|t|^{1+2\s}} \dfrac{2 \pi^{\frac{n-1}{2}}}{\Gamma(\frac{n-1}{2})} \int_0^\infty \dfrac{r^{n-2}}{(r^2+1)^{\frac{n+2\s}{2}}} \d r \\	
	& = c_{n,\s} \frac{\lambda}{|t|^{1+2\s}} 
	\dfrac{\pi^{\frac{n-1}{2}} \Gamma(\frac{1}{2}+\s)}{\Gamma(\frac{n}{2}+\s)} 
	= c_{n,\s} \frac{\lambda}{|t|^{1+2\s}} \frac{c_{1,\s}}{c_{n,\s}} = c_{1,\s} \frac{\lambda}{|t|^{1+2\s}},
	\end{align*}
	where we have used the explicit value of the normalizing constant for the fractional Laplacian,
	\begin{equation}
	\label{Eq:ConstantFracLaplacian}
	c_{n,\s} = \s\dfrac{2^{2\s} \Gamma(\frac{n}{2}+\s) }{\pi^{n/2} \Gamma(1-\s)},
	\end{equation}
	and the definition of the Beta and Gamma functions. The upper bound for $K_1$ is obtained in the same way. Note that the previous computation is an equality with $\lambda = 1$ in the case of the fractional Laplacian.
\end{proof}





In the proof of Theorem~\ref{Th:AsymptoticBehaviorSaddleSolution} we will use some properties of the layer solution, which are presented next. First, in \cite{CozziPassalacqua} it is proved that there exists a constant $C$ such that
\begin{equation}
	\label{Eq:PropertiesLayer}
	|u_0 (x)-\sign(x)| \leq C |x|^{-2\s}  \quad \text{ and } \quad |\dot{u}_0 (x)| \leq C |x|^{-1-2\s}  \quad \text{ for large }|x|.
\end{equation}
In our arguments we need also to show that the second derivative of the layer goes to zero at infinity. This is the first statement of the following lemma.

\begin{lemma}
	\label{Lemma:SecondDerivativeLayer}
	Let $K_1:\R \setminus \{0\} \to (0,+\infty)$ be a symmetric kernel satisfying \eqref{Eq:Ellipticity} and assume that it is decreasing in $(0,+\infty)$. Let $u_0$ be the layer solution associated to the kernel $K_1$, that is, $u_0$ solving \eqref{Eq:LayerSolution}. Then, 
	\begin{enumerate}[label=(\roman{*})]
		\item $\ddot{u}_0 (x) \to 0$ as $x\to \pm \infty$.	
		\item  $\ddot{u}_0 (x) < 0$ in $(0,+\infty)$.
	\end{enumerate}
\end{lemma}

We prove here the first statement of this lemma, and we postpone the proof of the second one until the next section, since we need to use a maximum principle for the linearized operator $L_{K_1} - f'(u_0)$.

\begin{proof}[Proof of point (i) of Lemma~\ref{Lemma:SecondDerivativeLayer}]
	By contradiction, suppose that there exists an unbounded sequence $\{x_j\}$ satisfying $|\ddot{u}_0(x_j)|>\varepsilon$ for some $\varepsilon>0$. Note that by the symmetry of $u_0$ we may assume that $x_j\to + \infty$. Now define $w_j (x) := \ddot{u}_0(x+x_j)$. By differentiating twice the equation of the layer solution, we see that $\ddot{u}_0$ solves
	$$
	L_{K_1} \ddot{u}_0 = f''(u_0)\dot{u}_0^2 + f'(u_0)\ddot{u}_0 \quad \text{ in }\R.
	$$
	Hence, as $x_j \to +\infty$ a standard compactness argument combined with the asymptotic behavior given by \eqref{Eq:PropertiesLayer} yields that $w_j$ converges on compact sets to a function $w$ that solves
	$$
	L_{K_1}  w = f'(1)w \quad \text{ in }\R.
	$$
	In addition, since $|\ddot{u}_0(x_j)|>\varepsilon$ we have $|w(0)|\geq \varepsilon$.
	
	At this point we use Lemma~4.3 of \cite{CozziPassalacqua} to deduce that, since $f'(1)<1$, then $w\to 0$ as $|x| \to +\infty$. Therefore, if $w$ is not identically zero, it has either a positive maximum or a negative minimum, but this contradicts the maximum principle (recall that $f'(1)<1$). We conclude that $w\equiv0$ in $\R$, but this is a contradiction with $|w(0)|\geq \varepsilon$.
\end{proof}




Now we have all the ingredients to establish the asymptotic behavior of the saddle-solution. The proof follows exactly the same compactness arguments used to prove the analogous result in the local case (see \cite{CabreTerraII}) and for the fractional Laplacian using the extension problem (see \cite{Cinti-Saddle, Cinti-Saddle2}). Thus we will omit some details. The main ingredients too establish this results are the translation invariance of the operator, the Liouville type and symmetry results of Theorems~\ref{Th:LiouvilleSemilinearWholeSpace} and \ref{Th:SymmHalfSpace} and a stability argument (recall the comments in Section~\ref{Sec:Preliminaries}). 



\begin{proof}[Proof of Theorem~\ref{Th:AsymptoticBehaviorSaddleSolution}]
By contradiction, assume that the result does not hold. Then, there exists an $\varepsilon>0$ and an unbounded sequence $\{x_k\}$, such that
\begin{equation}
\label{Eq:ContradictionAsymptotic}
|u(x_k)-U(x_k)|+|\nabla u(x_k)-\nabla U(x_k)|+|D^2u(x_k)-D^2U(x_k)| > \varepsilon.
\end{equation}
By the symmetry of $u$, we may assume without loss of generality that $x_k \in \overline{\ocal}$, and by continuity we can further assume $ x_k \notin \ccal$. 

Let $d_k:=\dist(x_k,\ccal)$. We distinguish two cases:

\textbf{Case 1: $\{d_k\}$ is an unbounded sequence.} In this situation, we may assume that $d_k \geq 2k$. Define
$$
w_k(x) := u(x+x_k), 
$$
which satisfies $0<w_k<1$ in $\overline{B_k}$ and
$$
L_K w_k = f(w_k) \ \textrm{ in } B_k.
$$
By letting $k\to +\infty$, by the uniform estimates for the operators of the class $\lcal_0$ and the Arzelà-Ascoli theorem, we have that, up to a subsequence, $w_k$ converges on compact sets to a function $w$ which is a pointwise solution to
$$
\beqc{\PDEsystem}
L_K  w &=& f(w) & \textrm{ in }\R^n\,,\\
w &\geq& 0 & \textrm{ in } \R^n\,.
\eeqc
$$

Then, by Theorem~\ref{Th:LiouvilleSemilinearWholeSpace}, either $w\equiv 0$ or $w\equiv 1$. First, note that $w$ cannot be zero. Indeed, since $w_k$ are stable with respect to perturbations supported in $B_k$ (see the comments in Section~\ref{Sec:Preliminaries}), $w$ is stable in $\R^n$, which means that the linearized operator $L_K-f'(w)$ is a positive operator. Nevertheless, if $w\equiv 0$, then the linearized operator $L_K-f'(w) = L_K-f'(0)$ is negative for sufficiently large balls, since $f'(0)>0$ and the first eigenvalue of $L_K$ is of order $R^{-2\s}$ in balls of radius $R$ (as in Lemma~\ref{Lemma:FirstOddEigenfunction}, see Proposition~9 of \cite{ServadeiValdinoci}). Therefore $w\equiv 1$. 

On the other hand, since $d_k\rightarrow +\infty$ and $U(x_k) =  u_0(d_k)$, we get by the properties of the layer solution that $U(x_k) \rightarrow 1$, $\nabla U(x_k) \rightarrow 0$ and $D^2U(x_k) \rightarrow 0$ ---see \eqref{Eq:PropertiesLayer} and Lemma~\ref{Lemma:SecondDerivativeLayer}. From this and condition \eqref{Eq:ContradictionAsymptotic} we get
$$
|u(x_k)-1|+|\nabla u(x_k)|+|D^2u(x_k)| > \varepsilon/2,
$$
for $k$ big enough. This yields that 
$$
|w_k(0)-1|+|\nabla w_k(0)|+|D^2w_k(0)| > \varepsilon/2,
$$
and this contradicts $w \equiv 1$. 

\textbf{Case 2: $\{d_k\}$ is a bounded sequence.}
In this situation, at least for a subsequence, we have that $d_k \rightarrow d$. Now, for each $x_k$ we define $x_k^0$ as its projection on $\ccal$. Therefore, we have that $ \nu_k^0 := (x_k-x_k^0)/d_k$ is the unit normal to $\ccal$. Through a subsequence, $ \nu_k^0 \rightarrow \nu$ with $|\nu|=1$.

We define
$$ w_k (x) := u(x+x_k^0), $$
which solves
$$ L_K  w_k = f(w_k) \ \text{in } \R^n. $$
Similarly as before, by letting $k\to +\infty$, up to a subsequence $w_k$ converges on compact sets to a function $w$ which is a pointwise solution to
$$
\beqc{\PDEsystem}
%L w &=& f(w)  \textrm{ in } &H:=\{x\cdot \nu >0\}\,,\\
%w &\geq& 0  \textrm{ in } &H\,,\\
%w &\text{ odd}& \hspace{-2mm} \text{with respect} \text{to} &H \,.
L_K  w &=& f(w)  &\textrm{ in } H:=\{x\cdot \nu >0\}\,,\\
w &\geq& 0  &\textrm{ in } H\,,\\
\,\,w \text{ is odd with respect to } H. \span\span\span \,
\eeqc
$$
For the details about the fact that $\ocal+x_k^0 \rightarrow H$, see \cite{CabreTerraI}.

As in the previous case, by stability $w$ cannot be zero, and thus $w>0$ in $H$ (by the strong maximum principle for odd functions with respect to a hyperplane, see \cite{ChenLiLi}). Hence, by Theorem \ref{Th:SymmHalfSpace}, $w$ only depends on $x\cdot \nu$ and is increasing. Finally, by the uniqueness of the layer solution, $w(x) = u_0(x\cdot \nu)$ and
\begin{align*}
u(x_k) &= w_k(x_k-x_k^0) = w(x_k-x_k^0) + \mathrm{o}(1) \\
&= u_0((x_k-x_k^0)\cdot \nu) + \mathrm{o}(1) = u_0((x_k-x_k^0)\cdot \nu_k^0) + \mathrm{o}(1) \\
&= u_0(d_k |\nu_k^0|^2) + \mathrm{o}(1) = u_0(d_k) + \mathrm{o}(1) = U (x_k) + \mathrm{o}(1),
\end{align*}
contradicting \eqref{Eq:ContradictionAsymptotic}. The same is done for $\nabla u$ and $D^2 u$.
\end{proof}

\begin{remark}
	\label{Remark:u>delta}
	The previous result yields that, for $\varepsilon>0$ the saddle-shaped solution satisfies $u\geq\delta$ in the set $\ocal_\varepsilon := \{(x',x'')\in \R^m\times\R^m \ : \ |x''|+\varepsilon <|x'| \}$, for some positive constant $\delta$. That is, thanks to the asymptotic result, and since $U(x)\geq u_0(\varepsilon/\sqrt{2})$ for $x\in \ocal_\varepsilon$, there exists a radius $R>0$ such that $u(x)\geq U(x)/2\geq u_0(\varepsilon/\sqrt{2})/2$ if $x\in \ocal_\varepsilon\setminus B_R$. Moreover, since $u$ is positive in the compact set $\overline{\ocal_\varepsilon}\cap\overline{B_R}$ it has a positive minimum in this set, say $m>0$. Therefore, if we choose $\delta = \min\{m,u_0(\varepsilon/\sqrt{2})/2\}$ we obtain the desired result.
\end{remark}