%%%%%%%%%%%%%%%%%%%%%%%%
\section{Preliminaries}
%%%%%%%%%%%%%%%%%%%%%%%%
\label{Sec:Preliminaries}

In this section we collect some preliminary results that will be used in the rest of this paper. First, we state the regularity results needed in the forthcoming sections. Then, we state a remark on stability that will be used later in this paper, and finally we recall the basic maximum principles for doubly radial odd functions proved in \cite{FelipeSanz-Perela:IntegroDifferentialI}.



%%%%%%%%%%%%%%%%%%%%%%%%%%%%%%%%%%%%%%%%%%%%%%%%%%%%%%%%
\subsection{Regularity theory for nonlocal operators in the class $\lcal_0$}
\label{Subsec:Regularity}
%%%%%%%%%%%%%%%%%%%%%%%%%%%%%%%%%%%%%%%%%%%%%%%%%%%%%%%%


In this subsection we present the regularity results that will be used in the paper. For further details, see \cite{RosOton-Survey,SerraC2s+alphaRegularity} and the references therein. 

First, note that for operators in the class $\lcal_0$, the minimal assumption on $w$ so that $L_K w$ is well defined in an open set $\Omega$ is that $w\in C^\alpha (\Omega)\cap L^1_\s(\R^n)$ for some $\alpha > 2\s$, where $w\in L^1_\s(\R^n)$ means that
$$
\int_{\R^n} \dfrac{|w(x)|}{1+|x|^{n+2\s}}\d x < +\infty\,.
$$

Now, we give a result on the interior regularity for linear equations.

\begin{proposition}[\cite{RosOton-Survey,SerraC2s+alphaRegularity}]
	\label{Prop:InteriorRegularity}
	Let $L_K \in\lcal_0(n,\s,\lambda, \Lambda)$ and let $w\in L^\infty (\R^n)$ be a weak solution to $L_K w = h$ in $B_1$. Then,
	\begin{equation}
	\label{Eq:C2sEstimate}
	\norm{w}_{C^{2\s} (B_{1/2})} \leq C\bpar{\norm{h}_{L^\infty (B_1)} + \norm{w}_{L^\infty  (\R^n)} }.
	\end{equation}
	Moreover, let $\alpha > 0$ and assume additionally that $w \in C^\alpha (\R^n)$. Then, if $\alpha +
	2\s$ is not an integer,
	\begin{equation}
	\label{Eq:Calpha->Calpha+2sEstimate}
	\norm{w}_{C^{\alpha + 2\s} (B_{1/2})} \leq C\bpar{\norm{h}_{C^{\alpha} (B_1)} + \norm{w}_{C^\alpha (\R^n)} },
	\end{equation}
	where $C$ is a constant that depends only on $n$, $\s$, $\lambda$ and $\Lambda$.
\end{proposition}


Throughout the paper we consider $u$ to be a saddle solution to \eqref{Eq:NonlocalAllenCahn} that satisfies $|u|\leq 1$ in $\R^n$. Hence, by applying \eqref{Eq:C2sEstimate} we find that for any $x_0\in \R^n$,
\begin{align*}
\norm{u}_{C^{2\s} (B_{1/2} (x_0))} &\leq C\bpar{\norm{f(u)}_{L^\infty (B_1(x_0))} + \norm{u}_{L^\infty  (\R^n)} } \\
&\leq C\bpar{1 + \norm{f}_{L^\infty ([-1,1])} }.
\end{align*}
Note that the estimate is independent of the point $x_0$, and thus since the equation is satisfied in the whole $\R^n$,
$$
\norm{u}_{C^{2\s}(\R^n)} \leq C\bpar{1 + \norm{f}_{L^\infty ([-1,1])} }.
$$
Then, we use estimate \eqref{Eq:Calpha->Calpha+2sEstimate} repeatedly and the same kind of arguments yield that, if $f\in C^{k}([-1,1])$, then $u\in C^{\alpha}(\R^n)$ for all $\alpha < k+ 2 \s$. Moreover, the following estimate holds:
$$
\norm{u}_{C^{\alpha}(\R^n)} \leq C\,,
$$
for some constant $C$ depending only on $n$, $\s$, $\lambda$, $\Lambda$, $k$, and $\norm{f}_{C^k([-1,1])}$.



%%%%%%%%%%%%%%%%%%%%%%%%%%%%%%%%%%%%%%%%%%%%%%%%%%%%%%%%
\subsection{A remark on stability}
\label{Subsec:RemarkStability}
%%%%%%%%%%%%%%%%%%%%%%%%%%%%%%%%%%%%%%%%%%%%%%%%%%%%%%%%




Recall that we say that a bounded solution $w$ to $L_K w = f(w)$ in $\Omega\subset \R^n$ is \emph{stable} in $\Omega$ if the second variation of the energy at $w$ is nonnegative. That is, if
\begin{equation}
\label{Eq:StablityCondition}  
\dfrac{1}{2} \int_{\R^n} \int_{\R^n} |\xi (x) - \xi(y)|^2 K(x - y) \d x \d y - \int_{\Omega} f'(w) \xi^2 \d x \geq 0
\end{equation}
for every $\xi \in C^\infty_c (\Omega)$.






Here we prove that if $w \leq 1$ is a positive solution to $L_K w = f(w)$ in a set $\Omega\subset \R^n$, with $f$ satisfying \eqref{Eq:Hypothesesf}, then $w$ is stable in $\Omega$. We will use this in Sections~\ref{Sec:SymmetryResults} and \ref{Sec:Asymptotic}. The proof of this fact is rather simple and we present it next. It is a consequence of the fact that, under these assumptions, $w$ is a positive supersolution of the linearized operator $L_K - f'(w)$ (a more detailed discussion can be found in \cite{HamelRosOtonSireValdinoci}). 

On the one hand, since $f$ is strictly concave in $(0,1)$ and $f(0)=0$, then $f'(w)w<f(w)$ in $\Omega$ (recall that $w$ is positive there). On the other hand, the following inequality holds for all functions $\varphi$ and $\xi$, with $\varphi>0$:
\begin{equation}
\label{Eq:IdentityStability}
\big (\varphi(x) - \varphi(y) \big) \bpar{\dfrac{\xi^2(x)}{\varphi(x)} - \dfrac{\xi^2(y)}{\varphi(y)} } \leq |\xi (x) - \xi(y)|^2\,.
\end{equation}
Indeed, developing the square and the products, this last inequality is equivalent to $2 \xi(x) \xi(y) \leq \xi^2(y) \varphi(x)/ \varphi(y) +  \xi^2 (x) \varphi(y) / \varphi(x)$, which in turn is equivalent to
$$
\bpar{\xi (x)\sqrt{\varphi(y) / \varphi(x)} - \xi(y) \sqrt{ \varphi(x)/ \varphi(y) } }^2 \geq 0\,.
$$
Using these two facts and the symmetry of $K$, for every $\xi\in C^\infty_c(\Omega)$ we have
\begin{align*}
\int_\Omega f'(w) \xi^2 \d x & \leq \int_\Omega  \dfrac{\xi^2}{w} f(w) \d x = \int_\Omega  \dfrac{\xi^2}{w} L_Kw \d x \\ 
&= \dfrac{1}{2} \int_{\R^{2m}} \int_{\R^{2m}} \big ( w(x) - w(y) \big) \bpar{\dfrac{\xi^2(x)}{w(x)} - \dfrac{\xi^2(y)}{w(y)} } K(x - y) \d x \d y
\\ 
&\leq \dfrac{1}{2} \int_{\R^{2m}} \int_{\R^{2m}} |\xi (x) - \xi(y)|^2 K(x - y) \d x \d y\,.
\end{align*}
Thus, $w$ is stable in $\Omega$. 


%%%%%%%%%%%%%%%%%%%%%%%%%%%%%%%%%%%%%%%%%%%%%%%%%%%%%%%%
\subsection{Maximum principles for doubly radial odd functions}
\label{Subsec:MaxPrinciples}
%%%%%%%%%%%%%%%%%%%%%%%%%%%%%%%%%%%%%%%%%%%%%%%%%%%%%%%%

In this last subsection, we state the basic maximum principles for doubly radial odd functions. Note that in the following result we only need assumptions on the functions at one side of the Simons cone thanks to their symmetry. This was proved in part I \cite{FelipeSanz-Perela:IntegroDifferentialI} and follows readily from the expression \eqref{Eq:OperatorOddF} by using the key inequality \eqref{Eq:KernelInequality} for the kernel $\overline{K}$.


\begin{proposition}[Maximum principle for odd functions with respect to $\ccal$]
	\label{Prop:MaximumPrincipleForOddFunctions} Let $\Omega \subset \ocal$ an open set and let $L_K$ be an integro-differential operator with a radially symmetric kernel $K$ satisfying the positivity condition \eqref{Eq:KernelInequality}.  Let $w\in C^{\alpha}(\Omega)\cap L^\infty(\R^{2m})$, with $\alpha > 2\s$, be a doubly radial function which is odd with respect to the Simons cone. 
	
	\begin{enumerate}[label=(\roman{*})]
		\item  (Weak maximum principle)
		Assume that
		$$
		\beqc{\PDEsystem}
		L_K w + c(x) w & \geq & 0 & \text{ in } \Omega\,,\\
		w & \geq & 0 & \text{ in } \ocal \setminus \Omega\,,
		\eeqc
		$$
		with $c \geq 0$, and that either
		$$
		\Omega \text{ is bounded} \quad \text{ or } \liminf_{x \in \ocal,\,|x|\to +\infty} w(x) \geq 0\,.
		$$
		Then, $w \geq 0$ in $\Omega$.
		
		\item (Strong maximum principle)  
		Assume that $L_K w + c(x) w\geq 0$ in $\Omega$, with $c(x)$ any function, and that $w\geq 0$ in $\ocal$. Then, either $w\equiv 0$ in $\ocal$ or $w > 0$ in $\Omega$.
	\end{enumerate} 
\end{proposition}

\begin{remark}
	\label{Remark:MaxPrincipleSingularity}
	Following the proof of this result in \cite{FelipeSanz-Perela:IntegroDifferentialI} it is easy to see that the regularity assumptions on $w$ in the previous results can be weakened. Indeed, we may allow $L_K w$ to take the value $+\infty$ at the points of $\Omega$ where $w$ is not regular enough for $L_K w$ to be finite. This will be used in the proof of Theorem~\ref{Th:Existence} in order to apply this maximum principle with a function that is no more regular than $C^\s$ in the interior of $\Omega$ (see Remark~\ref{Remark:CsRegularityFirstEigenfunction})
\end{remark}

%%%%%%%%%%%%%%%%%%%%%%%%%%%%%%%%%%%%%%%%%%%%%%%%%%%%%%%%%%%%%%%%%%%%%%
%%%%%%%%%%%%%%%%%%%%%%%%%%%%%%%%%%%%%%%%%%%%%%%%%%%%%%%%%%%%%%%%%%%%%%
