%%%%%%%%%%%%%%%%%%%%%%%%
\section{Preliminaries}
%%%%%%%%%%%%%%%%%%%%%%%%
\label{Sec:Preliminaries}

In this section we collect some preliminary results that will be used in the rest of this paper. First, we give the regularity results needed in the forthcoming sections. Then, we present the functional setting which we are going to work with and finally we state the two basic maximum principles for doubly radial odd functions, proved in \cite{FelipeSanz-Perela:IntegroDifferentialI}.


%%%%%%%%%%%%%%%%%%%%%%%%%%%%%%%%%%%%%%%%%%%%%%%%%%%%%%%%
\subsection{Regularity theory for nonlocal operators in the class $\lcal_0$}
\label{Subsec:Regularity}
%%%%%%%%%%%%%%%%%%%%%%%%%%%%%%%%%%%%%%%%%%%%%%%%%%%%%%%%


In this subsection we present the regularity results that will be used in the paper. For further reference, see \cite{RosOton-Survey,SerraC2s+alphaRegularity} and the references therein. 


We start with a result on the interior regularity for linear equations.

\begin{proposition}
	\label{Prop:InteriorRegularity}
	Let $L_K \in\lcal_0(n,\s,\lambda, \Lambda)$ and let $w\in L^\infty (\R^n)$ be a weak solution to $L_K w = h$ in $B_1$. Then,
	\begin{equation}
	\label{Eq:C2sEstimate}
	\norm{w}_{C^{2\s} (B_{1/2})} \leq C\bpar{\norm{h}_{L^\infty (B_1)} + \norm{w}_{L^\infty  (\R^n)} }\,.
	\end{equation}
	Moreover, let $\alpha > 0$ and assume additionally that $w \in C^\alpha (\R^n)$. Then, if $\alpha +
	2\s$ is not an integer,
	\begin{equation}
	\label{Eq:Calpha->Calpha+2sEstimate}
	\norm{w}_{C^{\alpha + 2\s} (B_{1/2})} \leq C\bpar{\norm{h}_{C^{\alpha} (B_1)} + \norm{w}_{C^\alpha (\R^n)} }\,,
	\end{equation}
	where $C$ is a constant that depends only on $n$, $\s$, $\lambda$ and $\Lambda$.
\end{proposition}


Throughout the paper we consider saddle solutions $u$ to \eqref{Eq:NonlocalAllenCahn} that satisfy $|u|\leq 1$ in $\R^n$. Hence, by applying \eqref{Eq:C2sEstimate} we find that for any $x_0\in \R^n$,
\begin{align*}
\norm{u}_{C^{2\s} (B_{1/2} (x_0))} &\leq C\bpar{\norm{f(u)}_{L^\infty (B_1(x_0))} + \norm{u}_{L^\infty  (\R^n)} } \\
&\leq C\bpar{1 + \norm{f}_{L^\infty ([-1,1])} }\,.
\end{align*}
Note that the estimate is independent of the point $x_0$, and thus since the equation is satisfied in the whole $\R^n$,
$$
\norm{u}_{C^{2\s}(\R^n)} \leq C\bpar{1 + \norm{f}_{L^\infty ([-1,1])} }\,.
$$
Then, we use estimate \eqref{Eq:Calpha->Calpha+2sEstimate} repeatedly and the same kind of arguments wield that, since $f\in C^{1}([-1,1])$, then $u\in C^{\alpha}(\R^n)$ for some $\alpha > 1+ 2 \s$. Moreover, the following estimate holds:
$$
\norm{u}_{C^{\alpha}(\R^n)} \leq C\,,
$$
for some constant $C$ depending only on $n$, $\s$, $\lambda$, $\Lambda$, and $\norm{f}_{C^1([-1,1])}$.


%%%%%%%%%%%%%%%%%%%%%%%%%%%%%%%%%%%%%%%%%%%%%%%%%%%%%%%%
\subsection{Functional setting}
\label{Subsec:Functional setting}
%%%%%%%%%%%%%%%%%%%%%%%%%%%%%%%%%%%%%%%%%%%%%%%%%%%%%%%%



In this section we define the functional spaces that we are going to consider in some parts of this paper. These were also the spaces considered in the previous paper \cite{FelipeSanz-Perela:IntegroDifferentialI}, and we refer the reader to that article for more details.

Given a set $\Omega \subset \R^n$ and a translation invariant and positive kernel $K$ satisfying \eqref{Eq:Symmetry&IntegrabilityOfK}, we define the space
$$
\H^K(\Omega) := \setcond{w \in L^2(\Omega)}{[w]^2_{\H^K(\Omega)} < + \infty},
$$
where
$$
[w]^2_{\H^K(\Omega)} := \dfrac{1}{2}\int\int_{\R^{2n} \setminus (\R^n\setminus\Omega)^2} |w(x) - w(y)|^2 K(x-y) \d x \d y\,.
$$
We also define
\begin{align*}
\H^K_0(\Omega) &:= \setcond{w \in \H^K(\Omega)}{ w = 0 \quad \textrm{a.e. in } \R^n \setminus \Omega} \\
&\ = \setcond{w \in \H^K(\R^n)}{ w = 0 \quad \textrm{a.e. in } \R^n \setminus \Omega}.
\end{align*}

Assume that $\Omega \subset \R^{2m}$ is a domain of double revolution. Then, we define
$$
\widetilde{\H}^K(\Omega) := \setcond{w \in \H^K(\Omega)}{w \textrm{ is doubly radial a.e.}}.
$$
and
$$
\widetilde{\H}^K_0(\Omega) := \setcond{w \in \H^K_0(\Omega)}{w \textrm{ is doubly radial a.e.}}.
$$
We will add the subscript `odd' and `even' to these spaces to consider only functions that are odd (respectively even) with respect to the Simons cone.

Recall that when $K$ satisfies \eqref{Eq:Ellipticity}, then $\H^K_0 (\Omega) = \H^\s_0 (\Omega)$, which is the space associated to the kernel of the fractional Laplacian, $K(y) = c_{n,\s}|y|^{-n-2\s}$. Furthermore, $\H^\s(\Omega) \subset H^\s(\Omega)$, the usual fractional Sobolev space (see \cite{HitchhikerGuide,CozziPassalacqua}). 


%%%%%%%%%%%%%%%%%%%%%%%%%%%%%%%%%%%%%%%%%%%%%%%%%%%%%%%%
\subsection{Maximum principles for doubly radial odd functions}
\label{Subsec:MaxPrinciples}
%%%%%%%%%%%%%%%%%%%%%%%%%%%%%%%%%%%%%%%%%%%%%%%%%%%%%%%%

In this last subsection we state the two basic maximum princples for doubly radial odd functions. Note that in both results we only need assumptions on the functions at one side of the Simons cone thanks to their symmetry. Both results where proved in \cite{FelipeSanz-Perela:IntegroDifferentialI} thanks to the key inequality \eqref{Eq:KernelInequality}.

The first result is a weak maximum principle for odd functions with respect to $\ccal$

\begin{proposition}[\cite{FelipeSanz-Perela:IntegroDifferentialI}]
\label{Prop:WeakMaximumPrincipleForOddFunctions}
 Let $\Omega \subset \ocal$ and let $L_K  \in \lcal_\star (2m,  \s)$.  Let $w\in C^{\alpha}_{\loc}(\Omega)\cap L^\infty(\R^{2m})$ with $\alpha > 2\s$ be a doubly radial function which is odd with respect to the Simons cone. Assume that
 $$
 \beqc{\PDEsystem}
 L_K w & \geq & 0 & \text{ in } \Omega\,,\\
 w & \geq & 0 & \text{ in } \ocal \setminus \Omega\,,
 \eeqc
 $$
 and that either $\Omega$ is bounded or 
 $$
 \liminf_{x \in \ocal,\,|x|\to +\infty} w(x) \geq 0\,.
 $$
 Then, $w \geq 0$ in $\Omega$.
\end{proposition}

The second result is the strong maximum principle for odd functions with respect to $\ccal$.

\begin{proposition}[\cite{FelipeSanz-Perela:IntegroDifferentialI}]
\label{Prop:StrongMaximumPrincipleForOddFunctions} 
Let $\Omega \subset \ocal$ and let $L_K  \in \lcal_\star (2m,  \s)$.  Let $w\in C^{\alpha}_{\loc}(\Omega)\cap L^\infty(\R^{2m})$ with $\alpha > 2\s$ be a doubly radial function which is odd with respect to the Simons cone. Assume that $L_K w \geq 0$ in $\Omega$, and that $w\geq 0$ in $\ocal$. Then, either $w\equiv 0$ or $w > 0$ in $\Omega$.
\end{proposition}






%%%%%%%%%%%%%%%%%%%%%%%%%%%%%%%%%%%%%%%%%%%%%%%%%%%%%%%%%%%%%%%%%%%%%%
%%%%%%%%%%%%%%%%%%%%%%%%%%%%%%%%%%%%%%%%%%%%%%%%%%%%%%%%%%%%%%%%%%%%%%
