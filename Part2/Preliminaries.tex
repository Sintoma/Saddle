%%%%%%%%%%%%%%%%%%%%%%%%
\section{Preliminaries}
%%%%%%%%%%%%%%%%%%%%%%%%
\label{Sec:Preliminaries}


%%%%%%%%%%%%%%%%%%%%%%%%%%%%%%%%%%%%%%%%%%%%%%%%%%%%%%%%
\subsection{Regularity theory for nonlocal operators in the class $\lcal_0$}
\label{Subsec:Regularity}
%%%%%%%%%%%%%%%%%%%%%%%%%%%%%%%%%%%%%%%%%%%%%%%%%%%%%%%%


This section is devoted to present the regularity results that will be used in the paper. For further reference, see \cite{RosOton-Survey,SerraC2s+alphaRegularity} and the references therein.

We start with a result on the interior regularity for linear equations.

\begin{proposition}
	\label{Prop:InteriorRegularity}
	Let $L_K \in\lcal_0$ and let $w\in L^\infty (\R^n)$ be a weak solution to $L_K w = h$ in $B_1$. Then,
	\begin{equation}
	\label{Eq:C2sEstimate}
	\norm{w}_{C^{2\s} (B_{1/2})} \leq C\bpar{\norm{h}_{L^\infty (B_1)} + \norm{w}_{L^\infty  (\R^n)} }\,.
	\end{equation}
	Moreover, let $\alpha > 0$ and assume additionally that $w \in C^\alpha (\R^n)$. Then, if $\alpha +
	2\s$ is not an integer,
	\begin{equation}
	\label{Eq:Calpha->Calpha+2sEstimate}
	\norm{w}_{C^{\alpha + 2\s} (B_{1/2})} \leq C\bpar{\norm{h}_{C^{\alpha} (B_1)} + \norm{w}_{C^\alpha (\R^n)} }\,,
	\end{equation}
	where $C$ is a constant that depends only on $n$, $\s$, $\lambda$ and $\Lambda$.
\end{proposition}


Throughout the paper we consider solutions $u$ to \eqref{Eq:NonlocalAllenCahn} that satisfy $|u|\leq 1$ in $\R^n$. Hence, by applying \eqref{Eq:C2sEstimate} we find that for any $x_0\in \R^n$,
\begin{align*}
\norm{u}_{C^{2\s} (B_{1/2} (x_0))} &\leq C\bpar{\norm{f(u)}_{L^\infty (B_1(x_0))} + \norm{u}_{L^\infty  (\R^n)} } \\
&\leq C\bpar{1 + \norm{f}_{L^\infty ([-1,1])} }\,.
\end{align*}
Note, that the estimate is independent of the point $x_0$, and thus since the equation is satisfied in the whole $\R^n$,
$$
\norm{u}_{C^{2\s}(\R^n)} \leq C\bpar{1 + \norm{f}_{L^\infty ([-1,1])} }\,.
$$
Then, we use estimate \eqref{Eq:Calpha->Calpha+2sEstimate} repeatedly and  the same kind of arguments lead to the following conclusion.

\begin{corollary}
	\label{Cor:C2regularity} Let $f\in C^{1}([-1,1])$, $L_K \in \lcal_0$ and let $-1 \leq u \leq 1$ be a
	bounded weak solution to \eqref{Eq:NonlocalAllenCahn}. Then $u\in C^{\alpha}(\R^n)$ for some
	$\alpha
	> 1+ 2 \s$. Moreover, the following estimate holds:
	\begin{equation}
	\norm{u}_{C^{\alpha}(\R^n)} \leq C\,,
	\end{equation}
	for some constant $C$ depending only on $n$, $\s$, $\lambda$, $\Lambda$, and $\norm{f}_{C^1([-1,1])}$.
\end{corollary}


Sometimes we will need estimates in balls. With the same argument as in Corollaries 2.4 and 2.5 of \cite{RosOtonSerra-Regularity}, we deduce from Proposition~\ref{Prop:InteriorRegularity} the following result.

\begin{corollary}
	\label{Cor:InteriorRegularityBalls}
	Let $L_K \in\lcal_0$ and let $w\in L^\infty (\R^n)$ be a weak solution to $Lw = h$ in $B_1$. Then,
	\begin{equation}
	\label{Eq:C2sEstimateBalls}
	\norm{w}_{C^{2\s} (B_{1/4})} \leq C\bpar{\norm{h}_{L^\infty (B_1)} + \norm{w}_{L^\infty  (B_1)} + \norm{\dfrac{w(x)}{(1+|x|)^{n+2\s}}}_{L^1(\R^n)} }\,.
	\end{equation}
	Moreover, let $\alpha > 0$ and assume additionally that $w \in C^\alpha (\R^n)$. Then, if $\alpha +
	2\s$ is not an integer,
	\begin{equation}
	\label{Eq:Calpha->Calpha+2sEstimateBalls}
	\norm{w}_{C^{\alpha + 2\s} (B_{1/4})} \leq C\bpar{\norm{h}_{C^{\alpha} (\overline{B_1})} + \norm{w}_{C^\alpha (\overline{B_1})} + \norm{\dfrac{w(x)}{(1+|x|)^{n+2\s}}}_{L^1(\R^n)} }\,,
	\end{equation}
	where $C$ is a constant that depends only on $n$, $\s$, $\lambda$ and $\Lambda$.
\end{corollary}

Therefore, assume now that $u$ solves $L_K u = f(u)$ in $B_R$ and that $|u|\leq 1$ in $\R^n$ with $f\in C^{\alpha}([-1,1])$ for some $\alpha > 0$. Then, the combination of \eqref{Eq:C2sEstimate} and \eqref{Eq:Calpha->Calpha+2sEstimateBalls} yields
\begin{equation}
\label{Eq:UniformC2alphaEstimateBalls}
\norm{u}_{C^{2s + \varepsilon}(B_{R/8})} \leq C \bpar{n,\ \s ,\ \lambda,\ \Lambda ,\ \norm{f}_{C^{\alpha}([-1,1])} }\,.
\end{equation}
for some $\varepsilon > 0$.

Sometimes the previous regularity results will be used together with a compactness argument. Since it will be used repeatedly through the paper, we find it useful to state it here for easy reference. The result is an easy consequence of the Arzelà-Ascoli theorem and the compact embedding $C^\alpha \subset \subset C^\beta$ when $\beta < \alpha$.

\begin{lemma}
	\label{Lemma:CompactnessLemma} Let $\Omega\subset \R^n$ a bounded domain, $L_K \in \lcal_0$ and let
	$w_k$ be a sequence of functions satisfying
	\begin{itemize}
		\item $w_k \in C^\alpha (\overline{\Omega})$ with $\alpha > 2\s$ and
		$$
		\norm{w_k}_{C^\alpha (\overline{\Omega})} \leq C
		$$
		with a constant $C$ independent of $k$.
		\item $L_K  w_k = h_k$ with $h_k \in C^{\alpha'}(\overline{\Omega})$ for some $\alpha' > 0$ and such
		that $h_k \to h \in C^{\alpha'}(\overline{\Omega})$ uniformly.
	\end{itemize}
	Then, for every $\beta \in (2\s, \alpha)$, a subsequence of $w_k$ converges to some $w \in C^\beta
	(\overline{\Omega})$ with the $C^\beta (\overline{\Omega})$ norm and satisfies $L_K w = h$ in
	$\Omega$.
\end{lemma}


\begin{proposition}[Weak maximum principle for odd functions with respect to $\ccal$]
\label{Prop:WeakMaximumPrincipleForOddFunctions} Let $u\in C^{\alpha}(\R^{2m})$ with $\alpha > 2\s$ be a doubly radial function which is odd with respect to the Simons cone. Let $\Omega \subset \ocal$ and let $L \in \lcal_\star$. Assume that
$$
\beqc{\PDEsystem}
Lu & \geq & 0 & \text{ in } \Omega\,,\\
u & \geq & 0 & \text{ in } \ocal \setminus \Omega\,,
\eeqc
$$
and that either $\Omega$ is bounded or 
$$
\liminf_{x \in \ocal,\,|x|\to \infty} u(x) \geq 0\,.
$$
Then, $u \geq 0$ in $\Omega$.
\end{proposition}


\begin{proposition}[Strong maximum principle for odd functions with respect to $\ccal$]
\label{Prop:StrongMaximumPrincipleForOddFunctions} Let $u\in C^{\alpha}(\R^{2m})$ with
$\alpha > 2\s$ be a doubly radial function which is odd with respect to the Simons cone.  Let
$\Omega \subset \ocal$ and assume that $Lu \geq 0$ in $\Omega$, where $L \in \lcal_\star$. Assume also that $u\geq 0$ in $\ocal$.
Then, either $u\equiv 0$ or $u > 0$ in $\Omega$.
\end{proposition}

Let us start by defining the functional spaces that we are going to consider in this paper. Given a set $\Omega \subset \R^n$ and a translation invariant and positive kernel $K$ satisfying \eqref{Eq:Symmetry&IntegrabilityOfK}, we define the space
$$
\H^K(\Omega) := \setcond{w \in L^2(\Omega)}{[w]^2_{\H^K(\Omega)} < + \infty},
$$
where
$$
[w]^2_{\H^K(\Omega)} := \dfrac{1}{2}\int\int_{\R^{2n} \setminus (\R^n\setminus\Omega)^2} |w(x) - w(y)|^2 K(x-y) \d x \d y\,.
$$
We also define
\begin{align*}
\H^K_0(\Omega) &:= \setcond{w \in \H^K(\Omega)}{ w = 0 \quad \textrm{a.e. in } \R^n \setminus \Omega} \\
&\ = \setcond{w \in \H^K(\R^n)}{ w = 0 \quad \textrm{a.e. in } \R^n \setminus \Omega}.
\end{align*}

Assume that $\Omega \subset \R^{2m}$ is a domain of double revolution. Then, we define
$$
\widetilde{\H}^K(\Omega) := \setcond{w \in \H^K(\Omega)}{w \textrm{ is doubly radial a.e.}}.
$$
and
$$
\widetilde{\H}^K_0(\Omega) := \setcond{w \in \H^K_0(\Omega)}{w \textrm{ is doubly radial a.e.}}.
$$
We will add the subscript `odd' and `even' to these spaces to consider only functions that are odd (respectively even) with respect to the Simons cone.


\begin{remark}
	\label{Remark:DecompositionHK}
	If $\widetilde{\H}^K_0(\Omega)$ is equipped with the scalar product
	$$
	\langle v,w \rangle_{\widetilde{\H}^K_0(\Omega)} := \dfrac{1}{2}\int_{\R^{2m}} \int_{\R^{2m}}  \big(v(x) - v(y)\big)\big(w(x) - w(y)\big) K(x-y) \d x \d y\,,
	$$
	then, it is easy to check that $\widetilde{\H}^K_0(\Omega)$ can be decomposed as the orthogonal
	direct sum of $\widetilde{\H}^K_{0,\, \mathrm{even}}(\Omega)$ and $\widetilde{\H}^K_{0,\,
		\mathrm{odd}}(\Omega)$.
\end{remark}

Note that when $K$ satisfies \eqref{Eq:Ellipticity}, then $\H^K_0 (\Omega) = \H^\s_0 (\Omega)$,
which is the space associated to the kernel of the fractional Laplacian, $K(y) = |y|^{-n-2\s}$.
Furthermore, $\H^\s(\Omega) \subset H^\s(\Omega)$, the usual fractional Sobolev space (see
\cite{HitchhikerGuide}).  For more comments on this, see~\cite{CozziPassalacqua}.


%%%%%%%%%%%%%%%%%%%%%%%%%%%%%%%%%%%%%%%%%%%%%%%%%%%%%%%%%%%%%%%%%%%%%%
%%%%%%%%%%%%%%%%%%%%%%%%%%%%%%%%%%%%%%%%%%%%%%%%%%%%%%%%%%%%%%%%%%%%%%
