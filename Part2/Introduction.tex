%%%%%%%%%%%%%%%%%%%%%%%
\section{Introduction}
%%%%%%%%%%%%%%%%%%%%%%%
\label{Sec:Introduction}
 
In this paper, which is the second part of \cite{FelipeSanz-Perela:IntegroDifferentialI}, we study saddle-shaped solutions to the semilinear equation
\begin{equation}
\label{Eq:NonlocalAllenCahn}
L_K u = f(u) \quad \textrm{ in } \R^{2m},
\end{equation}
where $L_K$ is a linear integro-differential operator of the form \eqref{Eq:DefOfLu} and $f$ is of Allen-Cahn type. These  solutions (see Definition~\ref{Def:SaddleShapedSol} below) are particularly interesting in relation to the nonlocal version of a conjecture by De Giorgi, with the aim of finding a counterexample in high dimensions. Moreover, this problem is related to the regularity theory of nonlocal minimal surfaces. For more comments on this we refer to Subsection~\ref{Subsec:DeGiorgi} and the references therein.

Previous to this article and its first part \cite{FelipeSanz-Perela:IntegroDifferentialI}, there are only three works devoted to saddle-shaped solutions to the equation \eqref{Eq:NonlocalAllenCahn} with $L_K$ being the fractional Laplacian. In  \cite{Cinti-Saddle,Cinti-Saddle2}, Cinti proved the existence of a saddle-shaped solution as well as some qualitative properties such as asymptotic behavior, monotonicity properties, and instability whenever $2m\leq 6$. In a previous paper by the authors \cite{Felipe-Sanz-Perela:SaddleFractional}, further properties of these solutions were proved, the main ones being uniqueness and, when $2m\geq 14$, stability. Concerning saddle-shaped solutions to the classical Allen-Cahn equation $-\laplacian u = f(u)$, the same results were proved in \cite{DangFifePeletier, Schatzman, CabreTerraI,CabreTerraII, Cabre-Saddle}. The possible stability in dimensions $8$, $10$, and $12$ is still an open problem (both in the local and fractional frameworks), as well as the possible minimality of this solution in dimensions $2m \geq 8$.

The present paper together with its first part \cite{FelipeSanz-Perela:IntegroDifferentialI} are the first ones studying saddle-shaped solutions for general integro-differential equations of the form \eqref{Eq:NonlocalAllenCahn}. In the three previous papers \cite{Cinti-Saddle, Cinti-Saddle2, Felipe-Sanz-Perela:SaddleFractional} the main tool used was the extension problem for the fractional Laplacian (see \cite{CaffarelliSilvestre}). Nevertheless, this technique has the limitation that it cannot be carried out for general integro-differential operators different from the fractional Laplacian. Therefore, some purely nonlocal techniques were developed in the previous paper \cite{FelipeSanz-Perela:IntegroDifferentialI} to study saddle-shaped solutions, and we exploit them in the present one.

In part~I \cite{FelipeSanz-Perela:IntegroDifferentialI}, we established an appropriate setting to study solutions to \eqref{Eq:NonlocalAllenCahn} that are doubly radial and odd with respect to the Simons cone, a property that is satisfied by saddle-shaped solutions (see Subsection~\ref{Subsec:Integro-differential setting}). In that paper we deduced an alternative expression for the operator $L_K$ when acting on doubly radial odd functions ---see \eqref{Eq:OperatorOddF}. This was used to deduce some maximum principles for odd functions under certain assumptions on the kernel $K$ of the operator $L_K$. Moreover, we proved an energy estimate for doubly radial and odd minimizers of the energy associated to the equation, as well as the existence of saddle-shaped solutions to \eqref{Eq:NonlocalAllenCahn}.

In the present paper, we further study saddle-shaped solutions to \eqref{Eq:NonlocalAllenCahn} by using the results obtained in part~I \cite{FelipeSanz-Perela:IntegroDifferentialI}. First, we prove existence of this type of solutions, Theorem~\ref{Th:Existence}, by using the monotone iteration method (as an alternative to the proof in \cite{FelipeSanz-Perela:IntegroDifferentialI} where variational methods are used). After this, we establish the asymptotic behavior of saddle-shaped solutions,  Theorem~\ref{Th:AsymptoticBehaviorSaddleSolution}. To do it, we use two ingredients: a Liouville type theorem and a one-dimensional symmetry result, both for semilinear equations like \eqref{Eq:NonlocalAllenCahn} under some hypotheses on $f$. These are Theorems~\ref{Th:LiouvilleSemilinearWholeSpace} and \ref{Th:SymmHalfSpace}, proved in Section~\ref{Sec:SymmetryResults}. In the study of the asymptotic behavior of saddle-shaped solutions we establish further properties of the so-called \emph{layer solution} $u_0$ (see Section~\ref{Sec:Asymptotic}). Finally, we show the uniqueness of the saddle-shaped solution, Theorem~\ref{Th:Uniqueness}, by using a maximum principle for the linearized operator $L_K - f'(u)$ (Proposition~\ref{Prop:MaximumPrincipleLinearized}).

As in part I \cite{FelipeSanz-Perela:IntegroDifferentialI}, equation \eqref{Eq:NonlocalAllenCahn} is driven by a linear integro-differential operator $L_K$ of the form
\begin{equation}
\label{Eq:DefOfLu}
L_K w(x) = \int_{\R^n} \{w(x) - w(y)\} K(x-y)\d y.
\end{equation}
%The integral in \eqref{Eq:DefOfLu} has to be understood in the principal value sense. 
The most canonical example of such operators is the fractional Laplacian, which corresponds to the kernel $K(z) = c_{n, \s} |z|^{-n-2\s}$, where $\s \in (0,1)$ and $c_{n, \s}$ is a normalizing positive constant ---see \eqref{Eq:ConstantFracLaplacian}.

Throughout the paper, we assume that $K$ is symmetric, i.e., $K(z) = K(-z)$, and that $L_K$ is uniformly elliptic, that is,
\begin{equation}
\label{Eq:Ellipticity}
\lambda \dfrac{c_{n,\s}}{|z|^{n+2\s}} \leq K(z) \leq \Lambda \dfrac{c_{n,\s}}{|z|^{n+2\s}}\,, 
\end{equation}
where $\lambda$ and $\Lambda$ are two positive constants. This condition is frequently adopted since it yields Hölder regularity of solutions (see \cite{RosOton-Survey,SerraC2s+alphaRegularity}). The family of linear operators satisfying this condition is the so-called $\lcal_0(n,\s,\lambda, \Lambda)$ ellipticity class. For short we will usually write $\lcal_0$ and we will make explicit the parameters only when needed. 

Following the previous article \cite{FelipeSanz-Perela:IntegroDifferentialI}, when dealing with doubly radial functions we will assume that the operator $L_K$ is rotation invariant, that is, $K$ is radially symmetric. This extra assumption allows us to rewrite the operator in a suitable form when acting on doubly radial odd functions, as explained below.

%%%%%%%%%%%%%%%%%%%%%%%%%%%%%%%%%%%%%%%%%%%%%%%%%%%%%%%%%%%%%%%%%%%%%%%%%%%%%%%%%%%%%%%%%%%%%%%%%%%%%%%%%%%%%%%%%%%%%%%%%%%%%%%%%%%%%%%%%%%%%%%%%%%%%%%%%%%%%%%%%%%%%%%%%%%%%%%%%%%%%%%%%%%%%%%%%%%%%%%%%%%%%%%%%%%%%%%%%%%%%%%%%%%%%%%%%%%%%%%%%%%%%%%%%%%%%%%%%%%%

\subsection{Integro-differential setting for odd functions with respect to the Simons cone}
\label{Subsec:Integro-differential setting}


%A property of $f$ that will be used through the paper is that, since $f$ is strictly concave in $(0,1)$ and $f(0)=0$, then 
%\begin{equation}
%\label{Eq:PropertyConcavityf}
%f'(\tau)\tau < f(\tau) \quad \textrm{ for all } \tau \in (0,1)\,.
%\end{equation}

In this subsection we recall the basic definitions and results established in part I \cite{FelipeSanz-Perela:IntegroDifferentialI}. First, we present the Simons cone, which is a central object along this paper. It is defined in $\R^{2m}$ by
%\begin{equation}
%\label{Eq:SimonsCone}
$$
\mathscr{C} := \setcond{x = (x', x'') \in \R^m \times \R^m = \R^{2m}}{|x'| = |x''|}\,.
%\end{equation}
$$
This cone is of importance in the theory of (local and nonlocal) minimal surfaces (see Subsection~\ref{Subsec:DeGiorgi}). 
%It has zero mean curvature at every point $x\in \ccal \setminus \{0\}$, in all even dimensions, and it is a minimizer of the perimeter functional when $2m\geq 8$. Concerning the nonlocal setting, $\ccal$ has also zero nonlocal mean curvature in all even dimensions, although it is not known if it is a minimizer of the nonlocal perimeter (see the introduction of \cite{Felipe-Sanz-Perela:SaddleFractional} and the references therein for more details).
We will use the letters $\ocal$ and $\ical$ to denote each of the parts in which $\R^{2m}$ is divided by the cone $\ccal$:
$$
\ocal:= \setcond{x = (x', x'') \in \R^{2m}}{|x'| > |x''|} \ \textrm{ and } \
\ical:= \setcond{x = (x', x'') \in \R^{2m}}{|x'| < |x''|}.
$$

Both $\ocal$ and $\ical$ belong to a family of sets in $\R^{2m}$ which are called of \emph{double revolution}. These are sets that are invariant under orthogonal transformations in the first $m$ variables, as well as under orthogonal transformations in the last $m$ variables. That is, $\Omega\subset \R^{2m}$ is a set of double revolution if $R\Omega = \Omega$ for every given transformation $R\in O(m)^2 = O(m) \times O(m)$, where  $O(m)$ is the orthogonal group of $\R^m$.


We say that a function $w:\R^{2m}  \to \R$ is \emph{doubly radial} if it depends only on the modulus of the first $m$ variables and on the modulus of the last $m$ ones, i.e., $w(x) = w(|x'|,|x''|)$. Equivalently, $w(Rx) = w(x)$ for every $R \in O(m)^2$.

We recall now the definition of $(\cdot)^\star$, an isometry that played a significant role in part~I \cite{FelipeSanz-Perela:IntegroDifferentialI}. It is defined by
\begin{equation}
\label{Eq:DefStar}
\begin{matrix}
(\cdot)^\star \colon & \R^{2m}= \R^{m}\times \R^{m}  &\to&  \R^{2m}= \R^{m}\times \R^{m}  \\
& x = (x',x'') &\mapsto & x^\star = (x'',x')\,.
\end{matrix}
\end{equation}
Note that this isometry is actually an involution that maps $\ocal$ into $\ical$ (and vice versa) and leaves the cone $\ccal$ invariant ---although not all points in $\ccal$ are fixed points of $(\cdot)^\star$. Taking into account this transformation, we say that a doubly radial function $w$ is \emph{odd with respect to the Simons cone} if $w(x) = -w(x^\star)$. Similarly, we say that a doubly radial function $w$ is \emph{even with respect to the Simons cone} if $w(x) = w(x^\star)$.



With these definitions at hand we can precisely define saddle-shaped solutions.
\begin{definition}
	\label{Def:SaddleShapedSol}
	We say that a bounded solution $u$ to \eqref{Eq:NonlocalAllenCahn} is a \emph{saddle-shaped solution} (or simply \emph{saddle solution}) if
	\begin{enumerate}
		\item $u$ is doubly radial.
		\item $u$ is odd with respect to the Simons cone.
		\item $u > 0$ in $\ocal = \{|x'| > |x''|\} $.
	\end{enumerate}
\end{definition}
Note that these solutions are even with respect to the coordinate axes and that their zero level set is the Simons cone $\mathscr{C} = \{|x'|=|x''|\}$. 



Let us collect now the main results of the previous paper \cite{FelipeSanz-Perela:IntegroDifferentialI} that will be used in the present one. Recall that if $K$ is a radially symmetric kernel we can rewrite the operator $L_K$ acting on a doubly radial function $w$ as
$$
L_K w(x) = \int_{\R^{2m}} \{w(x) - w(y)\} \overline{K}(x,y) \d y\,,
$$
where $\overline{K}$ is doubly radial in both variables and is defined by
\begin{equation}
\label{Eq:KbarDef}
\overline{K}(x,y) := \average_{O(m)^2} K(|Rx - y|)\d R\,.
\end{equation}
Here, $\d R$ denotes integration with respect to the Haar measure on $O(m)^2$ (see Section~2 of \cite{FelipeSanz-Perela:IntegroDifferentialI} for the details).

Moreover, if we consider doubly radial functions that are odd with respect to the Simons cone, we can use the involution $(\cdot)^\star$ to find that
\begin{equation}
\label{Eq:OperatorOddF}
L_K w (x) = \int_{\ocal} \{w(x) - w(y) \} \{\overline{K}(x, y) - \overline{K}(x, y^\star)  \} \d y +  2 w(x) \int_{\ocal} \overline{K}(x, y^\star) \d y \,.
\end{equation}
Furthermore,
\begin{equation}
\label{Eq:ZeroOrderTerm}
\frac{1}{C} \dist(x,\ccal)^{-2\s} \leq \int_{\ocal} \overline{K}(x, y^\star) \d y \leq C \dist(x,\ccal)^{-2\s},
\end{equation}
with $C>0$ depending only on $m, \s, \lambda$, and $\Lambda$ (see the details in part I \cite{FelipeSanz-Perela:IntegroDifferentialI}).


Note that the expression \eqref{Eq:OperatorOddF} has an integro-differential part plus a term of order zero with a positive coefficient. Thus, the most natural assumption to make in order to have an elliptic operator (when acting on doubly radial odd functions) is that the kernel of the integro-differential term is positive. That is, $\overline{K}(x, y) - \overline{K}(x, y^\star)>0$. One of the main results in part I \cite{FelipeSanz-Perela:IntegroDifferentialI}, stated next, established necessary and sufficient conditions on the original kernel $K$ for $L_K$ to have a positive kernel when acting on doubly radial odd functions. 

\begin{theorem}[\cite{FelipeSanz-Perela:IntegroDifferentialI}]
	\label{Th:SufficientNecessaryConditions}
	Let $K:(0,+\infty) \to (0,+\infty)$ and consider the radially symmetric kernel $K(|x-y|)$ in $\R^{2m}$. Define $\overline{K} : \R^{2m}\times \R^{2m} \to \R$ by \eqref{Eq:KbarDef}.
	
	If 
	\begin{equation}
	\label{Eq:SqrtConvex}	
	K(\sqrt{\tau}) \text{ is a strictly convex function of }\tau\,,
	\end{equation}
	then $L_K$ has a positive kernel in $\ocal$ when acting on doubly radial functions which are odd with respect to the Simons cone $\ccal$. More precisely, it holds
	\begin{equation}
	\label{Eq:KernelInequality}
	\overline{K}(x,y) > \overline{K}(x, y^\star) \quad \text{ for every }x,y \in \ocal\,.
	\end{equation}
	
	In addition, if $K\in C^2((0,+\infty))$, then \eqref{Eq:SqrtConvex} is not only a sufficient condition for \eqref{Eq:KernelInequality} to hold, but also a necessary one.
\end{theorem}

%%%%%%%%%%%%%%%%%%%%%%%%%%%%%%%%%%%%%%%%%%%%%%%%%%%%%%%%%%%%%%%%%%%%%%%%%%%%%%%%%%%%%%%%%%%%%%%%%%%%%%%%%%%%%%%%%%%%%%%%%%%%%%%%%%%%%%%%%%%%%%%%%%%%%%%%%%%%%%%%%%%%%%%%%%%%%%%%%%%%%%%%%%%%%%%%%%%%%%%%%%%%%%%%%%%%%%%%%%%%%%%%%%%%%%%%%%%%%%%%%%%%%%%%%%%%%%%%%%%%
\subsection{Main results}
\label{Subsec:Main results}

Through all the paper we will assume that $f$, the nonlinearity in \eqref{Eq:NonlocalAllenCahn}, is a $C^1$ function satisfying
\begin{equation}
\label{Eq:Hypothesesf}
f \textrm{ is odd, } \quad f(\pm 1)=0, \quad \text{ and } \quad f \textrm{ is strictly concave in }  (0,1).
\end{equation}
It is easy to see that these properties yield $f>0$ in $(0,1)$, $f'(0)>0$ and $f'(\pm 1) < 0$. 

The first main result of this paper concerns the existence of saddle-shaped solution.


\begin{theorem}[Existence of saddle-shaped solution]
	\label{Th:Existence}
	Let $f$ satisfy \eqref{Eq:Hypothesesf}. Let $K$ be a radially symmetric kernel satisfying the positivity condition \eqref{Eq:KernelInequality} and such that $L_K\in \lcal_0(2m, \s, \lambda, \Lambda)$. 
	
	Then, for every even dimension $2m \geq 2$, there exists a saddle-shaped solution $u$ to \eqref{Eq:NonlocalAllenCahn}. In addition, $u$ satisfies $|u|<1$ in $\R^{2m}$.
\end{theorem}



This theorem was already proved in part I \cite{FelipeSanz-Perela:IntegroDifferentialI} using variational techniques. Here, we show that existence can also be proved using, instead, the monotone iteration method. Let us remark that in both methods it is crucial to have the positivity condition \eqref{Eq:KernelInequality}.


The second main result of this paper is Theorem~\ref{Th:AsymptoticBehaviorSaddleSolution} below, on the asymptotic behavior of a saddle-shaped solution at infinity. To state it, let us introduce an important type of solutions in the study of the integro-differential Allen-Cahn equation: the layer solutions.


We say that a solution $v$ to $L_K v = f(v)$ in $\R^n$ is a \emph{layer solution} if $v$ is increasing in one direction, say $e\in \Sph^{n-1}$ and $v(x) \to \pm 1$ as $x\cdot e \to \pm \infty$ (not necessarily uniform). By a result of Cozzi and Passalacqua (Theorem~1 in \cite{CozziPassalacqua}), under the assumptions \eqref{Eq:Hypothesesf} on $f$, for every kernel $K_1$ such that $L_{K_1}\in \lcal_0(1,\s,\Lambda, \lambda)$ there exist a layer solution  to $L_{K_1} w = f(w)$ in $\R$ which is unique up to translations and is odd with respect to some point (in the case of the fractional Laplacian this result was proved in \cite{CabreSolaMorales,CabreSireII} by using the extension problem).

In $\R^n$, a special case of layer solutions are the one-dimensional ones. Actually, in relation with the available results concerning a conjecture by De Giorgi, in low dimensions all layer solutions are one-dimensional (see Subsection~\ref{Subsec:DeGiorgi}). One-dimensional layer solutions in $\R^n$ are in correspondence with the ones in $\R$ as explained next ---see also \cite{CozziPassalacqua}. Let $v$ be a layer solution to $L_K v = f(v)$ in $\R^n$ depending only on one direction, say $v(x) = w(x_n)$, and assume that $L_{K}\in \lcal_0(n,\s,\Lambda, \lambda)$. Then $w$ is a layer solution to $L_{K_1} w = f(w)$ in $\R$ with $K_1$ given by
$$
K_1(t) := \int_{\R^{n-1}} K\left(\theta,t\right) \d \theta = |t|^{n-1} \int_{\R^{n-1}} K\left(t\sigma,t\right) \d \sigma.
$$
Moreover $L_{K_1}\in \lcal_0(1,\s,\Lambda, \lambda)$. For more details see Proposition~\ref{Prop:KernelsDimension} in Section~\ref{Sec:Asymptotic} and \cite{CozziPassalacqua}. 

A particular layer solution, denoted by $u_0$, plays an important role in this paper. It is defined to be the unique solution of the following problem.
\begin{equation}
\label{Eq:LayerSolution}
\beqc{\PDEsystem}
L_{K_1}  u_0 &=& f(u_0) & \textrm{ in }\R\,,\\
\dot{u}_0 &>& 0 & \textrm{ in } \R\,,\\
u_0(x) & = &-u_0(-x)  & \textrm{ in }\R\,,\\
\ds \lim_{x \to \pm \infty} u_0(x) &=& \pm 1. & 
\eeqc
\end{equation}
Note that, by the previous comments, $v(x) = u_0(x_n)$ is a one-dimensional layer solution to $L_K v = f(v)$ in $\R^n$. Moreover, the same holds for $u_0(x\cdot e)$ for every $e\in \Sph^{n-1}$ whenever the kernel $K$ is radially symmetric.

The importance of the layer solution $u_0$ in relation with saddle-shaped solutions is that the associated function
\begin{equation}
\label{Eq:DefOfU}
U(x):= u_0 \left( \dfrac{|x'| - |x''|}{\sqrt{2}} \right)\,
\end{equation}
describes the asymptotic behavior of saddle solutions at infinity. Note that $(|x'| - |x''| )/\sqrt{2}$ is the signed distance to the Simons cone (see Lemma~4.2 in \cite{CabreTerraII}). Therefore, we can understand the function $U$ as the layer solution $u_0$ centered at each point of the Simons cone and oriented in the normal direction to the cone.

The precise statement on the asymptotic behavior of saddle-shaped solutions at infinity is the following.

\begin{theorem}
	\label{Th:AsymptoticBehaviorSaddleSolution}
	Let $f\in C^2(\R)$ satisfy \eqref{Eq:Hypothesesf}. Let $K$ be a radially symmetric kernel satisfying the positivity condition \eqref{Eq:KernelInequality} and such that $L_K\in \lcal_0(2m, \s, \lambda, \Lambda)$. Let $u$ be a saddle-shaped solution to \eqref{Eq:NonlocalAllenCahn} and let $U$ be the function defined by \eqref{Eq:DefOfU}.
	
	Then,
	$$
	\norm{u-U}_{L^\infty(\R^n\setminus B_R)}
	+\norm{\nabla u-\nabla U}_{L^\infty(\R^n\setminus B_R)}
	+\norm{D^2u-D^2U}_{L^\infty(\R^n\setminus B_R)} \to 0
	$$
	as $ R\to +\infty$.
\end{theorem}

To establish the asymptotic behavior of saddle-shaped solutions we use a compactness argument as in \cite{CabreTerraII, Cinti-Saddle, Cinti-Saddle2}, together with two important results established in Section~\ref{Sec:SymmetryResults}. The first one, Theorem~\ref{Th:LiouvilleSemilinearWholeSpace}, is a Liouville type result for nonnegative solutions to a semilinear equation in the whole space. 

\begin{theorem}
	\label{Th:LiouvilleSemilinearWholeSpace}
	Let $L_K \in \lcal_0(n,\s)$ and let $v$ be a bounded solution to
	\begin{equation}
	\label{Eq:PositiveWholeSpace}
	\beqc{\PDEsystem}
	L_K v &=& f(v) & \textrm{ in }\R^n\,,\\
	v &\geq& 0 & \textrm{ in } \R^n\,,
	\eeqc
	\end{equation}
	with a nonlinearity $f\in C^1$ satisfying
	\begin{itemize}
		\item $f(0) = f(1) = 0$,
		\item $f'(0)>0$,
		\item $f>0$ in $(0,1)$, and $f<0$ in $(1,+\infty)$.
	\end{itemize}
	Then, $v\equiv 0$ or $v \equiv 1$.
\end{theorem}

Similar classification results have been proved for the fractional Laplacian in \cite{ChenLiZhang,LiZhang} (either using the extension problem or not) with the method of moving spheres, which uses crucially the scale invariance of the operator $\fraclaplacian$. To the best of our knowledge, there is no similar result available in the literature for general kernels in the ellipticity class $\lcal_0$ (which are not necessarily scale invariant). Thus, we present here a proof based on the techniques used in \cite{BerestyckiHamelNadi} for a local equation with the classical Laplacian. It relies on the maximum principle, the translation invariance of the operator, a Harnack inequality and a stability argument. All these features are available for the operators in $\lcal_0$ (see Section~\ref{Sec:SymmetryResults}). Thus, the same arguments as in the local case can be carried out.

The second ingredient to prove the asymptotic behavior of saddle-shaped solutions is a symmetry result for equations in a half-space, stated next. Here and in the rest of the paper we use the notation $\R^n_+= \{(x_H,x_n)\in \R^{n-1}\times \R \ : \ x_n > 0\}$.  

\begin{theorem}
	\label{Th:SymmHalfSpace}
	Let $L_K\in \lcal_0(n,\s)$ and let $v$ be a bounded solution to one of these two problems:
	
	\begin{equation}
	\reqnomode
	\tag{P1}
	\label{Eq:P1}
	\beqc{\PDEsystem}
	L_K v &=& f(v)   &\textrm{ in } \,\R^n_+,\\
	v &>& 0   &\textrm{ in } \,\R^n_+,\\
	v(x_H,x_n) &=& -v(x_H,-x_n)   &\textrm{ in } \,\R^n.
	\eeqc
	\end{equation}
	
	\begin{equation}
	\reqnomode
	\tag{P2}
	\label{Eq:P2}
	\beqc{\PDEsystem}
	L_K v &=& f(v)   &\textrm{ in } \,\R^n_+,\\
	v &>& 0   &\textrm{ in } \,\R^n_+,\\
	v &=& 0   &\textrm{ in } \,\R^n \setminus \R^n_+.
	\eeqc
	\end{equation}
	
	\reqnomode
	
	Assume that, in $\R^n_+$, the kernel $K$ of the operator $L_K$ is decreasing in the direction of $x_n$, i.e., it satisfies
	$$
	K(x_H-y_H,x_n-y_n) \geq K(x_H-y_H,x_n+y_n) \,\,\,\,\text{for all } \,\, x,y\in \R^n_+.
	$$ 
	Suppose that $f\in C^1$ and
	\begin{itemize}
		\item $f(0) = f(1) = 0$,
		\item $f'(0)>0$, and $f'(t)\leq 0$ for all $t\in[1-\delta,1]$ for some $\delta>0$,
		\item $f>0$ in $(0,1)$, and
		\item $f$ is odd in the case of \eqref{Eq:P1}.
	\end{itemize}
	Then, $v$ depends only on $x_n$ and it is increasing in that direction.
\end{theorem}

The result for \eqref{Eq:P2} has been proved for the fractional Laplacian under some assumptions on $f$ (weaker than the ones in Theorem~\ref{Th:SymmHalfSpace}) in \cite{QuaasXia, BarriosEtAl-Monotonicity, BarriosEtAl-Symmetry, FallWethMonotonicity}. Instead, to the best of our knowledge \eqref{Eq:P1} has not been treated even for the fractional Laplacian. In our case, the fact that $f$ is of Allen-Cahn type allows us to use rather simple arguments that work for both problems \eqref{Eq:P1} and \eqref{Eq:P2} ---moving planes and sliding methods. Moreover, the fact that we replace the kernel of the operator by a general $K$ satisfying \eqref{Eq:Ellipticity} do not affect significantly the proof. Although \eqref{Eq:P2} will not be used in this paper, since the proof for this problem is analogous to the one for \eqref{Eq:P1}, we include it here for future reference, 

The last main result of this paper is the uniqueness of the saddle-shaped solution, stated next.

\begin{theorem}[Uniqueness of the saddle-shaped solution]
	\label{Th:Uniqueness}
	Let $f$ satisfy \eqref{Eq:Hypothesesf} and let $K$ be a radially symmetric kernel satisfying the positivity condition \eqref{Eq:KernelInequality} and such that $L_K\in \lcal_0(2m, \s, \lambda, \Lambda)$. 
	
	Then, for every dimension $2m \geq 2$, there exists a unique saddle-shaped solution to \eqref{Eq:NonlocalAllenCahn}.
\end{theorem}

To prove this result we need two ingredients. The first one is the asymptotic behavior of saddle solutions given in Theorem~\ref{Th:AsymptoticBehaviorSaddleSolution}. The second one is a maximum principle in $\ocal$ for the linearized operator $L_K - f'(u)$, which is given in Proposition~\ref{Prop:MaximumPrincipleLinearized}. To establish it, we will need to use a maximum principle in ``narrow'' sets, also proved in Section~\ref{Sec:MaximumPrinciple}. In the arguments, it is crucial again the positivity condition \eqref{Eq:KernelInequality}.

%%%%%%%%%%%%%%%%%%%%%%%%%%%%%%%%%%%%%%%%%%%%%%%%%%%%%%%%%%%%%%%%%%%%%%%%%%%%%%%%%%%%%%%%%%%%%%%%%%%%%%%%%%%%%%%%%%%%%%%%%%%%%%%%%%%%%%%%%%%%%%%%%%%%%%%%%%%%%%%%%%%%%%%%%%%%%%%%%%%%%%%%%%%%%%%%%%%%%%%%%%%%%%%%%%%%%%%%%%%%%%%%%%%%%%%%%%%%%%%%%%%%%%%%%%%%%%%%%%%%
\subsection{Saddle-shaped solutions in the context of a conjecture by De Giorgi}
\label{Subsec:DeGiorgi}

To conclude this introduction, let us make some comments on the importance of problem \eqref{Eq:NonlocalAllenCahn} and its relation with the theory of (classical and nonlocal) minimal surfaces and  a famous conjecture raised by De Giorgi.

A main open problem (even in the local case) is to determine whether the saddle-shaped solution is a minimizer of the energy functional associated to the equation, depending on the dimension $2m$. This question is deeply related to the regularity theory of local and nonlocal minimal surfaces, as explained next.

In the seventies, Modica and Mortola (see \cite{Modica,ModicaMortola}) proved that, considering an appropriately rescaled version of the (local) Allen-Cahn equation, the corresponding energy functionals $\Gamma$-converge to the perimeter functional. Thus, the blow-down sequence of minimizers of the Allen-Cahn energy converge to the characteristic function of a set of minimal perimeter. This same fact holds for the equation with the fractional Laplacian, though we have two different scenarios depending on the parameter $\s \in (0,1)$. If $\s \geq 1/2$, the rescaled energy functionals associated to the equation $\Gamma$-converge to the classical perimeter (see \cite{GiovanniBouchitteSeppecher,Gonzalez}), while in the case $\s \in (0,1/2)$ they $\Gamma$-converge to the fractional perimeter (see \cite{SavinValdinoci-GammaConvergence}). As a consequence, if the saddle-shaped solution was proved to be a minimizer in a certain dimension for some $\s \in (0,1/2)$, it would follow that the Simons cone $\ccal$ would be a minimizing nonlocal $(2\s)$-minimal surface in such dimensions. This last statement on the saddle-shaped solution is an open problem in any dimension (although it is known that the Simons cone is not a minimizer in dimension $2m=2$). The only available result related to this question is the recent one in our previous paper \cite{Felipe-Sanz-Perela:SaddleFractional}, which concerns stability (a weaker property than minimality). We proved that the saddle-shaped solution to the fractional Allen-Cahn equation is stable in dimensions $2m\geq 14$. As a consequence of this and a result in \cite{CabreCintiSerra-Stable}, the Simons cone is a stable nonlocal $(2\s)$-minimal surface in dimensions $ 2m\geq 14$ (see the details in \cite{Felipe-Sanz-Perela:SaddleFractional}).


Moreover, as explained below, saddle-shaped solutions are natural objects to build a counterexample to a famous conjecture raised by De Giorgi, that reads as follows. Let $u$ be a bounded solution to $-\Delta  u = u - u^3$ in $\R^n$ which is monotone in one direction, say $\partial_{x_n} u > 0$. Then, if $n\leq 8$, $u$ is one dimensional, i.e., $u$ depends only on one Euclidean variable. This conjecture was proved to be true in dimensions $n=2$ and  $n=3$ (see \cite{GhoussoubGui,AmbrosioCabre}), and in dimensions $4\leq n \leq 8$ with the extra assumption
\begin{equation}
\label{Eq:SavinCondition}
\lim_{x_n \to \pm \infty} u(x_H,x_n) = \pm 1 \quad \text{ for all } x_H\in \R^{n-1}\,,
\end{equation}
(see \cite{Savin-DeGiorgi}). A counterexample to the conjecture in dimensions $n\geq 9$ was given in \cite{delPinoKowalczykWei} by using the gluing method. 

An alternative approach to the one of \cite{delPinoKowalczykWei} to construct a counterexample to the conjecture was given by Jerison and Monneau in \cite{JerisonMonneau}. They showed that a counterexample in $\R^{n+1}$ can be constructed with a rather natural procedure if there exists a global minimizer of $-\Delta u = f(u)$ in $\R^n$ which is bounded and even with respect to each coordinate but is not one-dimensional. The saddle-shaped solution is of special interest in search of this counterexample, since it is even with respect to all the coordinate axis and it is canonically associated to the Simons cone, which in turn is the simplest nonplanar minimizing minimal surface. Therefore, by proving that the saddle solution to the classical Allen-Cahn equation is a minimizer in some dimension $2m$, one would obtain automatically a counterexample to the conjecture in $\R^{2m+1}$.

The corresponding conjecture in the fractional setting, where one replaces the operator $-\Delta$ by $\fraclaplacian$, has been widely studied in the last years. In this framework, the conjecture has been proven to be true for all $\s\in(0,1)$ in dimensions $n=2$ (see \cite{CabreSolaMorales,CabreSireI,SireValdinoci}) and $n=3$ (see \cite{CabreCinti-EnergyHalfL, CabreCinti-SharpEnergy,DipierroFarinaValdinoci}). The conjecture is also true in dimension $n=4$ in the case of $\s = 1/2$ (see \cite{FigalliSerra}) and if $\s\in(0,1/2)$ is close to $1/2$ (see \cite{CabreCintiSerra-Stable}). Assuming the additional hypothesis \eqref{Eq:SavinCondition}, the conjecture is true in dimensions $4\leq n \leq 8$ for $1/2 \leq \s < 1$ (see \cite{Savin-Fractional,Savin-Fractional2}), and also for $\s\in(0,1/2)$ if $\s$ is close to $1/2$ (see \cite{DipierroSerraValdinoci}). A counterexample to the De Giorgi conjecture for the fractional Allen-Cahn equation in dimensions $n \geq 9$ for $\s \in (1/2,1)$ has been very recently announced in \cite{ChanLiuWei}.

Concerning the conjecture with more general operators like $L_K$, fewer results are known. In dimension $n=2$ the conjecture is proved in \cite{HamelRosOtonSireValdinoci, Bucur, FazlySire}, under different assumptions on the kernel $K$ and even for more general nonlinear operators. Note also that the results of \cite{DipierroSerraValdinoci} also hold for a particular class of kernels in $\lcal_0$.

%%%%%%%%%%%%%%%%%%%%%%%%%%%%%%%%%%%%%%%%%%%%%%%%%%%%%%%%%%%%%%%%%%%%%%%%%%%%%%%%%%%%%%%%%%%%%%%%%%%%%%%%%%%%%%%%%%%%%%%%%%%%%%%%%%%%%%%%%%%%%%%%%%%%%%%%%%%%%%%%%%%%%%%%%%%%%%%%%%%%%%%%%%%%%%%%%%%%%%%%%%%%%%%%%%%%%%%%%%%%%%%%%%%%%%%%%%%%%%%%%%%%%%%%%%%%%%%%%%%%
\subsection{Plan of the article}
\label{Subsec:Plan}

The paper is organized as follows. In Section~\ref{Sec:Preliminaries} we present some preliminary results that will be used in the rest of the article. Section~\ref{Sec:Existence} contains the proof of Theorem~\ref{Th:Existence} on the existence of a saddle-shaped solution via the monotone iteration method. In Section~\ref{Sec:SymmetryResults} we establish the Liouville type and symmetry results, Theorems~\ref{Th:LiouvilleSemilinearWholeSpace} and \ref{Th:SymmHalfSpace}. Section~\ref{Sec:Asymptotic} is devoted to the layer solution $u_0$ of problem \eqref{Eq:NonlocalAllenCahn} and the proof of the asymptotic behavior of saddle-shaped solutions, Theorem~\ref{Th:AsymptoticBehaviorSaddleSolution}. Finally, Section~\ref{Sec:MaximumPrinciple} concerns the proof of a maximum principle in $\ocal$ for the linearized operator $L_K - f'(u)$ (Proposition~\ref{Prop:MaximumPrincipleLinearized}), as well as the proof of Theorem~\ref{Th:Uniqueness}, establishing the uniqueness of the saddle-shaped solution.

