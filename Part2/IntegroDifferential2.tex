\documentclass[12pt,reqno]{amsart}

%%%%%%%%%%%%%%%%%%%%%%%%%%%%%%%%%%%%%%%%%%%%%%%%%%%%%%%%%%%%%%%%%%%%%%%%%%%%%%%
%%%%%%%%%%%   PACKAGES %%%%%%%%%%%%%%%%%%%%%%%%%%%%%%%%%%%%%%%%%%%%%%%%%%%%%%%%
%%%%%%%%%%%%%%%%%%%%%%%%%%%%%%%%%%%%%%%%%%%%%%%%%%%%%%%%%%%%%%%%%%%%%%%%%%%%%%%

\usepackage[utf8]{inputenc}

%%---AMS packages---%% (automatically loaded in amsart or amsbook)
\usepackage{amsmath} %General package for maths (loads also amsbsy to make bold symbols)
\usepackage{amsthm} %Package for theorems
\usepackage{amssymb} %Symbols (loads also amsfonts)
\usepackage{amscd} %Package for rectangular diagrams
%\usepackage{amsrefs} %Gestió de referències


\usepackage{emptypage}
%\usepackage{pdfpages} %Include pdf documents

%\usepackage{dsfont} % Double stroke roman fonts
\usepackage{enumitem} %Enumerations
\usepackage{mathtools} %More symbols, etc.
\usepackage{mathrsfs} %Calligraphic symbols with \mathscr
%\usepackage{fancyhdr} %Customize the header
\usepackage{hyperref} %Makes references hyperlinks
\hypersetup{
	%true = make page number, not text, be link
    linktocpage = true,
    %set to all if you want both sections and subsections linked
    %linktoc=all,
    colorlinks=true,
    linkcolor=red,
}
\usepackage{esint} %To write averaged integrals

%%%FIGURES AND IMAGES
\usepackage{graphicx} %For using \includegraphics
%\usepackage{wrapfig} %For figures wrapped inside the text
%\usepackage{float} %Interface for defining floating objects
%\usepackage{caption} %Customize the captions in floating environments
%\usepackage{subfigure}
\usepackage{subcaption}
%\usepackage{tcolorbox} %For boxing
%\tcbuselibrary{theorems}
%\tcbuselibrary{skins}

\usepackage{pgf,tikz,pgfplots}
\usetikzlibrary{arrows}
\usetikzlibrary{intersections}


%%%EDITION HELP
\usepackage[colorinlistoftodos]{todonotes} %Package to add TODO notes in colors
\setlength{\marginparwidth}{2.3cm} % make todonotes appear wider
%\usepackage{showlabels} %Show the labels of equations and theorems
%\usepackage{showkeys} %Show the keys of all references
\usepackage{refcheck}


%%%%%%%%%%%%%%%%%%%%%%%%%%%%%%%%%%%%%%%%%%%%%%%%%%%%%%%%%%%%%%%%%%%%%%%%%%%%%%%
%%%%%%%%%%%   THEOREM ENVIRONMENTS %%%%%%%%%%%%%%%%%%%%%%%%%%%%%%%%%%%%%%%%%%%%
%%%%%%%%%%%%%%%%%%%%%%%%%%%%%%%%%%%%%%%%%%%%%%%%%%%%%%%%%%%%%%%%%%%%%%%%%%%%%%%

\newtheorem{theorem}{Theorem}[section]
\newtheorem*{theorem*}{Theorem}
\newtheorem{proposition}[theorem]{Proposition}
\newtheorem{lemma}[theorem]{Lemma}
\newtheorem{corollary}[theorem]{Corollary}
\newtheorem{conjecture}[theorem]{Conjecture}

\theoremstyle{definition}
\newtheorem{definition}[theorem]{Definition}
\newtheorem{example}[theorem]{Example}

\theoremstyle{remark}
\newtheorem{remark}[theorem]{Remark}
\newtheorem*{remark*}{Remark}
\newtheorem{observation}[theorem]{Observation}

%%%%%%%%%%%%%%%%%%%%%%%%%%%%%%%%%%%%%%%%%%%%%%%%%%%%%%%%%%%%%%%%%%%%%%%%%%%%%%%
%%%%%%%%%%%   MACROS/ABREVIATIONS  %%%%%%%%%%%%%%%%%%%%%%%%%%%%%%%%%%%%%%%%%%%%
%%%%%%%%%%%%%%%%%%%%%%%%%%%%%%%%%%%%%%%%%%%%%%%%%%%%%%%%%%%%%%%%%%%%%%%%%%%%%%%

\newcommand{\con}[1]{\mathbb{#1}}
\newcommand{\C}{\con{C}} %Complex
\newcommand{\R}{\con{R}} %Real
\newcommand{\Q}{\con{Q}} %Rational
\newcommand{\Z}{\con{Z}} %Integer
\newcommand{\N}{\con{N}} %Natural
\newcommand{\T}{\con{T}} %Torus
\newcommand{\Sph}{\con{S}} %Sphere
\renewcommand{\H}{\con{H}}
\newcommand{\ccal}{\mathscr{C}}
\newcommand{\ecal}{\mathcal{E}}
\newcommand{\ical}{\mathcal{I}}
\newcommand{\lcal}{\mathcal{L}}
\newcommand{\ncal}{\mathcal{N}}
\newcommand{\ocal}{\mathcal{O}}
\newcommand{\ucal}{\mathcal{U}}
\newcommand{\e}{\mathrm{e}}


\makeatletter
\newcommand{\leqnomode}{\tagsleft@true\let\veqno\@@leqno}
\newcommand{\reqnomode}{\tagsleft@false\let\veqno\@@eqno}
\makeatother

\newcommand{\limn}{\displaystyle\lim_{n \rightarrow \infty}} %limit n to infinity
\newcommand{\norm}[1]{\left | \left |{#1} \right | \right |}
\newcommand{\seminorm}[1]{\left [ {#1} \right ] }
\newcommand{\laplacian}{\Delta}
\newcommand{\s}{\gamma}
\newcommand{\fraclaplacian}{(-\Delta)^\s}
\newcommand{\Lr}{\mathcal{L}_{\mathrm{r}}}
\newcommand{\loc}{\mathrm{loc}}
\renewcommand{\d}{\,\mathrm{d}} %straight d with small space before
\newcommand{\dx}{\,\mathrm{d}x} %The usual differential of x
\newcommand{\ddt}[1][]{\dfrac{\d{#1}}{\d t}} %Derivative with respect t
\newcommand{\ddtpar}[1]{ \dfrac{\d}{\d t}\Big ( {#1} \Big)} %Derivative with respect t with parenthesis
\newcommand{\lp}[1]{\mathrm{L}^{#1}}%L^p spaces
\newcommand{\hp}[1]{\mathrm{H}^{#1}}%H^p spaces
\newcommand{\sobolev}[1]{\mathrm{W}^{#1}}%Sobolev spaces
\newcommand{\cp}[1]{\mathcal{C}^{#1}}%C^p spaces
\newcommand{\bpar}[1]{\left ( {#1}\right )}
\newcommand{\setcond}[2]{\left \{ #1 \ : \ #2  \right \}}
\newcommand\evalat[1]{_{\mkern1.5mu\big\vert_{\scriptstyle #1}}}
\newcommand{\normLinf}[1]{\left | \left |{#1} \right | \right |_{\lp{\infty}}}
%\newcommand{\averageline}{{\mathchoice
%{\kern1ex\vcenter{\hrule height.4pt width 6pt depth0pt} \kern-9.3pt} {}
%{\kern1ex\vcenter{\hrule height.4pt width 4.3pt depth0pt} \kern-7pt}
%{} {} }}
%\newcommand{\average}{\averageline \!\int}

\newcommand{\average}{\fint}

\newcommand\beqc[1]{\left\{\begin{array}{#1}}
\newcommand\eeqc{\end{array} \right.}
\def\PDEsystem{rcll}
\def\bmatrix{\begin{pmatrix}}
\def\ematrix{\end{pmatrix}}
\DeclareMathOperator{\Tr}{Tr}
\DeclareMathOperator{\tr}{tr}
\DeclareMathOperator{\Det}{Det}
\DeclareMathOperator{\dist}{dist}
\DeclareMathOperator{\diam}{diam}
\DeclareMathOperator{\sign}{sign}
\DeclareMathOperator{\sech}{sech}
\DeclareMathOperator{\PV}{P.V.}
\let\div\relax
\DeclareMathOperator{\div}{div}
\newcommand{\usub}{\underline{u}}
\newcommand{\usup}{\overline{u}}
\newcommand{\vsub}{\underline{v}}
\newcommand{\vsup}{\overline{v}}
\def\ds{\displaystyle}

%%%%%%%%%%%%%%%%%%%%%%%%%%%%%%%%%%%%%%%%%%%%%%%%%%%%%%%%%%%%%%%%%%%%%%%%%%%%%%%
%%%%%%%%%%% SETTINGS %%%%%%%%%%%%%%%%%%%%%%%%%%%%%%%%%%%%%%%%%%%%%%%%%%%%%%%%%%
%%%%%%%%%%%%%%%%%%%%%%%%%%%%%%%%%%%%%%%%%%%%%%%%%%%%%%%%%%%%%%%%%%%%%%%%%%%%%%%
\setlength{\textwidth}{155mm}
\setlength{\textheight}{210mm}
\oddsidemargin=5mm
\evensidemargin=5mm

\numberwithin{equation}{section}

\graphicspath {{Images/}}

%%%%%%%%%%%%%%%%%%%%%%%%%%%%%%%%%%%%%%%%%%%%%%%%%%%%%%%%%%%%%%%%%%%%%%%%%%%%%%%
%%%%%%%%%%% HYPHENATION %%%%%%%%%%%%%%%%%%%%%%%%%%%%%%%%%%%%%%%%%%%%%%%%%%%%%%%%%%
%%%%%%%%%%%%%%%%%%%%%%%%%%%%%%%%%%%%%%%%%%%%%%%%%%%%%%%%%%%%%%%%%%%%%%%%%%%%%%%
\hyphenation{auto-ma-ti-cally}

%%%%%%%%%%%%%%%%%%%%%%%%%%%%%%%%%%%%%%%%%%%%%%%%%%%%%%%%%%%%%%%%%%%%%%%%%%%%%%%
%%%%%%%%%%% GENERAL INFORMATION %%%%%%%%%%%%%%%%%%%%%%%%%%%%%%%%%%%%%%%%%%%%%%%
%%%%%%%%%%%%%%%%%%%%%%%%%%%%%%%%%%%%%%%%%%%%%%%%%%%%%%%%%%%%%%%%%%%%%%%%%%%%%%%

\title[Semilinear integro-differential equations II]{Semilinear integro-differential equations, II: one-dimensional and saddle-shaped solutions to the Allen-Cahn equation}
\author{Juan-Carlos Felipe-Navarro}
\address{J.C. Felipe-Navarro:
Universitat Polit\`ecnica de Catalunya and BGSMath, Departament de Matem\`{a}tiques, Diagonal 647, 08028 Barcelona, Spain}
\email{juan.carlos.felipe@upc.edu}

\author{Tomás Sanz-Perela}
\address{T. Sanz-Perela:
Universitat Polit\`ecnica de Catalunya and BGSMath, Departament de Matem\`{a}tiques, Diagonal 647, 08028 Barcelona, Spain}
\email{tomas.sanz@upc.edu}







\thanks{Both authors acknowledge financial support from the Spanish Ministry of Economy and Competitiveness (MINECO), through the María de Maeztu Programme for Units of Excellence in R\&D (MDM-2014-0445-16-4 and MDM-2014-0445 respectively). Moreover, both authors are supported by MINECO grants MTM2014-52402-C3-1-P and MTM2017-84214-C2-1-P, are members of the Barcelona Graduate School of Mathematics (BGSMath) and are part of the Catalan research group 2017 SGR 01392.}


\keywords{Integro-differential semilinear equation, saddle-shaped solution, odd symmetry, Simons cone, symmetry results}



%%%%%%%%%%%%%%%%%%%%%%%%%%%%%%%%%%%%%%%%%%%%%%%%%%%%%%%%%%%%%%%%%%%%%%%%%%%%%%%
%%%%%%%%%%%%%%%%%%%%%%%%%%%%%%%%%%%%%%%%%%%%%%%%%%%%%%%%%%%%%%%%%%%%%%%%%%%%%%%
%%%%%%%%%%%%%%%%%%%%%%%%%%%%%%%%%%%%%%%%%%%%%%%%%%%%%%%%%%%%%%%%%%%%%%%%%%%%%%%
%%%%%%%%%%%%%%%%%%%%%%%%%%%%%%%%%%%%%%%%%%%%%%%%%%%%%%%%%%%%%%%%%%%%%%%%%%%%%%%
%%%%%%%%%%%%%%%%%%%%%%%%%%%%%%%%%%%%%%%%%%%%%%%%%%%%%%%%%%%%%%%%%%%%%%%%%%%%%%%
\begin{document}


%%%%%%%%%%%%%%%%%%%%%%%%%%%%%%%%%%%%%%%%%%%%%%%%%%%%%%%%%%%%%%%%%%%%%%%%%%%%%%%
%%%%%%%%%%%%%%%%%%%%%%%%%%%%%%%%%%%%%%%%%%%%%%%%%%%%%%%%%%%%%%%%%%%%%%%%%%%%%%%
\begin{abstract}
This paper, which is the follow-up to part I, concerns saddle-shaped solutions to the semilinear equation $L_K u = f(u)$ in $\R^{2m}$, where $L_K$ is a linear elliptic integro-differential operator with a radially symmetric kernel and $f$ is of Allen-Cahn type. Saddle-shaped solutions are doubly radial, odd with respect to the Simons cone $\{(x', x'') \in \R^m \times \R^m \, : \, |x'| = |x''|\}$, and vanish only in this set.

Following the setting previously established in part I for doubly radial odd functions, we show existence, asymptotic behavior, and uniqueness of the saddle-shaped solution. For this, we prove, among others, some symmetry results and maximum principles in ``narrow'' sets. We also need to study one-dimensional solutions to $L_K u = f(u)$ in $\R^{n}$, that we do by relating this equation with an analogous one in $\R$, for a related operator.
\end{abstract}
%%%%%%%%%%%%%%%%%%%%%%%%%%%%%%%%%%%%%%%%%%%%%%%%%%%%%%%%%%%%%%%%%%%%%%%%%%%%%%%
%%%%%%%%%%%%%%%%%%%%%%%%%%%%%%%%%%%%%%%%%%%%%%%%%%%%%%%%%%%%%%%%%%%%%%%%%%%%%%%


\maketitle


\tableofcontents

%%%%%%%%%%%%%%%%%%%%%%%
\section{Introduction}
%%%%%%%%%%%%%%%%%%%%%%%
\label{Sec:Introduction}
 
In this paper, which is the second part of \cite{FelipeSanz-Perela:IntegroDifferentialI}, we study saddle-shaped solutions to the semilinear equation
\begin{equation}
\label{Eq:NonlocalAllenCahn}
L_K u = f(u) \quad \textrm{ in } \R^{2m},
\end{equation}
where $L_K$ is a linear integro-differential operator of the form \eqref{Eq:DefOfLu} and $f$ is of Allen-Cahn type. These  solutions (see Definition~\ref{Def:SaddleShapedSol} below) are particularly interesting in relation to the nonlocal version of a conjecture by De Giorgi, with the aim of finding a counterexample in high dimensions. Moreover, this problem is related to the regularity theory of nonlocal minimal surfaces. For more comments on this we refer to Subsection~\ref{Subsec:DeGiorgi} and the references therein.

Previous to this article and its first part \cite{FelipeSanz-Perela:IntegroDifferentialI}, there are only three works devoted to saddle-shaped solutions to the equation \eqref{Eq:NonlocalAllenCahn} with $L_K$ being the fractional Laplacian. In  \cite{Cinti-Saddle,Cinti-Saddle2}, Cinti proved the existence of a saddle-shaped solution as well as some qualitative properties such as asymptotic behavior, monotonicity properties, and instability whenever $2m\leq 6$. In a previous paper by the authors \cite{Felipe-Sanz-Perela:SaddleFractional}, further properties of these solutions were proved, the main ones being uniqueness and, when $2m\geq 14$, stability. Concerning saddle-shaped solutions to the classical Allen-Cahn equation $-\laplacian u = f(u)$, the same results were proved in \cite{DangFifePeletier, Schatzman, CabreTerraI,CabreTerraII, Cabre-Saddle}. The possible stability in dimensions $8$, $10$, and $12$ is still an open problem (both in the local and fractional frameworks), as well as the possible minimality of this solution in dimensions $2m \geq 8$.

The present paper together with its first part \cite{FelipeSanz-Perela:IntegroDifferentialI} are the first ones studying saddle-shaped solutions for general integro-differential equations of the form \eqref{Eq:NonlocalAllenCahn}. In the three previous papers \cite{Cinti-Saddle, Cinti-Saddle2, Felipe-Sanz-Perela:SaddleFractional} the main tool used was the extension problem for the fractional Laplacian (see \cite{CaffarelliSilvestre}). Nevertheless, this technique has the limitation that it cannot be carried out for general integro-differential operators different from the fractional Laplacian. Therefore, some purely nonlocal techniques were developed in the previous paper \cite{FelipeSanz-Perela:IntegroDifferentialI} to study saddle-shaped solutions, and we exploit them in the present one.

In part~I \cite{FelipeSanz-Perela:IntegroDifferentialI}, we established an appropriate setting to study solutions to \eqref{Eq:NonlocalAllenCahn} that are doubly radial and odd with respect to the Simons cone, a property that is satisfied by saddle-shaped solutions (see Subsection~\ref{Subsec:Integro-differential setting}). In that paper we deduced an alternative expression for the operator $L_K$ when acting on doubly radial odd functions ---see \eqref{Eq:OperatorOddF}. This was used to deduce some maximum principles for odd functions under certain assumptions on the kernel $K$ of the operator $L_K$. Moreover, we proved an energy estimate for doubly radial and odd minimizers of the energy associated to the equation, as well as the existence of saddle-shaped solutions to \eqref{Eq:NonlocalAllenCahn}.

In the present paper, we further study saddle-shaped solutions to \eqref{Eq:NonlocalAllenCahn} by using the results obtained in part~I \cite{FelipeSanz-Perela:IntegroDifferentialI}. First, we prove existence of this type of solutions, Theorem~\ref{Th:Existence}, by using the monotone iteration method (as an alternative to the proof in \cite{FelipeSanz-Perela:IntegroDifferentialI} where variational methods are used). After this, we establish the asymptotic behavior of saddle-shaped solutions,  Theorem~\ref{Th:AsymptoticBehaviorSaddleSolution}. To do it, we use two ingredients: a Liouville type theorem and a one-dimensional symmetry result, both for semilinear equations like \eqref{Eq:NonlocalAllenCahn} under some hypotheses on $f$. These are Theorems~\ref{Th:LiouvilleSemilinearWholeSpace} and \ref{Th:SymmHalfSpace}, proved in Section~\ref{Sec:SymmetryResults}. In the study of the asymptotic behavior of saddle-shaped solutions we establish further properties of the so-called \emph{layer solution} $u_0$ (see Section~\ref{Sec:Asymptotic}). Finally, we show the uniqueness of the saddle-shaped solution, Theorem~\ref{Th:Uniqueness}, by using a maximum principle for the linearized operator $L_K - f'(u)$ (Proposition~\ref{Prop:MaximumPrincipleLinearized}).

As in part I \cite{FelipeSanz-Perela:IntegroDifferentialI}, equation \eqref{Eq:NonlocalAllenCahn} is driven by a linear integro-differential operator $L_K$ of the form
\begin{equation}
\label{Eq:DefOfLu}
L_K w(x) = \int_{\R^n} \{w(x) - w(y)\} K(x-y)\d y.
\end{equation}
%The integral in \eqref{Eq:DefOfLu} has to be understood in the principal value sense. 
The most canonical example of such operators is the fractional Laplacian, which corresponds to the kernel $K(z) = c_{n, \s} |z|^{-n-2\s}$, where $\s \in (0,1)$ and $c_{n, \s}$ is a normalizing positive constant ---see \eqref{Eq:ConstantFracLaplacian}.

Throughout the paper, we assume that $K$ is symmetric, i.e., $K(z) = K(-z)$, and that $L_K$ is uniformly elliptic, that is,
\begin{equation}
\label{Eq:Ellipticity}
\lambda \dfrac{c_{n,\s}}{|z|^{n+2\s}} \leq K(z) \leq \Lambda \dfrac{c_{n,\s}}{|z|^{n+2\s}}\,, 
\end{equation}
where $\lambda$ and $\Lambda$ are two positive constants. This condition is frequently adopted since it yields Hölder regularity of solutions (see \cite{RosOton-Survey,SerraC2s+alphaRegularity}). The family of linear operators satisfying this condition is the so-called $\lcal_0(n,\s,\lambda, \Lambda)$ ellipticity class. For short we will usually write $\lcal_0$ and we will make explicit the parameters only when needed. 

Following the previous article \cite{FelipeSanz-Perela:IntegroDifferentialI}, when dealing with doubly radial functions we will assume that the operator $L_K$ is rotation invariant, that is, $K$ is radially symmetric. This extra assumption allows us to rewrite the operator in a suitable form when acting on doubly radial odd functions, as explained below.

%%%%%%%%%%%%%%%%%%%%%%%%%%%%%%%%%%%%%%%%%%%%%%%%%%%%%%%%%%%%%%%%%%%%%%%%%%%%%%%%%%%%%%%%%%%%%%%%%%%%%%%%%%%%%%%%%%%%%%%%%%%%%%%%%%%%%%%%%%%%%%%%%%%%%%%%%%%%%%%%%%%%%%%%%%%%%%%%%%%%%%%%%%%%%%%%%%%%%%%%%%%%%%%%%%%%%%%%%%%%%%%%%%%%%%%%%%%%%%%%%%%%%%%%%%%%%%%%%%%%

\subsection{Integro-differential setting for odd functions with respect to the Simons cone}
\label{Subsec:Integro-differential setting}


%A property of $f$ that will be used through the paper is that, since $f$ is strictly concave in $(0,1)$ and $f(0)=0$, then 
%\begin{equation}
%\label{Eq:PropertyConcavityf}
%f'(\tau)\tau < f(\tau) \quad \textrm{ for all } \tau \in (0,1)\,.
%\end{equation}

In this subsection we recall the basic definitions and results established in part I \cite{FelipeSanz-Perela:IntegroDifferentialI}. First, we present the Simons cone, which is a central object along this paper. It is defined in $\R^{2m}$ by
%\begin{equation}
%\label{Eq:SimonsCone}
$$
\mathscr{C} := \setcond{x = (x', x'') \in \R^m \times \R^m = \R^{2m}}{|x'| = |x''|}\,.
%\end{equation}
$$
This cone is of importance in the theory of (local and nonlocal) minimal surfaces (see Subsection~\ref{Subsec:DeGiorgi}). 
%It has zero mean curvature at every point $x\in \ccal \setminus \{0\}$, in all even dimensions, and it is a minimizer of the perimeter functional when $2m\geq 8$. Concerning the nonlocal setting, $\ccal$ has also zero nonlocal mean curvature in all even dimensions, although it is not known if it is a minimizer of the nonlocal perimeter (see the introduction of \cite{Felipe-Sanz-Perela:SaddleFractional} and the references therein for more details).
We will use the letters $\ocal$ and $\ical$ to denote each of the parts in which $\R^{2m}$ is divided by the cone $\ccal$:
$$
\ocal:= \setcond{x = (x', x'') \in \R^{2m}}{|x'| > |x''|} \ \textrm{ and } \
\ical:= \setcond{x = (x', x'') \in \R^{2m}}{|x'| < |x''|}.
$$

Both $\ocal$ and $\ical$ belong to a family of sets in $\R^{2m}$ which are called of \emph{double revolution}. These are sets that are invariant under orthogonal transformations in the first $m$ variables, as well as under orthogonal transformations in the last $m$ variables. That is, $\Omega\subset \R^{2m}$ is a set of double revolution if $R\Omega = \Omega$ for every given transformation $R\in O(m)^2 = O(m) \times O(m)$, where  $O(m)$ is the orthogonal group of $\R^m$.


We say that a function $w:\R^{2m}  \to \R$ is \emph{doubly radial} if it depends only on the modulus of the first $m$ variables and on the modulus of the last $m$ ones, i.e., $w(x) = w(|x'|,|x''|)$. Equivalently, $w(Rx) = w(x)$ for every $R \in O(m)^2$.

We recall now the definition of $(\cdot)^\star$, an isometry that played a significant role in part~I \cite{FelipeSanz-Perela:IntegroDifferentialI}. It is defined by
\begin{equation}
\label{Eq:DefStar}
\begin{matrix}
(\cdot)^\star \colon & \R^{2m}= \R^{m}\times \R^{m}  &\to&  \R^{2m}= \R^{m}\times \R^{m}  \\
& x = (x',x'') &\mapsto & x^\star = (x'',x')\,.
\end{matrix}
\end{equation}
Note that this isometry is actually an involution that maps $\ocal$ into $\ical$ (and vice versa) and leaves the cone $\ccal$ invariant ---although not all points in $\ccal$ are fixed points of $(\cdot)^\star$. Taking into account this transformation, we say that a doubly radial function $w$ is \emph{odd with respect to the Simons cone} if $w(x) = -w(x^\star)$. Similarly, we say that a doubly radial function $w$ is \emph{even with respect to the Simons cone} if $w(x) = w(x^\star)$.



With these definitions at hand we can precisely define saddle-shaped solutions.
\begin{definition}
	\label{Def:SaddleShapedSol}
	We say that a bounded solution $u$ to \eqref{Eq:NonlocalAllenCahn} is a \emph{saddle-shaped solution} (or simply \emph{saddle solution}) if
	\begin{enumerate}
		\item $u$ is doubly radial.
		\item $u$ is odd with respect to the Simons cone.
		\item $u > 0$ in $\ocal = \{|x'| > |x''|\} $.
	\end{enumerate}
\end{definition}
Note that these solutions are even with respect to the coordinate axes and that their zero level set is the Simons cone $\mathscr{C} = \{|x'|=|x''|\}$. 



Let us collect now the main results of the previous paper \cite{FelipeSanz-Perela:IntegroDifferentialI} that will be used in the present one. Recall that if $K$ is a radially symmetric kernel we can rewrite the operator $L_K$ acting on a doubly radial function $w$ as
$$
L_K w(x) = \int_{\R^{2m}} \{w(x) - w(y)\} \overline{K}(x,y) \d y\,,
$$
where $\overline{K}$ is doubly radial in both variables and is defined by
\begin{equation}
\label{Eq:KbarDef}
\overline{K}(x,y) := \average_{O(m)^2} K(|Rx - y|)\d R\,.
\end{equation}
Here, $\d R$ denotes integration with respect to the Haar measure on $O(m)^2$ (see Section~2 of \cite{FelipeSanz-Perela:IntegroDifferentialI} for the details).

Moreover, if we consider doubly radial functions that are odd with respect to the Simons cone, we can use the involution $(\cdot)^\star$ to find that
\begin{equation}
\label{Eq:OperatorOddF}
L_K w (x) = \int_{\ocal} \{w(x) - w(y) \} \{\overline{K}(x, y) - \overline{K}(x, y^\star)  \} \d y +  2 w(x) \int_{\ocal} \overline{K}(x, y^\star) \d y \,.
\end{equation}
Furthermore,
\begin{equation}
\label{Eq:ZeroOrderTerm}
\frac{1}{C} \dist(x,\ccal)^{-2\s} \leq \int_{\ocal} \overline{K}(x, y^\star) \d y \leq C \dist(x,\ccal)^{-2\s},
\end{equation}
with $C>0$ depending only on $m, \s, \lambda$, and $\Lambda$ (see the details in part I \cite{FelipeSanz-Perela:IntegroDifferentialI}).


Note that the expression \eqref{Eq:OperatorOddF} has an integro-differential part plus a term of order zero with a positive coefficient. Thus, the most natural assumption to make in order to have an elliptic operator (when acting on doubly radial odd functions) is that the kernel of the integro-differential term is positive. That is, $\overline{K}(x, y) - \overline{K}(x, y^\star)>0$. One of the main results in part I \cite{FelipeSanz-Perela:IntegroDifferentialI}, stated next, established necessary and sufficient conditions on the original kernel $K$ for $L_K$ to have a positive kernel when acting on doubly radial odd functions. 

\begin{theorem}[\cite{FelipeSanz-Perela:IntegroDifferentialI}]
	\label{Th:SufficientNecessaryConditions}
	Let $K:(0,+\infty) \to (0,+\infty)$ and consider the radially symmetric kernel $K(|x-y|)$ in $\R^{2m}$. Define $\overline{K} : \R^{2m}\times \R^{2m} \to \R$ by \eqref{Eq:KbarDef}.
	
	If 
	\begin{equation}
	\label{Eq:SqrtConvex}	
	K(\sqrt{\tau}) \text{ is a strictly convex function of }\tau\,,
	\end{equation}
	then $L_K$ has a positive kernel in $\ocal$ when acting on doubly radial functions which are odd with respect to the Simons cone $\ccal$. More precisely, it holds
	\begin{equation}
	\label{Eq:KernelInequality}
	\overline{K}(x,y) > \overline{K}(x, y^\star) \quad \text{ for every }x,y \in \ocal\,.
	\end{equation}
	
	In addition, if $K\in C^2((0,+\infty))$, then \eqref{Eq:SqrtConvex} is not only a sufficient condition for \eqref{Eq:KernelInequality} to hold, but also a necessary one.
\end{theorem}

%%%%%%%%%%%%%%%%%%%%%%%%%%%%%%%%%%%%%%%%%%%%%%%%%%%%%%%%%%%%%%%%%%%%%%%%%%%%%%%%%%%%%%%%%%%%%%%%%%%%%%%%%%%%%%%%%%%%%%%%%%%%%%%%%%%%%%%%%%%%%%%%%%%%%%%%%%%%%%%%%%%%%%%%%%%%%%%%%%%%%%%%%%%%%%%%%%%%%%%%%%%%%%%%%%%%%%%%%%%%%%%%%%%%%%%%%%%%%%%%%%%%%%%%%%%%%%%%%%%%
\subsection{Main results}
\label{Subsec:Main results}

Through all the paper we will assume that $f$, the nonlinearity in \eqref{Eq:NonlocalAllenCahn}, is a $C^1$ function satisfying
\begin{equation}
\label{Eq:Hypothesesf}
f \textrm{ is odd, } \quad f(\pm 1)=0, \quad \text{ and } \quad f \textrm{ is strictly concave in }  (0,1).
\end{equation}
It is easy to see that these properties yield $f>0$ in $(0,1)$, $f'(0)>0$ and $f'(\pm 1) < 0$. 

The first main result of this paper concerns the existence of saddle-shaped solution.


\begin{theorem}[Existence of saddle-shaped solution]
	\label{Th:Existence}
	Let $f$ satisfy \eqref{Eq:Hypothesesf}. Let $K$ be a radially symmetric kernel satisfying the positivity condition \eqref{Eq:KernelInequality} and such that $L_K\in \lcal_0(2m, \s, \lambda, \Lambda)$. 
	
	Then, for every even dimension $2m \geq 2$, there exists a saddle-shaped solution $u$ to \eqref{Eq:NonlocalAllenCahn}. In addition, $u$ satisfies $|u|<1$ in $\R^{2m}$.
\end{theorem}



This theorem was already proved in part I \cite{FelipeSanz-Perela:IntegroDifferentialI} using variational techniques. Here, we show that existence can also be proved using, instead, the monotone iteration method. Let us remark that in both methods it is crucial to have the positivity condition \eqref{Eq:KernelInequality}.


The second main result of this paper is Theorem~\ref{Th:AsymptoticBehaviorSaddleSolution} below, on the asymptotic behavior of a saddle-shaped solution at infinity. To state it, let us introduce an important type of solutions in the study of the integro-differential Allen-Cahn equation: the layer solutions.


We say that a solution $v$ to $L_K v = f(v)$ in $\R^n$ is a \emph{layer solution} if $v$ is increasing in one direction, say $e\in \Sph^{n-1}$ and $v(x) \to \pm 1$ as $x\cdot e \to \pm \infty$ (not necessarily uniform). By a result of Cozzi and Passalacqua (Theorem~1 in \cite{CozziPassalacqua}), under the assumptions \eqref{Eq:Hypothesesf} on $f$, for every kernel $K_1$ such that $L_{K_1}\in \lcal_0(1,\s,\Lambda, \lambda)$ there exist a layer solution  to $L_{K_1} w = f(w)$ in $\R$ which is unique up to translations and is odd with respect to some point (in the case of the fractional Laplacian this result was proved in \cite{CabreSolaMorales,CabreSireII} by using the extension problem).

In $\R^n$, a special case of layer solutions are the one-dimensional ones. Actually, in relation with the available results concerning a conjecture by De Giorgi, in low dimensions all layer solutions are one-dimensional (see Subsection~\ref{Subsec:DeGiorgi}). One-dimensional layer solutions in $\R^n$ are in correspondence with the ones in $\R$ as explained next ---see also \cite{CozziPassalacqua}. Let $v$ be a layer solution to $L_K v = f(v)$ in $\R^n$ depending only on one direction, say $v(x) = w(x_n)$, and assume that $L_{K}\in \lcal_0(n,\s,\Lambda, \lambda)$. Then $w$ is a layer solution to $L_{K_1} w = f(w)$ in $\R$ with $K_1$ given by
$$
K_1(t) := \int_{\R^{n-1}} K\left(\theta,t\right) \d \theta = |t|^{n-1} \int_{\R^{n-1}} K\left(t\sigma,t\right) \d \sigma.
$$
Moreover $L_{K_1}\in \lcal_0(1,\s,\Lambda, \lambda)$. For more details see Proposition~\ref{Prop:KernelsDimension} in Section~\ref{Sec:Asymptotic} and \cite{CozziPassalacqua}. 

A particular layer solution, denoted by $u_0$, plays an important role in this paper. It is defined to be the unique solution of the following problem.
\begin{equation}
\label{Eq:LayerSolution}
\beqc{\PDEsystem}
L_{K_1}  u_0 &=& f(u_0) & \textrm{ in }\R\,,\\
\dot{u}_0 &>& 0 & \textrm{ in } \R\,,\\
u_0(x) & = &-u_0(-x)  & \textrm{ in }\R\,,\\
\ds \lim_{x \to \pm \infty} u_0(x) &=& \pm 1. & 
\eeqc
\end{equation}
Note that, by the previous comments, $v(x) = u_0(x_n)$ is a one-dimensional layer solution to $L_K v = f(v)$ in $\R^n$. Moreover, the same holds for $u_0(x\cdot e)$ for every $e\in \Sph^{n-1}$ whenever the kernel $K$ is radially symmetric.

The importance of the layer solution $u_0$ in relation with saddle-shaped solutions is that the associated function
\begin{equation}
\label{Eq:DefOfU}
U(x):= u_0 \left( \dfrac{|x'| - |x''|}{\sqrt{2}} \right)\,
\end{equation}
describes the asymptotic behavior of saddle solutions at infinity. Note that $(|x'| - |x''| )/\sqrt{2}$ is the signed distance to the Simons cone (see Lemma~4.2 in \cite{CabreTerraII}). Therefore, we can understand the function $U$ as the layer solution $u_0$ centered at each point of the Simons cone and oriented in the normal direction to the cone.

The precise statement on the asymptotic behavior of saddle-shaped solutions at infinity is the following.

\begin{theorem}
	\label{Th:AsymptoticBehaviorSaddleSolution}
	Let $f\in C^2(\R)$ satisfy \eqref{Eq:Hypothesesf}. Let $K$ be a radially symmetric kernel satisfying the positivity condition \eqref{Eq:KernelInequality} and such that $L_K\in \lcal_0(2m, \s, \lambda, \Lambda)$. Let $u$ be a saddle-shaped solution to \eqref{Eq:NonlocalAllenCahn} and let $U$ be the function defined by \eqref{Eq:DefOfU}.
	
	Then,
	$$
	\norm{u-U}_{L^\infty(\R^n\setminus B_R)}
	+\norm{\nabla u-\nabla U}_{L^\infty(\R^n\setminus B_R)}
	+\norm{D^2u-D^2U}_{L^\infty(\R^n\setminus B_R)} \to 0
	$$
	as $ R\to +\infty$.
\end{theorem}

To establish the asymptotic behavior of saddle-shaped solutions we use a compactness argument as in \cite{CabreTerraII, Cinti-Saddle, Cinti-Saddle2}, together with two important results established in Section~\ref{Sec:SymmetryResults}. The first one, Theorem~\ref{Th:LiouvilleSemilinearWholeSpace}, is a Liouville type result for nonnegative solutions to a semilinear equation in the whole space. 

\begin{theorem}
	\label{Th:LiouvilleSemilinearWholeSpace}
	Let $L_K \in \lcal_0(n,\s)$ and let $v$ be a bounded solution to
	\begin{equation}
	\label{Eq:PositiveWholeSpace}
	\beqc{\PDEsystem}
	L_K v &=& f(v) & \textrm{ in }\R^n\,,\\
	v &\geq& 0 & \textrm{ in } \R^n\,,
	\eeqc
	\end{equation}
	with a nonlinearity $f\in C^1$ satisfying
	\begin{itemize}
		\item $f(0) = f(1) = 0$,
		\item $f'(0)>0$,
		\item $f>0$ in $(0,1)$, and $f<0$ in $(1,+\infty)$.
	\end{itemize}
	Then, $v\equiv 0$ or $v \equiv 1$.
\end{theorem}

Similar classification results have been proved for the fractional Laplacian in \cite{ChenLiZhang,LiZhang} (either using the extension problem or not) with the method of moving spheres, which uses crucially the scale invariance of the operator $\fraclaplacian$. To the best of our knowledge, there is no similar result available in the literature for general kernels in the ellipticity class $\lcal_0$ (which are not necessarily scale invariant). Thus, we present here a proof based on the techniques used in \cite{BerestyckiHamelNadi} for a local equation with the classical Laplacian. It relies on the maximum principle, the translation invariance of the operator, a Harnack inequality and a stability argument. All these features are available for the operators in $\lcal_0$ (see Section~\ref{Sec:SymmetryResults}). Thus, the same arguments as in the local case can be carried out.

The second ingredient to prove the asymptotic behavior of saddle-shaped solutions is a symmetry result for equations in a half-space, stated next. Here and in the rest of the paper we use the notation $\R^n_+= \{(x_H,x_n)\in \R^{n-1}\times \R \ : \ x_n > 0\}$.  

\begin{theorem}
	\label{Th:SymmHalfSpace}
	Let $L_K\in \lcal_0(n,\s)$ and let $v$ be a bounded solution to one of these two problems:
	
	\begin{equation}
	\reqnomode
	\tag{P1}
	\label{Eq:P1}
	\beqc{\PDEsystem}
	L_K v &=& f(v)   &\textrm{ in } \,\R^n_+,\\
	v &>& 0   &\textrm{ in } \,\R^n_+,\\
	v(x_H,x_n) &=& -v(x_H,-x_n)   &\textrm{ in } \,\R^n.
	\eeqc
	\end{equation}
	
	\begin{equation}
	\reqnomode
	\tag{P2}
	\label{Eq:P2}
	\beqc{\PDEsystem}
	L_K v &=& f(v)   &\textrm{ in } \,\R^n_+,\\
	v &>& 0   &\textrm{ in } \,\R^n_+,\\
	v &=& 0   &\textrm{ in } \,\R^n \setminus \R^n_+.
	\eeqc
	\end{equation}
	
	\reqnomode
	
	Assume that, in $\R^n_+$, the kernel $K$ of the operator $L_K$ is decreasing in the direction of $x_n$, i.e., it satisfies
	$$
	K(x_H-y_H,x_n-y_n) \geq K(x_H-y_H,x_n+y_n) \,\,\,\,\text{for all } \,\, x,y\in \R^n_+.
	$$ 
	Suppose that $f\in C^1$ and
	\begin{itemize}
		\item $f(0) = f(1) = 0$,
		\item $f'(0)>0$, and $f'(t)\leq 0$ for all $t\in[1-\delta,1]$ for some $\delta>0$,
		\item $f>0$ in $(0,1)$, and
		\item $f$ is odd in the case of \eqref{Eq:P1}.
	\end{itemize}
	Then, $v$ depends only on $x_n$ and it is increasing in that direction.
\end{theorem}

The result for \eqref{Eq:P2} has been proved for the fractional Laplacian under some assumptions on $f$ (weaker than the ones in Theorem~\ref{Th:SymmHalfSpace}) in \cite{QuaasXia, BarriosEtAl-Monotonicity, BarriosEtAl-Symmetry, FallWethMonotonicity}. Instead, to the best of our knowledge \eqref{Eq:P1} has not been treated even for the fractional Laplacian. In our case, the fact that $f$ is of Allen-Cahn type allows us to use rather simple arguments that work for both problems \eqref{Eq:P1} and \eqref{Eq:P2} ---moving planes and sliding methods. Moreover, the fact that we replace the kernel of the operator by a general $K$ satisfying \eqref{Eq:Ellipticity} do not affect significantly the proof. Although \eqref{Eq:P2} will not be used in this paper, since the proof for this problem is analogous to the one for \eqref{Eq:P1}, we include it here for future reference, 

The last main result of this paper is the uniqueness of the saddle-shaped solution, stated next.

\begin{theorem}[Uniqueness of the saddle-shaped solution]
	\label{Th:Uniqueness}
	Let $f$ satisfy \eqref{Eq:Hypothesesf} and let $K$ be a radially symmetric kernel satisfying the positivity condition \eqref{Eq:KernelInequality} and such that $L_K\in \lcal_0(2m, \s, \lambda, \Lambda)$. 
	
	Then, for every dimension $2m \geq 2$, there exists a unique saddle-shaped solution to \eqref{Eq:NonlocalAllenCahn}.
\end{theorem}

To prove this result we need two ingredients. The first one is the asymptotic behavior of saddle solutions given in Theorem~\ref{Th:AsymptoticBehaviorSaddleSolution}. The second one is a maximum principle in $\ocal$ for the linearized operator $L_K - f'(u)$, which is given in Proposition~\ref{Prop:MaximumPrincipleLinearized}. To establish it, we will need to use a maximum principle in ``narrow'' sets, also proved in Section~\ref{Sec:MaximumPrinciple}. In the arguments, it is crucial again the positivity condition \eqref{Eq:KernelInequality}.

%%%%%%%%%%%%%%%%%%%%%%%%%%%%%%%%%%%%%%%%%%%%%%%%%%%%%%%%%%%%%%%%%%%%%%%%%%%%%%%%%%%%%%%%%%%%%%%%%%%%%%%%%%%%%%%%%%%%%%%%%%%%%%%%%%%%%%%%%%%%%%%%%%%%%%%%%%%%%%%%%%%%%%%%%%%%%%%%%%%%%%%%%%%%%%%%%%%%%%%%%%%%%%%%%%%%%%%%%%%%%%%%%%%%%%%%%%%%%%%%%%%%%%%%%%%%%%%%%%%%
\subsection{Saddle-shaped solutions in the context of a conjecture by De Giorgi}
\label{Subsec:DeGiorgi}

To conclude this introduction, let us make some comments on the importance of problem \eqref{Eq:NonlocalAllenCahn} and its relation with the theory of (classical and nonlocal) minimal surfaces and  a famous conjecture raised by De Giorgi.

A main open problem (even in the local case) is to determine whether the saddle-shaped solution is a minimizer of the energy functional associated to the equation, depending on the dimension $2m$. This question is deeply related to the regularity theory of local and nonlocal minimal surfaces, as explained next.

In the seventies, Modica and Mortola (see \cite{Modica,ModicaMortola}) proved that, considering an appropriately rescaled version of the (local) Allen-Cahn equation, the corresponding energy functionals $\Gamma$-converge to the perimeter functional. Thus, the blow-down sequence of minimizers of the Allen-Cahn energy converge to the characteristic function of a set of minimal perimeter. This same fact holds for the equation with the fractional Laplacian, though we have two different scenarios depending on the parameter $\s \in (0,1)$. If $\s \geq 1/2$, the rescaled energy functionals associated to the equation $\Gamma$-converge to the classical perimeter (see \cite{GiovanniBouchitteSeppecher,Gonzalez}), while in the case $\s \in (0,1/2)$ they $\Gamma$-converge to the fractional perimeter (see \cite{SavinValdinoci-GammaConvergence}). As a consequence, if the saddle-shaped solution was proved to be a minimizer in a certain dimension for some $\s \in (0,1/2)$, it would follow that the Simons cone $\ccal$ would be a minimizing nonlocal $(2\s)$-minimal surface in such dimensions. This last statement on the saddle-shaped solution is an open problem in any dimension (although it is known that the Simons cone is not a minimizer in dimension $2m=2$). The only available result related to this question is the recent one in our previous paper \cite{Felipe-Sanz-Perela:SaddleFractional}, which concerns stability (a weaker property than minimality). We proved that the saddle-shaped solution to the fractional Allen-Cahn equation is stable in dimensions $2m\geq 14$. As a consequence of this and a result in \cite{CabreCintiSerra-Stable}, the Simons cone is a stable nonlocal $(2\s)$-minimal surface in dimensions $ 2m\geq 14$ (see the details in \cite{Felipe-Sanz-Perela:SaddleFractional}).


Moreover, as explained below, saddle-shaped solutions are natural objects to build a counterexample to a famous conjecture raised by De Giorgi, that reads as follows. Let $u$ be a bounded solution to $-\Delta  u = u - u^3$ in $\R^n$ which is monotone in one direction, say $\partial_{x_n} u > 0$. Then, if $n\leq 8$, $u$ is one dimensional, i.e., $u$ depends only on one Euclidean variable. This conjecture was proved to be true in dimensions $n=2$ and  $n=3$ (see \cite{GhoussoubGui,AmbrosioCabre}), and in dimensions $4\leq n \leq 8$ with the extra assumption
\begin{equation}
\label{Eq:SavinCondition}
\lim_{x_n \to \pm \infty} u(x_H,x_n) = \pm 1 \quad \text{ for all } x_H\in \R^{n-1}\,,
\end{equation}
(see \cite{Savin-DeGiorgi}). A counterexample to the conjecture in dimensions $n\geq 9$ was given in \cite{delPinoKowalczykWei} by using the gluing method. 

An alternative approach to the one of \cite{delPinoKowalczykWei} to construct a counterexample to the conjecture was given by Jerison and Monneau in \cite{JerisonMonneau}. They showed that a counterexample in $\R^{n+1}$ can be constructed with a rather natural procedure if there exists a global minimizer of $-\Delta u = f(u)$ in $\R^n$ which is bounded and even with respect to each coordinate but is not one-dimensional. The saddle-shaped solution is of special interest in search of this counterexample, since it is even with respect to all the coordinate axis and it is canonically associated to the Simons cone, which in turn is the simplest nonplanar minimizing minimal surface. Therefore, by proving that the saddle solution to the classical Allen-Cahn equation is a minimizer in some dimension $2m$, one would obtain automatically a counterexample to the conjecture in $\R^{2m+1}$.

The corresponding conjecture in the fractional setting, where one replaces the operator $-\Delta$ by $\fraclaplacian$, has been widely studied in the last years. In this framework, the conjecture has been proven to be true for all $\s\in(0,1)$ in dimensions $n=2$ (see \cite{CabreSolaMorales,CabreSireI,SireValdinoci}) and $n=3$ (see \cite{CabreCinti-EnergyHalfL, CabreCinti-SharpEnergy,DipierroFarinaValdinoci}). The conjecture is also true in dimension $n=4$ in the case of $\s = 1/2$ (see \cite{FigalliSerra}) and if $\s\in(0,1/2)$ is close to $1/2$ (see \cite{CabreCintiSerra-Stable}). Assuming the additional hypothesis \eqref{Eq:SavinCondition}, the conjecture is true in dimensions $4\leq n \leq 8$ for $1/2 \leq \s < 1$ (see \cite{Savin-Fractional,Savin-Fractional2}), and also for $\s\in(0,1/2)$ if $\s$ is close to $1/2$ (see \cite{DipierroSerraValdinoci}). A counterexample to the De Giorgi conjecture for the fractional Allen-Cahn equation in dimensions $n \geq 9$ for $\s \in (1/2,1)$ has been very recently announced in \cite{ChanLiuWei}.

Concerning the conjecture with more general operators like $L_K$, fewer results are known. In dimension $n=2$ the conjecture is proved in \cite{HamelRosOtonSireValdinoci, Bucur, FazlySire}, under different assumptions on the kernel $K$ and even for more general nonlinear operators. Note also that the results of \cite{DipierroSerraValdinoci} also hold for a particular class of kernels in $\lcal_0$.

%%%%%%%%%%%%%%%%%%%%%%%%%%%%%%%%%%%%%%%%%%%%%%%%%%%%%%%%%%%%%%%%%%%%%%%%%%%%%%%%%%%%%%%%%%%%%%%%%%%%%%%%%%%%%%%%%%%%%%%%%%%%%%%%%%%%%%%%%%%%%%%%%%%%%%%%%%%%%%%%%%%%%%%%%%%%%%%%%%%%%%%%%%%%%%%%%%%%%%%%%%%%%%%%%%%%%%%%%%%%%%%%%%%%%%%%%%%%%%%%%%%%%%%%%%%%%%%%%%%%
\subsection{Plan of the article}
\label{Subsec:Plan}

The paper is organized as follows. In Section~\ref{Sec:Preliminaries} we present some preliminary results that will be used in the rest of the article. Section~\ref{Sec:Existence} contains the proof of Theorem~\ref{Th:Existence} on the existence of a saddle-shaped solution via the monotone iteration method. In Section~\ref{Sec:SymmetryResults} we establish the Liouville type and symmetry results, Theorems~\ref{Th:LiouvilleSemilinearWholeSpace} and \ref{Th:SymmHalfSpace}. Section~\ref{Sec:Asymptotic} is devoted to the layer solution $u_0$ of problem \eqref{Eq:NonlocalAllenCahn} and the proof of the asymptotic behavior of saddle-shaped solutions, Theorem~\ref{Th:AsymptoticBehaviorSaddleSolution}. Finally, Section~\ref{Sec:MaximumPrinciple} concerns the proof of a maximum principle in $\ocal$ for the linearized operator $L_K - f'(u)$ (Proposition~\ref{Prop:MaximumPrincipleLinearized}), as well as the proof of Theorem~\ref{Th:Uniqueness}, establishing the uniqueness of the saddle-shaped solution.



%%%%%%%%%%%%%%%%%%%%%%%%
\section{Preliminaries}
%%%%%%%%%%%%%%%%%%%%%%%%
\label{Sec:Preliminaries}

In this section we collect some preliminary results that will be used in the rest of this paper. First, we state the regularity results needed in the forthcoming sections. Then, we present the functional setting which we are going to work with and finally we recall the two basic maximum principles for doubly radial odd functions proved in \cite{FelipeSanz-Perela:IntegroDifferentialI}.


%%%%%%%%%%%%%%%%%%%%%%%%%%%%%%%%%%%%%%%%%%%%%%%%%%%%%%%%
\subsection{Regularity theory for nonlocal operators in the class $\lcal_0$}
\label{Subsec:Regularity}
%%%%%%%%%%%%%%%%%%%%%%%%%%%%%%%%%%%%%%%%%%%%%%%%%%%%%%%%


In this subsection we present the regularity results that will be used in the paper. For further details, see \cite{RosOton-Survey,SerraC2s+alphaRegularity} and the references therein. 


We start with a result on the interior regularity for linear equations.

\begin{proposition}[\cite{RosOton-Survey,SerraC2s+alphaRegularity}]
	\label{Prop:InteriorRegularity}
	Let $L_K \in\lcal_0(n,\s,\lambda, \Lambda)$ and let $w\in L^\infty (\R^n)$ be a weak solution to $L_K w = h$ in $B_1$. Then,
	\begin{equation}
	\label{Eq:C2sEstimate}
	\norm{w}_{C^{2\s} (B_{1/2})} \leq C\bpar{\norm{h}_{L^\infty (B_1)} + \norm{w}_{L^\infty  (\R^n)} }.
	\end{equation}
	Moreover, let $\alpha > 0$ and assume additionally that $w \in C^\alpha (\R^n)$. Then, if $\alpha +
	2\s$ is not an integer,
	\begin{equation}
	\label{Eq:Calpha->Calpha+2sEstimate}
	\norm{w}_{C^{\alpha + 2\s} (B_{1/2})} \leq C\bpar{\norm{h}_{C^{\alpha} (B_1)} + \norm{w}_{C^\alpha (\R^n)} },
	\end{equation}
	where $C$ is a constant that depends only on $n$, $\s$, $\lambda$ and $\Lambda$.
\end{proposition}


Throughout the paper we consider saddle solutions $u$ to \eqref{Eq:NonlocalAllenCahn} that satisfy $|u|\leq 1$ in $\R^n$. Hence, by applying \eqref{Eq:C2sEstimate} we find that for any $x_0\in \R^n$,
\begin{align*}
\norm{u}_{C^{2\s} (B_{1/2} (x_0))} &\leq C\bpar{\norm{f(u)}_{L^\infty (B_1(x_0))} + \norm{u}_{L^\infty  (\R^n)} } \\
&\leq C\bpar{1 + \norm{f}_{L^\infty ([-1,1])} }.
\end{align*}
Note that the estimate is independent of the point $x_0$, and thus since the equation is satisfied in the whole $\R^n$,
$$
\norm{u}_{C^{2\s}(\R^n)} \leq C\bpar{1 + \norm{f}_{L^\infty ([-1,1])} }.
$$
Then, we use estimate \eqref{Eq:Calpha->Calpha+2sEstimate} repeatedly and the same kind of arguments wield that, if $f\in C^{k}([-1,1])$, then $u\in C^{\alpha}(\R^n)$ for some $\alpha > k+ 2 \s$. Moreover, the following estimate holds:
$$
\norm{u}_{C^{\alpha}(\R^n)} \leq C\,,
$$
for some constant $C$ depending only on $n$, $\s$, $\lambda$, $\Lambda$, $k$ and $\norm{f}_{C^k([-1,1])}$.


%%%%%%%%%%%%%%%%%%%%%%%%%%%%%%%%%%%%%%%%%%%%%%%%%%%%%%%%
\subsection{Functional setting}
\label{Subsec:Functional setting}
%%%%%%%%%%%%%%%%%%%%%%%%%%%%%%%%%%%%%%%%%%%%%%%%%%%%%%%%



In this section we define the functional spaces that we are going to consider in some parts of this paper. These were also the spaces used in the previous article \cite{FelipeSanz-Perela:IntegroDifferentialI}, and we refer the reader to that work for more details.

Given a set $\Omega \subset \R^n$ and a translation invariant and positive kernel $K$ satisfying \eqref{Eq:Symmetry&IntegrabilityOfK}, we define the space
$$
\H^K(\Omega) := \setcond{w \in L^2(\Omega)}{[w]^2_{\H^K(\Omega)} < + \infty},
$$
where
$$
[w]^2_{\H^K(\Omega)} := \dfrac{1}{2}\int\int_{\R^{2n} \setminus (\R^n\setminus\Omega)^2} |w(x) - w(y)|^2 K(x-y) \d x \d y\,.
$$
We also define
\begin{align*}
\H^K_0(\Omega) &:= \setcond{w \in \H^K(\Omega)}{ w = 0 \quad \textrm{a.e. in } \R^n \setminus \Omega} \\
&\ = \setcond{w \in \H^K(\R^n)}{ w = 0 \quad \textrm{a.e. in } \R^n \setminus \Omega}.
\end{align*}

Assume that $\Omega \subset \R^{2m}$ is a domain of double revolution. Then, we define
$$
\widetilde{\H}^K(\Omega) := \setcond{w \in \H^K(\Omega)}{w \textrm{ is doubly radial a.e.}}.
$$
and
$$
\widetilde{\H}^K_0(\Omega) := \setcond{w \in \H^K_0(\Omega)}{w \textrm{ is doubly radial a.e.}}.
$$
We will add the subscript `odd' and `even' to these spaces to consider only functions that are odd (respectively even) with respect to the Simons cone.

Recall that when $K$ satisfies \eqref{Eq:Ellipticity}, then $\H^K_0 (\Omega) = \H^\s_0 (\Omega)$, which is the space associated to the kernel of the fractional Laplacian, $K(y) = c_{n,\s}|y|^{-n-2\s}$. Furthermore, $\H^\s(\Omega) \subset H^\s(\Omega)$, the usual fractional Sobolev space (see \cite{HitchhikerGuide,CozziPassalacqua}). 


%%%%%%%%%%%%%%%%%%%%%%%%%%%%%%%%%%%%%%%%%%%%%%%%%%%%%%%%
\subsection{Maximum principles for doubly radial odd functions}
\label{Subsec:MaxPrinciples}
%%%%%%%%%%%%%%%%%%%%%%%%%%%%%%%%%%%%%%%%%%%%%%%%%%%%%%%%

In this last subsection, we state two basic maximum principles for doubly radial odd functions. Note that in both results we only need assumptions on the functions at one side of the Simons cone thanks to their symmetry. Both results where proved in \cite{FelipeSanz-Perela:IntegroDifferentialI} using the key inequality \eqref{Eq:KernelInequality} for the kernel $\overline{K}$.

The first result is a weak maximum principle for odd functions with respect to $\ccal$.

\begin{proposition}[\cite{FelipeSanz-Perela:IntegroDifferentialI}]
	\label{Prop:WeakMaximumPrincipleForOddFunctions} Let $\Omega \subset \ocal$ an open set and let $L_K  \in \lcal_\star (2m,  \s)$.  Let $w\in C^{\alpha}(\Omega)\cap L^\infty(\R^{2m})$, with $\alpha > 2\s$, be a doubly radial function which is odd with respect to the Simons cone. Assume that
	$$
	\beqc{\PDEsystem}
	L_K w & \geq & 0 & \text{ in } \Omega\,,\\
	w & \geq & 0 & \text{ in } \ocal \setminus \Omega\,,
	\eeqc
	$$
	and that either
	$$
	\Omega \text{ is bounded} \quad \text{ or } \liminf_{x \in \ocal,\,|x|\to +\infty} w(x) \geq 0\,.
	$$
	Then, $w \geq 0$ in $\Omega$.
\end{proposition}


The second result is the strong maximum principle for odd functions with respect to $\ccal$.

\begin{proposition}[\cite{FelipeSanz-Perela:IntegroDifferentialI}]
	\label{Prop:StrongMaximumPrincipleForOddFunctions} Let $\Omega \subset \ocal$ an open set and let $L_K  \in \lcal_\star (2m,  \s)$.  Let $w\in C^{\alpha}(\Omega)\cap L^\infty(\R^{2m})$, with $\alpha > 2\s$, be a doubly radial function which is odd with respect to the Simons cone. Assume that $L_K w \geq 0$ in $\Omega$, and that $w\geq 0$ in $\ocal$. Then, either $w\equiv 0$ or $w > 0$ in $\Omega$.
\end{proposition}

\begin{remark}
	\label{Remark:MaxPrincipleSingularity}
	The regularity assumptions on $w$ in the previous results can be weakened allowing $L_K w$ to take the value $+\infty$ at the points of $\Omega$ where $w$ is not regular enough for $L_K w$ to be finite. This will be used in the proof of Theorem~\ref{Th:Existence} in order to apply this maximum principle with a function that is no more regular than $C^\s$.
\end{remark}

%%%%%%%%%%%%%%%%%%%%%%%%%%%%%%%%%%%%%%%%%%%%%%%%%%%%%%%%%%%%%%%%%%%%%%
%%%%%%%%%%%%%%%%%%%%%%%%%%%%%%%%%%%%%%%%%%%%%%%%%%%%%%%%%%%%%%%%%%%%%%


%%%%%%%%%%%%%%%%%%%%%%%%%%%%%%%%%%%%%%%%%%%%%%%%%%
\section{Existence of saddle-shaped solution: monotone iteration method}
%%%%%%%%%%%%%%%%%%%%%%%%%%%%%%%%%%%%%%%%%%%%%%%%%%
\label{Sec:Existence}


In this section we give a proof of Theorem~\ref{Th:Existence} based on the maximum principle and the existence of a positive subsolution. To do this, we need a version of the monotone iteration procedure for doubly radial functions which are odd with respect to the Simons cone $\ccal$. Along this section we will call odd sub/supersolutions to problem \eqref{Eq:SemilinearSolutionInBall} the functions that are doubly radial, odd with respect to the Simons cone and satisfy the corresponding problem in \eqref{Eq:SemilinearSubSuperSolutionInBall}. In view of Remark~\ref{Remark:MaxPrincipleSingularity}, we do not need the operator to be finite in the whole set when applied to a subsolution (respectively supersolution), it can be $-\infty$ (respectively $+\infty$) at some points.

\begin{proposition}
	\label{Prop:MonotoneIterationOdd}
	Let $\s\in (0,1)$ and let $K$ be a radially symmetric kernel such that $L_K\in \lcal_0(2m, \s)$ and satisfying the positivity condition \eqref{Eq:KernelInequality}. Assume that $\vsub \leq \vsup$ are two bounded functions which are doubly radial and odd with respect to the Simons cone. Furthermore, assume that $\vsub\in C^\s(\R^{2m})$ and that $\vsub$ and $\vsup$ satisfy respectively   
	\begin{equation}
	\label{Eq:SemilinearSubSuperSolutionInBall}
	\beqc{\PDEsystem}
	L_K\vsub & \leq & f(\vsub) & \textrm{ in } B_R \cap \ocal\,, \\
	\vsub & \leq & \varphi & \textrm{ in } \ocal \setminus B_R\,, 
	\eeqc
	\quad \textrm{ and } \quad 
	\beqc{\PDEsystem}
	L_K\vsup & \geq & f(\vsup) & \textrm{ in } B_R \cap \ocal\,, \\
	\vsup & \geq & \varphi & \textrm{ in } \ocal \setminus B_R\,, 
	\eeqc
	\end{equation}
	with $f$ a $C^1$ odd function and $\varphi$ a doubly radial odd function.
	
	Then, there exists $v\in C^{2\s+\varepsilon}(B_R)$ for some $\varepsilon>0$, a solution to
	\begin{equation}
	\label{Eq:SemilinearSolutionInBall}
	\beqc{\PDEsystem}
	L_K v & = & f(v) & \textrm{ in } B_R\,, \\
	v &=& \varphi &  \textrm{ in } \R^{2m} \setminus B_R\,, 
	\eeqc
	\end{equation}
	such that $v$ is doubly radial, odd with respect to the Simons cone and  $\vsub \leq v \leq \vsup$ in $\ocal$.
\end{proposition}


\begin{proof}
	The proof follows the classical monotone iteration method for elliptic equations (see for instance \cite{Evans}). We just give here a sketch of the proof. 
	First, let $M \geq 0$ be such that $-M \leq \vsub \leq \vsup \leq M$ and set
	$$
	b := \max \left \{{0, - \min_{[-M,M]}f'}\right \}\geq 0\,.
	$$
	Then one defines 
	$$
	\widetilde{L}_K w := L_Kw + b w 	\quad \text{ and } \quad 	g(\tau) := f(\tau) + b \tau\,.
	$$
	Therefore, our problem is equivalent to find a solution to
	$$
	\beqc{\PDEsystem}
	\widetilde{L}_Kv & = & g(v) & \textrm{ in } B_R\,, \\
	v &=& \varphi &  \textrm{ in } \R^{2m} \setminus B_R\,, 
	\eeqc
	$$
	such that $v$ is doubly radial, odd with respect to the Simons cone and  $\vsub \leq v \leq \vsup$ in $\ocal$. Here the main point is that $g$ is also odd but satisfies $g'(\tau) \geq 0$ for $\tau \in [-M,M]$. Moreover, since $b \geq 0$, $\widetilde{L}_K$ satisfies the maximum principle for odd functions in $\ocal$ (as in Proposition~\ref{Prop:MaximumPrincipleForOddFunctions}).
	
	We define $v_0 = \vsub$ and, for $k\geq 1$, let $v_k$ be the solution to the linear problem
	$$
	\beqc{\PDEsystem}
	\tilde{L}_K v_k & = & g(v_{k-1}) & \textrm{ in } B_R\,, \\
	v_k &=& \varphi &  \textrm{ in } \R^{2m} \setminus B_R\,. 
	\eeqc
	$$
	It is easy to see by induction and the regularity results from Proposition~\ref{Prop:InteriorRegularity} that $v_k\in L^\infty(B_R) \cap C^{2\s+2\varepsilon}(B_R)$ for some $\varepsilon>0$. Moreover, given $\Omega\subset B_R$ a compact set, then $||v_k||_{C^{2\s+2\varepsilon}(\Omega)}$ is uniformly bounded in $k$.
	
	Then, using the maximum principle it is not difficult to show by induction that 
	$$
	\vsub = v_0 \leq v_1 \leq \ldots \leq v_k \leq v_{k+1} \leq \ldots \vsup \quad \text{ in }\ocal\,,
	$$
	and that each function $v
	_k$ is doubly radial and odd with respect to $\ccal$. Finally, by Arzelà-Ascoli theorem and the compact embedding of H\"older spaces we see that, up to a subsequence, $v_k$ converges uniformly on compacts in $C^{2\s+\varepsilon}$ norm to the desired solution.
\end{proof}

In order to construct a positive subsolution, we also need a characterization and some properties of the first odd eigenfunction and eigenvalue for the operator $L_K$, which are presented next. This eigenfunction is obtained though a minimization of the corresponding Rayleigh quotient in the appropriate space. Before presenting our result, let us recall some functional spaces that we are going to use next. 

Given a set $\Omega \subset \R^{2m}$ and a translation invariant and positive kernel $K$ satisfying \eqref{Eq:Symmetry&IntegrabilityOfK}, we define the space
$$
\H^K_0(\Omega) := \setcond{w \in L^2(\Omega)}{w = 0 \quad \textrm{a.e. in } \R^{2m} \setminus \Omega \quad \textrm{ and } [w]^2_{\H^K(\R^{2m})} < + \infty},
$$
where
\begin{equation}
	\label{Eq:SeminormHK}
[w]^2_{\H^K(\R^{2m})} := \dfrac{1}{2}\int_{\R^{2m}} \int_{\R^{2m}} |w(x) - w(y)|^2 K(x-y) \d x \d y\,.
\end{equation}
Recall that when $K$ satisfies the ellipticity assumption \eqref{Eq:Ellipticity}, then $\H^K_0 (\Omega) = \H^\s_0 (\Omega)$, which is the space associated to the kernel of the fractional Laplacian, $K(y) = c_{n,\s}|y|^{-n-2\s}$. We also define
$$
\widetilde{\H}^K_{0, \, \mathrm{odd}}(\Omega) := \setcond{w \in \H^K_0(\Omega)}{w \textrm{ is doubly radial a.e. and odd with respect to } \ccal}.
$$
Recall that when $K$ is radially symmetric and $w$ is doubly radial, we can replace the kernel $K(x-y)$ in the definition \eqref{Eq:SeminormHK} by the kernel $\overline{K}(x,y)$. This is readily deduced after a change of variables and taking the mean among all $R\in O(m)^2$ (see the details in Secton~3 of \cite{FelipeSanz-Perela:IntegroDifferentialI}).




\begin{lemma}
	\label{Lemma:FirstOddEigenfunction}
	Let $\Omega\subset \R^{2m} $ be a bounded set of double revolution and let  $K$ be a radially symmetric kernel such that $L_K\in \lcal_0(2m, \s, \lambda, \Lambda)$ and satisfying the positivity condition \eqref{Eq:KernelInequality}. Let us define 
	\begin{equation}
	\label{Eq:DefLambda1}
	\lambda_{1, \, \mathrm{odd}}(\Omega, L_K) := \inf_{w \in \widetilde{\H}^K_{0, \, \mathrm{odd}}(\Omega)} \dfrac{\dfrac{1}{2}  \ds\int_{\R^{2m}} \int_{\R^{2m}} |w(x) - w(y)|^2 \overline{K}(x,y) \d x \d y}{ \ds \int_\Omega w(x)^2 \d x}\,.
	\end{equation}
	
	Then, such infimum is attained at a function $\phi_1\in \widetilde{\H}^K_{0, \, \mathrm{odd}}(\Omega)\cap L^\infty(\Omega)$ which solves
	$$
	\beqc{\PDEsystem}
	L_K \phi_1 &=& \lambda_{1, \, \mathrm{odd}}(\Omega, L_K) \phi_1 & \textrm{ in } \Omega\,,\\
	\phi_1 & = & 0 & \textrm{ in } \R^{2m}\setminus \Omega\,,
	\eeqc
	$$
	and satisfies that $\phi_1 > 0$ in $\Omega \cap \ocal$.
	We call such function the \emph{first odd eigenfunction of $L_K$ in $\Omega$} and $\lambda_{1, \, \mathrm{odd}}(\Omega, L_K) $ the \emph{first odd eigenvalue}. 
	
	Moreover, in the case $\Omega = B_R$, there exists a constant $C$ depending only on $n$, $\s$ and $\Lambda$ such that
	$$
	\lambda_{1, \, \mathrm{odd}}(B_R, L_K) \leq C R^{-2\s}\,. 
	$$ 
\end{lemma}


\begin{proof}
	The first two statements are deduced exactly as in Proposition~9 of \cite{ServadeiValdinoci}, using the same arguments as in  Lemma~3.4. of \cite{FelipeSanz-Perela:IntegroDifferentialI} to guarantee that $\phi_1$ is nonnegative in $\ocal$. The fact that $\phi_1 > 0$ in $\Omega \cap \ocal$ follows from the strong maximum principle (see Proposition~\ref{Prop:MaximumPrincipleForOddFunctions}).
	
	We show the third statement. Let $\widetilde{w} (x):= w(Rx)$ for every $w\in \widetilde{\H}^K_{0, \, \mathrm{odd}}(B_R)$. Then,
	\begin{align*}
	& \min_{w \in \widetilde{\H}^K_{0, \, \mathrm{odd}}(B_R)} \dfrac{\dfrac{1}{2}  \ds\int_{\R^{2m}} \int_{\R^{2m}} |w(x) - w(y)|^2 \overline{K}(x,y) \d x \d y}{ \ds \int_{B_R} w(x)^2 \d x} \quad \quad \quad \quad \quad \quad \quad \quad \quad \quad \quad \quad\\
	&   \quad \quad \quad \quad \quad \quad \leq \min_{\widetilde{w} \in \widetilde{\H}^K_{0, \, \mathrm{odd}}(B_1)} \dfrac{\dfrac{c_{n, \s}\Lambda}{2}  \ds\int_{\R^{2m}} \int_{\R^{2m}} |\widetilde{w}(x/R) - \widetilde{w}(y/R)|^2 |x - y|^{-n-2 \s}\d x \d y}{ \ds \int_{B_R} \widetilde{w}(x/R)^2 \d x}
	\\
	& \quad \quad \quad \quad \quad \quad = R^{-2 \s }\min_{\widetilde{w} \in \widetilde{\H}^s_{0, \, \mathrm{odd}}(B_1)} \dfrac{\dfrac{c_{n, \s}\Lambda}{2}  \ds\int_{\R^{2m}} \int_{\R^{2m}} |\widetilde{w}(x) - \widetilde{w}(y)|^2 |x - y|^{-n-2 \s}\d x \d y}{ \ds \int_{B_1} \widetilde{w}(x)^2 \d x}
	\\
	& \quad \quad \quad \quad \quad \quad = \lambda_{1, \, \mathrm{odd}}(B_1, \fraclaplacian) \Lambda R^{-2 \s } \,.
	\end{align*}
\end{proof}

\begin{remark}
	\label{Remark:CsRegularityFirstEigenfunction}
	Note that, by standard regularity results for $L_K$, we have that $\phi_1 \in C^\s(\overline{\Omega})\cap C^\infty(\Omega)$, and the regularity up to the boundary is optimal (see \cite{RosOton-Survey} and the references therein for the details). Due to this and the fact that $\phi_1 >0$ in $\Omega\cap \ocal$ while $\phi_1=0$ in $\R^{2m}\setminus \Omega$, it is easy to check by using \eqref{Eq:OperatorOddF} that $-\infty <L_K \phi_1 < 0$ in $\ocal\setminus \overline{\Omega}$ and $L_K \phi_1 = -\infty$ in $\partial \Omega \cap \ocal$.
\end{remark}

With these ingredients, we can proceed with the proof of Theorem~\ref{Th:Existence}.

\begin{proof}[Proof of Theorem~\ref{Th:Existence}]
	The strategy is to build a suitable solution $u_R$ of 
	\begin{equation}
	\label{Eq:ProofExistenceProblemBR}
	\beqc{\PDEsystem}
	L_K u_R &=& f(u_R) & \textrm{ in } B_R\,,\\
	u_R &=& 0 & \textrm{ in }\R^{2m} \setminus B_R\,,
	\eeqc
	\end{equation}
	and then let $R\to+ \infty$ to get a saddle-shaped solution.
	
	Let $\phi_1^{R_0}$ be the first odd eigenfunction of $L_K$ in $B_{R_0} \subset \R^{2m}$, given by Lemma~\ref{Lemma:FirstOddEigenfunction}, and let  $\lambda_1^{R_0} := \lambda_{1, \, \mathrm{odd}}(B_{R_0}, L_K)$. Then, we claim that for $R_0$ big enough and $\varepsilon$ small enough, $\usub_R = \varepsilon\phi_1^{R_0} $ is an odd subsolution of \eqref{Eq:ProofExistenceProblemBR} for every $R\geq R_0$. To see this, first note that, without loss of generality, we can assume that $\norm{\phi_1^{R_0}}_{L^\infty(B_R)}=1$. Then, since $\varepsilon \phi_1^{R_0}>0$ in $B_{R_0}\cap \ocal$ and using \eqref{Eq:PropertyConcavityf}, we see that for every $x\in B_{R_0}\cap \ocal$,
	$$
	\dfrac{f(\varepsilon \phi_1^{R_0}(x))}{\varepsilon \phi_1^{R_0}(x)} > f'(\varepsilon \phi_1^{R_0}(x)) \geq f'(0)/2
	$$
	if $\varepsilon$ is small enough, independently of $x$. Therefore, since $f'(0)>0$, taking $R_0$ big enough so that $\lambda_1^{R_0} < f'(0)/2$ (see the last statement of Lemma~\ref{Lemma:FirstOddEigenfunction}), we have that for every $x\in B_{R_0}\cap \ocal$,  $f(\varepsilon \phi_1^{R_0}(x)) > \lambda_1 \varepsilon \phi_1^R(x)$. Thus,
	$$
	L_K \usub_R = \lambda_1^{R_0} \varepsilon \phi_1^{R_0} < f(\varepsilon\phi_1^{R_0}) = f(\usub_R) \quad \textrm{ in } B_{R_0}\cap \ocal\,.
	$$
	In addition, if $x\in (B_R\setminus B_{R_0})\cap\ocal$, by Remark~\ref{Remark:CsRegularityFirstEigenfunction} we have that
	$$
	L_K \usub_R < 0 = f(0) =  f(\usub_R) \quad \textrm{ in } (B_R\setminus B_{R_0})\cap \ocal\,.
	$$
	Hence, the claim is proved.
	
	Now, if we define $\usup_R := \chi_{\ocal \cap B_R} - \chi_{\ical \cap B_R}$, a simple computation shows that it is an odd supersolution to \eqref{Eq:ProofExistenceProblemBR}. Therefore, using the monotone iteration procedure given in Proposition~\ref{Prop:MonotoneIterationOdd} (taking into account Remarks~\ref{Remark:MaxPrincipleSingularity} and \ref{Remark:CsRegularityFirstEigenfunction} when using the maximum principle), we obtain a solution $u_R$ to \eqref{Eq:ProofExistenceProblemBR} such that it is doubly radial, odd with respect to the Simons cone and $\varepsilon \phi_1^{R_0} = \usub_R \leq u_R \leq \usup_R$ in $\ocal$. Note that, since $\usub_R > 0$ in $\ocal \cap B_{R_0}$, the same holds for $u_R$.
	
	Using a standard compactness argument as in \cite{FelipeSanz-Perela:IntegroDifferentialI}, we let $R\to +\infty$ to obtain a sequence $u_{R_j}$ converging on compacts in  $C^{2\s + \eta}(\R^{2m})$ norm, for some $\eta > 0$, to a solution $u \in C^{2\s + \eta}(\R^{2m})$ of $L_K u = f(u)$ in $\R^{2m}$. Note that $u$ is doubly radial, odd with respect to the Simons cone and $0\leq u \leq 1$ in $\ocal$. Let us show that $0 < u < 1$ in $\ocal$ and hence $u$ is a saddle-shaped solution. Indeed, the usual strong maximum principle yields $u<1$ in $\ocal$. Moreover, since $u_R\geq\varepsilon \phi_1^{R_0}>0$ in  $\ocal \cap B_{R_0}$ for $R>R_0$, also the limit $u\geq\varepsilon \phi_1^{R_0}>0$ in  $\ocal \cap B_{R_0}$. Therefore, by applying the strong maximum principle for odd functions (see Proposition~\ref{Prop:MaximumPrincipleForOddFunctions}) we obtain that $0 < u < 1$ in $\ocal$.
\end{proof}


The fact of being $u$ positive in $\ocal$ yields that $u$ is stable in this set, as explained in the following remark. 






%%%%%%%%%%%%%%%%%%%%%%%%%%%%%
\section{Symmetry results}
\label{Sec:SymmetryResults}
%%%%%%%%%%%%%%%%%%%%%%%%%%%%%




This section is devoted to prove the following two symmetry results. The first one is a result for positive solutions in the whole space

\begin{theorem}
	\label{Th:SymmetryWholeSpace}
	Let $L$ be an integro-differential operator with kernel $K$ satisfying 99. Let $u$ be a bounded solution to
	\begin{equation}
	\label{Eq:PositiveWholeSpace}
	\beqc{\PDEsystem}
	L u &=& f(u) & \textrm{ in }\R^n\,,\\
	u &\geq& 0 & \textrm{ in } \R^n\,,
	\eeqc
	\end{equation}
	with the nonlinearity $f\in C^1$ satisfying
	\begin{itemize}
		\item $f(0) = f(1) = 0$,
		\item $f'(0)>0$,
		\item $f>0$ in $(0,1)$, and
		\item $f<0$ in $(1,+\infty)$.
	\end{itemize}
	Then, $u\equiv 0$ or $u \equiv 1$.
\end{theorem}

The second one is a symmetry result for equations in a half-space. Here we use the notation $\R^n_+ = \{x_n > 0\}$ and $\R^n_- = \{x_n < 0\}$ .

\begin{theorem}
	\label{Th:SymmHalfSpace}
	Let $L$ be an integro-differential operator with kernel $K$ satisfying 99. Let $u$ be a bounded solution to one of these two problems
	
	\begin{equation}
	\leqnomode
	\tag{P4}
	\label{Eq:P4}
	\beqc{\PDEsystem}
	Lv &=& f(v)   &\textrm{ in } \,\R^n_+,\\
	v &>& 0   &\textrm{ in } \,\R^n_+,\\
	v(x',x_n) &=& -v(x',-x_n)   &\textrm{ in } \,\R^n.
	\eeqc
	\end{equation}

	\begin{equation}
	\leqnomode
	\tag{P3}
	\label{Eq:P3}
	\beqc{\PDEsystem}
	Lu &=& f(v)   &\textrm{ in } \,\R^n_+,\\
	v &>& 0   &\textrm{ in } \,\R^n_+,\\
	v &=& 0   &\textrm{ in } \,\overline{\R^n_-},
	\eeqc
	\end{equation}
	

	
	\reqnomode
	
	Assume that the kernel $K$ of the integral operator $L$ satisfies
	$$
	K(x-y) \geq K(x-y^*) \,\,\,\,\text{for all } \,\, x,y\in \R^n_+,
	$$
	where $y^*$ is the reflection of $y$ with respect to $\{x_n = 0\}$. Suppose that the nonlinearity $f$ is Lipsitchz and
	\begin{itemize}
		\item $f(0) = f(1) = 0$,
		\item $f'(0)>0$, and $f'(t)\leq 0$ for all $t\in[1-\delta,1]$ for some $\delta>0$,
		\item $f>0$ in $(0,1)$, and
		\item $f$ is odd in the case of \eqref{Eq:P4}.
	\end{itemize}
	Then, $v$ depends only on $x_n$ and it is increasing in that direction.
\end{theorem}


The result in the case of problem \eqref{Eq:P3} will not be used in this paper. However, since the proof is exactly the same as for \eqref{Eq:P4} we include it here for completeness and further reference.

\subsection{Preliminary: Parabolic Maximum principle}





\begin{theorem}
\label{Th:ParabolicmaxPrpBdd}
Assume that $v \in C^{2\s+\epsilon}(B_R\times(0,T])$ is bounded above and satisfies
\begin{equation*}
\beqc{\PDEsystem}
\partial_t v + L v &\leq& 0 & \textrm{ in }B_R\times(0,T]\,,\\
v_0:=v(x,0) &\leq& 0 & \textrm{ in } B_R\,,\\
v &\leq& 0 & \textrm{ in } \left( \R^n\setminus B_R\right) \times (0,T] \,.
\eeqc
\end{equation*}
Then, $v\leq 0$ in $\R^n\times (0,T]$.
\end{theorem}

\begin{proof}
By contradiction, assume that $v$ attains a positive absolute maximum at a point $(x_0,t_0)$. By the exterior conditions, $x_0$ must be in $B_R$. If $t_0\in(0,T)$, then $(x_0,t_0)$ is an interior global maximum and it must satisfy $v_t(x_0,t_0)=0$ and $Lv(x_0,t_0)>0$, which contradicts the equation. If $t_0 = T$, then $v_t(x_0,t_o)\geq 0$ and $Lv(x_0,t_0)>0$, which is also a contradiction with the equation.
\end{proof}

\begin{lemma}
\label{Lemma:NoBddSolL=1}
There is no bounded solution of $$Lv=1 \,\,\, \text{ in } \,\,\, \R^n.$$
\end{lemma}

\begin{proof}
Assume by contradiction that such solution exists. Then, by interior regularity (see Section~\ref{Sec:Preliminaries}) $v\in C^1(\R^n)$ and $|\nabla v|\leq C$ in $\R^n$. For every $i = 1,\ldots, n$, we differentiate the equation with respect to $x_i$ to obtain
\begin{equation*}
\beqc{\PDEsystem}
L v_{x_i} &=& 0 & \textrm{ in } \R^n\,,\\
|v_{x_i}| &\leq& C & \textrm{ in } \R^n\,.
\eeqc
\end{equation*}
By Liouville Theorem, $v_{x_i}$ is constant. \todo{Quizás está bien escribir en algún sitio o como minimo referenciarlo} Hence, $\nabla v$ is constant, and thus $v$ is affine. But since $u$ is bounded, $v$ must be constant too, and we arrive to a contradiction with $Lv=1$.
\end{proof}

\begin{lemma}
\label{Lemma:SolBall}
Let $L$ be an integro-differential operator with kernel $K$ satisfying 99-99 and let $R>0$ be given. Then, there exists $\phi_R$ a solution of
\begin{equation*}
\beqc{\PDEsystem}
L \phi_R &=& 1 & \textrm{ in } B_R\,,\\
\phi_R &=& 0 & \textrm{ in } \R^n\setminus B_R\,,
\eeqc
\end{equation*}
and satisfying
$$
M_R:= \sup_{B_R} \phi_R \to \infty \quad \text{as } R\to \infty\,.
$$
\end{lemma}
\begin{proof}
The existence of a weak solution $\phi_R$ is given by Riesz representation theorem. Moreover, by regularity results (see Section \ref{Subsec:Regularity}), $\phi_R$ is in fact a classical solution and by the maximum principle, $\phi_R>0$ in $B_R$. 

Since $M_R$ is increasing (use the maximum principle to compare $\phi_R$ and $\phi_{R'}$ with $R>R'$), it must have a limit $M$. Assume by contradiction that $M<+\infty$. Now, consider the new function $ \varphi_R := \phi_R/M_R$, which satisfies
\begin{equation}
\label{Eq:varphi}
\beqc{\PDEsystem}
L \varphi_R &=& 1/M_R & \textrm{ in } B_R\,,\\
\varphi_R &=& 0 & \textrm{ in } \R^n\setminus B_R\,, \\
\varphi_R &\leq & 1\,.
\eeqc
\end{equation}

Therefore, applying Lemma \ref{Lemma:CompactnessLemma} we deduce that $\varphi_R$ converges (up to a subsequence) to a function $\varphi$ that is solution to $L \varphi = 1/M$ in $\R^n$ and satisfies  $|\varphi| \leq 1$. This contradicts Lemma~\ref{Lemma:NoBddSolL=1} and therefore, $M_R \to +\infty$.
\end{proof}

\begin{lemma}
\label{Lemma:SolBallToZero}
Let $M_R$ be as in the previous lemma. Then, there exists a function $\psi_R\geq 0$ solution of
\begin{equation*}
\beqc{\PDEsystem}
L \psi_R &=& -1/M_R & \textrm{ in } B_R\,,\\
\psi_R &=& 1 & \textrm{ in } \R^n\setminus B_R\,,
\eeqc
\end{equation*}
such that
$$ \psi_R \to  0 \quad \text{as } R\to \infty\,. $$
\end{lemma}

\begin{proof}
We define $ \psi_R := 1-\phi_R/M_R = 1-\varphi_R$, where $\phi_R$ and $\varphi_R$ are defined in the previous proof. Thus, we only need to show the limit condition. We will see that $\varphi_R \to 1$ as $R\to\infty$. Recall that $\varphi_R$ solves problem \eqref{Eq:varphi}, and by the previous arguments, letting $R\to \infty$ we deduce that $\varphi_R$ converges to a function $\varphi\geq 0$ that solves $ L\varphi = 0 $ in $\R^n$. By Liouville Theorem, $\varphi$ must be constant, and since its $L^\infty$-norm is $1$ and $\varphi\geq 0$, we conclude $\varphi\equiv 1$.
\end{proof}

\begin{theorem}
\label{Th:ParaMaxPrp}
Assume $v$ is a bounded function such that
\begin{equation*}
\beqc{\PDEsystem}
\partial_t v + L v + c\,v &\leq& 0 & \textrm{ in }\R^n\times(0,+\infty)\,,\\
v_0:=v(x,0) &\leq& 0 & \textrm{ in } \R^n\,,
\eeqc
\end{equation*}
with $c=c(x)$ a bounded function. Then,
$$ v(x,t) \leq 0 \,\,\,\,\,\text{ in } \,\,\, \R^n\times[0,+\infty). $$
\end{theorem}

\begin{proof}
First of all, note that with the change of function $\tilde{v}(x,t) = \e^{-\alpha\,t} v(x,t)$ we can reduce the initial problem to
\begin{equation*}
\beqc{\PDEsystem}
\partial_t \tilde{v} + L \tilde{v} &\leq& 0 & \textrm{ in } \Omega \subseteq\R^n\times(0,+\infty)\,,\\
\tilde{v} &\leq& 0 & \textrm{ in }  \left(\R^n\times(0,+\infty)\right) \setminus  \Omega\,,\\
\tilde{v}_0 &\leq& 0 & \textrm{ in } \R^n\,,
\eeqc
\end{equation*}
if we take $\alpha > ||c||_{L^\infty}$ and $\Omega = \{(x,t)\in \R^n\times(0,+\infty) \ \textrm{such that } \ v(x,t) > 0\}$.

Now, consider the function $ w_R(x,t) := \norm{v}_{L^\infty} ( \psi_R + t/M_R )$, which satisfies
\begin{equation*}
\beqc{\PDEsystem}
\partial_t w_R + L w_R &=& 0 & \textrm{ in }B_R\times(0,T]\,,\\
w_R(x,0) &\geq& 0 & \textrm{ in } B_R\,,\\
w_R(x,t) &\geq& \norm{v}_{L^\infty}  & \textrm{ in } \left( \R^n\setminus B_R\right) \times (0,T] \,.
\eeqc
\end{equation*}
By the maximum principle in $(B_R\times[0,T])\cap \Omega$ (Lemma \ref{Th:ParabolicmaxPrpBdd}), we conclude that $ w_R\geq \tilde{v} $ in $B_R\times(0,T]$.

Finally, given an arbitrary point $(x_0,t_0)$, take $R_0>0$ and $T>0$ such that $(x_0,t_0)\in B_{R_0}\times [0,T]$. Thus,
$$ \tilde{v}(x_0,t_0) \leq w_R(x_o,t_0) = \norm{v}_{L^\infty} \left(  \psi_R(x_0) + \frac{t_0}{M_R} \right), \,\,\,\,\,\text{ for every }\,\,\, R\geq R_0.
$$
Letting $R \to \infty$ and using that $\psi_R(x_0) \to 0$ and $M_R \to \infty$ by Lemmas \ref{Lemma:SolBall} and \ref{Lemma:SolBallToZero}, we conclude $ \tilde{v}(x_0,t_0) \leq  0$, and therefore $ v(x_0,t_0) = \e^{\alpha\,t_0}\,\tilde{v}(x_0,t_0) \leq 0$.
\end{proof}

%%%%%%%%%%%%%%%%%%%%%%%%%%%%%%%%%%%%%%%%%%%%%%%%%%%%%%%%%%%%%%%%%%%%%%%%%%%%%%%%%%%%%%%%%%%%%%%%%%%%%%%%%%%%%%%%%%%%%%%%%%%%%%%%%%%%%%%%%%%%%%%%%%%

\subsection{A symmetry result for positive solutions in the whole space}



\begin{proof}[Proof of Theorem~\ref{Th:SymmetryWholeSpace}]
The proof follows the ideas of Berestycki, Hamel and Nadirashvili from Theorem 2.2. in \cite{BerestyckiHamelNadi} but adapted to the whole space and with a nonlocal operator.

Assume $u\not\equiv 0$. Then, by the strong maximum principle $u>0$. Our goal is to show that $u \equiv 1$, and this will be accomplished in two steps.

\textbf{Step 1:} We show that $m:=\inf_{\R^n} u >0$.

By contradiction, we will assume $m=0$. Then, there exists a sequence $\{x_k\}$ such that $u(x_k)\rightarrow 0$ as $k \rightarrow +\infty$. By the Harnack Inequality  from Di Casto, Kuusi and Palatucci in \cite{DiCastoKuusiPalatucci}, given any $R>0$ we have
\todo[inline, color=red]{MIRAR LO DE MATTEO!!!!}
\begin{equation}
\label{Eq:Harnack}
\sup_{B_R(x_k)}u \leq C_R \inf_{B_R(x_k)}u \leq C \, u(x_k) \rightarrow 0 \,\,\text{as}\,\, k\rightarrow +\infty.
\end{equation}


Since $f(0) = 0 $ and $f'(0)>0$, it is easy to show that $f(t)\geq f'(0)t/2$ if $t$ is small enough. Therefore, from this and \eqref{Eq:Harnack}  we deduce that there exists $M(R)\in\N$ such that
\begin{align}
\label{Eq:WholeSpace2}
L u - \frac{f'(0)}{2}u \geq 0 \,\,\textrm{ in }\ B_R(x_{M(R)})\,. 
\end{align}
On the other hand, let us define
$$ \lambda_R^{x_0} = \inf_{\substack{\varphi\in\cp{1}_{0}(B_R(x_0))\\ \varphi\not\equiv 0}} \frac{\ds \int_{\R^n}\int_{\R^n}|\varphi(x)-\varphi(y)|^2\,K(x-y) \d x \d y}{\ds \int_{\R^n}\varphi^2 \d x}, $$
which decreases to zero uniformly in $x_0$ as $R$ goes to infinity from being $L\in\mathcal{L}_0$ (see the proof of Lemma~\ref{Lemma:FirstOddEigenfunction} and also Proposition~9 of \cite{ServadeiValdinoci}). Therefore, there exists $R_0>0$ such that
$$ \lambda_R^x < \frac{f'(0)}{2} $$
for all $x\in \R^n$ and $R\geq R_0$. In particular, by choosing $x=x_{M(R_0)}$ there exists $w\in\cp{1}_0(B_{R_0}(x_{M(R_0)}))$ such that $w\not\equiv 0$ and
\begin{equation}
\label{Eq:Eigenfunction}
\int_{\R^n}\int_{\R^n}|w(x)-w(y)|^2\,K(x-y) \d x \d y < \frac{f'(0)}{2}\int_{\R^n}w^2 \d x.
\end{equation}

If we multiply \eqref{Eq:WholeSpace2} by $w^2/u\geq 0$ and integrate in $\R^n$, we get
\begin{align*}
0 &\leq \int_{\R^n} Lu\,\frac{w^2}{u} \d x - \frac{f'(0)}{2}\int_{\R^n} w^2 \d x \\
&= \int_{\R^n}\int_{\R^n}\left\{ u(x)-u(y) \right\} \left( \frac{w^2(x)}{u(x)}-\frac{w^2(y)}{u(y)} \right) K(x-y) \d x \d y - \frac{f'(0)}{2}\int_{\R^n} w^2 \d x \\
&\leq \int_{\R^n}\int_{\R^n} |w(x)-w(y)|^2 K(x-y) - \frac{f'(0)}{2}\int_{\R^n} w^2 \d x ,
\end{align*}
which contradicts \eqref{Eq:Eigenfunction}. Here we have used that the kernel is positive and that
$$
\left\{ u(x)-u(y) \right\} \left( \frac{w^2(x)}{u(x)}-\frac{w^2(y)}{u(y)} \right)  \leq |w (x) - w(y)|^2\,.
$$
Indeed, developing the squares and the products, this last inequality is equivalent to
$$
2 w(x) w(y) \leq \dfrac{u(x)}{u(y)} w^2(y) +  \dfrac{u(y)}{u(x)} \xi^2 (x)\,,
$$
which is equivalent to
$$
\bpar{w (x)\sqrt{\dfrac{u(y)}{u(x)}} - w(y) \sqrt{\dfrac{u(x)}{u(y)} } }^2 \geq 0\,.
$$


Then $\inf_{\R^n} u >0$.\\

\textbf{Step 2:} We show that $u\equiv 1$.

Now, choose $0<\xi_0<\min\{1,m\}$, which is well defined by Step~1, and let $\xi(t)$ be the solution of the ODE
$$
\beqc{\PDEsystem}
\dot{\xi}(t) &=& f(\xi(t)) & \textrm{ in }(0,\infty)\,,\\
\xi(0) &=& \xi_0\,.
\eeqc
$$
Since $f>0$ in $(0,1)$ and $f(1) = 0$ we have that $\dot{\xi}(t)>0$ for all $t\geq 0$ and $\ds \lim_{t\to 0} \xi(t) = 1$.

Now, note that both $u(x)$ and $\xi(t)$ solve the parabolic equation
$$ \partial_t w + Lw = f(w) \,\,\, \textrm{ in }\R^n\times (0,\infty)\,, $$
and satisfy
$$ u(x) \geq m \geq \xi_0 = \xi(0). $$
Thus, by the parabolic maximum principle, Theorem \ref{Th:ParaMaxPrp}, $u(x)\geq \xi(t)$ for all $x\in\R^n$ and $t\in(0,\infty)$. By letting $t \to \infty$ we obtain
$$ u(x) \geq 1 \,\, \textrm{ in }\R^n\,.  $$
In a similar way, taking $\tilde{\xi}_0>\norm{u}_{L^\infty} \geq 1$, using $f<0$ in $(1,\infty)$, $f(1)=0$ and the parabolic maximum principle, we obtain the upper bound $u\leq 1$.

\end{proof}


%%%%%%%%%%%%%%%%%%%%%%%%%%%%%%%%%%%%%%%%%%%%%%%%%%%%%%%%%%%%%%%%%%%%%%%%%%%%%%%%%%%%%%%%%%%%%%%%%%%%%%%%%%%%%%%%%%%%%%%%%%%%%%%%%%%%%%%%%%%%%%%%%%%%%%%%%

\subsection{A one-dimensional symmetry result for positive solutions ``in a half-space''} In this subsection...



\todo[inline]{Introduction...}

First we show that the solution is monotone.\todo{Escribir bien} We do it using a moving planes argument, and for this reason we need the following maximum principle in ``narrow'' domains. Recall that for a domain $\Omega \subset \R^n$, we define the quantity $R(\Omega)$ as the smallest positive $R$ for which
$$
\dfrac{|B_R(x)\setminus \Omega|}{|B_R(x)|}\geq \dfrac{1}{2} \quad \text{ for every } x \in \Omega.
$$
If no such radius exists, we define $R(\Omega) = +\infty$. Thus, we say that a domain $\Omega$ is ``narrow'' ir $R(\Omega)$ is small.


An important result that we need is the following ABP-type estimate. 
\todo[inline]{Esto aparece en Quaas y Xia para el fractional Laplacian, la demo es parecida y es como la de Cabre. Para L no he encontrado que la hicieran}
\begin{theorem}
	\label{Th:ABPEstimate}
	Let $\Omega \subset \R^n$ with $R(\Omega) < +\infty$. Let $L \in \lcal_0(\s)$\todo{Esta es buena notacion?} and let $v\in C^{\beta}(\Omega)$, with $\beta > 2\s$, such that $\sup_{\Omega} v < +\infty$ and satisfying
	$$
	\beqc{\PDEsystem}
	Lv - c(x)v &\leq & h & \text{ in } \Omega\,, \\
	v & \leq & 0 & \text{ in } \R^n\setminus \Omega\,,
	\eeqc
	$$
	with $c(x)\leq 0$ in $\Omega$ and $h\in L^\infty(\Omega)$.
	
	Then,
	$$
	\sup_\Omega v \leq C R(\Omega)^{2\s} \norm{h}_{L^{\infty}(\Omega)}\,,
	$$
	where $C$ is a constant depending on \todo{Poner}.
\end{theorem}

As a consequence of this result, one can deduce easily a maximum principle in ``narrow'' domains.

\begin{corollary}
	\label{Cor:MaxPpleNarrowDomains}
	Let $\Omega \subset \R^n$ with $R(\Omega) < +\infty$ and let $v\in C^{\beta}(\Omega)$, with $\beta > 2\s$, satisfy
	$$
	\beqc{\PDEsystem}
	Lv + c(x)v &\leq & 0 & \text{ in } \Omega\,, \\
	v & \leq & 0 & \text{ in } \R^n\setminus \Omega\,,
	\eeqc
	$$
	with $c(x)$ bounded below.
	
	Then, there exists a number $\bar{R} > 0$ such that $v \leq 0$ in $\Omega$ whenever $R(\Omega)< \bar{R}$.
	
\end{corollary}

\begin{proof}
	We write $c= c^+ - c^-$, and therefore $Lv -(-c^+)v \leq c^- v^+	$. By Theorem~\ref{Th:ABPEstimate} we get
	$$
	\sup_\Omega v \leq C R(\Omega)^{2\s} \norm{c^- v^+}_{L^\infty(\Omega)} \leq C R(\Omega)^{2\s} \norm{c^-}_{L^\infty(\Omega)} \sup_\Omega v 
	$$
	and if $C R(\Omega)^{2\s} \norm{c^-}_{L^\infty(\Omega)}  <1 $, we deduce that $v\leq 0$ in $\Omega$.
\end{proof}

The only ingredient needed to show Theorem~\ref{Th:ABPEstimate} is the following weak Harnack inequality, which follows easily from \cite{DiCastoKuusiPalatucci}

\todo[inline]{De hecho seguro que está en lo de Matteo, CHECK!!
	De momento no la pongo}

\begin{proof}[Proof of Theorem~\ref{Th:ABPEstimate}]
	We are going to show it for $v$ satisfying,
	$$
	\beqc{\PDEsystem}
	Lv &\leq & h & \text{ in } \Omega\,, \\
	v & \leq & 0 & \text{ in } \R^n\setminus \Omega\,,
	\eeqc
	$$
	since the case $c\neq 0$ can be reduced to this one. Indeed, if we consider $\Omega_0 = \{x \in \Omega \ : \ v > 0\}$, then $Lv \leq Lv - c(x)v \leq h$ in $\Omega_0$.
	
	Assume first that $\Omega$ is bounded. Then the supremum of $v$ must be achieved at an interior point $x_0\in \Omega$. Define	$M:= v(x_0) = \sup_\Omega v$ and consider the function $w := M - v^+$. Note that $0 \leq v \leq M$, $v(x_0) = 0$ and $v \equiv M$ in $\R^n \setminus \Omega$. If we extend $h$ to be $0$ outside $\Omega$, we see that $Lv \geq -h$ in $B_R(x_0)$. 
	
	Now, by choosing $R= 2R(\Omega)$, and using the weak Harnack inequality \todo{citar o poner bien} we get
	\begin{align*}
		M \dfrac{1}{2^\varepsilon} & \leq \bpar{M^{\varepsilon}\dfrac{|B_{R/2}(x_0)\setminus \Omega|}{|B_{R/2}(x_0)|}}^{1/\varepsilon}= \bpar{\dfrac{1}{|B_{R/2}(x_0)|} \int_{B_{R/2}(x_0)\setminus \Omega} v^\varepsilon}^{1/\varepsilon} \\ 
		& \leq \bpar{ \fint_{B_{R/2}(x_0)} v^\varepsilon}^{1/\varepsilon} \leq C \bpar{\inf_{B_{R}(x_0)} v + R^{2\s} \norm{h}_{L^{\infty}(\Omega)} }\,.
	\end{align*}
	The conclusion follows from the fact that $v(x_0)= \inf_{B_{R}(x_0)} v = 0$.
	
	In the case of $\Omega$ being unbounded, the proof is the same with minor changes. We define $M$ as before and we consider, for every $\delta > 0$, a point $x_0$ such that $M-\delta \leq v(x_0)$. We proceed as before and the desired estimate follows by letting $\delta \to 0$. 
\end{proof}
 

\todo[inline]{Ellos lo demuestran dentro de la demo de los moving planes. En realidad sirve para los dos problemas. De todas formas, como el argumento es interesante y corto, lo ponemos aqui por completitud y para futura referencia. Hay que poner que es de ellos justo en el enunciado¿¿¿???}

In order to apply the moving planes method, the previous maximum principle in narrow domains is not powerful enough. The reason for this is that we need a prescribed constant sign of a function outside the domain, but in the application of the moving planes argument, since our functions are odd with respect to a hyperplane, they cannot have a constant sign in the exterior of a narrow band. For this reason, we need another version of a maximum principle in ``narrow'' domains that applies to odd functions and only assumes a constant sign of the function at one side of a hyperplane.

\begin{proposition}
	\label{Prop:MaxPrpNarrowOdd}
	Let $H$ be a half-space in $\R^n$, and denote by $x^*$ the reflection of any point $x$ with respect to the hyperplane $\partial H$. Let $L$ be an integro-differential operator with a positive kernel $K$ satisfying
	\begin{equation}
	\label{Eq:KernelSymmetry}
	K(x-y) \geq K(x-y^*), \,\,\,\,\text{for all } \,\, x,y\in H.
	\end{equation}
	Assume that $v $ 99 satisfies 
	\begin{equation}
	\beqc{\PDEsystem}
	L v &\geq& c(x)\,v  &\textrm{ in } \Omega\subseteq H,\\
	v &\geq& 0 &\textrm{ in } H\setminus\Omega,\\
	v(x) &=& -v(x^*) &\textrm{ in } \R^n.
	\eeqc
	\end{equation}
	Then, there exist a number $\overline{R}$ such that $v \geq 0$ whenever $R(\Omega) \leq \overline{R}$.
\end{proposition}

\begin{proof}
	Let us begin by defining $\Omega_- = \{x\in \Omega \,:\,\, v<0\}$. We shall prove that $\Omega_-$ is empty. Assume by contradiction that it is not empty. Then, we split 
	$$ v = v_1+v_2\,, $$
	where
	\begin{equation*}
	v_1(x) = 
	\begin{cases}
	v(x)  &\textrm{ in } \Omega_-,\\
	0 &\textrm{ in } \R^n\setminus\Omega_-,
	\end{cases}
	\quad \text{ and } \quad
	v_2(x) = 
	\begin{cases}
	0  &\textrm{ in } \Omega_-,\\
	v(x) &\textrm{ in } \R^n\setminus\Omega_-.
	\end{cases}
	\end{equation*}
	Let us first show that $Lv_2\leq 0$ in $\Omega_-$. To see this, let us take $x\in\Omega_-$ and thus
	$$ 
	Lv_2(x) = \int_{\R^n\setminus\Omega_-} -v_2(y)K(x-y) \d y = -\int_{\R^n\setminus\Omega_-} v(y)K(x-y) \d y.  
	$$
	Now, we split $\R^n\setminus\Omega_-$ into
	$$ 
	A_1 = \Omega_-^*,\,\,\,\,\,\,\,\text{ and }\,\,\,\,\,\,\, A_2 = \left(H\setminus\Omega_-\right)\cup\left(H\setminus\Omega_-\right)^*,
	$$
	and we compute
	\begin{align*}
	-\int_{A_1} v(y)K(x-y) \d y = -\int_{\Omega_-} v(y^*)K(x-y^*) \d y  = \int_{\Omega_-} v(y)K(x-y^*) \d y \leq 0,
	\end{align*}
	where the last inequality comes from being $v$ negative in $\Omega_-$ and the kernel positive in all $\R^n$.
	On the other hand
	\begin{align*}
	-\int_{A_2} v(y)K(x-y) \d y = -\int_{H\setminus\Omega_-} v(y)K(x-y) \d y  -\int_{H\setminus\Omega_-} v(y^*)K(x-y^*) \d y \\ 
	= -\int_{H\setminus\Omega_-} v(y)\left\{K(x-y)-K(x-y^*)\right\} \d y \leq 0,
	\end{align*}
	where we have use the kernel condition \eqref{Eq:KernelSymmetry} and the odd symmetry of $v$. Thus, we get $Lv_2 \leq 0$ in $\Omega_-$, which means
	$$ Lv_1 = Lv-Lv_2 \geq Lv \geq c(x)\,v = c(x)\,v_1 \,\,\,\,\text{ in }\,\,\Omega_-. $$
	Therefore $v_1$ solves
	\begin{equation*}
	\beqc{\PDEsystem}
	Lv_1 &\geq& c(x)\,v_1   &\textrm{ in } \,\Omega_-,\\
	v_1 &=& 0 &\textrm{ in }\,\R^n\setminus\Omega_-,
	\eeqc
	\end{equation*}
	and we can apply the usual maximum principle for narrow domains (Corollary~\ref{Cor:MaxPpleNarrowDomains}) to $v_1$ in $\Omega_-$ in order to deduce that $v_1\geq 0$ in all $\R^n$. But this is a contradiction with the definition of $v_1$ and the fact that the set $\Omega_-$ is not empty.
\end{proof}

This maximum principle in narrow domains for odd functions with respect to a hyperplane is the principal tool to use the moving plane argument. This allows us to show the following result.

\todo[inline]{Mirar lo que hacen Barrios, Quaas, etc en MR3624935 (2017) por si hay que citarlo}
\begin{proposition}
	\label{Prop:MonotonyHalfSpace}
	Let $v$ be a bounded solution of one of the problems \eqref{Eq:P3} or \eqref{Eq:P4}, with $L$ an integro-differential operator with a positive kernel $K$ satisfying \eqref{Eq:KernelSymmetry} and $f$ a Lipschitz nonlinearity such that $f>0$ in $(0,||v||_{\lp{\infty}(\R^n_+)})$. Then,
	$$ 
	\frac{\partial v}{\partial x_n} > 0 \,\,\,\, \text{ in } \,\,\R^n_+.
	$$
\end{proposition}


\begin{proof}
	The proof is based on the moving planes method, and is exactly the same as the analogue proof of Theorem~3.1 in \cite{QuaasXia}, where Quaas and Xia establish an equivalent result for the fractional Laplacian. For this reason, we give here just a sketch. As usual, for $\lambda > 0$ one defines $w_\lambda (x) = v(x',2\lambda - x_n)-v(x',x_n)$ and since the nonlinearity is Lipschitz, $w_\lambda$ solves, in any of the two cases ---\eqref{Eq:P3} or \eqref{Eq:P4}---, the following problem:
	$$
	\beqc{\PDEsystem} 
	L w_\lambda &=& c_\lambda(x)\,w_\lambda  &\textrm{ in } \Sigma_\lambda\subseteq H_\lambda,\\ 
	w_\lambda &\geq& 0 &\textrm{ in } H_\lambda\setminus\Sigma_\lambda,\\ 
	w_\lambda(x) &=& w_\lambda(x_\lambda) &\textrm{ in } \R^n, 
	\eeqc 
	$$
	where $\Sigma_\lambda := \left\{ x = (x',x_n) \ : \ 0<x_n<\lambda \right\}$ and $H_\lambda := \left\{ x = (x',x_n) \ : \ x_n<\lambda \right\}$ and $c_\lambda$ is a bounded function. Note that $w_\lambda$ is odd with respect to $\partial H_\lambda$. Then, using the maximum principle in narrow domains  Proposition~\ref{Prop:MaxPrpNarrowOdd} one shows that, if $\lambda$ is small enough, $w_\lambda>0$ in $\Sigma_\lambda$. To conclude the proof, we define 
	$$ 
	\lambda^* := \sup\{\lambda \ : \ w_\eta>0 \,\, \text{ in } \,\, \Sigma_\lambda \,\, \text{ for all } \,\, \eta<\lambda\}. 
	$$
	Note that $\lambda^*$ is well defined by the previous argument. Then, to conclude the proof one has to show that $\lambda^*=\infty$. This is done by showing that, if $\lambda^*$ is finite, then there exists a small $\delta_0 > 0$ such that for every $\delta \in (0,\delta_0]$ we have
	$$
	w_{\lambda^* +  \delta} (x) > 0 \quad \text{ in } \Sigma_{\lambda^*-\varepsilon}\setminus \Sigma_{\varepsilon}
	$$
	for some small $\varepsilon$.
	This is established using a compactness argument exactly as in Lemma~3.1 \cite{QuaasXia}. Finally, by the maximum principle in narrow domains we can deduce that $w_{\lambda^* +  \delta} (x) > 0 $ in $\Sigma_{\lambda^*+\delta}$, contradicting the definition of $\lambda^*$.
\end{proof}

\todo[inline]{habría que citar la harnack q se usa en Lemma 3.1, seguramente la de Matteo funciona}










\begin{proposition}
\label{Prop:HalfSpaceLimUnif}
Let $u$ be a bounded solution of one of the following problems

\begin{equation}
\leqnomode
\tag{P1}
\label{Eq:P1}
\beqc{\PDEsystem}
L u &=& f(u)  &\textrm{ in }\R^n\,,\\
\ds \lim_{x_n \to \pm \infty} u(x',x_n) &=& \pm 1 \,\,\, &\textrm{ uniformly}.
\eeqc
\end{equation}

\begin{equation}
\leqnomode
\tag{P2}
\label{Eq:P2}
\beqc{\PDEsystem}
L u &=& f(u)  &\textrm{ in }\R^n_+ = \{ x_n>0\} \,,\\ 
u &=& 0  &\textrm{ in }\overline{\R^n_-} = \{ x_n\leq 0\}\,,\\
\ds \lim_{x_n \to + \infty} u(x',x_n) &=& 1 \,\,\, &\textrm{ uniformly}.
\eeqc
\end{equation}

\reqnomode

Assume that there exists $\delta > 0$ such that
$$ f'(t) \leq 0 \quad \text{ in } \quad [-1,-1+\delta]\cup[1-\delta,1], $$
for problem \eqref{Eq:P1} and
$$ f'(t) \leq 0 \quad \text{ in } \quad [1-\delta,1] $$
for problem \eqref{Eq:P2}.

Then, $u$ only depends on $x_n$ and is increasing in that direction.
\end{proposition}

\begin{proof}
It is based on the sliding method, exactly as in the proof of Theorem~1 in \cite{BerestyckiHamelMonneau}. The idea is, as usual,   to define $ v^t(x) := v(x+\nu t) $ for every $\nu\in\R^n$ with $|\nu|=1$ and $\nu_n>0$ and the aim is to show that $v^t(x)-v(x)\geq 0$ for all $t\geq 0$. Despite the fact that $L$ is a nonlocal operator, the proof is exactly the same as the one in \cite{BerestyckiHamelMonneau} ---it only relies on the maximum principle and the translation invariance of the operator. Therefore, we do not include here the details.
\end{proof}






With all these ingredients we can now proce Theorem~\ref{Th:SymmHalfSpace}

\begin{proof}[Proof of Theorem~\ref{Th:SymmHalfSpace}]
Note that by Proposition~\ref{Prop:HalfSpaceLimUnif} we only need to prove that 
$$
\ds \lim_{x_n\to \infty} v(x',x_n) = 1
$$
uniformly. Therefore we divide the proof in two steps: first, we prove that the limit exists and is $1$, and then we prove that it is uniform.


\textbf{Step 1:} Given $x'\in \R^{n-1}$, then  $\ds \lim_{x_n\to \infty} v(x',x_n) = 1$.

By Proposition \ref{Prop:MonotonyHalfSpace} we know that $v$ is strictly increasing in the direction $x_n$. Since $v$ is also bounded by hypothesis, we know that, given $x'\in\R^{n-1}$, the one variable function $v(x',\cdot)$ has a limit, that we call $\overline{v}(x')$. Note that, since $v(x',0) = 0$ and $v_{x_n}>0$, we deduce that $\overline{v}(x') > 0$.

Let $x_n^k$ be any increasing sequence tending to infinity. Define $v_k(x',x_n) := v(x',x_n+x_n^k)$. By the regularity theory of the operator $L$ (see Section~\ref{Sec:Preliminaries}) and a standard compactness argument, we see that, up to a subsequence, $v_k$ converge uniformly on compact sets to a function $v_\infty$ that is a solution to
\begin{equation}
\label{Eq:ProofSymmHalf-SemilinearEqWholeSpace}
\beqc{\PDEsystem}
Lv_\infty &=& f(v_\infty)   &\textrm{ in } \,\R^n,\\
v_\infty &\geq& 0   &\textrm{ in } \,\R^n.
\eeqc
\end{equation}
By Theorem \ref{Th:SymmetryWholeSpace}, either $v_\infty\equiv 0$ or $v_\infty \equiv 1$. But, by construction,
$$ v_\infty(x',0) = \lim_{k\to \infty} v_k(x',0) = \lim_{k\to \infty} v(x',x_n^k) = \overline{v}(x') > 0, $$
and therefore the only possibility is
$$ \lim_{x_n\to \infty} v(x',x_n) = 1 \quad \text{ for all } \ x'\in\R^{n-1}. $$

\textbf{Step 2:}The limit is uniform in $x'$.

Let us proceed by contradiction. Suppose that the limit is not uniform. This means that given any $\varepsilon>0$ small enough, there exists a sequence of points $(x_k',x_n^k)$ with $x_n^k\to \infty$ such that $v(x_k',x_n^k) = 1-\varepsilon$. Similarly as before, the sequence of functions $\tilde{v}_k(x',x_n) = v(x'+x_k',x_n+x_n^k)$ converge uniformly on compact sets to a function $\tilde{v}_\infty$ that solves also \eqref{Eq:ProofSymmHalf-SemilinearEqWholeSpace}. By Theorem \ref{Th:SymmetryWholeSpace}, it is clear that $\tilde{v}_\infty\equiv 0$ or $\tilde{v}_\infty \equiv 1$. But, by construction
$$ \tilde{v}_\infty(0,0) = \lim_{k\to \infty} \tilde{v}_k(0,0) = \lim_{k\to \infty} v(x'_k,x_n^k) = 1-\varepsilon, $$
which is a contradiction for $\varepsilon>0$ small enough. Thus, the limit is uniform. 

By applying Proposition~\ref{Prop:HalfSpaceLimUnif}, we get that $v$ depends only on $x_n$ and is increasing in that direction.
\end{proof}



%%%%%%%%%%%%%%%%%%%%%%%%%%%%%
\section{Asymptotic result for the saddle-shaped solution}
\label{Sec:Asymptotic}
%%%%%%%%%%%%%%%%%%%%%%%%%%%%%

In this section, we establish Theorem~\ref{Th:AsymptoticBehaviourSaddleSolution}, concerning the asymptotic behavior of the saddle-shaped solution. 



To study the asymptotic behavior of the saddle-shaped solution it is important to relate the Allen-Cahn equation in $\R^{2m}$ with the same equation in $\R$. In the local case, this is very easy, since if $v$ is a solution to $-\ddot{v} = f(v)$ in $\R$,  then $w(x) = v(x\cdot e)$ solves $-\Delta w = f(w)$ in $\R^n$ for every unitary vector $e\in \R^n$. This happens also for the fractional Laplacian, thanks to the local extension problem.

Nevertheless, for a general operator $L_K$ this is not true anymore and we need a way to associate a solution of a one-dimensional problem with a one-dimensional solution to a $n$-dimensional problem. This is given in the next result. Some of its points appear in \cite{CozziPassalacqua} with a different notation. We state and prove it here for completeness.

\begin{proposition}
	\label{Prop:KernelsDimension}
	Let $L_K $ a symmetric and translation invariant intego-differential operator with kernel $K:\R^n \to \R$. Define the one dimensional kernel $K_1$ by
	$$ K_1(t) := \int_{\R^{n-1}} K\left(\theta,t\right) \d \theta = |t|^{n-1} \int_{\R^{n-1}} K\left(t\sigma,t\right) \d \sigma. $$
	\begin{enumerate}[label=(\roman{*})]
		\item Let $v:\R\to\R$ and consider $w:\R^n\to\R$ defined by $w(x) = v(x_n)$. Then, $L_K w(x) = L_{K_1} v(x_n)$. If we assume moreover that $K$ is radially symmetric, then the same happens for $w(x) = v(x\cdot e)$ for every unitary vector $e\in \Sph^{n-1}$. That is, $L_K w(x) = L_{K_1} v(x \cdot e)$.
		\item $K_1(t)$ is nonincreasing/decreasing in $(0,+\infty)$ if $K$ is nonincreasing/decreasing in  the $x_n$-direction in $\{x_n>0\}$.
		\item If $L_K \in \mathcal{L}_0 (n,\s,\lambda,\Lambda)$, then   $L_{K_1} \in \mathcal{L}_0 (1,\s,\lambda,\Lambda)$.
		\item In particular, if $L_K $ is the fractional Laplacian in dimension $n$, then $L_{K_1}$ is the fractional Laplacian in dimension $1$.
		
	\end{enumerate}
\end{proposition}

\begin{proof}
	We start proving point $(i)$. We write $y=(y',y_n)$, with $y'\in \R^{n-1}$.
	\begin{align*}
	L_K w(x) &= \int_{\R^n} \left\{ w(x)-w(y) \right\} K(x-y) \d y \\
	&=\int_{\R^n} \left\{ v(x_n)-v(y_n) \right\} K\left(x'-y',x_n-y_n\right) \d y' \d y_n.
	\end{align*}
	Now we make the change of variables $\theta = x'-y'$. That is,
	\begin{align*}
	L_K w(x) &= \int_{\R} \left\{ v(x_n)-v(y_n) \right\} \int_{\R^{n-1}} K\left(\theta,x_n-y_n\right) \d \theta \d y_n \\
	&= \int_{\R} \left\{ v(x_n)-v(y_n) \right\} \int_{\R^{n-1}} K\left(\theta,x_n-y_n\right) \d\theta \d y_n \\
	&= \int_{\R} \left\{ v(x_n)-v(y_n) \right\} K_1(x_n-y_n ) \d y_n = L_{K_1}v(x_n).
	\end{align*}
	
	The alternative expression of the kernel $K_1$, that is useful in some cases, can be obtained from the change of variable $\theta = t\sigma$.	In the case of $K$ radially symmetric, the result is valid for $u(x) = v(x\cdot e)$ for every unitary vector $e\in \Sph^{n-1}$ after a change of variables using the previous computations.
	
	We establish now point $(ii)$. To do it, we bound the kernel $K_1$ using the ellipticity condition on $K$:
	\begin{align*}
	K_1(t) &= |t|^{n-1} \int_{\R^{n-1}} K\left(t(\sigma,1)\right) \d\sigma \geq |t|^{n-1} \int_{\R^n} c_{n,\s} \frac{\lambda}{|t|^{n+2\s}(|\sigma|^2+1)^{\frac{n+2s}{2}}} \d\sigma \\
	&= c_{n,\s} \frac{\lambda}{|t|^{1+2\s}} \int_{\R^{n-1}} \frac{\d\sigma}{(|\sigma|^2+1)^{\frac{n+2\s}{2}}} = c_{n,\s} \frac{\lambda}{|t|^{1+2\s}} \dfrac{2 \pi^{\frac{n-1}{2}}}{\Gamma(\frac{n-1}{2})} \int_0^\infty \dfrac{r^{n-2}}{(r^2+1)^{\frac{n+2\s}{2}}} \d r \\	
	& = c_{n,\s} \frac{\lambda}{|t|^{1+2\s}} 
	\dfrac{\pi^{\frac{n-1}{2}} \Gamma(\frac{1}{2}+\s)}{\Gamma(\frac{n}{2}+\s)} 
	= c_{n,\s} \frac{\lambda}{|t|^{1+2\s}} \frac{c_{1,\s}}{c_{n,\s}} = c_{1,\s} \frac{\lambda}{|t|^{1+2\s}},
	\end{align*}
	where we have used \eqref{Eq:ConstantFracLaplacian} and the definition of the Beta and Gamma functions. The upper bound for $K_1$ is obtained in the same way.
	
	Finally, point $(iii)$ follows from noticing that the previous computation is an equality with $\lambda = 1$ in the case of the fractional Laplacian.
\end{proof}


Thanks to this result, we can now define the concept of layer solution.

\begin{definition}
	Let $u_0$ the unique solution of the following problem.
	\begin{equation}
		\label{Eq:LayerSolution}
			\beqc{\PDEsystem}
			L_{K_1}  u_0 &=& f(w) & \textrm{ in }\R\,,\\
			u_0 &>& 0 & \textrm{ in } (0,+\infty)\,,\\
			u_0(x) & = &-u_0(-x)  & \textrm{ in }\R\,,\\
			\ds \lim_{x \to \pm \infty} u_0(x) &=& \pm 1. & 
			\eeqc
	\end{equation}
	Then, if $K$ and $K_1$ are related as in Proposition~\ref{Prop:KernelsDimension}, we say that $u_0$ is the \emph{layer solution} associated to the problem $L_K u = f(u)$ in $\R^{2m}$.
	
	We also define the associated function
	\begin{equation}
		\label{Eq:DefOfU}
		U(x):= u_0 \left( \dfrac{|x'| - |x''|}{\sqrt{2}} \right)\,.
	\end{equation}
\end{definition}
Note that $(|x'| - |x''| )/\sqrt{2}$ is the signed distance to the Simons cone.

The existence of the layer solution was proved in \cite{CabreSolaMorales,CabreSireII} for the fractional Laplacian and later in \cite{CozziPassalacqua} for a general class of operators including the ones considered in this paper. 



In the proof of Theorem~\ref{Th:AsymptoticBehaviourSaddleSolution} we will use some properties of the layer solution, which are presented next. First, in \cite{CozziPassalacqua} it is proved that there exists a constant $C$ such that
\begin{equation}
	\label{Eq:PropertiesLayer}
	|u_0 (x)-\sign(x)| \leq C |x|^{-2\s}  \quad \text{ and } \quad |\dot{u}_0 (x)| \leq C |x|^{-1-2\s}  \quad \text{ for large }|x|.
\end{equation}
In our arguments we need also to show that the second derivative of the layer goes to zero at infinity. This is the first statement of the following lemma.

\begin{lemma}
	\label{Lemma:SecondDerivativeLayer}
	Let $u_0$ be the layer solution associated to the kernel $K_1$, that is, $u_0$ solving \eqref{Eq:LayerSolution}. Then, 
	\begin{enumerate}[label=(\roman{*})]
		\item $\ddot{u}_0 (x) \to 0$ as $x\to \pm \infty$.	
		\item  $\ddot{u}_0 (x) < 0$ in $(0,+\infty)$.
	\end{enumerate}
\end{lemma}

We prove here the first statement of this lemma, and we postpone the proof of the second one until the next section, since we need to use a maximum principle for the linearized operator $L_{K_1} - f'(u_0)$.

\begin{proof}[Proof of point (i) of Lemma~\ref{Lemma:SecondDerivativeLayer}]
	By contradiction, suppose that there exists an unbounded sequence $\{x_j\}$ satisfying $|u_0(x_j)|>\varepsilon$ for some $\varepsilon>0$. Note that by the symmetry of $u_0$ we may assume that $x_j\to + \infty$. Now define $w_j (x) := \ddot{u}_0(x+x_j)$. By differentiating twice the equation of the layer solution, we see that $\ddot{u}_0$ solves
	$$
	L_{K_1} \ddot{u}_0 = f''(u_0)\dot{u}_0^2 + f'(u_0)\ddot{u}_0 \text{ in }\R.
	$$
	Hence, as $x_j \to +\infty$ a standard compactness argument combined with \eqref{Eq:PropertiesLayer} yield that $w_j$ converges on compact sets to a function $w$ solution of
	$$
	L_{K_1}  w = f'(1)w \quad \text{ in }\R.
	$$
	In addition, since $|u_0(x_j)|>\varepsilon$ we have $|w(0)|\geq \varepsilon$.
	
	At this point we use Lemma~4.3 of \cite{CozziPassalacqua} to deduce that, since $f'(1)<1$, then $w\to 0$ as $|x| \to +\infty$. Therefore, if $w$ is not identically zero, it has either a positive maximum or a negative minimum, but this contradicts the maximum principle (recall that $f'(1)<1$). We conclude that $w\equiv0$ in $\R$, but this is a contradiction with $|w(0)|\geq \varepsilon$.
\end{proof}




Now we have all the ingredients to establish the asymptotic behavior of the saddle-solution.



\begin{proof}[Proof of Theorem~\ref{Th:AsymptoticBehaviourSaddleSolution}]
By contradiction, assume that the result does not hold. Then, there exists an $\varepsilon>0$ and a sequence $\{x_k\}$, such that
\begin{equation}
\label{Eq:ContradictionAsymptotic}
|u(x_k)-U(x_k)|+|\nabla u(x_k)-\nabla U(x_k)|+|D^2u(x_k)-D^2U(x_k)| > \varepsilon.
\end{equation}
By the symmetry of $u$, we may assume without loss of generality that $x_k \in \overline{\ocal}$, and by continuity we can further assume $ x_k \notin \ccal$. 

Let $d_k:=\dist(x_k,\ccal)$. We distinguish two cases:

\textbf{Case 1: $\{d_k\}$ is an unbounded sequence.} In this situation, we may assume that $d_k \geq 2k$. Define
$$
w_k(x) := u(x+x_k), 
$$
which satisfies $0<w_k<1$ in $\overline{B_k}$ and
$$
L_K w_k = f(w_k) \ \textrm{ in } B_k.
$$
By letting $k\to +\infty$, by the uniform estimates for the operators of the class $\lcal_0$ and the Arzelà-Ascoli theorem, we have that, up to a subsequence, $w_k$ converges on compact sets to a function $w$ which is a pointwise solution of
$$
\beqc{\PDEsystem}
L_K  w &=& f(w) & \textrm{ in }\R^n\,,\\
w &\geq& 0 & \textrm{ in } \R^n\,.
\eeqc
$$

By Theorem~\ref{Th:SymmetryWholeSpace}, either $w\equiv 0$ or $w\equiv 1$. First, note that $w$ cannot be zero. Indeed, since $w_k$ are stable with respect to perturbations supported in $B_k$ (see Remark~\ref{Remark:Stability}), $w$ is stable in $\R^n$, which means that the linearized operator $L_K-f'(w)$ is a positive operator. Nevertheless, if $w\equiv 0$, then the linearized operator $L_K-f'(w) = L_K-f'(0)$ is negative for sufficiently large balls, since $f'(0)>0$ and the first eigenvalue of $L_K$ is of order $R^{-2\s}$ in balls of radius $R$ (as in Lemma~\ref{Lemma:FirstOddEigenfunction}). Therefore $w\equiv 1$. 

On the other hand, since $d_k\rightarrow +\infty$ and $U(x_k) =  u_0(d_k)$, we get by the properties of the layer solution that $U(x_k) \rightarrow 1$, $\nabla U(x_k) \rightarrow 0$ and $D^2U(x_k) \rightarrow 0$ ---see \eqref{Eq:PropertiesLayer} and Lemma~\ref{Lemma:SecondDerivativeLayer}. From this and condition \eqref{Eq:ContradictionAsymptotic} we get
$$
|u(x_k)-1|+|\nabla u(x_k)|+|D^2u(x_k)| > \varepsilon/2,
$$
for $k$ big enough. This yields that 
$$
|w_k(0)-1|+|\nabla w_k(0)|+|D^2w_k(0)| > \varepsilon/2,
$$
and this contradicts $w \equiv 1$. 

\textbf{Case 2: $\{d_k\}$ is a bounded sequence.}
In this situation, at least for a subsequence, we have that $d_k \rightarrow d$. Now, for each $x_k$ we define $x_k^0$ as its projection on $\ccal$. Therefore, we have that $ \nu_k^0 := (x_k-x_k^0)/d_k$ is the unit normal to $\ccal$. Through a subsequence, $ \nu_k^0 \rightarrow \nu$ with $|\nu|=1$.

We define
$$ w_k (x) := u(x+x_k^0), $$
which solves
$$ L_K  w_k = f(w_k) \ \text{in } \R^n. $$
Similarly as before, by letting $k\to +\infty$, up to a subsequence $w_k$ converges on compact sets to a function $w$ which is a pointwise solution to
$$
\beqc{\PDEsystem}
%L w &=& f(w)  \textrm{ in } &H:=\{x\cdot \nu >0\}\,,\\
%w &\geq& 0  \textrm{ in } &H\,,\\
%w &\text{ odd}& \hspace{-2mm} \text{with respect} \text{to} &H \,.
L_K  w &=& f(w)  &\textrm{ in } H:=\{x\cdot \nu >0\}\,,\\
w &\geq& 0  &\textrm{ in } H\,,\\
\,\,w \text{ odd with respect to } H. \span\span\span \,
\eeqc
$$
For the details about the fact that $\ocal \rightarrow H$, see \cite{CabreTerraI}.

As in the previous case, by stability $w$ cannot be zero, and then, by Theorem \ref{Th:SymmHalfSpace}, $w$ only depends on $x\cdot \nu$ and is increasing. Therefore, by the uniqueness of the layer solution, $w(x) = u_0(x\cdot \nu)$ and
\begin{align*}
u(x_k) &= w_k(x_k-x_k^0) = w(x_k-x_k^0) + \mathrm{o}(1) \\
&= u_0((x_k-x_k^0)\cdot \nu) + \mathrm{o}(1) \\
&= u_0((x_k-x_k^0)\cdot \nu_k^0) + \mathrm{o}(1) \\
&= u_0(d_k |\nu_k^0|^2) + \mathrm{o}(1) \\
&= u_0(d_k) + \mathrm{o}(1) = U (x_k) + \mathrm{o}(1),
\end{align*}
contradicting \eqref{Eq:ContradictionAsymptotic}. The same is done for $\nabla u$ and $D^2 u$.
\end{proof}

\begin{remark}
	\label{Remark:u>delta}
	The previous result yields that, for small $\delta$, the level set $\{0<u<\delta\}$ of the saddle-shaped solution is contained in a $\varepsilon$-neighborhood of the cone. Indeed, consider the set $\ocal_\varepsilon := \{(x',x'')\in \R^m\times\R^m \ : \ |x''|+\varepsilon <|x'| \}$. Since $u_0$ is monotone, for every small $\delta$ we can take $\varepsilon$ small enough and $R$ big enough so that $U(x)\geq 2\delta$ in $\ocal_\varepsilon \setminus B_R$. Thus, by taking $R$ big enough so that $|u-U|< \delta$ in $\ocal \setminus B_R$, we have that $u(x) \geq U(x)-\delta \geq \delta$ whenever $x\in \ocal_\varepsilon \setminus B_R$. Finally, since $\overline{B_R \cap \ocal}$ is a compact and $u$ is positive there, we can take $\varepsilon$ small enough so that $u\geq \delta$ in $B_R\cap \ocal_\varepsilon$.
\end{remark}

%%%%%%%%%%%%%%%%%%%%%%%%%%%%%
\section{Maximum principles for the linearized operator and uniqueness of the saddle-shaped solution}
%%%%%%%%%%%%%%%%%%%%%%%%%%%%%%
\label{Sec:MaximumPrinciple}

In this section we show that the linearized operator $L_K  -f'(u)$ satisfies the maximum principle in $\ocal$. This result combined with the asymptotic result of Theorem~\ref{Th:AsymptoticBehaviourSaddleSolution} yields the uniqueness of the saddle-shaped solution.

The maximum principle we establish is the following.

\begin{proposition}
	\label{Prop:MaximumPrincipleLinearized}
	Let $m\geq 1$, $\gamma \in (0,1)$, $\alpha > 2\gamma$ and let $v\in C^\alpha_{\mathrm{loc}}(\R^{2m})\cap L^\infty(\R^{2m})$ be a doubly radial function. Let $\Omega \subset \ocal$ a domain (not necessarily bounded) and let $L_K  \in \lcal_\star$. Assume that $v$ satisfies
	$$
	\beqc{\PDEsystem}
	L_K v - f'(u)v - c(x)v &\leq & 0 &\textrm{ in } \Omega\,,\\
	v &\leq & 0 &\textrm{ in } \ocal \setminus \Omega\,,\\
	- v(x^\star) & = & v(x) &\textrm{ in } \R^{2m},\\
	\ds \limsup_{x\in \Omega, \ |x|\to \infty} v(x) &\leq & 0\,,
	\eeqc
	$$
	with $c\leq 0$ in $\Omega$.
	Then, $v \leq 0$ in $\Omega$.
\end{proposition}

In order to prove this result we need a maximum principle in narrow domains, stated next.

\begin{proposition}
	\label{Prop:MaximumPrincipleNarrowDomainsOdd}
	Let $m\geq 1$, $\gamma \in (0,1)$, $\alpha > 2\gamma$ and let $v\in C^\alpha_{\mathrm{loc}}(\R^{2m})\cap L^\infty(\R^{2m})$ be a doubly radial function. Let $\varepsilon>0$ and let
	$$
	H \subset \{(x',x'')\in \R^m\times\R^m \ : \ |x''|<|x'|<|x''|+ \varepsilon\} \subset \ocal
	$$ 
	be a domain (not necessarily bounded). Let $L_K  \in \lcal_\star$ and assume that $v$ satisfies
	\begin{equation}
	\label{Eq:AssumptionsMaxPNarrow}
	\beqc{\PDEsystem}
	L_K v + c(x)v&\leq & 0 &\textrm{ in } H\,,\\
	v &\leq & 0 &\textrm{ in } \ocal \setminus H\,,\\
	- v(x^\star) & = & v(x) &\textrm{ in } \R^{2m},\\
	\ds \limsup_{x\in H, \ |x|\to \infty} v(x) &\leq & 0\,.
	\eeqc
	\end{equation}
	Under these assumptions there exists $\overline{\varepsilon}>0$ depending only on $\lambda, m, \s$ and $||c_-||_{L^\infty}$ such that, if $\varepsilon<\overline{\varepsilon}$, then $v \leq 0$ in $H$.
\end{proposition}

\begin{proof}
	Assume, by contradiction, that
	$$
	M := \sup_H v > 0\,.
	$$
	Under the assumptions \eqref{Eq:AssumptionsMaxPNarrow}, $M$ must be attained at an interior point $x_0 \in H$, that we can assume without loss of generality that is of the form $x_0 = (|x_0'|e,|x_0''|e)$, with $e=(1,0,...,0)\in\R^m$. Then,
	\begin{equation}
	\label{Eq:InequalitiesMaxPNarrowProof}
	0 \geq L_K  v(x_0) + c(x_0)v(x_0) \geq L_K  v(x_0) - \norm{c_-}_{L^\infty(H)}M\,.
	\end{equation} 
	Now, we compute $L_K  v(x_0)$. Since $v$ is doubly radial and odd with respect to the Simons cone, we can use the expression \eqref{Eq:OperatorOddF} to write
	\begin{align*}
	L_K v(x_0) &= \int_{\ocal} \big (M - v(y) \big) \big (\overline{K}(x_0,y) -\overline{K}(x_0,y^\star)\big) \d y + 2M\int_{\ocal} \overline{K}(x_0,y^\star)\d y\\
    &\geq2M \int_{\ocal} \overline{K}(x_0,y^\star)\d y,
	\end{align*}
    where the inequality follows from being $M$ the supremum of $v$ in $\ocal$ and the kernel inequality \eqref{Eq:KernelInequality}. Combining this last inequality with \eqref{Eq:InequalitiesMaxPNarrowProof}, we obtain
	$$
	0 \geq L_K  v(x_0) + c(x_0)v(x_0)  \geq M \left\{ 2 \int_{\ocal} \overline{K}(x_0,y^\star)\d y - \norm{c_-}_{L^\infty(H)}
	\right\}\,.
	$$
	
	Finally, if we use the lower bound of the integral term from Lemma~2.3. in \cite{FelipeSanz-Perela:IntegroDifferentialI} and the fact that $\dist(x_0,\ccal) \leq \varepsilon/\sqrt{2}$, we get
	\begin{align*}
	0 &\geq M \left\{ 2 \int_{\ocal} \overline{K}(x_0,y^\star)\d y - \norm{c_-}_{L^\infty(H)}
	\right\} \geq M \left(\frac{1}{C}\dist(x_0,\ccal)^{-2\s}-\norm{c_-}_{L^\infty(H)}\right) \\ &\geq M \left(\frac{1}{C}\varepsilon^{-2\s}-\norm{c_-}_{L^\infty(H)}\right).
	\end{align*}
	Therefore, for $\varepsilon$ small enough, we arrive at a contradiction that comes from assuming that the supremum is positive.
\end{proof}


Once this maximum principle in narrow domains is available, we can now proceed with the proof of Proposition~\ref{Prop:MaximumPrincipleLinearized}.

\begin{proof}[Proof of Proposition~\ref{Prop:MaximumPrincipleLinearized}]
    


	For the sake of simplicity, we will denote 
	$$
	\mathscr{L} w := L_K w - f'(u)w - cw\,.
	$$
	The crucial point in this proof is the fact that $u$ is a positive supersolution of the operator $\mathscr{L}$. Indeed, by \eqref{Eq:PropertyConcavityf} we get
	\begin{equation}
	\label{Eq:uSupersolLinearized}
	\mathscr{L} u = L_K u - f'(u)u - cu \geq f(u) - f'(u)u > 0 \quad \textrm{ in } \Omega \subset \ocal\,,
	\end{equation}
	where in the first inequality we have used that $u>0$ in $\ocal$ and that $c\leq 0$.
	
	Let $\varepsilon > 0$ be such that the maximum principle of Proposition~\ref{Prop:MaximumPrincipleNarrowDomainsOdd} is valid and define the following sets:
	$$
	\Omega_\varepsilon := \Omega \cap \{|x'| > |x''| + \varepsilon\}\quad \textrm{ and } \quad 
	\ncal_\varepsilon := \Omega \cap \{|x''| < |x'| < |x''| + \varepsilon\}\,.
	$$
	
	
    
    By contradiction, assume that there exists $x_0\in \Omega$ such that $v(x_0)> 0$.
	Set $w := v - \tau u$. By the asymptotic result, we have 
	\begin{equation}
		\label{Eq:u>delta}
		u \geq \delta > 0 \quad \textrm{ in } \overline{\Omega}_\varepsilon\,,
	\end{equation}
	for some $\delta >0$ (see Remark~\ref{Remark:u>delta}). Therefore,  $w < 0$ in $\overline{\Omega}_\varepsilon$ if $\tau$ is big enough,. Moreover, since $v\leq 0$ in $\ocal\setminus\Omega$, we have 
	$$
	w \leq 0 \quad \textrm{ in } \ocal \setminus \ncal_\varepsilon\,.
	$$
	Furthermore, we also have
	$$
	\limsup_{x\in \ncal_\varepsilon, \ |x|\to \infty} w(x) \leq 0
	$$
	and, by \eqref{Eq:uSupersolLinearized},
	$$
	\mathscr{L} w = \mathscr{L} v - \tau \mathscr{L} u \leq 0 \textrm{ in } \ncal_\varepsilon\,.
	$$
	Thus, since $w$ is odd with respect to $\ccal$, we can apply Proposition~\ref{Prop:MaximumPrincipleNarrowDomainsOdd} with $H = \ncal_\varepsilon$ to deduce that
	$$
	w \leq 0 \quad \textrm{ in } \Omega\,,
	$$
	if $\tau$ is big enough.
	
	Now, define 
	$$
	\underline{\tau}:= \inf \setcond{\tau > 0}{v - \tau u \leq 0 \ \textrm{ in } \Omega}.
	$$
	By the previous reasoning, $\underline{\tau}$ is well defined. Clearly, $v - \underline{\tau} u \leq 0 $ in $\Omega$. In addition, since $v(x_0)>0$, we have $-\underline{\tau} u(x_0) < v(x_0) - \underline{\tau} u (x_0) \leq 0$ and therefore, since $u(x_0)>0$, we deduce that  $\underline{\tau} > 0$.
	
	We claim that $v - \underline{\tau} u \not \equiv 0$. Indeed, if $v - \underline{\tau} u \equiv 0$ then $v = \underline{\tau} u$ and thus, since $\underline{\tau} > 0$, we get 
	$$
	0 \geq \mathscr{L} v(x_0) = \underline{\tau} \mathscr{L} u(x_0) > 0\,, 
	$$
	which is a contradiction.
	
	Then, since $v - \underline{\tau} u \not \equiv 0$, the strong maximum principle (Proposition~\ref{Prop:StrongMaximumPrincipleForOddFunctions}) yields
	$$
	v - \underline{\tau} u < 0 \quad \textrm{ in }\Omega\,.
	$$
	Therefore, by continuity, the assumption on $v$ at infinity and \eqref{Eq:u>delta}, there exists $0 < \eta <\underline{\tau}$ such that 
	$$
	\tilde{w} := v - (\underline{\tau} - \eta) u < 0 \quad \textrm{ in }\overline{\Omega}_\varepsilon\,.
	$$
	Using again the maximum principle in narrow domains with $\tilde{w}$ in $\ncal_\varepsilon$, we deduce that 
	$$
	v - (\underline{\tau} - \eta) u \leq 0 \quad \textrm{ in }\Omega\,,
	$$
	and this contradicts the definition of $\underline{\tau}$. Hence, $v\leq 0$ in $\Omega$.
\end{proof}


The same argument used in the previous proof can be used to establish the remaining statement of Lemma~\ref{Lemma:SecondDerivativeLayer}.

\begin{proof}[Proof of point (ii) of Lemma~\ref{Lemma:SecondDerivativeLayer}]
	
	Let $v = \ddot{u}_0$. First we show that $v\leq 0$ in $(0,+\infty)$. To see this, note that since $f$ is concave and by point (i) of Lemma~\ref{Lemma:SecondDerivativeLayer}, it follows that
	$$
	\beqc{\PDEsystem}
	L_{K_1} v - f'(u_0)v &\leq &0 & \text{ in } (0,+\infty)\,.\\
	v(x) &= &-v(-x) & \text{ for every } x\in \R\,,\\
	\ds \limsup_{x\to +\infty} v(x) &= & 0\,.
	\eeqc
	$$
	Now, we follow the proof of Proposition~\ref{Prop:MaximumPrincipleLinearized} but with the previous problem, replacing $u$ by $u_0$ and using that
	$$
	L_{K_1} u_0 - f'(u_0)u_0 > 0 \quad \text{ in } (0,+\infty)\,. 
	$$
	All the arguments are the same and yield that $v\leq 0$ in $(0,+\infty)$.
	
	The fact that the inequality is strict follows from the strong maximum principle for odd functions in $\R$, as follows. Suppose by contradiction that there exists a point $x_0\in (0,+\infty)$ such that $v(x_0) = 0$. Then,
	$$
	0 \geq L_{K_1} v (x_0) = - \int_{-\infty}^{+\infty}	v(y) K_1(x_0 - y) \d y = - \int_{-\infty}^{+\infty} v(y) \{ K_1(x_0 - y) - K_1(x_0 + y)\} \d y > 0\,,$$
	arriving at a contradiction. Here we have used that $v\not \equiv 0$ and point $(ii)$ from Proposition~\ref{Prop:KernelsDimension}, which yields $K_1(x - y) \geq  K_1(x + y)$ for every $x>0$ and $y>0$.
\end{proof}






With these ingredients available, we can finally establish the uniqueness of the saddle-shaped solution.



\begin{proof}[Proof of Theorem~\ref{Th:Uniqueness}]
	Let $u_1$ and $u_2$ be two saddle-shaped solutions. Define $v := u_1 - u_2$ which is a doubly radial function that is odd with respect to $\ccal$. Then,
	$$
	L_K v = f(u_1) - f(u_2) \leq f'(u_2) (u_1 - u_2) = f'(u_2) v \quad \textrm{ in } \ocal\,,
	$$
	since $f$ is concave in $(0,1)$. Moreover, by the asymptotic result (see Theorem~\ref{Th:AsymptoticBehaviourSaddleSolution}), we have
	$$
	\limsup_{x\in \ocal, \ |x|\to \infty} v(x) = 0\,.
	$$
	Then, by the maximum principle in $\ocal$ for the linearized operator $L_K  - f'(u_2)$ (see Proposition~\ref{Prop:MaximumPrincipleLinearized}), we are lead to $v \leq 0$ in $\ocal$, which means $u_1 \leq u_2$ in $\ocal$. Repeating the  argument with $-v = u_2 - u_1$ we deduce $u_1 \geq u_2$ in $\ocal$. Therefore, $u_1 = u_2$ in $\R^{2m}$.
\end{proof}






%%%%%%%%%%%%%%%%%%%%%%%%%%%%%%%%%%%%%%%%%%%%%%%%%%%%%%%%%%%%%%%%%%%%%%%%%%%%
%%%%%%%%%%%%%%%%%%%%%%%%%%%%%%%%%%%%%%%%%%%%%%%%%%%%%%%%%%%%%%%%%%%%%%%%%%%%
\section*{Acknowledgements}

The authors thank Xavier Cabré for his guidance and useful discussions on the topic of this paper. We also thank Matteo Cozzi for interesting discussions on some parts of this article.



%%%%%%%%%%%%%%%%%%%%%%%%%%%%%%%%%%%%%%%%%%%%%%%%%%%%%%%%%%%%%%%%%%%%%%%%%%%%
%%%%%%%%%%%%%%%%%%%%%%%%%%%%%%%%%%%%%%%%%%%%%%%%%%%%%%%%%%%%%%%%%%%%%%%%%%%%
\bibliographystyle{amsplain}
\bibliography{biblio}

\end{document}
that
