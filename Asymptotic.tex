%%%%%%%%%%%%%%%%%%%%%%%%%%%%%
\section{Asymptotic results}
\label{Sec:Asymptotic}
%%%%%%%%%%%%%%%%%%%%%%%%%%%%%











Layer solution


\begin{proposition}
	\label{Prop:KernelsDimension}
	Let $L$ and $\tilde{L}$ be two symmetric and translation invariant integral operators with kernels $K$ and $\tilde{K}$ in dimensions $n$ and $1$ respectively. Let also be $u:\R^n\to\R$ and $v:\R\to\R$ such that $u(x) = v(x_n)$.
	
	If,
	$$ \tilde{K}(t) = |t|^{n-1} \int_{\R^n} K\left(t(\sigma,1)\right) d\sigma, $$
	then
	\begin{enumerate}
		\item[(i)] $Lu(x) = \tilde{L}v(x_n)$,
		\item[(ii)] $L\in \mathcal{L}_0 (n,s,\lambda,\Lambda) \Rightarrow \tilde{L}\in \mathcal{L}_0 (n,s,\lambda,\Lambda)$.
	\end{enumerate}
\end{proposition}

Note that if $L$ is the fractional Laplacian in dimension $n$, then $\tilde{L}$ is also the fractional Laplacian, but in dimension $1$.

\begin{proof}
	\begin{enumerate}
		\item[(i)]
		\begin{align*}
		Lu(x) &= \int_{\R^n} \left\{ u(x)-u(y) \right\} K(x-y) dy \\
		&=\int_{\R^n} \left\{ v(x_n)-v(y_n) \right\} K\left((x'-y',x_n-y_n\right) dy'dy_n.
		\end{align*}
		Now we make the change of variables $y' = x'-(x_n-y_n)\sigma$. That is,
		\begin{align*}
		Lu(x) &= \int_{\R} \left\{ v(x_n)-v(y_n) \right\} \int_{\R^{n-1}} K\left((\sigma(x_n-y_n),x_n-y_n\right)) |x_n-y_n|^{n-1} d\sigma dy_n \\
		&= \int_{\R} \left\{ v(x_n)-v(y_n) \right\} |x_n-y_n|^{n-1} \int_{\R^{n-1}} K\left((x_n-y_n)(\sigma,1)\right) d\sigma dy_n \\
		&= \int_{\R} \left\{ v(x_n)-v(y_n) \right\} \tilde{K}(x_n-y_n dy_n = \tilde{L}v(x_n). \\
		\end{align*}
		\item[(ii)] We are only going to prove one of the bounds, since they are in fact the same. That is,
		\begin{align*}
		\tilde{K}(t) &= |t|^{n-1} \int_{\R^n} K\left(t(\sigma,1)\right) d\sigma \geq |t|^{n-1} \int_{\R^n} c_{n,s} \frac{\lambda}{|t|^{n+2s}(\sigma^2+
			1)^{\frac{n+2s}{2}}} d\sigma \\
		&= c_{n,s} \frac{\lambda}{|t|^{1+2s}} \int_{\R^n} \frac{d\sigma}{(\sigma^2+1)^{\frac{n+2s}{2}}} = c_{n,s} \frac{\lambda}{|t|^{1+2s}} \frac{c_{1,s}}{c_{n,s}} \\
		&= c_{1,s} \frac{\lambda}{|t|^{1+2s}},
		\end{align*}
		where the only inequality that appears comes from the lower bound of the kernel $K$ for being in $\mathcal{L}_0$.
	\end{enumerate}
\end{proof}







\begin{theorem}
\label{Thm:AsymptoticBehaviourSaddleSolution}
Let $f\in \cp{2,\alpha}(\R)$ of the Allen-Cahn type. Let $u$ be a saddle-shaped solution of
$$
Lu = f(u) \ \textrm{in } \R,
$$
with $L\in \lcal_\star$.

Then,
$$
\norm{u-\mathcal{U}}_{L^\infty(\R^n\setminus B_R)}+\norm{\nabla u-\nabla \mathcal{U}}_{L^\infty(\R^n\setminus B_R)}+\norm{D^2u-D^2\mathcal{U}}_{L^\infty(\R^n\setminus B_R)} \xrightarrow{\text{as } R\to \infty}{} 0,
$$
where
$$
\mathcal{U}(x) = u_0(z),
$$
with $z$ the distance to the cone and $u_0$ the layer 1D solution.
\end{theorem}

\begin{proof}
By contradiction, assume that the result does not hold. Then, there exists $\epsilon>0$ and a sequence $\{x_k\}$, that we may assume without loss of generality in $\overline{\ocal}$ and by continuity out of ${\ccal}$, such that
\begin{equation}
\label{Eq:ContradictionAsymptotic}
|u(x_k)-\mathcal{U}(x_k)|+|\nabla u(x_k)-\nabla \mathcal{U}(x_k)|+|D^2u(x_k)-D^2\mathcal{U}(x_k)| > \epsilon.
\end{equation}


We distinguish two cases:

CASE 1. $\{d_k:=\dist(x_k,\ccal)\}$ is an unbounded sequence. We may assume $d_k \geq 2k$.\\
In such case, we define
$$
w_k(x) = u(x+x_k)  \ \textrm{for } x\in B_{d_k}(0), 
$$
which satisfies $0<w_k<1$ and
$$
Lw_k = f(w_k) \ \textrm{in } B_k.
$$
By letting $k$ tend to infinity and using compactness \todo{Atencion} we have that $w_k$ converges to $w$, up to a subsequence, satisfying
$$
\beqc{\PDEsystem}
L w &=& f(w) & \textrm{ in }\R^n\,,\\
w &\geq& 0 & \textrm{ in } \R^n\,.
\eeqc
$$

By Theorem~\ref{Thm:SymmetryWholeSpace}, $w\equiv 0$ or $w\equiv 1$. First, $w$ cannot be zero. Indeed, since $w_k$ are stable in $B_k$, $w$ is stable in $\R^n$, which means that the linear operator $L-f'(w)$ is a positive operator. Nevertheless, if $w\equiv 0$, the linear operator $L-f'(w) = L-f'(0)$ is negative for sufficiently large balls, since $f'(0)>0$ and the first eigenvalue of $L$ is of order $R^{-2\s}$ in balls of radius $R$. Therefore $w\equiv 1$. 

On the other hand, since $d_k\rightarrow \infty$  and $\mathcal{U}(x_k) = u_0(z_k) = u_0(d_k)$ we get by the properties of the layer solution \todo{ver bien} that $\mathcal{U}(x_k) \rightarrow 1$, $\nabla \mathcal{U}(x_k) \rightarrow 0$ and $D^2\mathcal{U}(x_k) \rightarrow 0$. From this and condition \eqref{Eq: ContradictionAsymptotic} we get
$$
|u(x_k)-1|+|\nabla u(x_k)|+|D^2u(x_k)| > \epsilon/2,
$$
which means that $w \not\equiv 1$. Therefore we arrive to a contradiction with Theorem \ref{Thm:SymmetryWholeSpace}.

CASE 2. $\{d_k:=\dist(x_k,\ccal)\}$ is a bounded sequence.
Then, at least for a subsequence $d_k \rightarrow d$. Now, define for each $x_k$ its projection on $\ccal$, $x_k^0$ and then we have that $ \nu_k^0 := \frac{x_k-x_k^0}{d_k}$ is the unit normal to $\ccal$. Through a subsequence, $ \nu_k^0 \rightarrow \nu$ with $|\nu|=1$.

Define as before
$$ w_k (x) = u(x+x_k^0), $$
which solves
$$ L w_k = f(w_k) \ \text{in } \R^n. $$
By letting $k$ tend to infinity and using compactness \todo{Atencion} we have that $w_k$ converges to $w$, up to a subsequence, satisfying
$$
\beqc{\PDEsystem}
%L w &=& f(w)  \textrm{ in } &H:=\{x\cdot \nu >0\}\,,\\
%w &\geq& 0  \textrm{ in } &H\,,\\
%w &\text{ odd}& \hspace{-2mm} \text{with respect} \text{to} &H \,.
L w &=& f(w)  &\textrm{ in } H:=\{x\cdot \nu >0\}\,,\\
w &\geq& 0  &\textrm{ in } H\,,\\
\,\,w \text{ odd with respect to } H. \span\span\span \,
\eeqc
$$
For the detail of the proof of how $\ocal \rightarrow H$ see \cite{CabreTerra1}.

As in the previous case, by stability $w$ cannot be zero, and then, by Theorem \ref{Thm:SymmHalfSpace}, $w$ only depends on $x\cdot \nu$ and is increasing. Therefore, by the uniqueness of the layer solution, $w(x) = u_0(x\cdot \nu)$ and
\begin{align*}
u(x_k) &= w_k(x_k-x_k^0) = w(x_k-x_k^0) + o(1) \\
&= u_0((x_k-x_k^0)\cdot \nu) + o(1) \\
&= u_0((x_k-x_k^0)\cdot \nu_k^0) + o(1) \\
&= u_0(d_k |\nu_k^0|^2) + o(1) \\
&= u_0(d_k) + o(1) = \ucal (x_k) + o(1),
\end{align*}
contradicting \eqref{Eq:ContradictionAsymptotic}. The same is done for $\nabla u$ and $D^2 u$.
\end{proof}