\documentclass[12pt]{report}


\usepackage{amsmath} %General package for maths (loads also amsbsy to make bold symbols)
\usepackage{amsthm} %Package for theorems
\usepackage{amssymb} %Symbols (loads also amsfonts)


% Title Page
\title{}
\author{}


\begin{document}
	
	\begin{center}
		\textbf{	Comments Revision JFA-19-244}
	\end{center}

With respect to the last version of the paper (after the referee revision), we have performed two minor modifications in our manuscript. They are explained next.

During the last revision, we realized that some comments in Remark~3.4 could be somewhat misleading and we decided to rewrite it in a more clear manner. The issue concerns estimates (3.8) and (3.9) in the previous version of the document. Such estimates require the Lipschitz regularity on the kernel of the operator $L_K$ in order to hold. Although this regularity is fulfilled by the operators that we consider in this paper (that is, satisfying the convexity assumption (1.12) apart from the ellipticity), we had not specified this in such remark. We have amended this modifying the whole Remark~3.4, devoting some lines to the regularity of the kernel $K$. In addition, for the sake of simplicity, we have removed estimate (3.8), which was not needed (the previous estimate (3.6) suffices for our purposes).

Moreover, we also realized that the regularity of the kernel was not specified in Theorems 1.3 and 1.4, where we only assumed the positivity condition (1.13). Instead of adding a Lipschitz regularity assumption on the kernel, we have preferred to change the hypotheses, replacing the positivity condition (1.13) by the convexity assumption (1.12) ---as mentioned above, (1.12) yields the required regularity of the kernel. In order to clarify this, we have added a footnote below  Theorem 1.3, where we explain that the results also hold when, instead of the convexity condition (1.12), we assume the positivity condition (1.13) plus some regularity on the kernel. Note that, as it is explained in the paper, we believe that in fact conditions (1.12) and (1.13) are equivalent, although we are only able to prove it when the kernel is a $C^2$ function (see Theorem 1.1).  
\end{document}          

