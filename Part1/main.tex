\documentclass[12pt,reqno]{amsart}

%%%%%%%%%%%%%%%%%%%%%%%%%%%%%%%%%%%%%%%%%%%%%%%%%%%%%%%%%%%%%%%%%%%%%%%%%%%%%%%
%%%%%%%%%%%   PACKAGES %%%%%%%%%%%%%%%%%%%%%%%%%%%%%%%%%%%%%%%%%%%%%%%%%%%%%%%%
%%%%%%%%%%%%%%%%%%%%%%%%%%%%%%%%%%%%%%%%%%%%%%%%%%%%%%%%%%%%%%%%%%%%%%%%%%%%%%%

\usepackage[utf8]{inputenc}

%%---AMS packages---%% (automatically loaded in amsart or amsbook)
\usepackage{amsmath} %General package for maths (loads also amsbsy to make bold symbols)
\usepackage{amsthm} %Package for theorems
\usepackage{amssymb} %Symbols (loads also amsfonts)
\usepackage{amscd} %Package for rectangular diagrams
%\usepackage{amsrefs} %Gestió de referències


\usepackage{emptypage}
%\usepackage{pdfpages} %Include pdf documents

%\usepackage{dsfont} % Double stroke roman fonts
\usepackage{enumitem} %Enumerations
\usepackage{mathtools} %More symbols, etc.
\usepackage{mathrsfs} %Calligraphic symbols with \mathscr
%\usepackage{fancyhdr} %Customize the header
\usepackage{hyperref} %Makes references hyperlinks
\hypersetup{
	%true = make page number, not text, be link
    linktocpage = true,
    %set to all if you want both sections and subsections linked
    %linktoc=all,
    colorlinks=true,
    linkcolor=red,
}
\usepackage{esint} %To write averaged integrals

%%%FIGURES AND IMAGES
\usepackage{graphicx} %For using \includegraphics
%\usepackage{wrapfig} %For figures wrapped inside the text
%\usepackage{float} %Interface for defining floating objects
%\usepackage{caption} %Customize the captions in floating environments
%\usepackage{subfigure}
\usepackage{subcaption}
%\usepackage{tcolorbox} %For boxing
%\tcbuselibrary{theorems}
%\tcbuselibrary{skins}

\usepackage{pgf,tikz,pgfplots}
\usetikzlibrary{arrows}
\usetikzlibrary{intersections}
\usetikzlibrary{fadings}


%%%EDITION HELP
\usepackage[colorinlistoftodos]{todonotes} %Package to add TODO notes in colors
\setlength{\marginparwidth}{2.3cm} % make todonotes appear wider
%\usepackage{showlabels} %Show the labels of equations and theorems
%\usepackage{showkeys} %Show the keys of all references
\usepackage{refcheck}


%%%%%%%%%%%%%%%%%%%%%%%%%%%%%%%%%%%%%%%%%%%%%%%%%%%%%%%%%%%%%%%%%%%%%%%%%%%%%%%
%%%%%%%%%%%   THEOREM ENVIRONMENTS %%%%%%%%%%%%%%%%%%%%%%%%%%%%%%%%%%%%%%%%%%%%
%%%%%%%%%%%%%%%%%%%%%%%%%%%%%%%%%%%%%%%%%%%%%%%%%%%%%%%%%%%%%%%%%%%%%%%%%%%%%%%

\newtheorem{theorem}{Theorem}[section]
\newtheorem*{theorem*}{Theorem}
\newtheorem{proposition}[theorem]{Proposition}
\newtheorem{lemma}[theorem]{Lemma}
\newtheorem{corollary}[theorem]{Corollary}
\newtheorem{conjecture}[theorem]{Conjecture}

\theoremstyle{definition}
\newtheorem{definition}[theorem]{Definition}
\newtheorem{example}[theorem]{Example}

\theoremstyle{remark}
\newtheorem{remark}[theorem]{Remark}
\newtheorem*{remark*}{Remark}
\newtheorem{observation}[theorem]{Observation}

%%%%%%%%%%%%%%%%%%%%%%%%%%%%%%%%%%%%%%%%%%%%%%%%%%%%%%%%%%%%%%%%%%%%%%%%%%%%%%%
%%%%%%%%%%%   MACROS/ABREVIATIONS  %%%%%%%%%%%%%%%%%%%%%%%%%%%%%%%%%%%%%%%%%%%%
%%%%%%%%%%%%%%%%%%%%%%%%%%%%%%%%%%%%%%%%%%%%%%%%%%%%%%%%%%%%%%%%%%%%%%%%%%%%%%%

\newcommand{\con}[1]{\mathbb{#1}}
\newcommand{\C}{\con{C}} %Complex
\newcommand{\R}{\con{R}} %Real
\newcommand{\Q}{\con{Q}} %Rational
\newcommand{\Z}{\con{Z}} %Integer
\newcommand{\N}{\con{N}} %Natural
\newcommand{\T}{\con{T}} %Torus
\newcommand{\Sph}{\con{S}} %Sphere
\renewcommand{\H}{\con{H}}
\newcommand{\ccal}{\mathscr{C}}
\newcommand{\ecal}{\mathcal{E}}
\newcommand{\ical}{\mathcal{I}}
\newcommand{\lcal}{\mathcal{L}}
\newcommand{\ncal}{\mathcal{N}}
\newcommand{\ocal}{\mathcal{O}}
\newcommand{\ucal}{\mathcal{U}}
\newcommand{\e}{\mathrm{e}}


\makeatletter
\newcommand{\leqnomode}{\tagsleft@true\let\veqno\@@leqno}
\newcommand{\reqnomode}{\tagsleft@false\let\veqno\@@eqno}
\makeatother

\newcommand{\limn}{\displaystyle\lim_{n \rightarrow \infty}} %limit n to infinity
\newcommand{\norm}[1]{\left | \left |{#1} \right | \right |}
\newcommand{\seminorm}[1]{\left [ {#1} \right ] }
\newcommand{\laplacian}{\Delta}
\newcommand{\s}{\gamma}
\newcommand{\fraclaplacian}{(-\Delta)^\s}
\newcommand{\Lr}{\mathcal{L}_{\mathrm{r}}}
\newcommand{\loc}{\mathrm{loc}}
\renewcommand{\d}{\,\mathrm{d}} %straight d with small space before
\newcommand{\dx}{\,\mathrm{d}x} %The usual differential of x
\newcommand{\ddt}[1][]{\dfrac{\d{#1}}{\d t}} %Derivative with respect t
\newcommand{\ddtpar}[1]{ \dfrac{\d}{\d t}\Big ( {#1} \Big)} %Derivative with respect t with parenthesis
\newcommand{\lp}[1]{\mathrm{L}^{#1}}%L^p spaces
\newcommand{\hp}[1]{\mathrm{H}^{#1}}%H^p spaces
\newcommand{\sobolev}[1]{\mathrm{W}^{#1}}%Sobolev spaces
\newcommand{\cp}[1]{\mathcal{C}^{#1}}%C^p spaces
\newcommand{\bpar}[1]{\left ( {#1}\right )}
\newcommand{\setcond}[2]{\left \{ #1 \ : \ #2  \right \}}
\newcommand\evalat[1]{_{\mkern1.5mu\big\vert_{\scriptstyle #1}}}
\newcommand{\normLinf}[1]{\left | \left |{#1} \right | \right |_{\lp{\infty}}}
%\newcommand{\averageline}{{\mathchoice
%{\kern1ex\vcenter{\hrule height.4pt width 6pt depth0pt} \kern-9.3pt} {}
%{\kern1ex\vcenter{\hrule height.4pt width 4.3pt depth0pt} \kern-7pt}
%{} {} }}
%\newcommand{\average}{\averageline \!\int}

\newcommand{\average}{\fint}

\newcommand\beqc[1]{\left\{\begin{array}{#1}}
\newcommand\eeqc{\end{array} \right.}
\def\PDEsystem{rcll}
\def\bmatrix{\begin{pmatrix}}
\def\ematrix{\end{pmatrix}}
\DeclareMathOperator{\Tr}{Tr}
\DeclareMathOperator{\tr}{tr}
\DeclareMathOperator{\Det}{Det}
\DeclareMathOperator{\dist}{dist}
\DeclareMathOperator{\diam}{diam}
\DeclareMathOperator{\sign}{sign}
\DeclareMathOperator{\sech}{sech}
\DeclareMathOperator{\PV}{P.V.}
\let\div\relax
\DeclareMathOperator{\div}{div}
\newcommand{\usub}{\underline{u}}
\newcommand{\usup}{\overline{u}}
\def\ds{\displaystyle}

%%%%%%%%%%%%%%%%%%%%%%%%%%%%%%%%%%%%%%%%%%%%%%%%%%%%%%%%%%%%%%%%%%%%%%%%%%%%%%%
%%%%%%%%%%% SETTINGS %%%%%%%%%%%%%%%%%%%%%%%%%%%%%%%%%%%%%%%%%%%%%%%%%%%%%%%%%%
%%%%%%%%%%%%%%%%%%%%%%%%%%%%%%%%%%%%%%%%%%%%%%%%%%%%%%%%%%%%%%%%%%%%%%%%%%%%%%%
\setlength{\textwidth}{155mm}
\setlength{\textheight}{210mm}
\oddsidemargin=5mm
\evensidemargin=5mm

\numberwithin{equation}{section}

\graphicspath {{Images/}}

%%%%%%%%%%%%%%%%%%%%%%%%%%%%%%%%%%%%%%%%%%%%%%%%%%%%%%%%%%%%%%%%%%%%%%%%%%%%%%%
%%%%%%%%%%% HYPHENATION %%%%%%%%%%%%%%%%%%%%%%%%%%%%%%%%%%%%%%%%%%%%%%%%%%%%%%%%%%
%%%%%%%%%%%%%%%%%%%%%%%%%%%%%%%%%%%%%%%%%%%%%%%%%%%%%%%%%%%%%%%%%%%%%%%%%%%%%%%
\hyphenation{auto-ma-ti-cally}

%%%%%%%%%%%%%%%%%%%%%%%%%%%%%%%%%%%%%%%%%%%%%%%%%%%%%%%%%%%%%%%%%%%%%%%%%%%%%%%
%%%%%%%%%%% GENERAL INFORMATION %%%%%%%%%%%%%%%%%%%%%%%%%%%%%%%%%%%%%%%%%%%%%%%
%%%%%%%%%%%%%%%%%%%%%%%%%%%%%%%%%%%%%%%%%%%%%%%%%%%%%%%%%%%%%%%%%%%%%%%%%%%%%%%

\title[Semilinear integro-differential equations I]{Semilinear integro-differential equations I: odd solutions with respect to cones}
\author{Juan-Carlos Felipe-Navarro}
\address{J.C. Felipe-Navarro:
Universitat Polit\`ecnica de Catalunya and BGSMath, Departament de Matem\`{a}tiques, Diagonal 647, 08028 Barcelona, Spain}
\email{juan.carlos.felipe@upc.edu}

\author{Tomás Sanz-Perela}
\address{T. Sanz-Perela:
Universitat Polit\`ecnica de Catalunya and BGSMath, Departament de Matem\`{a}tiques, Diagonal 647, 08028 Barcelona, Spain}
\email{tomas.sanz@upc.edu}







\thanks{The first author is supported by MINECO grant MDM-2014-0445-16-4. The second author is supported by MINECO grant MDM-2014-0445. Both authors are supported by MINECO grants MTM2014-52402-C3-1-P and MTM2017-84214-C2-1-P, are members of the Barcelona Graduate School of Mathematics and part of the Catalan research group 2014 SGR 1083.}

\keywords{Integro-differential semilinear equation, odd symmetry, Simons cone, maximum principle for odd functions, energy estimate, saddle-shaped solution}




%%%%%%%%%%%%%%%%%%%%%%%%%%%%%%%%%%%%%%%%%%%%%%%%%%%%%%%%%%%%%%%%%%%%%%%%%%%%%%%
%%%%%%%%%%%%%%%%%%%%%%%%%%%%%%%%%%%%%%%%%%%%%%%%%%%%%%%%%%%%%%%%%%%%%%%%%%%%%%%
%%%%%%%%%%%%%%%%%%%%%%%%%%%%%%%%%%%%%%%%%%%%%%%%%%%%%%%%%%%%%%%%%%%%%%%%%%%%%%%
%%%%%%%%%%%%%%%%%%%%%%%%%%%%%%%%%%%%%%%%%%%%%%%%%%%%%%%%%%%%%%%%%%%%%%%%%%%%%%%
%%%%%%%%%%%%%%%%%%%%%%%%%%%%%%%%%%%%%%%%%%%%%%%%%%%%%%%%%%%%%%%%%%%%%%%%%%%%%%%
\begin{document}


%%%%%%%%%%%%%%%%%%%%%%%%%%%%%%%%%%%%%%%%%%%%%%%%%%%%%%%%%%%%%%%%%%%%%%%%%%%%%%%
%%%%%%%%%%%%%%%%%%%%%%%%%%%%%%%%%%%%%%%%%%%%%%%%%%%%%%%%%%%%%%%%%%%%%%%%%%%%%%%
\begin{abstract}
This is the first of two papers concerning solutions to the semilinear equation $L_K u = f(u)$ in $\R^{2m}$, where $L_K$ is a linear integro-differential operator which is uniformly elliptic and with a radially symmetric kernel. In this first paper, we are interested in solutions which are odd with respect to the Simons cone $\{(x', x'') \in \R^m \times \R^m \, : \, |x'| = |x''|\}$. 

We rewrite the operator acting on functions with these symmetry properties, finding an expression that is more suitable in order to use classical techniques such as variational arguments or maximum principles. Moreover, we find necessary and sufficient conditions on the kernel of the operator so that a version of the maximum principle for odd functions holds. Under these conditions, we prove an energy estimate and the existence of saddle-shaped solutions to a nonlocal version of the Allen-Cahn equation.

The setting established in this paper will be used in part~II in order to prove further properties of saddle-shaped solutions, such as asymptotic behavior, a maximum principle for the linearized operator and uniqueness.
\end{abstract}
%%%%%%%%%%%%%%%%%%%%%%%%%%%%%%%%%%%%%%%%%%%%%%%%%%%%%%%%%%%%%%%%%%%%%%%%%%%%%%%
%%%%%%%%%%%%%%%%%%%%%%%%%%%%%%%%%%%%%%%%%%%%%%%%%%%%%%%%%%%%%%%%%%%%%%%%%%%%%%%


\maketitle


\tableofcontents

%%%%%%%%%%%%%%%%%%%%%%%
\section{Introduction}
%%%%%%%%%%%%%%%%%%%%%%%
\label{Sec:Introduction}
 
In this paper, which is the second part of \cite{FelipeSanz-Perela:IntegroDifferentialI}, we study saddle-shaped solutions to the semilinear equation
\begin{equation}
\label{Eq:NonlocalAllenCahn}
L_K u = f(u) \quad \textrm{ in } \R^{2m},
\end{equation}
where $L_K$ is a linear integro-differential operator of the form \eqref{Eq:DefOfLu} and $f$ is of Allen-Cahn type. These  solutions (see Definition~\ref{Def:SaddleShapedSol} below) are particularly interesting in relation to the nonlocal version of a conjecture by De Giorgi, with the aim of finding a counterexample in high dimensions. Moreover, this problem is related to the regularity theory of nonlocal minimal surfaces. For more comments on this we refer to Subsection~\ref{Subsec:DeGiorgi} and the references therein.

Previous to this article and its first part \cite{FelipeSanz-Perela:IntegroDifferentialI}, there are only three works devoted to saddle-shaped solutions to the equation \eqref{Eq:NonlocalAllenCahn} with $L_K$ being the fractional Laplacian. In  \cite{Cinti-Saddle,Cinti-Saddle2}, Cinti proved the existence of a saddle-shaped solution as well as some qualitative properties such as asymptotic behavior, monotonicity properties, and instability whenever $2m\leq 6$. In a previous paper by the authors \cite{Felipe-Sanz-Perela:SaddleFractional}, further properties of these solutions were proved, the main ones being uniqueness and, when $2m\geq 14$, stability. Concerning saddle-shaped solutions to the classical Allen-Cahn equation $-\laplacian u = f(u)$, the same results were proved in \cite{DangFifePeletier, Schatzman, CabreTerraI,CabreTerraII, Cabre-Saddle}. The possible stability in dimensions $8$, $10$, and $12$ is still an open problem (both in the local and fractional frameworks), as well as the possible minimality of this solution in dimensions $2m \geq 8$.

The present paper together with its first part \cite{FelipeSanz-Perela:IntegroDifferentialI} are the first ones studying saddle-shaped solutions for general integro-differential equations of the form \eqref{Eq:NonlocalAllenCahn}. In the three previous papers \cite{Cinti-Saddle, Cinti-Saddle2, Felipe-Sanz-Perela:SaddleFractional} the main tool used was the extension problem for the fractional Laplacian (see \cite{CaffarelliSilvestre}). Nevertheless, this technique has the limitation that it cannot be carried out for general integro-differential operators different from the fractional Laplacian. Therefore, some purely nonlocal techniques were developed in the previous paper \cite{FelipeSanz-Perela:IntegroDifferentialI} to study saddle-shaped solutions, and we exploit them in the present one.

In part~I \cite{FelipeSanz-Perela:IntegroDifferentialI}, we established an appropriate setting to study solutions to \eqref{Eq:NonlocalAllenCahn} that are doubly radial and odd with respect to the Simons cone, a property that is satisfied by saddle-shaped solutions (see Subsection~\ref{Subsec:Integro-differential setting}). In that paper we deduced an alternative expression for the operator $L_K$ when acting on doubly radial odd functions ---see \eqref{Eq:OperatorOddF}. This was used to deduce some maximum principles for odd functions under certain assumptions on the kernel $K$ of the operator $L_K$. Moreover, we proved an energy estimate for doubly radial and odd minimizers of the energy associated to the equation, as well as the existence of saddle-shaped solutions to \eqref{Eq:NonlocalAllenCahn}.

In the present paper, we further study saddle-shaped solutions to \eqref{Eq:NonlocalAllenCahn} by using the results obtained in part~I \cite{FelipeSanz-Perela:IntegroDifferentialI}. First, we prove existence of this type of solutions, Theorem~\ref{Th:Existence}, by using the monotone iteration method (as an alternative to the proof in \cite{FelipeSanz-Perela:IntegroDifferentialI} where variational methods are used). After this, we establish the asymptotic behavior of saddle-shaped solutions,  Theorem~\ref{Th:AsymptoticBehaviorSaddleSolution}. To do it, we use two ingredients: a Liouville type theorem and a one-dimensional symmetry result, both for semilinear equations like \eqref{Eq:NonlocalAllenCahn} under some hypotheses on $f$. These are Theorems~\ref{Th:LiouvilleSemilinearWholeSpace} and \ref{Th:SymmHalfSpace}, proved in Section~\ref{Sec:SymmetryResults}. In the study of the asymptotic behavior of saddle-shaped solutions we establish further properties of the so-called \emph{layer solution} $u_0$ (see Section~\ref{Sec:Asymptotic}). Finally, we show the uniqueness of the saddle-shaped solution, Theorem~\ref{Th:Uniqueness}, by using a maximum principle for the linearized operator $L_K - f'(u)$ (Proposition~\ref{Prop:MaximumPrincipleLinearized}).

As in part I \cite{FelipeSanz-Perela:IntegroDifferentialI}, equation \eqref{Eq:NonlocalAllenCahn} is driven by a linear integro-differential operator $L_K$ of the form
\begin{equation}
\label{Eq:DefOfLu}
L_K w(x) = \int_{\R^n} \{w(x) - w(y)\} K(x-y)\d y.
\end{equation}
%The integral in \eqref{Eq:DefOfLu} has to be understood in the principal value sense. 
The most canonical example of such operators is the fractional Laplacian, which corresponds to the kernel $K(z) = c_{n, \s} |z|^{-n-2\s}$, where $\s \in (0,1)$ and $c_{n, \s}$ is a normalizing positive constant ---see \eqref{Eq:ConstantFracLaplacian}.

Throughout the paper, we assume that $K$ is symmetric, i.e., $K(z) = K(-z)$, and that $L_K$ is uniformly elliptic, that is,
\begin{equation}
\label{Eq:Ellipticity}
\lambda \dfrac{c_{n,\s}}{|z|^{n+2\s}} \leq K(z) \leq \Lambda \dfrac{c_{n,\s}}{|z|^{n+2\s}}\,, 
\end{equation}
where $\lambda$ and $\Lambda$ are two positive constants. This condition is frequently adopted since it yields Hölder regularity of solutions (see \cite{RosOton-Survey,SerraC2s+alphaRegularity}). The family of linear operators satisfying this condition is the so-called $\lcal_0(n,\s,\lambda, \Lambda)$ ellipticity class. For short we will usually write $\lcal_0$ and we will make explicit the parameters only when needed. 

Following the previous article \cite{FelipeSanz-Perela:IntegroDifferentialI}, when dealing with doubly radial functions we will assume that the operator $L_K$ is rotation invariant, that is, $K$ is radially symmetric. This extra assumption allows us to rewrite the operator in a suitable form when acting on doubly radial odd functions, as explained below.

%%%%%%%%%%%%%%%%%%%%%%%%%%%%%%%%%%%%%%%%%%%%%%%%%%%%%%%%%%%%%%%%%%%%%%%%%%%%%%%%%%%%%%%%%%%%%%%%%%%%%%%%%%%%%%%%%%%%%%%%%%%%%%%%%%%%%%%%%%%%%%%%%%%%%%%%%%%%%%%%%%%%%%%%%%%%%%%%%%%%%%%%%%%%%%%%%%%%%%%%%%%%%%%%%%%%%%%%%%%%%%%%%%%%%%%%%%%%%%%%%%%%%%%%%%%%%%%%%%%%

\subsection{Integro-differential setting for odd functions with respect to the Simons cone}
\label{Subsec:Integro-differential setting}


%A property of $f$ that will be used through the paper is that, since $f$ is strictly concave in $(0,1)$ and $f(0)=0$, then 
%\begin{equation}
%\label{Eq:PropertyConcavityf}
%f'(\tau)\tau < f(\tau) \quad \textrm{ for all } \tau \in (0,1)\,.
%\end{equation}

In this subsection we recall the basic definitions and results established in part I \cite{FelipeSanz-Perela:IntegroDifferentialI}. First, we present the Simons cone, which is a central object along this paper. It is defined in $\R^{2m}$ by
%\begin{equation}
%\label{Eq:SimonsCone}
$$
\mathscr{C} := \setcond{x = (x', x'') \in \R^m \times \R^m = \R^{2m}}{|x'| = |x''|}\,.
%\end{equation}
$$
This cone is of importance in the theory of (local and nonlocal) minimal surfaces (see Subsection~\ref{Subsec:DeGiorgi}). 
%It has zero mean curvature at every point $x\in \ccal \setminus \{0\}$, in all even dimensions, and it is a minimizer of the perimeter functional when $2m\geq 8$. Concerning the nonlocal setting, $\ccal$ has also zero nonlocal mean curvature in all even dimensions, although it is not known if it is a minimizer of the nonlocal perimeter (see the introduction of \cite{Felipe-Sanz-Perela:SaddleFractional} and the references therein for more details).
We will use the letters $\ocal$ and $\ical$ to denote each of the parts in which $\R^{2m}$ is divided by the cone $\ccal$:
$$
\ocal:= \setcond{x = (x', x'') \in \R^{2m}}{|x'| > |x''|} \ \textrm{ and } \
\ical:= \setcond{x = (x', x'') \in \R^{2m}}{|x'| < |x''|}.
$$

Both $\ocal$ and $\ical$ belong to a family of sets in $\R^{2m}$ which are called of \emph{double revolution}. These are sets that are invariant under orthogonal transformations in the first $m$ variables, as well as under orthogonal transformations in the last $m$ variables. That is, $\Omega\subset \R^{2m}$ is a set of double revolution if $R\Omega = \Omega$ for every given transformation $R\in O(m)^2 = O(m) \times O(m)$, where  $O(m)$ is the orthogonal group of $\R^m$.


We say that a function $w:\R^{2m}  \to \R$ is \emph{doubly radial} if it depends only on the modulus of the first $m$ variables and on the modulus of the last $m$ ones, i.e., $w(x) = w(|x'|,|x''|)$. Equivalently, $w(Rx) = w(x)$ for every $R \in O(m)^2$.

We recall now the definition of $(\cdot)^\star$, an isometry that played a significant role in part~I \cite{FelipeSanz-Perela:IntegroDifferentialI}. It is defined by
\begin{equation}
\label{Eq:DefStar}
\begin{matrix}
(\cdot)^\star \colon & \R^{2m}= \R^{m}\times \R^{m}  &\to&  \R^{2m}= \R^{m}\times \R^{m}  \\
& x = (x',x'') &\mapsto & x^\star = (x'',x')\,.
\end{matrix}
\end{equation}
Note that this isometry is actually an involution that maps $\ocal$ into $\ical$ (and vice versa) and leaves the cone $\ccal$ invariant ---although not all points in $\ccal$ are fixed points of $(\cdot)^\star$. Taking into account this transformation, we say that a doubly radial function $w$ is \emph{odd with respect to the Simons cone} if $w(x) = -w(x^\star)$. Similarly, we say that a doubly radial function $w$ is \emph{even with respect to the Simons cone} if $w(x) = w(x^\star)$.



With these definitions at hand we can precisely define saddle-shaped solutions.
\begin{definition}
	\label{Def:SaddleShapedSol}
	We say that a bounded solution $u$ to \eqref{Eq:NonlocalAllenCahn} is a \emph{saddle-shaped solution} (or simply \emph{saddle solution}) if
	\begin{enumerate}
		\item $u$ is doubly radial.
		\item $u$ is odd with respect to the Simons cone.
		\item $u > 0$ in $\ocal = \{|x'| > |x''|\} $.
	\end{enumerate}
\end{definition}
Note that these solutions are even with respect to the coordinate axes and that their zero level set is the Simons cone $\mathscr{C} = \{|x'|=|x''|\}$. 



Let us collect now the main results of the previous paper \cite{FelipeSanz-Perela:IntegroDifferentialI} that will be used in the present one. Recall that if $K$ is a radially symmetric kernel we can rewrite the operator $L_K$ acting on a doubly radial function $w$ as
$$
L_K w(x) = \int_{\R^{2m}} \{w(x) - w(y)\} \overline{K}(x,y) \d y\,,
$$
where $\overline{K}$ is doubly radial in both variables and is defined by
\begin{equation}
\label{Eq:KbarDef}
\overline{K}(x,y) := \average_{O(m)^2} K(|Rx - y|)\d R\,.
\end{equation}
Here, $\d R$ denotes integration with respect to the Haar measure on $O(m)^2$ (see Section~2 of \cite{FelipeSanz-Perela:IntegroDifferentialI} for the details).

Moreover, if we consider doubly radial functions that are odd with respect to the Simons cone, we can use the involution $(\cdot)^\star$ to find that
\begin{equation}
\label{Eq:OperatorOddF}
L_K w (x) = \int_{\ocal} \{w(x) - w(y) \} \{\overline{K}(x, y) - \overline{K}(x, y^\star)  \} \d y +  2 w(x) \int_{\ocal} \overline{K}(x, y^\star) \d y \,.
\end{equation}
Furthermore,
\begin{equation}
\label{Eq:ZeroOrderTerm}
\frac{1}{C} \dist(x,\ccal)^{-2\s} \leq \int_{\ocal} \overline{K}(x, y^\star) \d y \leq C \dist(x,\ccal)^{-2\s},
\end{equation}
with $C>0$ depending only on $m, \s, \lambda$, and $\Lambda$ (see the details in part I \cite{FelipeSanz-Perela:IntegroDifferentialI}).


Note that the expression \eqref{Eq:OperatorOddF} has an integro-differential part plus a term of order zero with a positive coefficient. Thus, the most natural assumption to make in order to have an elliptic operator (when acting on doubly radial odd functions) is that the kernel of the integro-differential term is positive. That is, $\overline{K}(x, y) - \overline{K}(x, y^\star)>0$. One of the main results in part I \cite{FelipeSanz-Perela:IntegroDifferentialI}, stated next, established necessary and sufficient conditions on the original kernel $K$ for $L_K$ to have a positive kernel when acting on doubly radial odd functions. 

\begin{theorem}[\cite{FelipeSanz-Perela:IntegroDifferentialI}]
	\label{Th:SufficientNecessaryConditions}
	Let $K:(0,+\infty) \to (0,+\infty)$ and consider the radially symmetric kernel $K(|x-y|)$ in $\R^{2m}$. Define $\overline{K} : \R^{2m}\times \R^{2m} \to \R$ by \eqref{Eq:KbarDef}.
	
	If 
	\begin{equation}
	\label{Eq:SqrtConvex}	
	K(\sqrt{\tau}) \text{ is a strictly convex function of }\tau\,,
	\end{equation}
	then $L_K$ has a positive kernel in $\ocal$ when acting on doubly radial functions which are odd with respect to the Simons cone $\ccal$. More precisely, it holds
	\begin{equation}
	\label{Eq:KernelInequality}
	\overline{K}(x,y) > \overline{K}(x, y^\star) \quad \text{ for every }x,y \in \ocal\,.
	\end{equation}
	
	In addition, if $K\in C^2((0,+\infty))$, then \eqref{Eq:SqrtConvex} is not only a sufficient condition for \eqref{Eq:KernelInequality} to hold, but also a necessary one.
\end{theorem}

%%%%%%%%%%%%%%%%%%%%%%%%%%%%%%%%%%%%%%%%%%%%%%%%%%%%%%%%%%%%%%%%%%%%%%%%%%%%%%%%%%%%%%%%%%%%%%%%%%%%%%%%%%%%%%%%%%%%%%%%%%%%%%%%%%%%%%%%%%%%%%%%%%%%%%%%%%%%%%%%%%%%%%%%%%%%%%%%%%%%%%%%%%%%%%%%%%%%%%%%%%%%%%%%%%%%%%%%%%%%%%%%%%%%%%%%%%%%%%%%%%%%%%%%%%%%%%%%%%%%
\subsection{Main results}
\label{Subsec:Main results}

Through all the paper we will assume that $f$, the nonlinearity in \eqref{Eq:NonlocalAllenCahn}, is a $C^1$ function satisfying
\begin{equation}
\label{Eq:Hypothesesf}
f \textrm{ is odd, } \quad f(\pm 1)=0, \quad \text{ and } \quad f \textrm{ is strictly concave in }  (0,1).
\end{equation}
It is easy to see that these properties yield $f>0$ in $(0,1)$, $f'(0)>0$ and $f'(\pm 1) < 0$. 

The first main result of this paper concerns the existence of saddle-shaped solution.


\begin{theorem}[Existence of saddle-shaped solution]
	\label{Th:Existence}
	Let $f$ satisfy \eqref{Eq:Hypothesesf}. Let $K$ be a radially symmetric kernel satisfying the positivity condition \eqref{Eq:KernelInequality} and such that $L_K\in \lcal_0(2m, \s, \lambda, \Lambda)$. 
	
	Then, for every even dimension $2m \geq 2$, there exists a saddle-shaped solution $u$ to \eqref{Eq:NonlocalAllenCahn}. In addition, $u$ satisfies $|u|<1$ in $\R^{2m}$.
\end{theorem}



This theorem was already proved in part I \cite{FelipeSanz-Perela:IntegroDifferentialI} using variational techniques. Here, we show that existence can also be proved using, instead, the monotone iteration method. Let us remark that in both methods it is crucial to have the positivity condition \eqref{Eq:KernelInequality}.


The second main result of this paper is Theorem~\ref{Th:AsymptoticBehaviorSaddleSolution} below, on the asymptotic behavior of a saddle-shaped solution at infinity. To state it, let us introduce an important type of solutions in the study of the integro-differential Allen-Cahn equation: the layer solutions.


We say that a solution $v$ to $L_K v = f(v)$ in $\R^n$ is a \emph{layer solution} if $v$ is increasing in one direction, say $e\in \Sph^{n-1}$ and $v(x) \to \pm 1$ as $x\cdot e \to \pm \infty$ (not necessarily uniform). By a result of Cozzi and Passalacqua (Theorem~1 in \cite{CozziPassalacqua}), under the assumptions \eqref{Eq:Hypothesesf} on $f$, for every kernel $K_1$ such that $L_{K_1}\in \lcal_0(1,\s,\Lambda, \lambda)$ there exist a layer solution  to $L_{K_1} w = f(w)$ in $\R$ which is unique up to translations and is odd with respect to some point (in the case of the fractional Laplacian this result was proved in \cite{CabreSolaMorales,CabreSireII} by using the extension problem).

In $\R^n$, a special case of layer solutions are the one-dimensional ones. Actually, in relation with the available results concerning a conjecture by De Giorgi, in low dimensions all layer solutions are one-dimensional (see Subsection~\ref{Subsec:DeGiorgi}). One-dimensional layer solutions in $\R^n$ are in correspondence with the ones in $\R$ as explained next ---see also \cite{CozziPassalacqua}. Let $v$ be a layer solution to $L_K v = f(v)$ in $\R^n$ depending only on one direction, say $v(x) = w(x_n)$, and assume that $L_{K}\in \lcal_0(n,\s,\Lambda, \lambda)$. Then $w$ is a layer solution to $L_{K_1} w = f(w)$ in $\R$ with $K_1$ given by
$$
K_1(t) := \int_{\R^{n-1}} K\left(\theta,t\right) \d \theta = |t|^{n-1} \int_{\R^{n-1}} K\left(t\sigma,t\right) \d \sigma.
$$
Moreover $L_{K_1}\in \lcal_0(1,\s,\Lambda, \lambda)$. For more details see Proposition~\ref{Prop:KernelsDimension} in Section~\ref{Sec:Asymptotic} and \cite{CozziPassalacqua}. 

A particular layer solution, denoted by $u_0$, plays an important role in this paper. It is defined to be the unique solution of the following problem.
\begin{equation}
\label{Eq:LayerSolution}
\beqc{\PDEsystem}
L_{K_1}  u_0 &=& f(u_0) & \textrm{ in }\R\,,\\
\dot{u}_0 &>& 0 & \textrm{ in } \R\,,\\
u_0(x) & = &-u_0(-x)  & \textrm{ in }\R\,,\\
\ds \lim_{x \to \pm \infty} u_0(x) &=& \pm 1. & 
\eeqc
\end{equation}
Note that, by the previous comments, $v(x) = u_0(x_n)$ is a one-dimensional layer solution to $L_K v = f(v)$ in $\R^n$. Moreover, the same holds for $u_0(x\cdot e)$ for every $e\in \Sph^{n-1}$ whenever the kernel $K$ is radially symmetric.

The importance of the layer solution $u_0$ in relation with saddle-shaped solutions is that the associated function
\begin{equation}
\label{Eq:DefOfU}
U(x):= u_0 \left( \dfrac{|x'| - |x''|}{\sqrt{2}} \right)\,
\end{equation}
describes the asymptotic behavior of saddle solutions at infinity. Note that $(|x'| - |x''| )/\sqrt{2}$ is the signed distance to the Simons cone (see Lemma~4.2 in \cite{CabreTerraII}). Therefore, we can understand the function $U$ as the layer solution $u_0$ centered at each point of the Simons cone and oriented in the normal direction to the cone.

The precise statement on the asymptotic behavior of saddle-shaped solutions at infinity is the following.

\begin{theorem}
	\label{Th:AsymptoticBehaviorSaddleSolution}
	Let $f\in C^2(\R)$ satisfy \eqref{Eq:Hypothesesf}. Let $K$ be a radially symmetric kernel satisfying the positivity condition \eqref{Eq:KernelInequality} and such that $L_K\in \lcal_0(2m, \s, \lambda, \Lambda)$. Let $u$ be a saddle-shaped solution to \eqref{Eq:NonlocalAllenCahn} and let $U$ be the function defined by \eqref{Eq:DefOfU}.
	
	Then,
	$$
	\norm{u-U}_{L^\infty(\R^n\setminus B_R)}
	+\norm{\nabla u-\nabla U}_{L^\infty(\R^n\setminus B_R)}
	+\norm{D^2u-D^2U}_{L^\infty(\R^n\setminus B_R)} \to 0
	$$
	as $ R\to +\infty$.
\end{theorem}

To establish the asymptotic behavior of saddle-shaped solutions we use a compactness argument as in \cite{CabreTerraII, Cinti-Saddle, Cinti-Saddle2}, together with two important results established in Section~\ref{Sec:SymmetryResults}. The first one, Theorem~\ref{Th:LiouvilleSemilinearWholeSpace}, is a Liouville type result for nonnegative solutions to a semilinear equation in the whole space. 

\begin{theorem}
	\label{Th:LiouvilleSemilinearWholeSpace}
	Let $L_K \in \lcal_0(n,\s)$ and let $v$ be a bounded solution to
	\begin{equation}
	\label{Eq:PositiveWholeSpace}
	\beqc{\PDEsystem}
	L_K v &=& f(v) & \textrm{ in }\R^n\,,\\
	v &\geq& 0 & \textrm{ in } \R^n\,,
	\eeqc
	\end{equation}
	with a nonlinearity $f\in C^1$ satisfying
	\begin{itemize}
		\item $f(0) = f(1) = 0$,
		\item $f'(0)>0$,
		\item $f>0$ in $(0,1)$, and $f<0$ in $(1,+\infty)$.
	\end{itemize}
	Then, $v\equiv 0$ or $v \equiv 1$.
\end{theorem}

Similar classification results have been proved for the fractional Laplacian in \cite{ChenLiZhang,LiZhang} (either using the extension problem or not) with the method of moving spheres, which uses crucially the scale invariance of the operator $\fraclaplacian$. To the best of our knowledge, there is no similar result available in the literature for general kernels in the ellipticity class $\lcal_0$ (which are not necessarily scale invariant). Thus, we present here a proof based on the techniques used in \cite{BerestyckiHamelNadi} for a local equation with the classical Laplacian. It relies on the maximum principle, the translation invariance of the operator, a Harnack inequality and a stability argument. All these features are available for the operators in $\lcal_0$ (see Section~\ref{Sec:SymmetryResults}). Thus, the same arguments as in the local case can be carried out.

The second ingredient to prove the asymptotic behavior of saddle-shaped solutions is a symmetry result for equations in a half-space, stated next. Here and in the rest of the paper we use the notation $\R^n_+= \{(x_H,x_n)\in \R^{n-1}\times \R \ : \ x_n > 0\}$.  

\begin{theorem}
	\label{Th:SymmHalfSpace}
	Let $L_K\in \lcal_0(n,\s)$ and let $v$ be a bounded solution to one of these two problems:
	
	\begin{equation}
	\reqnomode
	\tag{P1}
	\label{Eq:P1}
	\beqc{\PDEsystem}
	L_K v &=& f(v)   &\textrm{ in } \,\R^n_+,\\
	v &>& 0   &\textrm{ in } \,\R^n_+,\\
	v(x_H,x_n) &=& -v(x_H,-x_n)   &\textrm{ in } \,\R^n.
	\eeqc
	\end{equation}
	
	\begin{equation}
	\reqnomode
	\tag{P2}
	\label{Eq:P2}
	\beqc{\PDEsystem}
	L_K v &=& f(v)   &\textrm{ in } \,\R^n_+,\\
	v &>& 0   &\textrm{ in } \,\R^n_+,\\
	v &=& 0   &\textrm{ in } \,\R^n \setminus \R^n_+.
	\eeqc
	\end{equation}
	
	\reqnomode
	
	Assume that, in $\R^n_+$, the kernel $K$ of the operator $L_K$ is decreasing in the direction of $x_n$, i.e., it satisfies
	$$
	K(x_H-y_H,x_n-y_n) \geq K(x_H-y_H,x_n+y_n) \,\,\,\,\text{for all } \,\, x,y\in \R^n_+.
	$$ 
	Suppose that $f\in C^1$ and
	\begin{itemize}
		\item $f(0) = f(1) = 0$,
		\item $f'(0)>0$, and $f'(t)\leq 0$ for all $t\in[1-\delta,1]$ for some $\delta>0$,
		\item $f>0$ in $(0,1)$, and
		\item $f$ is odd in the case of \eqref{Eq:P1}.
	\end{itemize}
	Then, $v$ depends only on $x_n$ and it is increasing in that direction.
\end{theorem}

The result for \eqref{Eq:P2} has been proved for the fractional Laplacian under some assumptions on $f$ (weaker than the ones in Theorem~\ref{Th:SymmHalfSpace}) in \cite{QuaasXia, BarriosEtAl-Monotonicity, BarriosEtAl-Symmetry, FallWethMonotonicity}. Instead, to the best of our knowledge \eqref{Eq:P1} has not been treated even for the fractional Laplacian. In our case, the fact that $f$ is of Allen-Cahn type allows us to use rather simple arguments that work for both problems \eqref{Eq:P1} and \eqref{Eq:P2} ---moving planes and sliding methods. Moreover, the fact that we replace the kernel of the operator by a general $K$ satisfying \eqref{Eq:Ellipticity} do not affect significantly the proof. Although \eqref{Eq:P2} will not be used in this paper, since the proof for this problem is analogous to the one for \eqref{Eq:P1}, we include it here for future reference, 

The last main result of this paper is the uniqueness of the saddle-shaped solution, stated next.

\begin{theorem}[Uniqueness of the saddle-shaped solution]
	\label{Th:Uniqueness}
	Let $f$ satisfy \eqref{Eq:Hypothesesf} and let $K$ be a radially symmetric kernel satisfying the positivity condition \eqref{Eq:KernelInequality} and such that $L_K\in \lcal_0(2m, \s, \lambda, \Lambda)$. 
	
	Then, for every dimension $2m \geq 2$, there exists a unique saddle-shaped solution to \eqref{Eq:NonlocalAllenCahn}.
\end{theorem}

To prove this result we need two ingredients. The first one is the asymptotic behavior of saddle solutions given in Theorem~\ref{Th:AsymptoticBehaviorSaddleSolution}. The second one is a maximum principle in $\ocal$ for the linearized operator $L_K - f'(u)$, which is given in Proposition~\ref{Prop:MaximumPrincipleLinearized}. To establish it, we will need to use a maximum principle in ``narrow'' sets, also proved in Section~\ref{Sec:MaximumPrinciple}. In the arguments, it is crucial again the positivity condition \eqref{Eq:KernelInequality}.

%%%%%%%%%%%%%%%%%%%%%%%%%%%%%%%%%%%%%%%%%%%%%%%%%%%%%%%%%%%%%%%%%%%%%%%%%%%%%%%%%%%%%%%%%%%%%%%%%%%%%%%%%%%%%%%%%%%%%%%%%%%%%%%%%%%%%%%%%%%%%%%%%%%%%%%%%%%%%%%%%%%%%%%%%%%%%%%%%%%%%%%%%%%%%%%%%%%%%%%%%%%%%%%%%%%%%%%%%%%%%%%%%%%%%%%%%%%%%%%%%%%%%%%%%%%%%%%%%%%%
\subsection{Saddle-shaped solutions in the context of a conjecture by De Giorgi}
\label{Subsec:DeGiorgi}

To conclude this introduction, let us make some comments on the importance of problem \eqref{Eq:NonlocalAllenCahn} and its relation with the theory of (classical and nonlocal) minimal surfaces and  a famous conjecture raised by De Giorgi.

A main open problem (even in the local case) is to determine whether the saddle-shaped solution is a minimizer of the energy functional associated to the equation, depending on the dimension $2m$. This question is deeply related to the regularity theory of local and nonlocal minimal surfaces, as explained next.

In the seventies, Modica and Mortola (see \cite{Modica,ModicaMortola}) proved that, considering an appropriately rescaled version of the (local) Allen-Cahn equation, the corresponding energy functionals $\Gamma$-converge to the perimeter functional. Thus, the blow-down sequence of minimizers of the Allen-Cahn energy converge to the characteristic function of a set of minimal perimeter. This same fact holds for the equation with the fractional Laplacian, though we have two different scenarios depending on the parameter $\s \in (0,1)$. If $\s \geq 1/2$, the rescaled energy functionals associated to the equation $\Gamma$-converge to the classical perimeter (see \cite{GiovanniBouchitteSeppecher,Gonzalez}), while in the case $\s \in (0,1/2)$ they $\Gamma$-converge to the fractional perimeter (see \cite{SavinValdinoci-GammaConvergence}). As a consequence, if the saddle-shaped solution was proved to be a minimizer in a certain dimension for some $\s \in (0,1/2)$, it would follow that the Simons cone $\ccal$ would be a minimizing nonlocal $(2\s)$-minimal surface in such dimensions. This last statement on the saddle-shaped solution is an open problem in any dimension (although it is known that the Simons cone is not a minimizer in dimension $2m=2$). The only available result related to this question is the recent one in our previous paper \cite{Felipe-Sanz-Perela:SaddleFractional}, which concerns stability (a weaker property than minimality). We proved that the saddle-shaped solution to the fractional Allen-Cahn equation is stable in dimensions $2m\geq 14$. As a consequence of this and a result in \cite{CabreCintiSerra-Stable}, the Simons cone is a stable nonlocal $(2\s)$-minimal surface in dimensions $ 2m\geq 14$ (see the details in \cite{Felipe-Sanz-Perela:SaddleFractional}).


Moreover, as explained below, saddle-shaped solutions are natural objects to build a counterexample to a famous conjecture raised by De Giorgi, that reads as follows. Let $u$ be a bounded solution to $-\Delta  u = u - u^3$ in $\R^n$ which is monotone in one direction, say $\partial_{x_n} u > 0$. Then, if $n\leq 8$, $u$ is one dimensional, i.e., $u$ depends only on one Euclidean variable. This conjecture was proved to be true in dimensions $n=2$ and  $n=3$ (see \cite{GhoussoubGui,AmbrosioCabre}), and in dimensions $4\leq n \leq 8$ with the extra assumption
\begin{equation}
\label{Eq:SavinCondition}
\lim_{x_n \to \pm \infty} u(x_H,x_n) = \pm 1 \quad \text{ for all } x_H\in \R^{n-1}\,,
\end{equation}
(see \cite{Savin-DeGiorgi}). A counterexample to the conjecture in dimensions $n\geq 9$ was given in \cite{delPinoKowalczykWei} by using the gluing method. 

An alternative approach to the one of \cite{delPinoKowalczykWei} to construct a counterexample to the conjecture was given by Jerison and Monneau in \cite{JerisonMonneau}. They showed that a counterexample in $\R^{n+1}$ can be constructed with a rather natural procedure if there exists a global minimizer of $-\Delta u = f(u)$ in $\R^n$ which is bounded and even with respect to each coordinate but is not one-dimensional. The saddle-shaped solution is of special interest in search of this counterexample, since it is even with respect to all the coordinate axis and it is canonically associated to the Simons cone, which in turn is the simplest nonplanar minimizing minimal surface. Therefore, by proving that the saddle solution to the classical Allen-Cahn equation is a minimizer in some dimension $2m$, one would obtain automatically a counterexample to the conjecture in $\R^{2m+1}$.

The corresponding conjecture in the fractional setting, where one replaces the operator $-\Delta$ by $\fraclaplacian$, has been widely studied in the last years. In this framework, the conjecture has been proven to be true for all $\s\in(0,1)$ in dimensions $n=2$ (see \cite{CabreSolaMorales,CabreSireI,SireValdinoci}) and $n=3$ (see \cite{CabreCinti-EnergyHalfL, CabreCinti-SharpEnergy,DipierroFarinaValdinoci}). The conjecture is also true in dimension $n=4$ in the case of $\s = 1/2$ (see \cite{FigalliSerra}) and if $\s\in(0,1/2)$ is close to $1/2$ (see \cite{CabreCintiSerra-Stable}). Assuming the additional hypothesis \eqref{Eq:SavinCondition}, the conjecture is true in dimensions $4\leq n \leq 8$ for $1/2 \leq \s < 1$ (see \cite{Savin-Fractional,Savin-Fractional2}), and also for $\s\in(0,1/2)$ if $\s$ is close to $1/2$ (see \cite{DipierroSerraValdinoci}). A counterexample to the De Giorgi conjecture for the fractional Allen-Cahn equation in dimensions $n \geq 9$ for $\s \in (1/2,1)$ has been very recently announced in \cite{ChanLiuWei}.

Concerning the conjecture with more general operators like $L_K$, fewer results are known. In dimension $n=2$ the conjecture is proved in \cite{HamelRosOtonSireValdinoci, Bucur, FazlySire}, under different assumptions on the kernel $K$ and even for more general nonlinear operators. Note also that the results of \cite{DipierroSerraValdinoci} also hold for a particular class of kernels in $\lcal_0$.

%%%%%%%%%%%%%%%%%%%%%%%%%%%%%%%%%%%%%%%%%%%%%%%%%%%%%%%%%%%%%%%%%%%%%%%%%%%%%%%%%%%%%%%%%%%%%%%%%%%%%%%%%%%%%%%%%%%%%%%%%%%%%%%%%%%%%%%%%%%%%%%%%%%%%%%%%%%%%%%%%%%%%%%%%%%%%%%%%%%%%%%%%%%%%%%%%%%%%%%%%%%%%%%%%%%%%%%%%%%%%%%%%%%%%%%%%%%%%%%%%%%%%%%%%%%%%%%%%%%%
\subsection{Plan of the article}
\label{Subsec:Plan}

The paper is organized as follows. In Section~\ref{Sec:Preliminaries} we present some preliminary results that will be used in the rest of the article. Section~\ref{Sec:Existence} contains the proof of Theorem~\ref{Th:Existence} on the existence of a saddle-shaped solution via the monotone iteration method. In Section~\ref{Sec:SymmetryResults} we establish the Liouville type and symmetry results, Theorems~\ref{Th:LiouvilleSemilinearWholeSpace} and \ref{Th:SymmHalfSpace}. Section~\ref{Sec:Asymptotic} is devoted to the layer solution $u_0$ of problem \eqref{Eq:NonlocalAllenCahn} and the proof of the asymptotic behavior of saddle-shaped solutions, Theorem~\ref{Th:AsymptoticBehaviorSaddleSolution}. Finally, Section~\ref{Sec:MaximumPrinciple} concerns the proof of a maximum principle in $\ocal$ for the linearized operator $L_K - f'(u)$ (Proposition~\ref{Prop:MaximumPrincipleLinearized}), as well as the proof of Theorem~\ref{Th:Uniqueness}, establishing the uniqueness of the saddle-shaped solution.



%%%%%%%%%%%%%%%%%%%%%%%%
\section{Rotation invariant operators acting on doubly radial odd functions}
%%%%%%%%%%%%%%%%%%%%%%%%
\label{Sec:OperatorOddF}

This section is devoted to study rotation invariant operators of the class $\lcal_0$ when they act on doubly radial odd functions. First, we deduce an alternative expression for the operator in terms of a doubly radial kernel $\overline{K}$. Then, we present necessary and sufficient conditions on the kernel $K$ in order to $L_K$ belong to the class $\lcal_\star$, i.e., in order to \eqref{Eq:KernelInequality} hold (we establish Theorem~\ref{Th:SufficientNecessaryConditions}). Finally, we show two maximum principles for doubly radial odd functions.


%%%%%%%%%%%%%%%%%%%%%%%%%%%%%%%%%%%%%%%%%%%%%%%%%%%%%%%%%%%%%%%%%%%%%%%%%%%%%%%%%%%%%%%%%%%%%%%%%%%
%%%%%%%%%%%%%%%%%%%%%%%%%%%%%%%%%%%%%%%%%%%%%%%%%%%%%%%%%%%%%%%%%%%%%%%%%%%%%%%%%%%%%%%%%%%%%%%%%%%
\subsection{Alternative expressions for the operator $L_K$}
%%%%%%%%%%%%%%%%%%%%%%%%%%%%%%%%%%%%%%%%%%%%%%%%%%%%%%%%%%%%%%%%%%%%%%%%%%%%%%%%%%%%%%%%%%%%%%%%%%%
%%%%%%%%%%%%%%%%%%%%%%%%%%%%%%%%%%%%%%%%%%%%%%%%%%%%%%%%%%%%%%%%%%%%%%%%%%%%%%%%%%%%%%%%%%%%%%%%%%%

The main purpose of this subsection is to deduce an alternative expression for a rotation invariant operator $L_K \in \lcal_0$ acting on doubly radial functions. This expression is more suitable to work with and it will be used throughout the paper. Our first remark is that if $w$ is invariant by $O(m)^2$, the same holds for $L_Kw$. Indeed, for every $R \in O(m)^2$,
\begin{align*}
L_K w (Rx)
& = \int_{\R^{2m}} \{w(Rx) - w(y)\} K(|Rx - y|)  \d y\\
& = \int_{\R^{2m}} \{w(Rx) - w(R\tilde{y})\} K(|Rx - R\tilde{y}|) \d \tilde{y}\\
& = \int_{\R^{2m}} \{w(x) - w(\tilde{y})\} K(|x-\tilde{y}|) \d \tilde{y}\\
& = L_K w (x)\,.
\end{align*}
Here we have used the change $y = R\tilde{y}$ and the fact that $w(R \cdot) = w(\cdot)$ for every $R\in O(m)^2$.

Next, we present an alternative expression for the operator $L_K $ acting on doubly radial functions. This expression involves the new kernel $\overline{K}$, which is also doubly radial.

\begin{lemma} \label{Lemma:AlternativeOperatorExpression}
Let $L_K \in \lcal_0(2m,\s)$ have a radially symmetric kernel $K$, and let $w$ be a doubly radial function such that $L_K w$ is well-defined. Then, $L_K w$ can be expressed as
$$
L_K w(x) = \int_{\R^{2m}} \{w(x) - w(y)\} \overline{K}(x,y) \d y
$$
where $\overline{K}$ is symmetric, invariant by $O(m)^2$ in both arguments, and it is defined by
\begin{equation*}
%\label{Eq:KbarDef}
\overline{K}(x,y) := \average_{O(m)^2} K(|Rx - y|)\d R\,.
\end{equation*}
Here, $\d R$ denotes integration with respect to the Haar measure on $O(m)^2$.
\end{lemma}

Recall (see for instance \cite{Nachbin}) that the Haar measure on $O(m)^2$ exists and it is unique up to a
multiplicative constant. Let us state next the properties of this measure that will be used in the rest of the
paper. In the following, the Haar measure is denoted by $\mu$. First, since $O(m)^2$ is a compact
group, it is unimodular (see Chapter~II, Proposition~ 13 of \cite{Nachbin}). As a consequence, the
measure $\mu$ is left and right invariant, that is, $\mu(R\Sigma) = \mu(\Sigma) = \mu(\Sigma R) $
for every subset $\Sigma \subset O(m)^2$ and every $R\in O(m)^2$. Moreover, it holds
\begin{equation}
\label{Eq:Unimodular}
\average_{O(m)^2} g(R^{-1}) \d R = \average_{O(m)^2} g(R) \d R
\end{equation}	
for every $g\in L^1(O(m)^2)$ ---see \cite{Nachbin} for the details.

\begin{proof}[Proof of Lemma~\ref{Lemma:AlternativeOperatorExpression}]
Since $L_K w (x) = L_K w (Rx)$ for every $R\in O(m)^2$, by taking the mean over all the transformations in $O(m)^2$, we get
\begin{align*}
L_K w(x) &= \average_{O(m)^2} L_K w(Rx)\d R =  \average_{O(m)^2} \int_{\R^{2m}} \{w(x) - w(y)\}K(|Rx - y|) \d y \d R\\
&= \int_{\R^{2m}} \{w(x) - w(y)\}  \average_{O(m)^2} K(|Rx - y|) \d R  \d y \\
&= \int_{\R^{2m}} \{w(x) - w(y)\}  \overline{K}(x,y) \d y\,.
\end{align*}
Now, we show that $\overline{K}$ is symmetric. Using property \eqref{Eq:Unimodular}, we get
\begin{align*}
\overline{K}(y,x) &= \average_{O(m)^2} K(|R y - x|)\d R = \average_{O(m)^2} K(|R^{-1} (R y - x)|)\d R \\
&= \average_{O(m)^2} K(|R^{-1}x-y)|)\d R = \overline{K}(x,y)\,.
\end{align*}
It remains to show that
$\overline{K}$ is invariant by $O(m)^2$ in its two arguments. By the symmetry, it is enough to
check it for the first one. Let $\tilde{R} \in O(m)^2$. Then,
$$
\overline{K} (\tilde{R}x, y) = \average_{O(m)^2} K(|R \tilde{R} x - y|)\d R  = \average_{O(m)^2} K(|R x - y|)\d R = \overline{K} (x, y)\,,
$$
where we have used the right invariance of the Haar measure.
\end{proof}



In the following lemma we present some properties of the involution $(\cdot)^\star$ defined by \eqref{Eq:DefStar} and its relation with the doubly radial kernel $\overline{K}$ and the transformations of $O(m)^2$. In particular, in the proof of Theorem~\ref{Th:SufficientNecessaryConditions} it will be useful to consider the following transformation. For every $R\in O(m)^2$, we define  $R_\star\in O(m)^2$ by 
\begin{equation}
	\label{Eq:DefRStar}
	R_\star := (R(\cdot)^\star)^\star\,.
\end{equation}
Equivalently, if $R = (R_1, R_2)$ with $R_1$, $R_2 \in O(m)$, then $R_\star = (R_2, R_1)$.

\begin{lemma}
\label{Lemma:PropertiesStar}
Let $(\cdot)^\star: \R^{2m} \to \R^{2m}$ be the involution defined by $x^\star = (x',x'')^\star = (x'', x')$
---see \eqref{Eq:DefStar}.
Then,
\begin{enumerate}
\item
The Haar integral on $O(m)^2$ has the following invariance:
\begin{equation}
\label{Eq:InvarianceByStar}
\int_{O(m)^2} g(R_\star) \d R = \int_{O(m)^2} g(R) \d R \,,
\end{equation}
for every $g \in L^1(O(m)^2)$.
\item $\overline{K}(x^\star,y) = \overline{K} (x,y^\star)$. As a consequence, $\overline{K}(x^\star,y^\star) = \overline{K} (x,y)$.
\end{enumerate}
\end{lemma}

\begin{proof}
The first statement is easy to check by using Fubini:
\begin{align*}
\int_{O(m)^2} g(R_\star) \d R & = \int_{O(m)} \!\! \d R_1 \int_{O(m)} \!\! \d R_2 \ \ g(R_2, R_1)  =  \int_{O(m)} \!\! \d R_2 \int_{O(m)} \!\! \d R_1 \ \ g(R_2, R_1) \\
& =  \int_{O(m)} \!\! \d R_1 \int_{O(m)} \!\! \d R_2 \ \ g(R_1, R_2)  =  \int_{O(m)^2} g(R) \d R\,.
\end{align*}

To show the second statement, we use the definition of $R_\star$ and \eqref{Eq:InvarianceByStar}
to see that
\begin{align*}
\overline{K}(x^\star,y) &= \average_{O(m)^2} K(|Rx^\star - y|) \d R = \average_{O(m)^2} K(|(Rx^\star - y)^\star|) \d R \\
&= \average_{O(m)^2} K(|(R x^\star)^\star - y^\star|) \d R = \average_{O(m)^2} K(|R_\star x - y^\star|) \d R \\
&= \average_{O(m)^2} K(|Rx - y^\star|) \d R = \overline{K}(x,y^\star)\,.
\end{align*}
As a consequence, we have that $\overline{K}(x^\star,y^\star) = \overline{K}(x,(y^\star)^\star) = \overline{K}(x,y)$.
\end{proof}

To conclude this subsection, we present two alternative expressions for the operator $L_K$ when it acts on doubly radial odd functions. These expressions are suitable in the rest of the paper and also in the forthcoming one \cite{FelipeSanz-Perela:IntegroDifferentialII}, since the integrals appearing in the expression are computed only in $\ocal$, and this is important to prove maximum principle and other properties.

\begin{lemma}
	\label{Lemma:OperatorOddF}
	Let $w$ be a doubly radial function which is odd with respect to the Simons cone. Let $L_K \in \lcal_0(2m,\s,\lambda, \Lambda)$ be a rotation invariant operator and let $L_K^\ocal$ be defined by \eqref{Eq:OperatorOddF}. 
	
	Then, for every $x\in\ocal$,
	$$ 	L_K w (x) = L_K^\ocal w(x).   $$
	Indeed,
	\begin{align*}
	L_K w (x) &= \int_{\ocal} \{w(x) - w(y) \} \overline{K}(x, y) \d y +  \int_{\ocal} \{w(x) + w(y) \} \overline{K}(x, y^\star) \d y \\
	&= \int_{\ocal} \{w(x) - w(y) \} \{\overline{K}(x, y) - \overline{K}(x, y^\star)  \} \d y +  2 w(x) \int_{\ocal} \overline{K}(x, y^\star) \d y \,.
	\end{align*}
	Moreover,
	\begin{equation}
	\label{Eq:ZeroOrderTerm}
		\dfrac{1}{C} \dist(x,\ccal)^{-2\s} \leq \int_{\ocal} \overline{K}(x, y^\star) \d y \leq C \dist(x,\ccal)^{-2\s},
	\end{equation}
	with $C>0$ depending only on $m, \s, \lambda$, and $\Lambda$.
\end{lemma}

\begin{proof}
	The first statement is just a computation. Indeed,  using the change of variables  $\bar{y} = y^\star$ and the odd symmetry of $w$, we see that
	\begin{align*}
	\int_{\ical}  \{w(x) - w(y) \} \overline{K}(x, y)\d y &= \int_{\ocal} \{w(x) - w(y^\star) \}\overline{K}(x, y^\star)\d y \\
	&= \int_{\ocal} \{w(x) + w(y) \}\overline{K}(x, y^\star)\d y\,.
	\end{align*}
	Hence,
	\begin{align*}
	L_K w (x) &= \int_{\R^{2m}}  \{w(x) - w(y) \} \overline{K}(x, y)\d y \\
	&= \int_{\ocal}  \{w(x) - w(y) \} \overline{K}(x, y)\d y +\int_{\ical}  \{w(x) - w(y) \} \overline{K}(x, y)\d y \\
	&= \int_{\ocal} \{w(x) - w(y) \} \overline{K}(x, y) \d y +  \int_{\ocal} \{w(x) + w(y) \} \overline{K}(x, y^\star) \d y \,.
	\end{align*}
	By adding and subtracting $w(x)\overline{K}(x, y^\star)$ in the last integrand, we immediately deduce
	$$
	L_K w (x) =  \int_{\ocal} \{w(x) - w(y) \} \{\overline{K}(x, y) - \overline{K}(x, y^\star)  \} \d y +  2 w(x) \int_{\ocal} \overline{K}(x, y^\star) \d y\,.
	$$
	Note that we can add and subtract the term $w(x)\overline{K}(x, y^\star)$  since it is integrable with respect to $y$ in $\ocal$. This is a consequence of \eqref{Eq:ZeroOrderTerm}.
	
	Let us show now \eqref{Eq:ZeroOrderTerm}. In the following arguments we will use the letters $C$ and $c$ to denote positive constants, depending only on $m, \s, \lambda$, and $\Lambda$, that may change its value in each inequality. 
	
	The upper bound in \eqref{Eq:ZeroOrderTerm} is the simplest one since we only need to use the ellipticity of the kernel and the inclusion $\ical \subset \{y\in\R^{2m}:|x-y|\geq \dist(x,\ccal)\}$ for every $x\in \ocal$. Indeed,
	\begin{align*}
	\int_{\ocal} \overline{K}(x, y^\star) \d y &=  \int_{\ocal} K(|x-y^\star|) \d y = \int_{\ical} K(|x-y|) \d y \leq \int_{|x-y|\geq \dist(x,\ccal)} K(|x-y|) \d y \\
	&\leq C \int_{|x-y|\geq \dist(x,\ccal)} |x-y|^{-2m-2\s} \d y = C \int_{\dist(x,\ccal)}^\infty \rho^{-1-2s} \d \rho \\
	&= C \dist(x,\ccal)^{-2s}\,.
	\end{align*}

	In order to prove the lower bound in \eqref{Eq:ZeroOrderTerm}, let us define $x_0 = x/\dist(x,\ccal)$. Note that $\dist (x_0, \ccal) = 1$. Then, we have
	\begin{align*}
	\int_{\ocal} \overline{K}(x, y^\star) \d y &=  \int_{\ical} K(|x-y|) \d y \\
	&\geq c \int_{\ical} |x-y|^{-2m-2\s} \d y = c \int_{\ical} |x_0\dist(x,\ccal)-y|^{-2m-2\s} \d y \\
	&= c \,\dist(x,\ccal)^{-2s}\, \int_{\ical} |x_0-\tilde{y}|^{-2m-2\s} \d \tilde{y}\,.
	\end{align*}
	
	To conclude the proof, we claim that
	$$ \int_{\ical} |x_0-y|^{-2m-2\s} \d y \geq c>0\quad \textrm{for every } x_0 \textrm{ such that } \dist(x_0,\ccal) = 1,$$
	with a constant $c$ independent of $x_0$. To establish this claim we will use three facts. The first one is that, since $\dist(x_0,\ccal) = 1$, there exists a point $\overline{x_0}\in \ccal$ realizing this distance. Thus, we can easily deduce that $B_{3k+3}(\overline{x_0}) \setminus \overline{ B_{3k+2}(\overline{x_0})} \subset  B_{3k+4}(x_0)\setminus \overline{ B_{3k+1}(x_0)}$ for every $k\geq 0$. The second fact is the identity $\ical = \ical \cap B_1^c(x_0)$, that also follows from $\dist(x_0,\ccal) = 1$. Finally, the last fact is a property of the Simons cone: $|B_R(z) \cap \ical| = 1/2 |B_R|$ for every $z\in \ccal$ (see Lemma 2.5 in \cite{Felipe-Sanz-Perela:SaddleFractional} for the proof). Combining these three facts, we get
	\begin{align*}
	\int_{\ical} |x_0-y|^{-2m-2\s} \d y &= \sum_{k=0}^\infty \int_{I\cap \left( B_{3k+4}(x_0)\setminus B_{3k+1}(x_0)\right)} |x_0-y|^{-2m-2\s} \d y\\
	&\geq \sum_{k=0}^\infty \int_{I\cap \left( B_{3k+4}(x_0)\setminus B_{3k+1}(x_0)\right)} (3k+4)^{-2m-2\s} \d y \\
	&= \sum_{k=0}^\infty (3k+4)^{-2m-2\s} |I\cap \left( B_{3k+4}(x_0)\setminus B_{3k+1}(x_0)\right)|\\
	&\geq \sum_{k=0}^\infty (3k+4)^{-2m-2\s} |I\cap \left( B_{3k+3}(\overline{x_0}) \setminus B_{3k+2}(\overline{x_0}) \right)|\\
	&= c \sum_{k=0}^\infty (3k+4)^{-2m-2\s} \left\{(3k+3)^{2m} -(3k+2)^{2m}\right\}\\
	&\geq c \sum_{k=0}^\infty (3k+4)^{-1-2\s} = c.
	\end{align*}
\end{proof}


%%%%%%%%%%%%%%%%%%%%%%%%%%%%%%%%%%%%%%%%%%%%%%%%%%%%%%%%%%%%%%%%%%%%%%%%
\subsection{Necessary and sufficient conditions for ellipticity}
%%%%%%%%%%%%%%%%%%%%%%%%%%%%%%%%%%%%%%%%%%%%%%%%%%%%%%%%%%%%%%%%%%%%%%%%




In this subsection, we establish Theorem~\ref{Th:SufficientNecessaryConditions}. As we have mentioned in the introduction, the kernel inequality \eqref{Eq:KernelInequality} is crucial in the rest of the results of this paper, as well as in the ones in \cite{FelipeSanz-Perela:IntegroDifferentialII}. We will see in the next subsection that this inequality guarantees that the operator $L_K$ has a maximum principle for odd functions (see Proposition~\ref{Prop:WeakMaximumPrincipleForOddFunctions}).

First, we give a sufficient condition on a radially symmetric kernel $K$ so that $\overline{K}$ satisfies \eqref{Eq:KernelInequality}. It is the following result.

\begin{proposition}
\label{Prop:KernelInequalitySufficientCondition} 
Let $K:(0,+\infty) \to \R$ define a positive radially symmetric kernel $K(|x-y|)$ in $\R^{2m}$. Define $\overline{K} : \R^{2m}\times \R^{2m} \to \R$ by \eqref{Eq:KbarDef'}. Assume that $K(\sqrt{\cdot})$ is strictly convex in $(0,+\infty)$. Then, the associated kernel $\overline{K}$ satisfies
	\begin{equation}
	\label{Eq:KernelInequalityBis}
	\overline{K}(x,y) > \overline{K}(x, y^\star) \quad \text{ for every }x,y \in \ocal\,.
	\end{equation}
\end{proposition}

\begin{proof}
Since $\overline{K}$ is invariant by $O(m)^2$, it is enough to show \eqref{Eq:KernelInequalityBis} for points $x, y\in \ocal$ of the form $x = (|x'|e, |x''|e)$ and $y = (|y'|e, |y''|e)$, with $e \in \Sph^{m-1}$ an arbitrary unitary vector.

Now, define
\begin{equation}
\label{Eq:DefQ}
	\begin{split}
	Q_1 &:= \setcond{R = (R_1,R_2) \in O(m)^2}{e\cdot R_1 e > |e\cdot R_2 e|},\\
	Q_2 &:= \setcond{R = (R_1,R_2) \in O(m)^2}{e\cdot R_2 e > |e\cdot R_1 e|} = (Q_1)_\star,\\
	Q_3 &:= \setcond{R = (R_1,R_2) \in O(m)^2}{e\cdot R_1 e < -|e\cdot R_2 e|} = -Q_1,\\
	Q_4 &:= \setcond{R = (R_1,R_2) \in O(m)^2}{e\cdot R_2 e < - |e\cdot R_1 e|} = -(Q_1)_\star.
	\end{split}
\end{equation}
Recall that given $R=(R_1,R_2)\in O(m)^2$, then $R_\star=(R_2,R_1)\in O(m)^2$ ---see \eqref{Eq:DefRStar}. Moreover, note that the sets $Q_i$ are disjoint, have the same measure and cover all $O(m)^2$ up to a set of measure zero. 

%Moreover, they satisfy the following:
%\begin{itemize}
%\item If $R = (R_1, R_2)\in Q_2$, then $R_\star = (R_2, R_1) \in Q_1$.
%\item If $R = (R_1, R_2)\in Q_3$, then $-R = (-R_1, -R_2) \in Q_1$.
%\item If $R = (R_1, R_2)\in Q_4$, then $-R_\star = (-R_2, -R_1) \in Q_1$.
%\end{itemize}
Therefore,
\begin{align*}
4\overline{K} (x, y) &= 4\average_{O(m)^2} K(|x - R y|)\d R \\
& = \average_{Q_1} K(|x - R y|)\d R + \average_{Q_2} K(|x - R y|)\d R \\
& \quad \quad
+ \average_{Q_3} K(|x - R y|)\d R +
\average_{Q_4} K(|x - R y|)\d R \\
&= \average_{Q_1} \{K(|x - R y|) + K(|x + R y|) \\
&\quad \quad + K(|x - R_\star y|) + K(|x + R_\star y|)\}\d R
\end{align*}
and
\begin{align*}
4\overline{K} (x, y^\star) &= 4\average_{O(m)^2} K(|x - R y^\star|)\d R \\
& = \average_{Q_1} \{K(|x - R y^\star|) + K(|x + R y^\star|) \\
&\quad \quad + K(|x - R_\star y^\star|) + K(|x + R_\star y^\star|)\}\d R.
\end{align*}
Thus, if we prove
\begin{equation}
\label{Eq:InequalityIntegrandKernelInequalityProof}
\begin{split}
K(|x - R y|) + K(|x + R y|) + K(|x - R_\star y|) + K(|x + R_\star y|)
\quad \quad \quad \quad \quad \quad \quad \quad
\\
\geq
K(|x - R y^\star|) + K(|x + R y^\star|)+K(|x - R_\star y^\star|) + K(|x + R_\star y^\star|)\,,
\end{split}
\end{equation}
for every $R\in Q_1$, we immediately deduce \eqref{Eq:KernelInequalityBis} with a non strict inequality. To see that it is indeed a strict one, we will show that the inequality in \eqref{Eq:InequalityIntegrandKernelInequalityProof} is strict for every $R \in Q_1$.


For a short notation, we call
\begin{equation}
	\label{Eq:DefAlphaBeta}
	\alpha := e \cdot R_1 e  \quad \text{ and } \quad \beta := e \cdot R_2 e\,.
\end{equation}
Now, note that since  $x = (|x'|e, |x''|e)$ and $y = (|y'|e, |y''|e)$, we have
\begin{align*}
|x \pm Ry|^2&= |x' \pm R_1y'|^2 + |x'' \pm R_2y''|^2 \\
&= |x'|^2 + |y'|^2 \pm 2 x'\cdot R_1 y' +  |x''|^2 + |y''|^2 \pm 2 x''\cdot R_2 y''\\
&= |x|^2 + |y|^2 \pm 2 |x'||y'| \alpha \pm 2 |x''||y''| \beta.
\end{align*}
Similarly,
$$
|x \pm R_\star y|^2 =  |x|^2 + |y|^2 \pm 2 |x'||y'| \beta \pm 2 |x''||y''|\alpha,
$$
$$
|x \pm R y^\star|^2 =  |x|^2 + |y|^2 \pm 2 |x'||y''| \alpha \pm 2 |x''||y'|\beta,
$$
and
$$
|x \pm R_\star y^\star|^2 = |x|^2 + |y|^2 \pm 2 |x'||y''| \beta \pm 2 |x''||y'| \alpha.
$$

We define now
$$
g(\tau) := K \bpar{\sqrt{|x|^2 + |y|^2 + 2 \tau }} + K \bpar{\sqrt{|x|^2 + |y|^2 - 2 \tau}}.
$$
Thus, proving \eqref{Eq:InequalityIntegrandKernelInequalityProof} is equivalent to show that, for every $\alpha$, $\beta \in [-1,1]$ such that $\alpha > |\beta|$, it holds
\begin{equation}
\label{Eq:InequalityIntegrandKernelInequalityProof2}
\begin{split}
g\Big(|x'||y'| \alpha + |x''||y''| \beta \Big)
+ g\Big(|x'||y'| \beta + |x''||y''| \alpha \Big) \hspace{2cm}
\\ \geq
g\Big(|x'||y''| \alpha + |x''||y'|\beta \Big)
+ g\Big(|x'||y''| \beta + |x''||y'| \alpha \Big)\,.
\end{split}
\end{equation}

Let
$$
\begin{array}{cc}
A_{\alpha,\beta} := |x'||y'|  \alpha + |x''||y''|\beta \,, &
B_{\alpha,\beta} := |x'||y''| \alpha + |x''||y'| \beta \,, \\
C_{\alpha,\beta} := |x''||y'| \alpha + |x'||y''| \beta \,, &
D_{\alpha,\beta} := |x''||y''|\alpha + |x'||y'|  \beta \,.
\end{array}
$$
With this notation and taking into account that $g$ is even,
\eqref{Eq:InequalityIntegrandKernelInequalityProof2} is equivalent to
\begin{equation}
\label{Eq:InequalityIntegrandKernelInequalityProof3}
g(|A_{\alpha,\beta}|) + g(|D_{\alpha,\beta}|) \geq g(|C_{\alpha,\beta}|) + g(|B_{\alpha,\beta}|)\,,
\end{equation}
for every $\alpha$, $\beta \in [-1,1]$ such that $\alpha > |\beta|$. Note that $g$ is defined in the open interval $I = (-(|x|^2 + |y|^2)/2,\ (|x|^2 + |y|^2)/2)$ and that $A_{\alpha,\beta}$, $B_{\alpha,\beta}$, $C_{\alpha,\beta}$, $D_{\alpha,\beta} \in I$.

To show \eqref{Eq:InequalityIntegrandKernelInequalityProof3}, we use Proposition~\ref{Prop:EquivalenceK(sqrt)Convex<->Inequality} of the Appendix~\ref{Sec:AuxiliaryResults}. There, it is stated that in order to establish \eqref{Eq:InequalityIntegrandKernelInequalityProof3} it is enough to check that
$$
\begin{cases}
|A_{\alpha,\beta}| \geq |B_{\alpha,\beta}|,\ \ |A_{\alpha,\beta}| \geq |C_{\alpha,\beta}|, \ \ |A_{\alpha,\beta}| \geq |D_{\alpha,\beta}|\,, \\
|A_{\alpha,\beta}| + |D_{\alpha,\beta}| \geq |B_{\alpha,\beta}| + |C_{\alpha,\beta}|\,.
\end{cases}
$$
The verification of these inequalities is a simple but tedious computation and it is presented in Appendix~\ref{Sec:AuxiliaryResults2} ---see point (1) of Lemma~\ref{Lemma:ComputationABCD}. Once this is proved, we deduce \eqref{Eq:InequalityIntegrandKernelInequalityProof3} by Proposition~\ref{Prop:EquivalenceK(sqrt)Convex<->Inequality}.

To finish, we see that the inequality in \eqref{Eq:InequalityIntegrandKernelInequalityProof3} is always strict for every $\alpha$, $\beta \in [-1,1]$ such that $\alpha > |\beta|$ (that corresponds to $Q_1$). By contradiction, assume that equality holds in \eqref{Eq:InequalityIntegrandKernelInequalityProof3}. Thus, by Proposition~\ref{Prop:EquivalenceK(sqrt)Convex<->Inequality}, if follows that the sets $\{|A_{\alpha,\beta}|, |D_{\alpha,\beta}|\} $ and $\{|B_{\alpha,\beta}|,|C_{\alpha,\beta}|\}$ coincide. This fact and point (2) of Lemma~\ref{Lemma:ComputationABCD} yield $\alpha = \beta = 0$, a contradiction. Thus, the inequality in \eqref{Eq:InequalityIntegrandKernelInequalityProof3} is strict, as well as the inequality in \eqref{Eq:InequalityIntegrandKernelInequalityProof}. This leads to \eqref{Eq:KernelInequalityBis}.
\end{proof}


Now, we give a necessary condition on the kernel $K$ so that inequality \eqref{Eq:KernelInequality} holds.

\begin{proposition}
\label{Prop:KernelInequalityNecessaryCondition} Let $K:(0,+\infty) \to \R$ define a positive radially symmetric kernel $K(|x-y|)$ in $\R^{2m}$. Define $\overline{K} : \R^{2m}\times \R^{2m} \to \R$ by \eqref{Eq:KbarDef'}. 

If
\begin{equation}
\label{Eq:KernelInequalityAE}
\overline{K}(x,y) > \overline{K}(x, y^\star) \quad \text{ for almost every }x,y \in \mathcal{O}\,,
\end{equation}
then $K(\sqrt{\cdot})$ cannot be concave in any interval $I\subset [0,+\infty)$.
\end{proposition}

\begin{proof}
We prove it by contraposition. In fact, we will show that if there exists an interval where $K(\sqrt{\cdot})$ is concave, then we can find an open set in $\ocal \times \ocal$ with positive measure where \eqref{Eq:KernelInequalityAE} is not satisfied.

Let $\ell_2>\ell_1>0$ be such that $K(\sqrt{\cdot})$ is concave in $(\ell_1,\ell_2)$ and define the set $\Omega_{\ell_1,\ell_2}\subset \R^{4m}$ as the points $(x,y)\in \ocal\times \ocal$ satisfying
\begin{equation}
\label{Eq:OmegaSetDefinition}
\beqc{\PDEsystem}
(|x'|-|y'|)^2+(|x''|-|y''|)^2&>&\ell_1,\\
(|x'|+|y'|)^2+(|x''|+|y''|)^2&<&\ell_2.
\eeqc
\end{equation}

First, it is easy to see that $\Omega_{\ell_1,\ell_2}$ is a nonempty open set. In fact, points of the form $(x',0,y',0)\in (\R^m)^4$ such that $(|x'|-|y'|)^2>\ell_1$ and $(|x'|+|y'|)^2 <\ell_2$ belong to $\Omega_{\ell_1,\ell_2}$. Then, if we prove that $\overline{K}(x,y) \leq \overline{K}(x, y^\star)$ in $\Omega_{\ell_1,\ell_2}$ we are done.

Given $x,y\in \Omega_{\ell_1,\ell_2}$, we are going to show, as in the previous proof, that
\begin{equation}
\label{Eq:InequalityIntegrandKernelInequalityProof4}
\begin{split}
K(|x - R y|) + K(|x + R y|) + K(|x - R_\star y|) + K(|x + R_\star y|)
\quad \quad \quad \quad \quad \quad
\\
\leq
K(|x - R y^\star|) + K(|x + R y^\star|)+K(|x - R_\star y^\star|) + K(|x + R_\star y^\star|)\,, 
\end{split}
\end{equation}
for any $R\in Q_1$, where $Q_1$ is defined in \eqref{Eq:DefQ} (see the proof of Proposition~\ref{Prop:KernelInequalitySufficientCondition}). As before, we can assume that $x$ and $y$ are of the form $x = (|x'|e, |x''|e)$ and $y = (|y'|e, |y''|e)$, with $e \in \Sph^{m-1}$ an arbitrary unitary vector. Then, by defining $\alpha$ and $\beta$ as in \eqref{Eq:DefAlphaBeta}, we see that proving \eqref{Eq:InequalityIntegrandKernelInequalityProof4} is equivalent to establish that
\begin{equation}
\label{Eq:InequalityIntegrandKernelInequalityProof5}
g(A_{\alpha,\beta}) + g(D_{\alpha,\beta}) \leq g(B_{\alpha,\beta}) + g(C_{\alpha,\beta})\,,
\end{equation}
for every $\alpha, \beta \in [-1,1]$ such that $\alpha>|\beta|$, where
$$
\begin{array}{cc}
A_{\alpha,\beta} = |x'||y'|  \alpha + |x''||y''|\beta \,, &
B_{\alpha,\beta} = |x'||y''| \alpha + |x''||y'| \beta \,, \\
C_{\alpha,\beta} = |x''||y'| \alpha + |x'||y''| \beta \,, &
D_{\alpha,\beta} = |x''||y''|\alpha + |x'||y'|  \beta \,.
\end{array}
$$
and
\begin{align*}
g(\tau) &= K\left( \sqrt{|x|^2+|y|^2+2\tau} \right) + K\left( \sqrt{|x|^2+|y|^2-2\tau} \right).
\end{align*}




Now, by \eqref{Eq:OmegaSetDefinition}, we have $\ell_1 < |x|^2+|y|^2 <\ell_2$. As a consequence of this and the concavity of $K(\sqrt{\cdot})$ in $(\ell_1,\ell_2)$, it is easy to see (by using Lemma~\ref{Lemma:ConvexFunctions} in the Appendix~\ref{Sec:AuxiliaryResults}) that $g$ is concave in $ \left( -\overline{\ell}, \overline{\ell}\right) $, and decreasing in $(0,\overline{\ell})$, where 
$$
\overline{\ell} := \min{\left\{\frac{\ell_2-|x|^2-|y|^2}{2},\frac{|x|^2+|y|^2-\ell_1}{2}\right\}}.$$
Note that, since $\ell_1 < |x|^2+|y|^2 <\ell_2$, we have $\overline{\ell}>0$.


We claim that $A_{\alpha,\beta}, B_{\alpha,\beta}, C_{\alpha,\beta}$ and $D_{\alpha,\beta}$ belong to $(-\overline{\ell},\overline{\ell})$ for every $\alpha, \beta \in [-1,1]$ such that $\alpha>|\beta|$. Indeed, it is easy to check that for every $\alpha, \beta \in [-1,1]$ such that $\alpha>|\beta|$, the numbers $A_{\alpha,\beta}, B_{\alpha,\beta}, C_{\alpha,\beta}$ and $D_{\alpha,\beta}$ belong to the open interval $(-|x'||y'|-|x''||y''|,|x'||y'|+|x''||y''|)$. Furthermore, since $x,y \in \Omega_{\ell_1,\ell_2}$, we obtain from \eqref{Eq:OmegaSetDefinition} that
$$
\beqc{\PDEsystem}
|x'||y'|+|x''||y''|&<&\dfrac{\ell_2-|x|^2-|y|^2}{2}\\
|x'||y'|+|x''||y''|&<&\dfrac{|x|^2+|y|^2-\ell_1}{2}
\eeqc 
$$
and thus $ |x'||y'|+|x''||y''|<\overline{\ell}$ and the claim is proved.

Finally, by applying Lemma~\ref{Lemma:ConvexFunctions} to the function $-g$ in $(0,\overline{\ell})$ (using again  point (1) of Lemma~\ref{Lemma:ComputationABCD}), we obtain that inequality \eqref{Eq:InequalityIntegrandKernelInequalityProof5} is satisfied, which yields \eqref{Eq:InequalityIntegrandKernelInequalityProof4}. Finally, by integrating \eqref{Eq:InequalityIntegrandKernelInequalityProof4} with respect to all the rotations $R\in Q_1$ we get $$ \overline{K}(x,y) \leq \overline{K}(x, y^\star),$$ for every $(x,y)\in \Omega_{\ell_1,\ell_2}$, contradicting \eqref{Eq:KernelInequalityAE}.
\end{proof}

From the two previous results, Theorem~\ref{Th:SufficientNecessaryConditions} follows immediately.

\begin{proof}[Proof of Theorem~\ref{Th:SufficientNecessaryConditions}]
	The first statement is exactly the same as Proposition~\ref{Prop:KernelInequalitySufficientCondition}. Assume now that $K$ is a $C^2$ function and that \eqref{Eq:KernelInequality} holds. Then, by Proposition~\ref{Prop:KernelInequalityNecessaryCondition}, $K(\sqrt{\cdot})$ is not concave in any interval of $[0,+\infty)$. By the regularity of $K$, this automatically yields that $K(\sqrt{\cdot})$ is strictly convex.
\end{proof}

\begin{remark}
	Note that a priori we cannot relax the $C^2$ assumption in the necessary condition of Theorem~\ref{Th:SufficientNecessaryConditions}, since there are $C^1$ functions that are neither convex nor concave in any interval (they can be constructed as a primitive of a Weierstrass function, whose graph is a non rectifiable curve with fractal dimension). Besides these ``exotic'' examples, there are also simple radially symmetric kernels $K$ that are not $C^1$ for which we do not know if the kernel inequality \eqref{Eq:KernelInequality} holds. For instance, given $0<\s<1$, if we consider the kernel
	$$ K(\tau) = \frac{1}{\tau^{2m+2\s}} \chi_{(0,1)}(\tau)+\frac{1}{10\tau^{2m+2\s}-9} \chi_{[1,+\infty)}(\tau), $$
	it is easy to check that $K$ is continuous and decreasing but $K(\sqrt{\tau})$ is not convex in $(0,+\infty)$ even though it does not have any interval of concavity (see Figure~\ref{Fig:Grafica}).
	\begin{figure}
	\centering
	\begin{tikzpicture}
	\begin{axis}[
	axis x line=middle, axis y line=left,
	every axis x label/.style={at={(current axis.right of origin)},anchor=west},
	every axis y label/.style={at={(current axis.north west)},above=2mm},
	ymin=0, ymax=3, ylabel=$K(\sqrt{\tau})$,
	xmin=0.4, xmax=2, xlabel=$\tau$
	]
	\addplot[domain=0.4:1, samples=100] {1/(x^1.5)};
	\addplot[domain=1:2, samples=100] {0.1/(x^1.5-0.9)};
	\end{axis}
	\end{tikzpicture}
	\caption{An example of kernel $K(\sqrt{\tau})$ ($m=1$ and $\s=1/2$) which is not strictly convex in $(0,+\infty)$ but does not have any interval of concavity. }
	\label{Fig:Grafica}
	\end{figure}
\end{remark}

%%%%%%%%%%%%%%%%%%%%%%%%%%%%%%%%%%%%%%%%%%%%%%%%%%%%%%%%%%%%%%%%%%%%%%
%%%%%%%%%%%%%%%%%%%%%%%%%%%%%%%%%%%%%%%%%%%%%%%%%%%%%%%%%%%%%%%%%%%%%%
\subsection{Maximum principles for doubly radial odd functions}
%%%%%%%%%%%%%%%%%%%%%%%%%%%%%%%%%%%%%%%%%%%%%%%%%%%%%%%%%%%%%%%%%%%%%%
%%%%%%%%%%%%%%%%%%%%%%%%%%%%%%%%%%%%%%%%%%%%%%%%%%%%%%%%%%%%%%%%%%%%%%

In this subsection we prove a weak and a strong maximum principles for doubly radial functions that are odd with respect to the Simons cone. The formulation of these maximum principles is very suitable since all the hypotheses refer to the set $\ocal$ and not $\R^{2m}$. The key ingredient in the proofs is the kernel inequality \eqref{Eq:KernelInequality}.


We first establish a weak maximum principle.

\begin{proposition}[Weak maximum principle for odd functions with respect to $\ccal$]
\label{Prop:WeakMaximumPrincipleForOddFunctions} Let $\Omega \subset \ocal$ an open set and let $L_K  \in \lcal_\star (2m,  \s)$.  Let $u\in C^{\alpha}(\Omega)\cap L^\infty(\R^{2m})$, with $\alpha > 2\s$, be a doubly radial function which is odd with respect to the Simons cone. Assume that
$$
\beqc{\PDEsystem}
L_K u + c(x) u & \geq & 0 & \text{ in } \Omega\,,\\
u & \geq & 0 & \text{ in } \ocal \setminus \Omega\,,
\eeqc
$$
with $c \geq 0$, and that either
$$
\Omega \text{ is bounded} \quad \text{ or } \liminf_{x \in \ocal,\,|x|\to +\infty} u(x) \geq 0\,.
$$
Then, $u \geq 0$ in $\Omega$.
\end{proposition}

\begin{proof}
By contradiction, assume that $u$ takes negative values in $\Omega$. Under the hypotheses we are assuming, a negative minimum must be achieved. Thus, there exists $x_0\in \Omega$ such that
$$
u(x_0) = \min_{\Omega} u =: m < 0\,.
$$
Then, using the expression of $L_K$ for odd functions (see Lemma~\ref{Lemma:OperatorOddF}), we have
$$
L_K u (x_0) = \int_{\ocal} \{m - u(y) \} \{\overline{K}(x_0, y) - \overline{K}(x_0, y^\star)  \} \d y +  2 m \int_{\ocal} \overline{K}(x_0, y^\star) \d y\,.
$$
Now, since $m - u(y) \leq 0$ in $\ocal$, $m<0$, $c\geq 0$ and $\overline{K}(x_0, y) \geq \overline{K}(x_0, y^\star)>0$ (since $L_K\in \lcal_\star$), we get
$$
0 \leq L_K  u(x_0) + c(x_0) u(x_0) \leq m \left(2\int_{\ocal} \overline{K}(x_0, y^\star) \d y + c(x_0)\right)  < 0\,,
$$
a contradiction.
\end{proof}

\begin{remark}
Note that since the operator $L_K$ includes itself a positive zero order term in addition to the integro-differential part, the condition $c\geq 0$ in the previous proposition can be lightly relaxed. Indeed, if we follow the proof of the result, we can deduce that the hypothesis on $c$ that we can assume is 
$$ c(x) > -2\int_{\ocal} \overline{K}(x, y^\star) \d y. $$
This hypothesis seems hard to be checked for applications apart from the case $c\geq 0$. Nevertheless, recall that by Lemma~\ref{Lemma:OperatorOddF} we have an explicit lower bound for the quantity $ \int_{\ocal} \overline{K}(x, y^\star) \d y $ in terms of the function $\dist(x,\ccal)$ that could be used to check the previous condition.
\end{remark}

The following is a strong maximum principle for odd functions.

\begin{proposition}[Strong maximum principle for odd functions with respect to $\ccal$]
\label{Prop:StrongMaximumPrincipleForOddFunctions} Let $\Omega \subset \ocal$ an open set and let $L_K  \in \lcal_\star (2m,  \s)$.  Let $u\in C^{\alpha}(\Omega)\cap L^\infty(\R^{2m})$, with $\alpha > 2\s$, be a doubly radial function which is odd with respect to the Simons cone. Assume that $L_K u + c(x) u\geq 0$ in $\Omega$, with $c(x)$ any function, and that $u\geq 0$ in $\ocal$. Then, either $u\equiv 0$ or $u > 0$ in $\Omega$.
\end{proposition}

\begin{proof}
Assume that $u \not \equiv 0$. We shall prove that $u > 0$ in $\Omega$. By contradiction, assume that there exists a point $x_0\in \Omega$ such that $u(x_0)= 0$. Then, using the expression of $L_K $ for odd functions given in Lemma~\ref{Lemma:OperatorOddF}, the kernel inequality \eqref{Eq:KernelInequality} and the fact that $u\geq 0$ in $\ocal$, we obtain
$$
0 \leq L_K u(x_0) + c(x_0) u(x_0) = - \int_{\ocal} u(y)\big \{\overline{K}(x_0, y) - \overline{K}(x_0, y^\star) \big \}\d y < 0\,,
$$
a contradiction.
\end{proof}


%%%%%%%%%%%%%%%%%%%%%%%%%%%%%%%%%%%%%%%%%%%%%%%%%%%%%%%%%%%%%%%%%%%%%%
%%%%%%%%%%%%%%%%%%%%%%%%%%%%%%%%%%%%%%%%%%%%%%%%%%%%%%%%%%%%%%%%%%%%%%


%%%%%%%%%%%%%%%%%%%%%%%%
\section{The energy functional for doubly radial odd functions}
%%%%%%%%%%%%%%%%%%%%%%%%
\label{Sec:EnergyForOddF}


This section is devoted to the energy functional associated to the semilinear equation \eqref{Eq:NonlocalAllenCahn}. We first define appropriately the functional spaces where we are going to apply classic techniques of calculus of variations. Next we rewrite the energy in terms of the new kernel $\overline{K}$ and we give an alternative expression for the energy of doubly radial odd functions. Finally, we establish some results that are useful when using variational techniques, and that will be exploited in the next section.


Let us start by defining the functional spaces that we are going to consider in the rest of the paper. Given a set $\Omega \subset \R^n$ and a translation invariant and positive kernel $K$ satisfying \eqref{Eq:Symmetry&IntegrabilityOfK}, we define the space
$$
\H^K(\Omega) := \setcond{w \in L^2(\Omega)}{[w]^2_{\H^K(\Omega)} < + \infty},
$$
where
$$
[w]^2_{\H^K(\Omega)} := \dfrac{1}{2}\int\int_{(\R^{n})^2 \setminus (\R^n\setminus\Omega)^2} |w(x) - w(y)|^2 K(x-y) \d x \d y\,.
$$
We also define
\begin{align*}
	\H^K_0(\Omega) &:= \setcond{w \in \H^K(\Omega)}{ w = 0 \quad \textrm{a.e. in } \R^n \setminus \Omega} \\
	&\ = \setcond{w \in \H^K(\R^n)}{ w = 0 \quad \textrm{a.e. in } \R^n \setminus \Omega}.
\end{align*}

Assume that $\Omega \subset \R^{2m}$ is a set of double revolution. Then, we define
$$
\widetilde{\H}^K(\Omega) := \setcond{w \in \H^K(\Omega)}{w \textrm{ is doubly radial a.e.}}.
$$
and
$$
\widetilde{\H}^K_0(\Omega) := \setcond{w \in \H^K_0(\Omega)}{w \textrm{ is doubly radial a.e.}}.
$$
We will add the subscript `odd' and `even' to these spaces to consider only functions that are odd (respectively even) with respect to the Simons cone.


\begin{remark}
\label{Remark:DecompositionHK}
If $\widetilde{\H}^K_0(\Omega)$ is equipped with the scalar product
$$
\langle v,w \rangle_{\widetilde{\H}^K_0(\Omega)} := \dfrac{1}{2}\int_{\R^{2m}} \int_{\R^{2m}}  \big(v(x) - v(y)\big)\big(w(x) - w(y)\big) K(x-y) \d x \d y\,,
$$
then, it is easy to check that $\widetilde{\H}^K_0(\Omega)$ can be decomposed as the orthogonal
direct sum of $\widetilde{\H}^K_{0,\, \mathrm{even}}(\Omega)$ and $\widetilde{\H}^K_{0,\,
\mathrm{odd}}(\Omega)$.
\end{remark}

Note that when $K$ satisfies \eqref{Eq:Ellipticity}, then $\H^K_0 (\Omega) = \H^\s_0 (\Omega)$,
which is the space associated to the kernel of the fractional Laplacian, $K(z) = c_{n,\s} |z|^{-n-2\s}$.
Furthermore, $\H^\s(\Omega) \subset H^\s(\Omega)$, the usual fractional Sobolev space (see
\cite{HitchhikerGuide}).  For more comments on this, see~\cite{CozziPassalacqua}, and the references therein.

Once presented the functional setting of our problem, we proceed with the study of the energy functional associated to equation \eqref{Eq:NonlocalAllenCahn}. 


Given a kernel $K$ satisfying \eqref{Eq:Symmetry&IntegrabilityOfK} and a potential $G$, and given a function $w\in \H^K(\Omega)$, with $\Omega\subset \R^{n}$, we write the energy defined in \eqref{Eq:Energy} as
$$
\ecal(w, \Omega) = \ecal_\mathrm{K}(w,\Omega) + \ecal_\mathrm{P}(w,\Omega)\,,
$$
where
$$
\ecal_\mathrm{K}(w, \Omega) := \dfrac{1}{2} [w]^2_{\H^K(\Omega)} \quad \text{ and } \quad  \ecal_\mathrm{P}(w, \Omega) := \int_{\Omega} G(w) \d x
\,.
$$
We will call $\ecal_\mathrm{K}$ and $\ecal_\mathrm{P}$ the \emph{kinetic} and \emph{potential} energy respectively.


Note that, for functions $w\in \H^K_0(\Omega)$, it holds $\ecal_\mathrm{K}(w,\Omega) = \ecal_\mathrm{K}(w,\R^n)$. Moreover, if $G\geq 0$, the energy satisfies $\ecal(w, \Omega) \leq \ecal(w, \Omega')$  whenever $ \Omega \subset \Omega'$.

Sometimes it is useful to rewrite the kinetic energy as
\begin{equation}
	\label{Eq:KineticEnergyInteractions}
	\begin{split}
	\ecal_\mathrm{K}(w, \Omega) := \dfrac{1}{4} \left \{ \int_\Omega \int_\Omega |w(x) - w(y)|^2 K(x-y) \d x \d y \right. \qquad \qquad \\
	+\left. 2 \int_\Omega \int_{\R^n \setminus \Omega} |w(x) - w(y)|^2 K(x-y) \d x \d y \right \}.
	\end{split}	
\end{equation}
Roughly speaking, we have split the kinetic energy into two parts: ``interactions inside-inside'' and ``interactions inside-outside''. Our goal is to rewrite the kinetic energy in terms of the doubly radial kernel $\overline{K}$ and with integrals computed only in $\ocal$, in the same spirit as in the previous section for the operator $L_K$. In particular, we are interested in finding an expression similar to \eqref{Eq:KineticEnergyInteractions} for the kinetic energy. To do this, we introduce the following notation for the interaction. Given $A$, $B\subset \ocal$ sets of double revolution, we define
\begin{equation}
	\label{Eq:DefIw}
	\begin{split}
	I_w(A,B) := 2\int_A  \int_B  \ |w(x)-w(y)|^2 \left\{ \overline{K}(x,y) - \overline{K}(x,y^\star) \right\} \d x \d y  \\
	+\, 4 \int_A  \int_B  \left\{w^2(x)+w^2(y)\right\} \overline{K}(x,y^\star) \d x \d y\,.
	\end{split}
\end{equation}
Thanks to this notation, we rewrite the kinetic energy as follows.


\begin{lemma}
	\label{Lemma:ShortExpressionEnergy}
Let $\Omega\subset \R^{2m}$ be a set of double revolution that is symmetric with respect to the Simons cone, i.e., $\Omega^\star = \Omega$, and let $w\in \widetilde{\H}^K_{0,\, \mathrm{odd}}(\Omega)$. Let $K$ be a radially symmetric kernel satisfying \eqref{Eq:Symmetry&IntegrabilityOfK}. Then, 
\begin{align}
\label{Eq:ShortExpressionEnergy}
\ecal_\mathrm{K}(w, \Omega) = \frac{1}{4} \big \{I_w(\Omega\cap\ocal,\Omega\cap\ocal) +  2I_w(\Omega\cap\ocal,\ocal\setminus\Omega) \big \},
\end{align}
where $I_w(\cdot, \cdot)$ is the interaction defined in \eqref{Eq:DefIw}.
\end{lemma}

\begin{proof}
	
First, in \eqref{Eq:KineticEnergyInteractions} we consider the change $ y = R\tilde{y}$. Since $w$ is doubly radial and $\Omega$ is of double revolution, by taking the average among all $R\in O(m)^2$ as in Lemma~\ref{Lemma:AlternativeOperatorExpression}, we obtain 
$$
\ecal_\mathrm{K}(w, \Omega) = \frac{1}{4} \int_{\Omega} \int_{\Omega} |w(x)-w(y)|^2 \overline{K}(x,y) \d x \d y + \frac{1}{2} \int_{\Omega} \int_{\R^n \setminus \Omega} |w(x)-w(y)|^2 \overline{K}(x,y) \d x \d y\,.
$$
Now we split $\Omega$ into $\Omega \cap \ocal$ and $\Omega \setminus \ocal$. By using the change of variables given by $(\cdot)^\star$ and the symmetries of $\Omega$ and $w$, we get
\begin{align*}
4\ecal_\mathrm{K}(w, \Omega) =   & \hspace{1.3mm} 2 \int_{\Omega\cap \ocal} \int_{\Omega \cap \ocal} |w(x)-w(y)|^2 \overline{K}(x,y) + |w(x)+w(y)|^2 \overline{K}(x,y^\star) \d x \d y  \\
& +  4\int_{\Omega\cap \ocal} \int_{\ocal \setminus \Omega} |w(x)-w(y)|^2 \overline{K}(x,y) + |w(x)+w(y)|^2 \overline{K}(x,y^\star) \d x \d y  \\
= & \hspace{1.3mm} 2 \int_{\Omega\cap \ocal} \int_{\Omega \cap \ocal}  \ |w(x)-w(y)|^2 \left\{ \overline{K}(x,y) - \overline{K}(x,y^\star) \right\} \d x \d y \\
&+4 \int_{\Omega\cap \ocal} \int_{\Omega\cap \ocal} \left\{w^2(x)+w^2(y)\right\} \overline{K}(x,y^\star) \d x \d y\,\\
&+ 4\int_{\Omega\cap \ocal} \int_{\ocal \setminus \Omega}  \ |w(x)-w(y)|^2 \left\{ \overline{K}(x,y) - \overline{K}(x,y^\star) \right\} \d x \d y\\
  & +8 \int_{\Omega\cap \ocal} \int_{\ocal \setminus \Omega}  \left\{w^2(x)+w^2(y)\right\} \overline{K}(x,y^\star) \d x \d y\,\\
= & \hspace{1.3mm}  I_w(\Omega\cap\ocal,\Omega\cap\ocal) + 2 I_w(\Omega\cap\ocal,\ocal\setminus\Omega).
\end{align*}
\end{proof}

Using the previous expression for the energy, we can establish now the following lemma regarding the decrease of the energy under some operations.
\begin{lemma}
\label{Lemma:DecreaseEnergy} 
Let $\Omega\subset \R^{2m}$ be a set of double revolution that is symmetric with respect to the Simons cone, and let $K$ be a kernel such that $L_K \in \lcal_\star$. Given $u\in\widetilde{\H}^K_{\mathrm{odd}}(\Omega)$, we define
\begin{equation*}
v(x) = \begin{cases}
\hspace{3.2mm}|u(x)| \,\,\, &\text{if } \,\,\, x\in\ocal,\\
-|u(x)| \,\,\, &\text{if } \,\,\, x\in\ical\, ,
\end{cases}
\quad 
\text{ and }
\quad
w(x) = \begin{cases}
\hspace{3.6mm}\min\{1,u(x)\} \,\,\, &\text{if } \,\,\, x\in\ocal,\\
\,\,\,\max\{-1,u(x)\} \,\,\, &\text{if } \,\,\, x\in\ical\,.
\end{cases}
\end{equation*}

If $G$ satisfies \eqref{Eq:HipothesesG}, then
$$ \ecal(v,\Omega) \leq \ecal(u,\Omega) \quad 
\text{ and }
\quad \ecal(w,\Omega) \leq \ecal(u,\Omega) \,.  $$
\end{lemma}

\begin{proof}
We first establish the result for $v$. Let us show that  $\ecal_\mathrm{K}(v) \leq \ecal_\mathrm{K}(u)$. Note that $v\in \widetilde{\H}^K_{\mathrm{odd}}(\Omega)$. Thus, by using the expression of the kinetic energy given in \eqref{Eq:ShortExpressionEnergy} and the fact that $\overline{K}(x,y) > \overline{K}(x,y^\star)> 0$ if $x,y\in \ocal$ ---see \eqref{Eq:KernelInequality}---, we only need to check that $|v(x)-v(y)|^2\leq |u(x)-u(y)|^2$ and $v^2(x)\leq u^2(x)$ whenever $x,y\in\ocal$. The first condition follows from the equivalence
$$ \big||u(x)|-|u(y)|\big|^2\leq |u(x)-u(y)|^2 \Longleftrightarrow u(x)u(y) \leq |u(x)u(y)|,  
$$
while the second one is trivial and it is in fact an equality. Concerning the potential energy, since $G$ is an even function we have that $\ecal_\mathrm{P}(v) = \ecal_\mathrm{P}(u)$, and therefore we get the desired result for $v$ by adding the kinetic and potential energies.

We show now the result for $w$. Let us show that  $\ecal_\mathrm{K}(w) \leq \ecal_\mathrm{K}(u)$. As before, $w\in \widetilde{\H}^K_{\mathrm{odd}}(\Omega)$ and thus, in view of \eqref{Eq:ShortExpressionEnergy} and the kernel inequality \eqref{Eq:KernelInequality}, we only need to check that $|w(x)-w(y)|^2\leq |u(x)-u(y)|^2$ and $w^2(x) \leq u^2(x)$ whenever $x,y\in\ocal$. The first inequality is trivial whenever $u(x)\leq 1$ and $u(y)\leq 1$, or $u(x)\geq 1$ and $u(y)\geq 1$. If $u(x)\geq 1$ and $u(y)\leq 1$, then $ |u(x)-u(y)|^2-|w(x)-w(y)|^2 = |u(x)-u(y)|^2-|1-u(y)|^2 = (u(x)-1))^2+2(u(x)-1)(1-u(y)) \geq 0$. The second inequality follows from the fact that $w^2(x) = u^2(x)$ when $u(x)\leq 1$, while $w^2(x) = 1 \leq u^2(x)$ if $u(x)\geq 1$. Concerning the potential energy, since $G$ is such that $G(x)\geq G(1) = G(-1) = 0$ if $|x|\leq 1$, then clearly $\ecal_\mathrm{P}(w) \leq \ecal_\mathrm{P}(u)$, and therefore we get the desired result by adding the kinetic and potential energies.
\end{proof}



Next we present a result that will be used later, and concerns weak solutions to semilinear Dirichlet problems. Its main consequence is that a function $u\in \widetilde{\H}^K_{0}(\Omega)$ that minimizes the energy $\ecal$, but only among doubly radial functions, is actually a weak solution to a semilinear Dirichlet problem in $\Omega$. We remark that to show the following result we do not need to use the kernel $\overline{K}$.

\begin{proposition}
	\label{Prop:WeakSolutionForAllTestFunctions}
	Let $\Omega \subset \R^{2m}$ be a bounded set of double revolution and let $L_K \in \lcal_0$ with kernel $K$ radially symmetric. Let $u\in \widetilde{\H}^K_{0}(\Omega)$ such that
	$$
	\int_{\R^{2m}}\int_{\R^{2m}} \{u(x)-u(y)\}\{\xi(x)-\xi(y)\} K(|x-y|) \d x \d y = \int_{\R^{2m}} f(u(x)) \xi(x) \d x
	$$
	for every $\xi \in C^\infty_c(\Omega)$ that is doubly radial. Then, $u$ is a weak solution to
	$$
	\beqc{\PDEsystem}
	L_K u &=& f(u) & \text{in } \Omega\,,\\
	u &=& 0 & \text{in } \R^{2m}\setminus \Omega\,,
	\eeqc
	$$
	i.e.,
	$$
	\int_{\R^{2m}}\int_{\R^{2m}} \{u(x)-u(y)\}\{\eta(x)-\eta(y)\} K(|x-y|) \d x \d y = \int_{\R^{2m}} f(u(x)) \eta(x) \d x
	$$
	for every $\eta \in C^\infty_c(\Omega)$ (not necessarily doubly radial).
\end{proposition}

\begin{proof}
	Let $\eta \in C^\infty_c(\Omega)$. We define its associated doubly radial function as
	$$
	\overline{\eta}(x) := \average_{O(m)^2}\eta(R x)\d R\,.
	$$
	
	Now, on the one hand, given $R\in O(m)^2$ and using the change $x = R\tilde{x}$, $y = R \tilde{y}$ and the fact that $u$ is doubly radial, we get
	\begin{align*}
	&\int_{\R^{2m}}\int_{\R^{2m}} \{u(x)-u(y)\}\{\eta(x)-\eta(y)\} K(|x-y|) \d x \d y = \\
	&\quad \quad \quad = \int_{\R^{2m}}\int_{\R^{2m}} \{u(x)-u(y)\}\{\eta(R x)-\eta(R y)\} K(|x-y|) \d x \d y\,.
	\end{align*}
	Taking the average in the previous equality among all $R\in O(m)^2$ we obtain
	\begin{align*}
	& \int_{\R^{2m}}\int_{\R^{2m}} \{u(x)-u(y)\}\{\eta(x)-\eta(y)\} K(|x-y|) \d x \d y = \\
	&\quad \quad \quad =\average_{O(m)^2} \int_{\R^{2m}}\int_{\R^{2m}} \{u(x)-u(y)\}\{\eta(R x)-\eta(R y)\} K(|x-y|) \d x \d y \d R \\
	%&\quad \quad \quad= \int_{\R^{2m}}\int_{\R^{2m}} \{u(x)-u(y)\}\left \{\average_{O(m)^2}\eta(R x)\d R-\average_{O(m)^2} \eta(Ry) \d R \right \} K(|x-y|) \d x \d y \\
	&\quad \quad \quad= \int_{\R^{2m}}\int_{\R^{2m}} \{u(x)-u(y)\}\left \{\overline{\eta}(x) -\overline{\eta}(y)  \right \} K(|x-y|) \d x \d y \,.
	\end{align*}
	
	On the other hand, using also the change $x = R\tilde{x}$, we have
	$$
	\int_{\Omega} f(u(x)) \eta(x) \d x = \int_{\Omega} f(u(R^{-1}x)) \eta(x) \d x = \int_{\Omega} f(u(x)) \eta(Rx) \d x\,.
	$$
	Similarly as before, taking the average among all $R\in O(m)^2$, we get
	$$
	\int_{\Omega} f(u(x)) \eta(x) \d x = \average_{O(m)^2} \int_{\Omega} f(u(x)) \eta(Rx) \d x \d R = \int_{\Omega} f(u(x))\overline{\eta}(x) \d x\,.
	$$
	
	Hence, since $\overline{\eta} \in C^\infty_c(\Omega)$ is doubly radial, we have
	\begin{align*}
		&\int_{\R^{2m}}\int_{\R^{2m}} \{u(x)-u(y)\}\{\eta(x)-\eta(y)\} K(|x-y|) \d x \d y - \int_{\Omega} f(u(x)) \eta(x) \d x \\
		&\quad \quad= \int_{\R^{2m}}\int_{\R^{2m}} \{u(x)-u(y)\}\left \{\overline{\eta}(x) -\overline{\eta}(y)  \right \} K(|x-y|) \d x \d y - \int_{\Omega} f(u(x))\overline{\eta}(x) \d x \\
		&\quad \quad= 0\,,
	\end{align*}
	and thus the result is proved.
\end{proof}



\begin{remark}
	\label{Remark:InteriorRegularity}
	This proposition combined with some regularity results for operators in the class $\lcal_0(n,\s,\lambda, \Lambda)$ yield that bounded minimizers among doubly radial functions of the energy $\ecal(\cdot,\Omega)$ are classical solutions to $L_K u = f(u)$ in $\Omega$. Indeed, if $w\in L^\infty (\R^n)$ is a weak solution to $L_K w = h$ in $B_1\subset \R^n$, then
	\begin{equation}
	\label{Eq:C2sEstimate}
	\norm{w}_{C^{2\s} (\overline{B_{1/2}})} \leq C\bpar{\norm{h}_{L^\infty (B_1)} + \norm{w}_{L^\infty  (\R^n)}}.
	\end{equation} 
	If, in addition, $w \in C^\alpha (\R^n)$ with $\alpha + 2\s$ not an integer, then
	\begin{equation}
	\label{Eq:Calpha->Calpha+2sEstimate}
	\norm{w}_{C^{\alpha + 2\s} (\overline{B_{1/2}})} \leq C\bpar{\norm{h}_{C^{\alpha} (\overline{B_1})} + \norm{w}_{C^\alpha (\R^n)} },
	\end{equation}
	where the previous two constants $C$ depend only on $n$, $\s$, $\lambda$, and $\Lambda$ (see \cite{RosOton-Survey,SerraC2s+alphaRegularity} and the references therein).
	
	From the previous estimates and using the same argument as in Corollaries 2.4 and 2.5 of \cite{RosOtonSerra-Regularity}, \eqref{Eq:C2sEstimate} and \eqref{Eq:Calpha->Calpha+2sEstimate} yield, respectively, the estimates
	\begin{equation}
	\label{Eq:C2sEstimateBalls}
	\norm{w}_{C^{2\s} (\overline{B_{1/4}})} \leq C\bpar{\norm{h}_{L^\infty (B_1)} + \norm{w}_{L^\infty  (B_1)} + \norm{\dfrac{w(x)}{(1+|x|)^{n+2\s}}}_{L^1(\R^n)} },
	\end{equation}
	and 
	\begin{equation}
	\label{Eq:Calpha->Calpha+2sEstimateBalls}
	\norm{w}_{C^{\alpha + 2\s} (\overline{B_{1/4}})} \leq C\bpar{\norm{h}_{C^{\alpha} (\overline{B_1})} + \norm{w}_{C^\alpha (\overline{B_1})} + \norm{\dfrac{w(x)}{(1+|x|)^{n+2\s}}}_{L^1(\R^n)} }.
	\end{equation}
	Therefore, by applying these estimates (maybe a translated and rescaled version of them) to a weak solution $u\in L^\infty(\R^{2m})$ of $L_K u = f(u)$ in $\Omega$, with $f$ a $C^1$ nonlinearity, we easily conclude that $u$ is a classical solution, that is, the equation makes sense pointwise.
\end{remark}






%%%%%%%%%%%%%%%%%%%%%%%%%%%%%%%%%%%%%%%%%%%%%%%%%%%%%%%%%%%%%%%%%%%%%%
%%%%%%%%%%%%%%%%%%%%%%%%%%%%%%%%%%%%%%%%%%%%%%%%%%%%%%%%%%%%%%%%%%%%%%
\section{An energy estimate for doubly radial odd minimizers}
\label{Sec:EnergyEstimate}

In this section we present an estimate for the energy in $B_S$ of minimizers in the space $\widetilde{\H}^K_{0, \mathrm{odd}}(B_R)$. That is, we prove Theorem~\ref{Th:EnergyEstimate}. In order to establish this result, we follow the ideas of Savin and Valdinoci in \cite{SavinValdinoci-EnergyEstimate}, where they show the same estimate but for minimizers without any symmetry. The strategy is to compare $u$ with a suitable competitor which is constructed combining $u$ with an auxiliary function. In our case, since $u$ is a minimizer among functions in $\widetilde{\H}^K_{0, \mathrm{odd}}(B_R)$, we need to adapt the auxiliary function used in \cite{SavinValdinoci-EnergyEstimate} so that the resulting competitor is admissible, i.e., is a doubly radial function which is odd with  respect to the Simons cone $\ccal$.

The auxiliary function needed to build the competitor is defined as follows. For points $x\in \ocal$, we set
$$ \Psi_S(x) := \max\left\{-1+2\min\{(|x|-S-1)_+,1\},-\dist(x,\ccal) \right\},  $$
and we define it in $\ical$ by considering its odd reflection. It is clear that it is a bounded function with $||\Psi_S||_{L^\infty(\R^{2m})}=1$. In our arguments we will also use the following function and set:
$$ d_S(x) := \max\left\{1,\min\{S+1-|x|,\dist(x,\ccal)\} \right\},  $$
and
\begin{equation}
\label{Eq:DefOmegaS}
\Omega_S := \left( B_{S+2}\setminus \overline{B_s} \right) \cup \left( B_{S+2} \cap \{\dist(x,\ccal)< 1\}\right).
\end{equation} 

\begin{figure}
	\centering
	\begin{subfigure}{0.49\textwidth}
		\centering
		\definecolor{lila_custom}{RGB}{201,69,254}
\definecolor{naranja_custom}{RGB}{255,148,0}


\begin{tikzpicture}[y=0.80pt, x=0.80pt, yscale=-1.000000, xscale=1.000000, inner sep=0pt, outer sep=0pt]


%region-down
\path[scale=0.938,fill=lila_custom, opacity = 0.7, line width=0.400pt] (248.0459,242.6193) ..
controls (275.7060,284.1286) and (276.2013,310.5581) .. (276.7985,330.5197) ..
controls (233.8448,330.5197) and (216.3370,330.5469) .. (160.2334,330.5469) ..
controls (186.2822,304.6380) and (219.4410,271.2569) .. (248.0459,242.6193) --
cycle;



%big circle
\path[draw=black,line join=miter,line cap=butt,line width=1.4pt]
(243.0000,193.8560) .. controls (275.7522,226.7370) and (289.9164,269.7115) ..
(289.9164,309.8622);

%big circle2
\path[draw=naranja_custom,line join=miter,line cap=butt,line width=1.8pt]
(243.0000+11,193.8560+12) .. controls (275.7522-9,226.7370-6) and (289.9164,258.7115) ..
(289.9164,309.8622);

%x-axis    
\path[draw=black,line join=miter,line cap=butt,line width=1.5pt]
(126.5055,309.8063) -- (330.4872,309.8063);

%x-axis cap    
\path[draw=black,fill=black,even odd rule,line width=0.497pt]
(330.4872,309.8063) -- (328.1493,312.1289) -- (336.3320,309.8063) --
(328.1493,307.4838) -- cycle;

%y-axis
\path[draw=black,line join=miter,line cap=butt,line width=1.5pt]
(127.0562,310.75) -- (127.0562,99.5223);

%y-axis cap    
\path[draw=black,fill=black,even odd rule,line width=0.497pt] (127.0562,99.5223)
-- (129.3788,101.8602) -- (127.0562,93.6775) -- (124.7336,101.8602) -- cycle;



%medium circle
\path[draw=black,line join=miter,line cap=butt,line width=1.4pt]
(221.4938,215.3997) .. controls (250.7785,244.9692) and (259.4986,284.9193) ..
(259.4986,309.8622);

%small circle    
\path[draw=black,line join=miter,line cap=butt,line width=1.4pt]
(199.9500,236.9435) .. controls (222.0483,259.0975) and (229.1002,286.5606) ..
(229.1002,309.8622);

%cone   
\path[draw=black,line join=miter,line cap=butt,miter limit=4.00,even odd
rule,line width=1.57pt] (126.7629,310.0643) -- (326.1869,110.8844);

%cone+1    
\path[draw=black,line join=miter,line cap=butt,even odd rule,line width=0.704pt]
(150.2563,309.8481) -- (318.0555,142.0518);

%cone-1 
%\path[draw=black,line join=miter,line cap=butt,even odd rule,line width=0.704pt]
%(127.1847,287.2416) -- (299.4946,114.9526); 


%Linea Cota
\path[draw=black,line join=miter,line cap=butt,line width=0.5pt]
(298.3343,142.4919) -- (306.3936,150.7321);


%Flecha Izquierda Cota
\path[draw=black,fill=black,even odd rule,line width=0.200pt]
(298.9657,143.1375) -- (300.2914,143.1522) -- (297.3269,141.4619) --
(298.9510,144.4632) -- cycle;

%Flecha Derecha Cota
\path[draw=black,fill=black,even odd rule,line width=0.200pt]
(305.7622,150.0865) -- (304.4364,150.0718) -- (307.4010,151.7621) --
(305.7769,148.7608) -- cycle;

%Cota naranja
\path[draw=black,line join=miter,line cap=butt,line width=0.5pt]
(300, 230) .. controls (300, 235) and (295, 250) ..
(285, 250);

%Flecha cota naranja
\path[draw=black,fill=black,even odd rule,line width=0.200pt]
(285, 250) -- (285+0.9375, 250+0.9375) -- (285-2.3437, 250) --
(285+0.9375, 250-0.9375) -- cycle;

\node at (315,138) {\normalsize $\mu^{-1}$};
\node at (127,85) {\normalsize $|x''|$};
\node at (350, 311) {\normalsize $|x'|$};
\node at (228,320) {\normalsize $S$};
\node at (255, 320) {\normalsize $S\!+\!1$};
\node at (292, 320) {\normalsize $S\!+\!2$};
\node at (334, 101) {\normalsize $\ccal$};
\node at (298, 220) {\normalsize $\Psi_S\! = \!1$};
\node at (220, 286) {\normalsize $\Psi_S\! = \!-1$};


\end{tikzpicture}


	\end{subfigure}
	\begin{subfigure}{0.49\textwidth}
		\centering
		\definecolor{azul_custom}{RGB}{66,240,209}
\definecolor{lila_custom}{RGB}{201,69,254}
\definecolor{naranja_custom}{RGB}{255,148,0}


\begin{tikzpicture}[y=0.80pt, x=0.80pt, yscale=-1.000000, xscale=1.000000, inner sep=0pt, outer sep=0pt]


%region
\path[fill=azul_custom,line cap=round,miter limit=4.00,line width=1.216pt]
(210.3958,249.7906) .. controls (188.0101,272.1763) and (172.1936,287.9233) ..
(150.1544,309.9625) .. controls (148.2221,309.9625) and (137.7690,309.7049) ..
(127.0089,309.7371) .. controls (127.0089,300.7477) and (127.0826,301.1491) ..
(127.0826,287.3606) .. controls (136.2973,278.1061) and (177.3049,237.1011) ..
(187.7910,226.5786) .. controls (170.0299,214.0930) and (149.7638,207.7938) ..
(127.0312,207.7938) .. controls (127.0312,190.9391) and (126.9993,161.0751) ..
(126.9993,146.8284) .. controls (170.9394,146.8284) and (208.6039,162.4681) ..
(243.0000,193.8560) .. controls (271.9385,225.9716) and (289.9164,263.3638) ..
(289.9164,309.8622) .. controls (260.7310,309.8622) and (257.9656,309.8622) ..
(229.1006,309.8622) .. controls (229.1006,279.3951) and (221.7362,269.4328) ..
(210.3958,249.7907) -- cycle;


%x-axis    
\path[draw=black,line join=miter,line cap=butt,line width=1.5pt]
(126.5055,309.8063) -- (330.4872,309.8063);

%x-axis cap    
\path[draw=black,fill=black,even odd rule,line width=0.497pt]
(330.4872,309.8063) -- (328.1493,312.1289) -- (336.3320,309.8063) --
(328.1493,307.4838) -- cycle;

%y-axis
\path[draw=black,line join=miter,line cap=butt,line width=1.5pt]
(127.0562,310.75) -- (127.0562,99.5223);

%y-axis cap    
\path[draw=black,fill=black,even odd rule,line width=0.497pt] (127.0562,99.5223)
-- (129.3788,101.8602) -- (127.0562,93.6775) -- (124.7336,101.8602) -- cycle;


%big circle
\path[draw=black,line join=miter,line cap=butt,line width=1.4pt]
(127.0312,146.9770) .. controls (166.9982,146.9770) and (210.2478,160.9750) ..
(243.0000,193.8560) .. controls (275.7522,226.7370) and (289.9164,269.7115) ..
(289.9164,309.8622);

%medium circle
\path[draw=black,line join=miter,line cap=butt,line width=1.4pt]
(127.0312,177.3949) .. controls (151.7088,177.3949) and (192.2090,185.8302) ..
(221.4938,215.3997) .. controls (250.7785,244.9692) and (259.4986,284.9193) ..
(259.4986,309.8622);

%small circle    
\path[draw=black,line join=miter,line cap=butt,line width=1.4pt]
(127.0312,207.7933) .. controls (150.2506,207.7933) and (177.3815,214.3180) ..
(199.9500,236.9435) .. controls (222.0483,259.0975) and (229.1002,286.5606) ..
(229.1002,309.8622);

%cone   
\path[draw=black,line join=miter,line cap=butt,miter limit=4.00,even odd
rule,line width=1.57pt] (126.7629,310.0643) -- (326.1869,110.8844);

%cone+1    
\path[draw=black,line join=miter,line cap=butt,even odd rule,line width=0.704pt]
(150.2563,309.8481) -- (318.0555,142.0518);

%cone-1 
\path[draw=black,line join=miter,line cap=butt,even odd rule,line width=0.704pt]
(127.1847,287.2416) -- (299.4946,114.9526); 

%Linea Cota
\path[draw=black,line join=miter,line cap=butt,line width=0.5pt]
(298.3343,142.4919) -- (306.3936,150.7321);


%Flecha Izquierda Cota
\path[draw=black,fill=black,even odd rule,line width=0.200pt]
(298.9657,143.1375) -- (300.2914,143.1522) -- (297.3269,141.4619) --
(298.9510,144.4632) -- cycle;

%Flecha Derecha Cota
\path[draw=black,fill=black,even odd rule,line width=0.200pt]
(305.7622,150.0865) -- (304.4364,150.0718) -- (307.4010,151.7621) --
(305.7769,148.7608) -- cycle;

\node at (315,138) {\normalsize $\mu^{-1}$};
\node at (127,85) {\normalsize $|x''|$};
\node at (350, 311) {\normalsize $|x'|$};
\node at (228,320) {\normalsize $S$};
\node at (255, 320) {\normalsize $S\!+\!1$};
\node at (292, 320) {\normalsize $S\!+\!2$};
\node at (334, 101) {\normalsize $\ccal$};
\node at (264, 250) {\normalsize $\Omega_S$};

\end{tikzpicture}











	\end{subfigure}
	\caption{(a) The $1$ and $-1$ level sets of $\Psi_S$. (b) The set $\Omega_S$.}
	\label{Fig:PsiSandOmegaS}
\end{figure}

Note that both $\Psi_S$ and $d_S$ are Lipschitz functions, with Lipschitz norm independent of $S$. Moreover $\Psi_S$ is odd and $d_S$ even with respect to the Simons cone. Regarding the set $\overline{\Omega_S}$, we can see it as the preimage of $1$ through $d_S$ inside $\overline{B_{S+2}}$ (see Figure~\ref{Fig:PsiSandOmegaS}). Furthermore, its measure is well known to be of order $2m-1$ (see the proof of the energy estimate in \cite{CabreTerraI}). That is,
\begin{equation}
\label{Eq:MeasureOmegaS}
|\Omega_S| \leq C\,S^{2m-1}.
\end{equation}


Now we show some auxiliary results concerning the previous definitions, needed in the proof of Theorem~\ref{Th:EnergyEstimate}.

\begin{lemma}
\label{Lemma: AdaptedLipschitzConditionWith_dFunction}
Given $S>0$, if either $(x,y) \in \left(\Omega_S\cap \ocal\right) \times \ical$ or $(x,y)\in \left(B_{S+2}\cap \ocal\right) \times \ocal$, then
$$ |\Psi_S(x) - \Psi_S(y)| \leq C \frac{|x-y|}{d_S(x)} \ \ \ \ \ \textrm{whenever} \ \ |x-y|\leq d_S(x), $$
with $C>0$ independent of $S$.
\end{lemma}

\begin{proof}
Note first that if $x\in \Omega_S \cap \ocal$, then $d_S(x)=1$ and the result is trivial by the Lipschitz continuity of $\Psi_S$. Hence, we only need to establish the result for the case $x\in B_S\cap \{\dist(x,\ccal)\geq 1\} \cap \ocal$ and $y\in \ocal$. Under these hypotheses, we have that $\Psi_S(x)=-1$ (see Figure~\ref{Fig:PsiSandOmegaS}) and $d_S(x) = \min\{S+1-|x|,\dist(x,\ccal)\}$. Moreover, since $x\in B_S$ and $\dist(x,\ccal)\geq 1$ we get $d_S(x) \leq S+1-|x|$. Therefore, if $|x-y|\leq d_S(x)$ we obtain
$$ |y|\leq |x-y| + |x| \leq d_S(x)+|x| \leq S+1. $$

Now we distinguish two cases, either $\{\dist(\cdot,\ccal)\geq 1\}$ or $\{\dist(\cdot,\ccal)\leq 1\}$. Assume first that $y\in B_{S+1} \cap \{\dist(\cdot,\ccal)\geq 1\}\cap \ocal$. Then, $\Psi_S(y)=-1$ and the result is trivial from being also $\Psi_S(x)=-1$. Thus, it only remains to show the result in the case $x\in B_S \cap \{\dist(\cdot,\ccal)\geq 1\}\cap \ocal$ and $y\in B_{S+1} \cap \{\dist(\cdot,\ccal)\leq 1\}\cap \ocal$. Note that under these assumptions, $\Psi_S(x)=-1$ and $\Psi_S(y)=-\dist(y,\ccal)$.


Given $x,y \in \R^{2m}$ it is easy to prove by using the triangular inequality and the definition of distance to the cone that
\begin{equation} \label{Eq:TriangularCone}
\dist(x,\ccal) \leq |x-y| + \dist(y,\ccal).
\end{equation}
Therefore we have
\begin{equation} \label{Eq:TriangularCone2}
1-|x-y|-\dist(y,\ccal) \leq 1-\dist(x,\ccal) \leq 0
\end{equation}
Now, multiplying \eqref{Eq:TriangularCone} by $|1-\dist(y,\ccal)|$ and using \eqref{Eq:TriangularCone2} we obtain
\begin{align*}
|1-\dist(y,\ccal)|\,\dist(x,\ccal) &\leq |1-\dist(y,\ccal)| \left(|x-y| + \dist(y,\ccal)\right) \\
%&= \left(1-\dist(y,\ccal)\right) \left(|x-y| + \dist(y,\ccal)\right) \\
&= |x-y|+\dist(y,\ccal) \left\{ -|x-y|+ 1- \dist(y,\ccal) \right\} \\
&\leq |x-y|.
\end{align*}

Hence,
$$ |\Psi_S(x)-\Psi_S(y)| = |1-\dist(y,\ccal)| \leq \frac{|x-y|}{\dist(x,\ccal)} \leq  \frac{|x-y|}{d_S(x)},$$
completing the proof.
\end{proof}

Another auxiliary result that we will need in the proof of Theorem~\ref{Th:EnergyEstimate} is the following estimate for the function $d_S$. 

\begin{lemma}
\label{Lemma:Integrability_dFunction}
Given $\s \in (0,1)$ and $B_{S+2}\subset \R^{2m}$ with $S>0$, we have
$$ \int_{B_{S+2}} d_S(x)^{-2\s} \d x \leq \begin{cases}
C \ S^{2m-2\s}\ \ \ \ &\textrm{if } \ \ \s\in(0,1/2),\\
C\ \log(S)\,S^{2m-2\s}\ \ \ \ &\textrm{if } \ \ \s=1/2,\\
C \ S^{2m-1}\ \ \ \ &\textrm{if } \ \ \s\in(1/2,1),\\
\end{cases} $$
with $C>0$ independent of $S$ and only depending on $m$ and $\s$.
\end{lemma}


\begin{proof}
In order to prove this result we first note that $d_S(x)=1$ in $\Omega_S$. Thus, the contribution to the integral of this part is just the measure of the set $\Omega_S$ (see equation \eqref{Eq:MeasureOmegaS}). That is,
$$\int_{\Omega_S} d_S(x)^{-2\s} \d x = |\Omega_S| \leq C\,S^{2m-1}.$$

For the other part of the integral we can write
\begin{align*}
\int_{B_{S+2}\setminus \Omega_S} d_S(x)^{-2\s} \d x &= \int_{B_{S}\cap \dist\{x,\ccal\}>1} d_S(x)^{-2\s} \d x \\
& \leq \int_{B_{S}\cap \dist\{x,\ccal\}>1} \left( S+1-|x| \right)^{-2\s} \d x + \int_{B_{S}\cap \dist\{x,\ccal\}>1} \dist(x,\ccal)^{-2\s} \d x.
\end{align*}
The desired estimate for the first integral can be found in \cite{SavinValdinoci-EnergyEstimate}. Therefore, in order to complete the proof it only remains to estimate the second integral. It can be estimated by writing it in $(y,z)$ variables, where
$$
y = \dfrac{|x'|+|x''|}{\sqrt{2}} \, \quad \text{ and } z = \dfrac{|x'|-|x''|}{\sqrt{2}}\,.
$$
In this case, $z$ is the signed distance to the cone. Thus,
\begin{align*}
\int_{B_{S}\cap \dist\{x,\ccal\}>1} \dist(x,\ccal)^{-2\s} \d x &\leq C \int \int_{B_{S}\cap \{y\geq|z|>1\}} |z|^{-2\s} \, (y^2-z^2)^{m-1} \d y\d z \\
& \leq C \int \int_{B_{S}\cap \{y\geq|z|>1\}} |z|^{-2\s} \, y^{2m-2} \d y\d z \\
& \leq C\, \int_1^S \d z \int_0^S \d y\ z^{-2\s} \, y^{2m-2} \\
& \leq C\, \left(\int_1^S z^{-2\s} \d z \right)  \left(  \int_0^S \d y \, y^{2m-2} \right) \\
& \leq \begin{cases}
C \ S^{2m-2\s}\ \ \ \ &\textrm{if } \ \ \s\in(0,1/2),\\
C\ \log(S)\,S^{2m-2\s}\ \ \ \ &\textrm{if } \ \ \s=1/2,\\
C \ S^{2m-1}\ \ \ \ &\textrm{if } \ \ \s\in(1/2,1).\\
\end{cases}
\end{align*}
\end{proof}

Note that $(y,z)$ variables are very useful when dealing with doubly radial odd functions ---see \cite{CabreTerraI, CabreTerraII, Cabre-Saddle, Cinti-Saddle,Cinti-Saddle2, Felipe-Sanz-Perela:SaddleFractional}.

The last auxiliary result we need in order to establish the energy estimate is the following inequality.

\begin{lemma}
\label{Lemma: InteractionInequalityMinimumFunction}
Let $A\subset B_R \subset \R^{2m}$ be a set of double revolution such that $A^\star = A$. Let $\omega, \phi, \varphi \in \widetilde{\H}^K(B_R)$, with $K$ radially symmetric, be such that
$$\begin{cases}
\omega = \phi \leq \varphi \ \ \ \ \textrm{in } \ \ \ \ocal \setminus A\,,\\
\omega = \varphi \leq \phi \ \ \ \ \textrm{in } \ \ \ \ocal \cap A\,.
\end{cases}$$
Then, if $K$ satisfies \eqref{Eq:KernelInequality}, it holds
\begin{align*}
I_\omega(\ocal\cap A, \ocal \setminus A) \leq I_\phi(\ocal\cap A, \ocal \setminus A) + I_\varphi(\ocal\cap A, \ocal \setminus A)\,,
\end{align*}
where $I_w(\cdot, \cdot)$ is the interaction defined in \eqref{Eq:DefIw}.
\end{lemma}

\begin{proof}
A simple computation shows that if $x\in \ocal \cap A$ and $y\in \ocal \setminus A$ we have that
$$ |\phi(x)-\phi(y)|^2+|\varphi(x)-\varphi(y)|^2\geq |\omega(x)-\omega(y)|^2. $$
Indeed,
\begin{align*}
|\phi(x)-\phi(y)|^2+|\varphi(x)&-\varphi(y)|^2 - |\omega(x)-\omega(y)|^2 \\
&= |\phi(x)-\phi(y)|^2+|\varphi(x)-\varphi(y)|^2 - |\varphi(x)-\phi(y)|^2 \\
&= \phi^2(x)-2\phi(x)\phi(y)+\varphi^2(y)-2\varphi(x)\varphi(y)+2\varphi(x)\phi(y) \\
&= \left( \phi(x) - \varphi(y)\right) ^2+2\left( \phi(x)-\varphi(x) \right) \left( \varphi(y)-\phi(y) \right) \\
&\geq 0.
\end{align*}
Therefore, by using this inequality and the reflexion property of the kernel, \eqref{Eq:KernelInequality}, we obtain
\begin{align*}
I_\phi(\ocal\cap A, \ocal \setminus A) &+ I_\varphi(\ocal\cap A, \ocal \setminus A) - I_\omega(\ocal\cap A, \ocal \setminus A) =\\
&\hspace{-26mm}= \int_{\ocal\cap A} \d x \int_{\ocal\setminus A} \d y \Big( 2\left\{\phi^2(x)+\phi^2(y)+\varphi^2(x)+\varphi^2(y)-\omega^2(x)-\omega^2(y) \right\} \overline{K}(x,y^\star) \Big.\\
&\hspace{-18mm}\Big.+  \left\{|\phi(x)-\phi(y)|^2+|\varphi(x)-\varphi(y)|^2-|\omega(x)-\omega(y)|^2 \right\} \left\{\overline{K}(x,y)-\overline{K}(x,y^\star)\right\}\Big)\\
&\hspace{-26mm}\geq 2\int_{\ocal\cap A} \d x \int_{\ocal\setminus A} \d y \left\{
\phi^2(x)+\varphi^2(y)\right\} \overline{K}(x,y^\star) \geq 0.
\end{align*}
\end{proof}



With all these ingredients we can establish now the sharp energy estimate.

\begin{proof}[Proof of Theorem~\ref{Th:EnergyEstimate}]

Note that, by Lemma~\ref{Lemma:DecreaseEnergy}  we can assume without loss of generality that if $u$ is a minimizer of $\ecal$ in $B_R$, then $-1 \leq u \leq 1$, $u \geq 0$ in $\ocal$, and $u \leq 0$ in $\ical$. 

\textbf{Step 1. We show that $0\leq u < 1$ in $\ocal$.} 

In order to prove it we first need to show that $u$ is a weak solution of
\begin{equation}
\label{Eq:ProofEnergyEstimateProblemBR}
	\beqc{\PDEsystem}
	L_K  u &=& f(u) & \textrm{ in } B_R\,,\\
	u &=& 0 & \textrm{ in }\R^{2m} \setminus B_R.
	\eeqc
\end{equation}
To see this, we consider on the one hand perturbations $u +  \varepsilon \xi$, with $\xi \in \widetilde{\H}^K_{0, \,\mathrm{odd}}(B_R)$ and such that $\xi$ has compact support in $B_R$. Then, since $u$ is a minimizer among functions in $\widetilde{\H}^K_{0, \,\mathrm{odd}}(B_R)$, we get
$$
0 = \dfrac{\d}{\d \varepsilon}\evalat{\varepsilon = 0} \ecal(u +  \varepsilon \xi, B_R) = \langle u,\xi \rangle_{\widetilde{\H}^K_0(B_R)} - \langle f(u),\xi \rangle_{L^2(B_R)}\,.
$$
On the other hand, take $\xi \in \widetilde{\H}^K_{0, \,\mathrm{even}}(B_R)$. Since $u$ is odd with respect to the Simons cone, so is $f(u)$. Then, by Remark~\ref{Remark:DecompositionHK} and the same decomposition in $L^2(B_R)$, we find that
$$
\langle v_R,\xi \rangle_{\widetilde{\H}^K_0(B_R)} = 0 \quad \textrm{ and } \quad  \langle f(v_R),\xi \rangle_{L^2(B_R)} = 0\,.
$$
Therefore, we have that
$$
\langle u,\xi \rangle_{\widetilde{\H}^K_0(B_R)} = \langle f(u),\xi \rangle_{L^2(B_R)}
$$
for every $\xi \in\widetilde{\H}^K_0(B_R)$ with compact support in  $B_R$. Thus,
$$
\int_{\R^{2m}}\int_{\R^{2m}} \{u(x)-u(y)\}\{\xi(x)-\xi(y)\} K(|x-y|) \d x \d y = \int_{\R^{2m}} f(u(x)) \xi(x) \d x
$$
for every $\xi \in C^\infty_0(\Omega)$ that is doubly radial. 

By Proposition~\ref{Prop:WeakSolutionForAllTestFunctions}, $u$ is a weak solution of \eqref{Eq:ProofEnergyEstimateProblemBR}, and in view of the regularity theory for operators in $\lcal_0$ (see Remark~\ref{Remark:InteriorRegularity}), since $u$ is bounded, it is a classical solution.

From being $u$ a classical solution it is easy to show that it cannot be $1$ or $-1$ and therefore that it satisfies $0\leq u < 1$ in $\ocal$. In order to see this, let us suppose that there exists $x_0\in\R^{2m}$ such that $|u(x_0)|=1$. Without loss of generality we can assume that $x_0\in\ocal\cap B_R$. Then, from equation \eqref{Eq:ProofEnergyEstimateProblemBR} and the fact of being $x_0$ an absolute maximum, we can arrive at a contradiction:
\begin{align*}
0 &= f(1) = f(u(x_0)) = L_K u(x_0) = \int_\ocal (1-u(y)) \overline{K}(x,y) + (1+u(y)) \overline{K}(x,y^\star)  \d y \\
&\geq \int_\ocal (1-u(y)) \overline{K}(x,y^\star) + (1+u(y)) \overline{K}(x,y^\star)  \d y = 2\int_\ocal \overline{K}(x,y^\star) \d y\\
&>0.
\end{align*}

\textbf{Step 2. We build a suitable competitor for $u$ and compare their energies.}

Now, for $x\in \ocal$ we define
$$ 
v(x) := \min\{u(x),\Psi_S(x)\}, 
$$
and we define it in $\ical$ by considering its odd reflection with respect to the Simons cone. Let us also define
$$
A = \{v=\Psi_S\} = \{\Psi_S \leq u\}. 
$$
Then, it is easy to check that we have the inclusions
\begin{equation}
\label{Eq:EnergyEstimateProofInclusionsA}
	B_{S+1} \subset A \subset B_{S+2}\,.
\end{equation}
Indeed, note first that we only need to prove it inside $\ocal$, by the symmetry of $A$ with respect to the Simons cone. On the one hand, if $ x\in B_{S+1}\cap \ocal$, then $\Psi_S(x) = \max\{-1,-\dist(x,\ccal)\} \leq 0 \leq u(x)$, which yields $v(x) = \Psi_S(x)$ Thus, $x\in A\cap \ocal$. On the other hand, if $ x\in A\cap \ocal$ then $\Psi_S(x) \leq u(x) < 1$. This can only happen if $x\in B_{S+2}$.

Note that both $u$ and $v$ are equal outside $B_{S+2} \subset B_R$, and therefore $v$ is an admissible competitor. By comparing the energies of $u$ and $v$ we will obtain the desired estimate. 

Let us decompose the energy of $u$ in $B_R$ in terms of interactions between sets that involve $A$. That is, using expression \eqref{Eq:ShortExpressionEnergy}, we get
\begin{align*}
\ecal(u,B_R) &= \frac{1}{2}I_u(\ocal \cap A, \ocal \cap A) + I_u(\ocal \cap A, \ocal \setminus A) \\
& \hspace{5mm} + \frac{1}{2}I_u\big((\ocal \setminus A) \cap B_R, (\ocal \setminus A) \cap B_R\big) + I_u\big((\ocal \setminus A) \cap B_R, \ocal \setminus B_R\big) \\
& \hspace{5mm} + \int_A G(u) + \int_{B_R\setminus A} G(u)
\end{align*}
Since $u$ is a minimizer, $v=\Psi_S$ in $A$ and $u=v$ outside of $A$,  from the previous expression we obtain
\begin{align*}
0 &\leq \ecal(v,B_R)-\ecal(u,B_R) = \frac{1}{2}I_{\Psi_S}(\ocal \cap A, \ocal \cap A) - \frac{1}{2}I_u(\ocal \cap A, \ocal \cap A)\\
& \hspace{5mm} + I_v(\ocal \cap A, \ocal \setminus A) - I_u(\ocal \cap A, \ocal \setminus A) + \int_A G(\Psi_S) - \int_{A} G(u).
\end{align*}
Since $v = \min\{u,\Psi_S\}$ in $\ocal$ we can apply Lemma \ref{Lemma: InteractionInequalityMinimumFunction} with $\omega = v$, $\Psi_S = \varphi$, and $u= \phi$, to get $I_v(\ocal \cap A, \ocal\setminus A) \leq I_u(\ocal \cap A, \ocal\setminus A) + I_{\Psi_S}(\ocal \cap A, \ocal\setminus A)$. Therefore,
\begin{align*}
\frac{1}{2}I_u(\ocal \cap A, \ocal \cap A) + \int_{A} G(u) &\leq \frac{1}{2}I_{\Psi_S}(\ocal \cap A, \ocal \cap A) + I_{\Psi_S}(\ocal \cap A, \ocal \setminus A) + \int_A G(\Psi_S)  \\
&= \ecal(\Psi_S, A) \leq \ecal(\Psi_S,B_{S+2})
\end{align*}
From this and using \eqref{Eq:EnergyEstimateProofInclusionsA}, we deduce an estimate for the energy of $u$ in $B_S$ as follows.
\begin{align*}
\ecal(u,B_S) &\leq \frac{1}{2}I_u(\ocal \cap A, \ocal \cap A) + \int_{A} G(u) + I_u(\ocal \setminus B_{S+1}, \ocal \cap B_S) \\
& \leq  \ecal(\Psi_S,B_{S+2}) + I_u(\ocal \setminus B_{S+1}, \ocal \cap B_S).
\end{align*}
Thus, to obtain the desired energy estimate we only have to bound the right-hand side of the last inequality.


\textbf{Step 3. We estimate the remaining terms.}

In the following arguments, we use the definition of the energy that involves the original kernel $K$ and not $\overline{K}$.

\textbf{3.1. Estimate for $\ecal(\Psi_S,B_{S+2})$.}
First, by using the change of variables given by $(\cdot)^\star$ and the ellipticity of $K$, condition \eqref{Eq:Ellipticity}, we obtain
\begin{align*}
\ecal(\Psi_S,B_{S+2}) &= \frac{1}{4} \int_{B_{S+2}} \d x \int_{B_{S+2}} \d y\ |\Psi_S(x)-\Psi_S(y)|^2K(|x-y|)  \\
&\hspace{5mm} +\frac{1}{2} \int_{B_{S+2}} \d x \int_{\R^{2m} \setminus B_{S+2}} \d y \ |\Psi_S(x)-\Psi_S(y)|^2K(|x-y|) + \int_{B_{S+2}} G(\Psi_S) \\
&\leq \frac{1}{2} \int_{B_{S+2}} \d x \int_{\R^{2m}} \d y\ |\Psi_S(x)-\Psi_S(y)|^2K(|x-y|) + \int_{B_{S+2}} G(\Psi_S) \\
&= \int_{B_{S+2} \cap \ocal} \d x \int_{\R^{2m}} \d y\ |\Psi_S(x)-\Psi_S(y)|^2K(|x-y|) \d x\d y + \int_{B_{S+2}} G(\Psi_S) \\
&\leq \Lambda\, c_{n,\s} \int_{B_{S+2} \cap \ocal} \d x \int_{\R^{2m}} \d y \ \frac{|\Psi_S(x)-\Psi_S(y)|^2}{|x-y|^{n+2\s}} + \int_{B_{S+2}} G(\Psi_S).
\end{align*}
Now, we split the domain of integration of the kinetic energy into three parts.
\begin{align*}
\ecal(\Psi_S,B_{S+2}) &\leq \Lambda\, c_{n,\s} \int_{B_{S+2} \cap \ocal} \d x \int_{\ocal} \d y \frac{|\Psi_S(x)-\Psi_S(y)|^2}{|x-y|^{n+2\s}} \\
&\hspace{5mm} + \Lambda\, c_{n,\s} \int_{\Omega_S \cap \ocal} \d x \int_{\ical} \d y \frac{|\Psi_S(x)-\Psi_S(y)|^2}{|x-y|^{n+2\s}} \\
&\hspace{5mm} + \Lambda\, c_{n,\s} \int_{(B_{S+2}\setminus \Omega_S) \cap \ocal} \d x \int_{\ical} \d y \frac{|\Psi_S(x)-\Psi_S(y)|^2}{|x-y|^{n+2\s}} + \int_{B_{S+2}} G(\Psi_S) \\
&=:\Lambda\, c_{n,\s} (I_1+I_2+I_3)+I_G,
\end{align*}
where $\Omega_S$ is defined in \eqref{Eq:DefOmegaS}. 

Let us estimate this four integrals. To estimate $I_1$, we use Lemma~\eqref{Lemma: AdaptedLipschitzConditionWith_dFunction} and the fact that $\Psi_S$ is bounded by $1$.
\begin{align*}
I_1 &= \int_{B_{S+2} \cap \ocal} \int_{\ocal} \frac{|\Psi_S(x)-\Psi_S(y)|^2}{|x-y|^{n+2\s}} \d y\d x\\
&= \int_{B_{S+2} \cap \ocal} \int_{\ocal\cap\{|x-y|\leq d_S(x)\}} \frac{|\Psi_S(x)-\Psi_S(y)|^2}{|x-y|^{n+2\s}} \d y\d x\\
&\hspace{5mm} + \int_{B_{S+2} \cap \ocal} \int_{\ocal\cap\{|x-y|\geq d_S(x)\}} \frac{|\Psi_S(x)-\Psi_S(y)|^2}{|x-y|^{n+2\s}} \d y\d x\\
&\leq C \int_{B_{S+2} \cap \ocal} d_S(x)^{-2}\left(\int_{\ocal\cap\{|x-y|\leq d_S(x)\}} |x-y|^{2-n-2\s} \d y\right)\d x\\
&\hspace{5mm} + C \int_{B_{S+2} \cap \ocal} \left(\int_{\ocal\cap\{|x-y|\geq d_S(x)\}} |x-y|^{-n-2\s} \d y \right)\d x\\
&\leq C \int_{B_{S+2} \cap \ocal} d_S(x)^{-2}\left(\int_0^{d_S(x)} \rho^{1-2\s} \d \rho\right)\d x + C \int_{B_{S+2} \cap \ocal}  \left(\int_{d_S(x)}^\infty \rho^{-1-2\s} \d\rho\right) \d x\\
&\leq C \int_{B_{S+2} \cap \ocal} d_S(x)^{-2\s} \d x.
\end{align*}
The bound of $I_2$ is essentially the same using also Lemma~\ref{Lemma: AdaptedLipschitzConditionWith_dFunction} and the inclusion $\Omega_S \subset B_{S+2}$. That is,
\begin{align*}
I_2 &\leq C \int_{\Omega_S \cap \ocal} d_S(x)^{-2}\left(\int_0^{d_S(x)} \rho^{1-2\s} \d \rho\right)\d x + C \int_{\Omega_S \cap \ocal} \left( \int_{d_S(x)}^\infty \rho^{-1-2\s} \d\rho \right) \d x\\
&\leq C \int_{\Omega _S\cap \ocal} d_S(x)^{-2\s} \d x \leq C \int_{B_{S+2} \cap \ocal} d_S(x)^{-2\s} \d x.
\end{align*}
For the case of $I_3$, we use the fact that given $x\in (B_{S+2}\setminus \Omega_S)\cap \ocal$, then $\dist(x,\ccal)\geq d_S(x)$ and therefore $\ical \subset \R^{2m}\setminus B_{d_S(x)}(x)$. We obtain
\begin{align*}
I_3 &= \int_{(B_{S+2}\setminus \Omega_S) \cap \ocal} \d x \int_{\ical} \d y \frac{|\Psi_S(x)-\Psi_S(y)|^2}{|x-y|^{n+2\s}} \leq C \int_{(B_{S+2}\setminus \Omega_S) \cap \ocal} \d x \int_{\R^{2m}\setminus B_{d_S(x)}(x)} \d y \ |x-y|^{-n-2\s} \\
&\leq C \int_{B_{S+2} \cap \ocal} \left(\int_{d_S(x)}^\infty \rho^{-1-2\s} \d \rho \right)\d x \leq C \int_{B_{S+2} \cap \ocal} d_S(x)^{-2\s} \d x.
\end{align*}
Finally, we estimate $I_G$. Since $\Psi_S$ is either $1$ or $-1$ in $B_{S+2}\setminus \Omega_S$, and $G(-1)=G(1)=0$, we have
\begin{align*}
I_G = \int_{B_{S+2}} G(\Psi_S) = \int_{\Omega_S} G(\Psi_S) \leq C | \Omega_S| \leq C\,S^{2m-1}\,,
\end{align*}
where we have used \eqref{Eq:MeasureOmegaS}.Therefore, we obtain
\begin{align*}
\ecal(\Psi_S,B_{S+2}) &\leq C \left(\int_{B_{S+2} \cap \ocal} d_S(x)^{-2\s} \d x + S^{2m-1} \right)\leq C\left(\int_{B_{S} \cap \ocal} d_S(x)^{-2\s} \d x + S^{2m-1} \right).
\end{align*}

%%%%%%%%%

\textbf{3.2. Estimate for $I_u(\ocal \setminus B_{S+1}, \ocal \cap B_S)$.} First, we claim that $|x-y|\geq d_S(x)$ whenever $x\in B_S\cap \ocal$ and $y\in \R^{2m}\setminus B_{S+1}$. Indeed, if $x\in B_S$, then it is easy to see that $d_S(x) \leq S+1-|x|$ and therefore we have $|x-y|\geq |y|-|x|\geq |y|+d_S(x)-S-1 \geq  d_S(x)$, since $|y| \geq S+1$. Thus, using this inequality and the ellipticity of $K$, we get
\begin{align*}
I_u(\ocal \setminus B_{S+1}, \ocal \cap B_S) &\leq C \int_{B_S \cap \ocal} \d x \int_{\R^{2m}\setminus B_{S+1}} \d y \ \frac{|u(x)-u(y)|^2}{|x-y|^{2m+2\s}} \\
& \leq C \int_{B_S \cap \ocal} \d x \int_{|x-y|\geq d_S(x)} \d y \ |x-y|^{-2m-2\s} \\
&\leq C \int_{B_S \cap \ocal} d_S(x)^{-2\s} \d x.
\end{align*}

\textbf{Step 4. Conclusion.}

Finally, by adding up the estimates of Step~3 and applying Lemma~\ref{Lemma:Integrability_dFunction}, we obtain the desired result. That is,
\begin{align*}
\ecal(u,B_S) &\leq \ecal(\Psi_S,B_{S+2}) + I_u(\ocal \setminus B_{S+1}, \ocal \cap B_S) \leq C\left(\int_{B_S \cap \ocal} d_S(x)^{-2\s} \d x + S^{2m-1} \right)\\
&\leq \begin{cases}
C \ S^{2m-2\s}\ \ \ &\textrm{if } \ \ \s\in(0,1/2),\\
C\ \log(S)\,S^{2m-2\s}\ \ \ \ &\textrm{if } \ \ \s=1/2,\\
C \ S^{2m-1}\ \ \ \ &\textrm{if } \ \ \s\in(1/2,1).\\
\end{cases}
\end{align*}
\end{proof}


%%%%%%%%%%%%%%%%%%%%%%%%%%%%%%%%%%%%%%%%%%%%%%%%%%
\section{Existence of saddle-shaped solution: monotone iteration method}
%%%%%%%%%%%%%%%%%%%%%%%%%%%%%%%%%%%%%%%%%%%%%%%%%%
\label{Sec:Existence}


In this section we give a proof of Theorem~\ref{Th:Existence} based on the maximum principle and the existence of a positive subsolution. To do this, we need a version of the monotone iteration procedure for doubly radial functions which are odd with respect to the Simons cone $\ccal$. Along this section we will call odd sub/supersolutions to problem \eqref{Eq:SemilinearSolutionInBall} the functions that are doubly radial, odd with respect to the Simons cone and satisfy the corresponding problem in \eqref{Eq:SemilinearSubSuperSolutionInBall}. In view of Remark~\ref{Remark:MaxPrincipleSingularity}, we do not need the operator to be finite in the whole set when applied to a subsolution (respectively supersolution), it can be $-\infty$ (respectively $+\infty$) at some points.

\begin{proposition}
	\label{Prop:MonotoneIterationOdd}
	Let $\s\in (0,1)$ and let $K$ be a radially symmetric kernel such that $L_K\in \lcal_0(2m, \s)$ and satisfying the positivity condition \eqref{Eq:KernelInequality}. Assume that $\vsub \leq \vsup$ are two bounded functions which are doubly radial and odd with respect to the Simons cone. Furthermore, assume that $\vsub\in C^\s(\R^{2m})$ and that $\vsub$ and $\vsup$ satisfy respectively   
	\begin{equation}
	\label{Eq:SemilinearSubSuperSolutionInBall}
	\beqc{\PDEsystem}
	L_K\vsub & \leq & f(\vsub) & \textrm{ in } B_R \cap \ocal\,, \\
	\vsub & \leq & \varphi & \textrm{ in } \ocal \setminus B_R\,, 
	\eeqc
	\quad \textrm{ and } \quad 
	\beqc{\PDEsystem}
	L_K\vsup & \geq & f(\vsup) & \textrm{ in } B_R \cap \ocal\,, \\
	\vsup & \geq & \varphi & \textrm{ in } \ocal \setminus B_R\,, 
	\eeqc
	\end{equation}
	with $f$ a $C^1$ odd function and $\varphi$ a doubly radial odd function.
	
	Then, there exists $v\in C^{2\s+\varepsilon}(B_R)$ for some $\varepsilon>0$, a solution to
	\begin{equation}
	\label{Eq:SemilinearSolutionInBall}
	\beqc{\PDEsystem}
	L_K v & = & f(v) & \textrm{ in } B_R\,, \\
	v &=& \varphi &  \textrm{ in } \R^{2m} \setminus B_R\,, 
	\eeqc
	\end{equation}
	such that $v$ is doubly radial, odd with respect to the Simons cone and  $\vsub \leq v \leq \vsup$ in $\ocal$.
\end{proposition}


\begin{proof}
	The proof follows the classical monotone iteration method for elliptic equations (see for instance \cite{Evans}). We just give here a sketch of the proof. 
	First, let $M \geq 0$ be such that $-M \leq \vsub \leq \vsup \leq M$ and set
	$$
	b := \max \left \{{0, - \min_{[-M,M]}f'}\right \}\geq 0\,.
	$$
	Then one defines 
	$$
	\widetilde{L}_K w := L_Kw + b w 	\quad \text{ and } \quad 	g(\tau) := f(\tau) + b \tau\,.
	$$
	Therefore, our problem is equivalent to find a solution to
	$$
	\beqc{\PDEsystem}
	\widetilde{L}_Kv & = & g(v) & \textrm{ in } B_R\,, \\
	v &=& \varphi &  \textrm{ in } \R^{2m} \setminus B_R\,, 
	\eeqc
	$$
	such that $v$ is doubly radial, odd with respect to the Simons cone and  $\vsub \leq v \leq \vsup$ in $\ocal$. Here the main point is that $g$ is also odd but satisfies $g'(\tau) \geq 0$ for $\tau \in [-M,M]$. Moreover, since $b \geq 0$, $\widetilde{L}_K$ satisfies the maximum principle for odd functions in $\ocal$ (as in Proposition~\ref{Prop:MaximumPrincipleForOddFunctions}).
	
	We define $v_0 = \vsub$ and, for $k\geq 1$, let $v_k$ be the solution to the linear problem
	$$
	\beqc{\PDEsystem}
	\tilde{L}_K v_k & = & g(v_{k-1}) & \textrm{ in } B_R\,, \\
	v_k &=& \varphi &  \textrm{ in } \R^{2m} \setminus B_R\,. 
	\eeqc
	$$
	It is easy to see by induction and the regularity results from Proposition~\ref{Prop:InteriorRegularity} that $v_k\in L^\infty(B_R) \cap C^{2\s+2\varepsilon}(B_R)$ for some $\varepsilon>0$. Moreover, given $\Omega\subset B_R$ a compact set, then $||v_k||_{C^{2\s+2\varepsilon}(\Omega)}$ is uniformly bounded in $k$.
	
	Then, using the maximum principle it is not difficult to show by induction that 
	$$
	\vsub = v_0 \leq v_1 \leq \ldots \leq v_k \leq v_{k+1} \leq \ldots \vsup \quad \text{ in }\ocal\,,
	$$
	and that each function $v
	_k$ is doubly radial and odd with respect to $\ccal$. Finally, by Arzelà-Ascoli theorem and the compact embedding of H\"older spaces we see that, up to a subsequence, $v_k$ converges uniformly on compacts in $C^{2\s+\varepsilon}$ norm to the desired solution.
\end{proof}

In order to construct a positive subsolution, we also need a characterization and some properties of the first odd eigenfunction and eigenvalue for the operator $L_K$, which are presented next. This eigenfunction is obtained though a minimization of the corresponding Rayleigh quotient in the appropriate space. Before presenting our result, let us recall some functional spaces that we are going to use next. 

Given a set $\Omega \subset \R^{2m}$ and a translation invariant and positive kernel $K$ satisfying \eqref{Eq:Symmetry&IntegrabilityOfK}, we define the space
$$
\H^K_0(\Omega) := \setcond{w \in L^2(\Omega)}{w = 0 \quad \textrm{a.e. in } \R^{2m} \setminus \Omega \quad \textrm{ and } [w]^2_{\H^K(\R^{2m})} < + \infty},
$$
where
\begin{equation}
	\label{Eq:SeminormHK}
[w]^2_{\H^K(\R^{2m})} := \dfrac{1}{2}\int_{\R^{2m}} \int_{\R^{2m}} |w(x) - w(y)|^2 K(x-y) \d x \d y\,.
\end{equation}
Recall that when $K$ satisfies the ellipticity assumption \eqref{Eq:Ellipticity}, then $\H^K_0 (\Omega) = \H^\s_0 (\Omega)$, which is the space associated to the kernel of the fractional Laplacian, $K(y) = c_{n,\s}|y|^{-n-2\s}$. We also define
$$
\widetilde{\H}^K_{0, \, \mathrm{odd}}(\Omega) := \setcond{w \in \H^K_0(\Omega)}{w \textrm{ is doubly radial a.e. and odd with respect to } \ccal}.
$$
Recall that when $K$ is radially symmetric and $w$ is doubly radial, we can replace the kernel $K(x-y)$ in the definition \eqref{Eq:SeminormHK} by the kernel $\overline{K}(x,y)$. This is readily deduced after a change of variables and taking the mean among all $R\in O(m)^2$ (see the details in Secton~3 of \cite{FelipeSanz-Perela:IntegroDifferentialI}).




\begin{lemma}
	\label{Lemma:FirstOddEigenfunction}
	Let $\Omega\subset \R^{2m} $ be a bounded set of double revolution and let  $K$ be a radially symmetric kernel such that $L_K\in \lcal_0(2m, \s, \lambda, \Lambda)$ and satisfying the positivity condition \eqref{Eq:KernelInequality}. Let us define 
	\begin{equation}
	\label{Eq:DefLambda1}
	\lambda_{1, \, \mathrm{odd}}(\Omega, L_K) := \inf_{w \in \widetilde{\H}^K_{0, \, \mathrm{odd}}(\Omega)} \dfrac{\dfrac{1}{2}  \ds\int_{\R^{2m}} \int_{\R^{2m}} |w(x) - w(y)|^2 \overline{K}(x,y) \d x \d y}{ \ds \int_\Omega w(x)^2 \d x}\,.
	\end{equation}
	
	Then, such infimum is attained at a function $\phi_1\in \widetilde{\H}^K_{0, \, \mathrm{odd}}(\Omega)\cap L^\infty(\Omega)$ which solves
	$$
	\beqc{\PDEsystem}
	L_K \phi_1 &=& \lambda_{1, \, \mathrm{odd}}(\Omega, L_K) \phi_1 & \textrm{ in } \Omega\,,\\
	\phi_1 & = & 0 & \textrm{ in } \R^{2m}\setminus \Omega\,,
	\eeqc
	$$
	and satisfies that $\phi_1 > 0$ in $\Omega \cap \ocal$.
	We call such function the \emph{first odd eigenfunction of $L_K$ in $\Omega$} and $\lambda_{1, \, \mathrm{odd}}(\Omega, L_K) $ the \emph{first odd eigenvalue}. 
	
	Moreover, in the case $\Omega = B_R$, there exists a constant $C$ depending only on $n$, $\s$ and $\Lambda$ such that
	$$
	\lambda_{1, \, \mathrm{odd}}(B_R, L_K) \leq C R^{-2\s}\,. 
	$$ 
\end{lemma}


\begin{proof}
	The first two statements are deduced exactly as in Proposition~9 of \cite{ServadeiValdinoci}, using the same arguments as in  Lemma~3.4. of \cite{FelipeSanz-Perela:IntegroDifferentialI} to guarantee that $\phi_1$ is nonnegative in $\ocal$. The fact that $\phi_1 > 0$ in $\Omega \cap \ocal$ follows from the strong maximum principle (see Proposition~\ref{Prop:MaximumPrincipleForOddFunctions}).
	
	We show the third statement. Let $\widetilde{w} (x):= w(Rx)$ for every $w\in \widetilde{\H}^K_{0, \, \mathrm{odd}}(B_R)$. Then,
	\begin{align*}
	& \min_{w \in \widetilde{\H}^K_{0, \, \mathrm{odd}}(B_R)} \dfrac{\dfrac{1}{2}  \ds\int_{\R^{2m}} \int_{\R^{2m}} |w(x) - w(y)|^2 \overline{K}(x,y) \d x \d y}{ \ds \int_{B_R} w(x)^2 \d x} \quad \quad \quad \quad \quad \quad \quad \quad \quad \quad \quad \quad\\
	&   \quad \quad \quad \quad \quad \quad \leq \min_{\widetilde{w} \in \widetilde{\H}^K_{0, \, \mathrm{odd}}(B_1)} \dfrac{\dfrac{c_{n, \s}\Lambda}{2}  \ds\int_{\R^{2m}} \int_{\R^{2m}} |\widetilde{w}(x/R) - \widetilde{w}(y/R)|^2 |x - y|^{-n-2 \s}\d x \d y}{ \ds \int_{B_R} \widetilde{w}(x/R)^2 \d x}
	\\
	& \quad \quad \quad \quad \quad \quad = R^{-2 \s }\min_{\widetilde{w} \in \widetilde{\H}^s_{0, \, \mathrm{odd}}(B_1)} \dfrac{\dfrac{c_{n, \s}\Lambda}{2}  \ds\int_{\R^{2m}} \int_{\R^{2m}} |\widetilde{w}(x) - \widetilde{w}(y)|^2 |x - y|^{-n-2 \s}\d x \d y}{ \ds \int_{B_1} \widetilde{w}(x)^2 \d x}
	\\
	& \quad \quad \quad \quad \quad \quad = \lambda_{1, \, \mathrm{odd}}(B_1, \fraclaplacian) \Lambda R^{-2 \s } \,.
	\end{align*}
\end{proof}

\begin{remark}
	\label{Remark:CsRegularityFirstEigenfunction}
	Note that, by standard regularity results for $L_K$, we have that $\phi_1 \in C^\s(\overline{\Omega})\cap C^\infty(\Omega)$, and the regularity up to the boundary is optimal (see \cite{RosOton-Survey} and the references therein for the details). Due to this and the fact that $\phi_1 >0$ in $\Omega\cap \ocal$ while $\phi_1=0$ in $\R^{2m}\setminus \Omega$, it is easy to check by using \eqref{Eq:OperatorOddF} that $-\infty <L_K \phi_1 < 0$ in $\ocal\setminus \overline{\Omega}$ and $L_K \phi_1 = -\infty$ in $\partial \Omega \cap \ocal$.
\end{remark}

With these ingredients, we can proceed with the proof of Theorem~\ref{Th:Existence}.

\begin{proof}[Proof of Theorem~\ref{Th:Existence}]
	The strategy is to build a suitable solution $u_R$ of 
	\begin{equation}
	\label{Eq:ProofExistenceProblemBR}
	\beqc{\PDEsystem}
	L_K u_R &=& f(u_R) & \textrm{ in } B_R\,,\\
	u_R &=& 0 & \textrm{ in }\R^{2m} \setminus B_R\,,
	\eeqc
	\end{equation}
	and then let $R\to+ \infty$ to get a saddle-shaped solution.
	
	Let $\phi_1^{R_0}$ be the first odd eigenfunction of $L_K$ in $B_{R_0} \subset \R^{2m}$, given by Lemma~\ref{Lemma:FirstOddEigenfunction}, and let  $\lambda_1^{R_0} := \lambda_{1, \, \mathrm{odd}}(B_{R_0}, L_K)$. Then, we claim that for $R_0$ big enough and $\varepsilon$ small enough, $\usub_R = \varepsilon\phi_1^{R_0} $ is an odd subsolution of \eqref{Eq:ProofExistenceProblemBR} for every $R\geq R_0$. To see this, first note that, without loss of generality, we can assume that $\norm{\phi_1^{R_0}}_{L^\infty(B_R)}=1$. Then, since $\varepsilon \phi_1^{R_0}>0$ in $B_{R_0}\cap \ocal$ and using \eqref{Eq:PropertyConcavityf}, we see that for every $x\in B_{R_0}\cap \ocal$,
	$$
	\dfrac{f(\varepsilon \phi_1^{R_0}(x))}{\varepsilon \phi_1^{R_0}(x)} > f'(\varepsilon \phi_1^{R_0}(x)) \geq f'(0)/2
	$$
	if $\varepsilon$ is small enough, independently of $x$. Therefore, since $f'(0)>0$, taking $R_0$ big enough so that $\lambda_1^{R_0} < f'(0)/2$ (see the last statement of Lemma~\ref{Lemma:FirstOddEigenfunction}), we have that for every $x\in B_{R_0}\cap \ocal$,  $f(\varepsilon \phi_1^{R_0}(x)) > \lambda_1 \varepsilon \phi_1^R(x)$. Thus,
	$$
	L_K \usub_R = \lambda_1^{R_0} \varepsilon \phi_1^{R_0} < f(\varepsilon\phi_1^{R_0}) = f(\usub_R) \quad \textrm{ in } B_{R_0}\cap \ocal\,.
	$$
	In addition, if $x\in (B_R\setminus B_{R_0})\cap\ocal$, by Remark~\ref{Remark:CsRegularityFirstEigenfunction} we have that
	$$
	L_K \usub_R < 0 = f(0) =  f(\usub_R) \quad \textrm{ in } (B_R\setminus B_{R_0})\cap \ocal\,.
	$$
	Hence, the claim is proved.
	
	Now, if we define $\usup_R := \chi_{\ocal \cap B_R} - \chi_{\ical \cap B_R}$, a simple computation shows that it is an odd supersolution to \eqref{Eq:ProofExistenceProblemBR}. Therefore, using the monotone iteration procedure given in Proposition~\ref{Prop:MonotoneIterationOdd} (taking into account Remarks~\ref{Remark:MaxPrincipleSingularity} and \ref{Remark:CsRegularityFirstEigenfunction} when using the maximum principle), we obtain a solution $u_R$ to \eqref{Eq:ProofExistenceProblemBR} such that it is doubly radial, odd with respect to the Simons cone and $\varepsilon \phi_1^{R_0} = \usub_R \leq u_R \leq \usup_R$ in $\ocal$. Note that, since $\usub_R > 0$ in $\ocal \cap B_{R_0}$, the same holds for $u_R$.
	
	Using a standard compactness argument as in \cite{FelipeSanz-Perela:IntegroDifferentialI}, we let $R\to +\infty$ to obtain a sequence $u_{R_j}$ converging on compacts in  $C^{2\s + \eta}(\R^{2m})$ norm, for some $\eta > 0$, to a solution $u \in C^{2\s + \eta}(\R^{2m})$ of $L_K u = f(u)$ in $\R^{2m}$. Note that $u$ is doubly radial, odd with respect to the Simons cone and $0\leq u \leq 1$ in $\ocal$. Let us show that $0 < u < 1$ in $\ocal$ and hence $u$ is a saddle-shaped solution. Indeed, the usual strong maximum principle yields $u<1$ in $\ocal$. Moreover, since $u_R\geq\varepsilon \phi_1^{R_0}>0$ in  $\ocal \cap B_{R_0}$ for $R>R_0$, also the limit $u\geq\varepsilon \phi_1^{R_0}>0$ in  $\ocal \cap B_{R_0}$. Therefore, by applying the strong maximum principle for odd functions (see Proposition~\ref{Prop:MaximumPrincipleForOddFunctions}) we obtain that $0 < u < 1$ in $\ocal$.
\end{proof}


The fact of being $u$ positive in $\ocal$ yields that $u$ is stable in this set, as explained in the following remark. 








\appendix


%%%%%%%%%%%%%%%%%%%%%%%%%%%%%%%%%%%%%%%%%%%%%%%%%%%%%%%%%%%%%%%%%%%%%%%%%%%%
%%%%%%%%%%%%%%%%%%%%%%%%%%%%%%%%%%%%%%%%%%%%%%%%%%%%%%%%%%%%%%%%%%%%%%%%%%%%
\section{Some auxiliary results on convex functions}
\label{Sec:AuxiliaryResults}
%%%%%%%%%%%%%%%%%%%%%%%%%%%%%%%%%%%%%%%%%%%%%%%%%%%%%%%%%%%%%%%%%%%%%%%%%%%%
%%%%%%%%%%%%%%%%%%%%%%%%%%%%%%%%%%%%%%%%%%%%%%%%%%%%%%%%%%%%%%%%%%%%%%%%%%%%

In this appendix we present some auxiliary results concerning convex functions. The main result, used in the proof of Theorem~\ref{Th:SufficientNecessaryConditions}, is the following.


%Recall that for measurable functions $f:\R\to \R$, convexity in an open interval $I$ is equivalent to midpoint convexity, i.e.,
%$$
%\dfrac{f(x) + f(y)}{2} \geq f \left( \dfrac{x+y}{2}\right) \quad \textrm{ for every } x,\, y \in I\,,
%$$
%and the same is true for strict convexity with an strict inequality
%(see Chapter~1 of \cite{Niculescu} and the references therein).

\begin{proposition}
	\label{Prop:EquivalenceK(sqrt)Convex<->Inequality}
	Let $K:(0, +\infty) \to (0,+\infty)$ be a measurable function. Then, the following statements are equivalent:
	\begin{enumerate}
		\item[i)] $K(\sqrt{\cdot})$ is strictly convex in $(0, +\infty)$.
		\item[ii)] For every positive constants $c_1$ and $c_2$, the function $g:(0,1/c_2)\to \R$ defined by
		\begin{equation}
		\label{Eq:DefinitiongFromK}
		g(z) := K(c_1 \sqrt{1 + c_2z}) + K(c_1 \sqrt{1 - c_2z})
		\end{equation}
		satisfies 
		\begin{equation}
		\label{Eq:InequalityConvexFunctions}
		g(A) + g(D) \geq g(B) + g(C)
		\end{equation}
		whenever $A$, $B$, $C$ and $D$ belong to $(0, 1/c_2)$ and satisfy
		$$
		A = \max\{A,\, B,\, C,\, D\} \quad \text{ and } \quad A + D \geq B + C.
		$$
		
		In addition, still assuming $A = \max\{A,\, B,\, C,\, D\}$ and $A + D \geq B + C$, equality holds in \eqref{Eq:InequalityConvexFunctions} if and only if the sets $\{A,D\}$ and $\{C,B\}$ coincide.	
	\end{enumerate}
\end{proposition}



To prove this proposition, we need a lemma on convex functions. 


\begin{lemma}
	\label{Lemma:ConvexFunctions}
	Let $0<M\leq +\infty$ and let $h:(0,M)\to \R$ be a measurable nondecreasing function. Then, the following statements are equivalent.
	
	\begin{enumerate}[label=(\alph*)]
		\item $h$ is convex in $(0,M)$.
		
		\item For every $0\leq L\leq 2M$, the function $\tilde{h}_L (x) := h(x) + h(L-x)$ is convex in $(\max \{L-M,0\}, \min \{L,M\})$.
		
		\item For every $A$, $B$, $C$, $D$ in the interval $(0,M)$ such that
		$$
		A = \max\{A,\, B,\, C,\, D\} \quad \text{ and } \quad A + D \geq B + C\,,
		$$
		it holds
		\begin{equation}
			\label{Eq:InequalityConvexFunctionsbis}
			h(A) + h(D) \geq h(B) + h(C)\,.
		\end{equation}
	\end{enumerate}
\end{lemma}

\begin{proof}
	$(a)\Rightarrow (c)$.	Since $B$ and $C$ are interchangeable and $h$ is nondecreasing, we may assume that $A \geq B \geq C \geq D$. Now, let $M_C$ be the maximum slope of the supporting lines of $h$ at $C$, and let $m_B$ be the minimum slope of the supporting lines of $h$ at $B$. By the convexity and monotonicity of $h$, it holds $m_B \geq M_C\geq 0$ and also
	$$
	h(x) \geq h(B) + m_B (x-B) \quad \text{ and } \quad h(x) \geq h(C) + M_C (x-C) 
	$$
	for every $x \in (0,M)$.
	
	Hence, since $A-B \geq C-D\geq 0$, we have
	$$
	h(A)-h(B) \geq m_B(A-B) \geq M_C (C-D) \geq h(C) - h(D)\,.
	$$
	
	$(c) \Rightarrow (b)$. Let $x$, $y\in (\max \{L-M,0\}, \min \{L,M\})$ and assume that $x>y$. By taking $A=x$, $B=C=(x+y)/2$, and $D = y$ in \eqref{Eq:InequalityConvexFunctionsbis}, we get 
	$$
	\dfrac{h(x) + h(y)}{2} \geq h \left( \dfrac{x+y}{2}\right). 
	$$ 
	Similarly, by taking $A= L-y$, $B=C=L -(x+y)/2$, and $D = L-x$ in \eqref{Eq:InequalityConvexFunctions}, we get 
	$$
	\dfrac{h(L-x) + h(L-y)}{2} \geq h \left(L - \dfrac{x+y}{2}\right). 
	$$
	By adding up the previous two inequalities we obtain
	$$
	\dfrac{\tilde{h}_L(x) + \tilde{h}_L(y)}{2} \geq \tilde{h}_L \left( \dfrac{x+y}{2}\right). 
	$$ 
	
	$(b) \Rightarrow (a)$. Let $x_0$, $y_0 \in (0,M)$ and choose $L = x_0 + y_0 \leq 2M$. By $(b)$ we have 
	$$
	\dfrac{h(x) + h(x_0 + y_0-x) + h(y) + h(x_0 + y_0-y)}{2} \geq h \left( \dfrac{x+y}{2}\right) + h \left(x_0 + y_0 - \dfrac{x+y}{2}\right),
	$$
	for every $x$ and $y$ in the interval $(\max \{L-M,0\}, \min \{L,M\})$. By choosing $x=x_0$ and $y=y_0$ we obtain
	$$
	h(x_0) + h(y_0)\geq 2 h \left( \dfrac{x_0+y_0}{2}\right). 
	$$
\end{proof}


\begin{remark}
	\label{Remark:StrictConvexity}
	We can replace convexity by strict convexity in $(a)$ and $(b)$ and then the inequality in \eqref{Eq:InequalityConvexFunctionsbis} is strict unless the sets $\{A,D\}$ and $\{C,B\}$ coincide.
\end{remark}

\begin{remark}
	\label{Remark:h_Lincreasing}
	Note that the function $\tilde{h}_L$ is even with respect to $L/2$. Thus, if it is convex, it is nondecreasing in $(L/2, \min \{L,M\})$.
\end{remark}

\begin{remark}
	\label{Remark:hypothesisNondecreasing}
	The assumption of $h$ being nondecreasing it is only used to deduce $(c)$ from $(a)$. It is not required to show the equivalence between $(a)$ and $(b)$, neither to deduce $(a)$ from $(c)$.
\end{remark}

With this result available we can show now Proposition~\ref{Prop:EquivalenceK(sqrt)Convex<->Inequality}

\begin{proof}
	$i) \Rightarrow ii)$ We take $M = +\infty$ and $h(\cdot) = K(\sqrt{\cdot})$ in Lemma~\ref{Lemma:ConvexFunctions}. Since $h$ is strictly convex, the function $\tilde{h}_L$ is strictly convex in $(0,L)$ for every $L> 0$ (recall that we do not need to assume that $h$ is monotone to deduce this, see Remark~\ref{Remark:hypothesisNondecreasing}). Moreover, by Remark~\ref{Remark:h_Lincreasing}, $\tilde{h}_L$ is nondecreasing in $(L/2,L)$. Thus, the function $\phi(\cdot) = \tilde{h}_L(\cdot + L/2)$ is strictly convex in $(-L/2,L/2)$ and nondecreasing in $(0,L/2)$. If we choose $L=2c_1^2$, we have that $\phi((L/2)c_2 \cdot) = g(\cdot)$, where $g$ is defined by \eqref{Eq:DefinitiongFromK}. Therefore, $g$ is strictly convex in $(-1/c_2, 1/c_2)$ and nondecreasing in $(0,1/c_2)$ and the result follows by applying  Lemma~\ref{Lemma:ConvexFunctions} to $g$ in $(0,1/c_2)$ (taking into account Remark~\ref{Remark:StrictConvexity}).
	
	
	$ii) \Rightarrow i)$ By Lemma~\ref{Lemma:ConvexFunctions} applied to $g$ we deduce that $g$ is strictly convex in $(0,1/c_2)$ ---recall that by Remark~\ref{Remark:hypothesisNondecreasing} we do not need to assume that $g$ is monotone. Thus, $\varphi(\cdot) = g(\cdot/(c_1^2 c_2))$ is strictly convex in $(-c_1^2, c_1^2)$. Hence, if we call $h(\cdot) := K(\sqrt{\cdot})$ and $L:= 2c_1^2$, we have that $\varphi(\cdot - c_1^2) = h(\cdot) + h(L-\cdot) =:  \tilde{h}_L(\cdot)$, and thus $\tilde{h}_L$ is strictly convex in $(0,L)$. Note that since $c_1>0$ is arbitrary, $\tilde{h}_L$ is strictly convex in $(0,L)$ for all $L>0$. Therefore, by Lemma~\ref{Lemma:ConvexFunctions}, with  $M = +\infty$, we conclude that $h(\cdot) = K(\sqrt{\cdot})$ is strictly convex in $(0,+\infty)$.
\end{proof}













%
%
%
%
%The first one is the following.
%
%\begin{lemma}
%\label{Lemma:Convex<->AllReflectionsConvex} Let $h: I \subset \R \to \R$ be a function defined in
%an open interval $I$ such that it is measurable. Then, $h(z)$ is convex in $I$ if and only if
%$\widetilde{h}_c(z) := h(c+z) + h(c-z)$ is convex in $I_c := (-\dist\{c, \partial I\}, \dist\{c,
%\partial I\})$ for every $c\in I$. The statement remains true if we replace convexity by strict
%convexity, concavity or strict concavity.
%\end{lemma}
%
%\begin{proof}
%First, let us assume that $h$ is convex in $I$. We call
%$$
%x_+ = c + x\,, \quad x_- = c - x\,, \quad y_+ = c + y\,, \quad \textrm{ and } \quad y_- = c - y\,.
%$$
%Then, if $x$, $y\in I_c$, we have that $x_+$, $x_-$, $y_+$, $y_- \in I$. Hence, for all $\tau\in(0,1)$,
%\begin{align*}
%\tau\widetilde{h}_c(x) + (1-\tau)\widetilde{h}_c(y)
%&=  \tau h(x_+) + (1-\tau )h(y_+) + \tau  h(x_-) + (1-\tau )h(y_-) \\
%&\geq h(\tau x_+ + (1-\tau )y_+) + h(\tau x_- + (1-\tau )y_-) \\
%&= h(c + \tau x + (1-\tau )y) + h(c-\tau x + (1-\tau )y) \\
%& = \widetilde{h}_c(\tau x + (1-\tau )y)\,.
%\end{align*}
%Therefore, $\widetilde{h}_c(z)$ is convex in $I_c$ for every $c\in I$.
%
%Assume now that $\widetilde{h}_c(z)$ is convex in $I_c$ for every $c\in I$. By contradiction,
%suppose that $h$ is not convex in $I$. Then, there exist some $x$, $y\in I$ such that
%\begin{equation}
%\label{Eq:ContradictionConvexity}
%\dfrac{h(x) + h(y)}{2} < h \left (\dfrac{x+y}{2}\right )\,.
%\end{equation}
%
%Let $c = (x+y)/2$ and thus
%$$
%\widetilde{h}_c(z) = h\left( \dfrac{x+y}{2} + z\right) +  h\left( \dfrac{x+y}{2} - z\right)\,.
%$$
%Define $ x_0 := (x-y)/2$ and $y_0:= (y-x)/2$. It is clear that $x_0$, $y_0\in I_c$. Therefore,
%$$
%h(x) + h(y) = \dfrac{1}{2} \left( \widetilde{h}_c(x_0) + \widetilde{h}_c(y_0)\right )
%\geq \widetilde{h}_c \left( \dfrac{x_0 + y_0}{2}\right )
%= 2 h \left (\dfrac{x+y}{2}\right )\,,
%$$
%and this contradicts \eqref{Eq:ContradictionConvexity}. Hence, $h$ is convex in $I$.
%\end{proof}
%
%From this result we deduce an immediate corollary.
%
%\begin{corollary}
%\label{Cor:gConvex<->K(sqrt)convex} Let $K:(0,+\infty) \to (0,+\infty)$ be a measurable function.
%Then, given any $c_1,c_2>0$, the function
%$$
%g(z) := K \left (c_1 \sqrt{1 + c_2 z}\right) +  K \left (c_1 \sqrt{1 - c_2 z}\right)
%$$
%is  (strictly) convex in $(-1/c_2, 1/c_2)$ for every $c_1>0$ if and only if $K(\sqrt{z})$ is
%(strictly) convex in $(0, +\infty)$.
%\end{corollary}
%\begin{proof}
%Since we can rewrite $g$ as
%$$
%g(z) = K \left (\sqrt{c_1^2 + c_1^2c_2 z}\right) +  K \left (\sqrt{c_1^2 - c_1^2c_2 z}\right),
%$$
%it is clear that $g$ is  (strictly) convex in $(-1/c_2, 1/c_2)$ for every $c_1>0$ if and only if
%$$
%K \left(\sqrt{c_1^2 + z}\right) +  K \left(\sqrt{c_1^2 - z}\right)
%$$
%is (strictly) convex in $(-c_1^2, c_1^2)$ for every $c_1>0$. Then, by applying
%Lemma~\ref{Lemma:Convex<->AllReflectionsConvex} with $I = (0,+\infty)$, this is equivalent to the
%convexity of $K(\sqrt{z})$ in $(0, +\infty)$.
%\end{proof}
%
%The following is a characterization of the nondecreasing convex functions.
%
%\begin{lemma}
%\label{Lemma:InequalitConvexFunctions} Let $g \colon I\subset \R \to \R$ be a measurable and nondecreasing function defined in
%an open interval $I$. Then, we have the
%following equivalences:
%\begin{enumerate}
%\item[i)] $g$ is strictly convex in $I$.
%\item[ii)] For any given real numbers $A$, $B$, $C$, $D$ $\in I$ such that
%\begin{equation}
%\label{Eq:AssumptionsInequalitiesABCD}
%\begin{cases}
%A = \max\{A,B,C,D\}\,, \\
%A + D \geq B + C\,,
%\end{cases}
%\end{equation}
%it is satisfied that
%$$
%g(A) + g(D) \geq g(B) + g(C),
%$$
%and the equality holds if and only if
%$$ A = B \ \ \ \ \textrm{ and} \ \ \ \ C=D, $$
%or
%$$ A = C \ \ \ \ \textrm{ and} \ \ \ \ B=D. $$
%\end{enumerate}
%
%\end{lemma}
%\begin{proof}
%$i)\, \Rightarrow \,ii)$ First let us point out some properties of $g$. Since $g$ is strictly convex, then $g$ is continuous in $I$ and differentiable except for, at most, a countable set of points.  Moreover, its left derivative $g'_-$ is well defined at every point of $I$ and the fundamental theorem of calculus holds. Furthermore, since the convexity is strict, $g'_-$ is increasing. For more details on these results on convex functions and their differentiability properties, see Chapter~V of \cite{Rockafellar1970}, in particular Corollaries 24.2.1 and 26.3.1.
%
%Without loss of generality, we can assume that $B \geq C$. Now, we distinguish two cases: either $D \geq C$ or $D < C$. In the first case, note that since $g$ is strictly convex and nondecreasing, it is in fact increasing and therefore $g(A) + g(D) \geq g(B) + g(C)$ holds trivially. By the same reason, we obtain the strict inequality if $A>B$ or $D>C$. Suppose now that $D < C$. In this case, we can assume $A > B \geq C > D$, since if $A = B$ we arrive at a contradiction with \eqref{Eq:AssumptionsInequalitiesABCD}. Now, by using the fundamental theorem of calculus we obtain
%\begin{align*}
%g(A) + g(D) - g(B) - g(C) &= \int_B^A g'_-(x) \d x - \int_D^C g'_-(x) \d x \\
%&\geq \int_B^{B+C-D} g'_-(x) \d x - \int_D^C g'_-(x) \d x  \\
%&= \int_D^C g'_-(x+B-D) - g'_-(x) \d x  \\
%& > 0\,,
%\end{align*}
%since $g'_-$ is nonnegative and increasing.
%
%$ii)\, \Rightarrow \,i)$ Given $x\neq y$ in $I$, that we can suppose $x>y$ without loss of generality, we take $A=x$, $B=C=(x+y)/2$ and $D=y$. Then we get $ g(x)+g(y) > 2g\left( (x+y)/2 \right)$.
%\end{proof}
%
%\begin{remark}
%\label{Remark:InequalitConvexFunctions} Note that the condition of strict convexity is only needed
%in order to characterize when the equality is satisfied. That is, with only a convexity condition
%we also obtain the inequality of statement $ii)$, although we are not able to determine when equality is satisfied.
%\end{remark}
%\begin{remark}
%\label{Remark:LeftImplicationDoNotRequireNondecreasing}
%The deduction of $i)$ from $ii)$ does not require $g$ to be nondecreasing.
%\end{remark}
%
%An equivalent version of the previous result but for nonincreasing functions is the following.
%
%\begin{corollary}
%\label{Cor:hDecreasingConvex} Let $h \colon I\subset \R \to \R$ be a measurable and nonincreasing function defined in an open
%interval $I$. Then, we have the following
%equivalences:
%\begin{enumerate}
%\item[i)] $h$ is convex in $I$.
%\item[ii)] For any given real numbers $a$, $b$, $c$, $d$ $\in I$ such that
%\begin{equation*}
%%\label{Eq:AssumptionsInequalitiesabcd}
%\begin{cases}
%a \geq b \geq c \geq d \,, \\
%a + d \leq b + c\,,
%\end{cases}
%\end{equation*}
%it is satisfied that
%$$ h(a) + h(d) \geq h(b) + h(c)\,.$$
%\end{enumerate}
%\end{corollary}
%
%\begin{proof}
%	The deduction of $i)$ from $ii)$ is exactly the same as in Lemma~\ref{Lemma:InequalitConvexFunctions}. Assume now that $i)$ holds and let us define $g(z) = h(a-z)$. It is clear that since $h$ is measurable and nonincreasing, then $g$ is measurable and nondecreasing. On the other hand, let $A=a-d$, $B=a-c$, $C=a-b$ and $D=0$. Then,we have that condition $a \geq b \geq c \geq d$ is equivalent to $A\geq B \geq C \geq D$ and condition $a+d\leq b+c$ is equivalent to $A+D\geq B+C$. Therefore, we can apply Lemma~\ref{Lemma:InequalitConvexFunctions}, taking into account Remark~\ref{Remark:InequalitConvexFunctions}, and the desired equivalence is obtained.
%\end{proof}
%
%The last result we need to establish Proposition~\ref{Prop:EquivalenceK(sqrt)Convex<->Inequality} is the following.
%
%\begin{lemma}
%\label{Lemma:gNondecreasing}
%Let $K:(0,+\infty) \to (0,+\infty)$ be a measurable function such that
%$$ \lim_{z\to+\infty} K(z) = 0 $$
%and $ K(\sqrt{z}) $ is convex/strictly convex in $(0,+\infty)$. Then, given any $c_1,c_2>0$ the
%function
%$$
%g(z) := K(c_1 \sqrt{1 + c_2 z}) +  K(c_1 \sqrt{1 - c_2 z})
%$$
%is nondecreasing/increasing in $(0, 1/c_2)$.
%\end{lemma}
%
%\begin{proof}
%First, note that from the hypothesis of $K$ we can deduce that $K(\sqrt{z})$, and also
%$K(c_1\sqrt{z})$, is convex and nonincreasing.
%
%Now, given $x\geq y$ with $x$, $y\in (0,1/c_2)$, we have
%\begin{align*}
%g(x)-g(y) &= K(c_1\sqrt{1+c_2 x}) + K(c_1\sqrt{1-c_2 x}) \\
%&\ \ \ \ - K(c_1\sqrt{1+c_2 y}) -K(c_1\sqrt{1-c_2 y}) \geq 0,
%\end{align*}
%where we have applied Corollary~\ref{Cor:hDecreasingConvex} with $h(z) = K(c_1\sqrt{z})$,
%$a=1+c_2x$, $b=1+c_2y$, $c=1-c_2y$ and $d=1-c_2x$.
%\end{proof}
%
%\begin{remark}
%	\label{Remark:Concavity}
%If we assume that $K$ is nonincreasing and concave, then we can prove in a similar way that $g$ is
%nondecreasing.
%\end{remark}
%
%With all these results at hand we can now prove Proposition~\ref{Prop:EquivalenceK(sqrt)Convex<->Inequality}.
%
%\begin{proof}[Proof of Proposition~\ref{Prop:EquivalenceK(sqrt)Convex<->Inequality}]
%$i)\, \Rightarrow \,ii)$ By Lemma~\ref{Lemma:gNondecreasing} and
%Corollary~\ref{Cor:gConvex<->K(sqrt)convex}, $g$ is strictly convex in
%$(-1/c_2,1/c_2)$ and nondecreasing in $(0,1/c_2)$ for all $c_2>0$. Therefore, point $ii)$ follows from
%Lemma~\ref{Lemma:InequalitConvexFunctions}.
%
%$ii)\, \Rightarrow \,i)$ By Lemma~\ref{Lemma:InequalitConvexFunctions} and in view of
%Remark~\ref{Remark:LeftImplicationDoNotRequireNondecreasing}, $g$ is strictly convex in
%$(-1/c_2,1/c_2)$ for all $c_2>0$. Hence, using Corollary~\ref{Cor:gConvex<->K(sqrt)convex} we deduce
%that $K(\sqrt{z})$ is strictly convex in $(0, +\infty)$.
%\end{proof}


%%%%%%%%%%%%%%%%%%%%%%%%%%%%%%%%%%%%%%%%%%%%%%%%%%%%%%%%%%%%%%%%%%%%%%%%%%%%
%%%%%%%%%%%%%%%%%%%%%%%%%%%%%%%%%%%%%%%%%%%%%%%%%%%%%%%%%%%%%%%%%%%%%%%%%%%%
\section{An auxiliary computation}
\label{Sec:AuxiliaryResults2}
%%%%%%%%%%%%%%%%%%%%%%%%%%%%%%%%%%%%%%%%%%%%%%%%%%%%%%%%%%%%%%%%%%%%%%%%%%%%
%%%%%%%%%%%%%%%%%%%%%%%%%%%%%%%%%%%%%%%%%%%%%%%%%%%%%%%%%%%%%%%%%%%%%%%%%%%%

In this appendix we present an auxiliary computation that is needed in Section~\ref{Sec:OperatorOddF} in order to complete the proof of Proposition~\ref{Prop:KernelInequalitySufficientCondition}.

\begin{lemma}
\label{Lemma:ComputationABCD} Let $\alpha$, $\beta$ be two real numbers satisfying $\alpha \geq
|\beta|$. Let $x=(x',x'')$, $y=(y',y'')\in \ocal \subset \R^{2m}$. Define
$$
\begin{array}{cc}
	A = |x'||y'|  \alpha + |x''||y''|\beta \,, \ \ \ \ \ &
	B = |x'||y''| \alpha + |x''||y'| \beta \,, \\
	C = |x''||y'| \alpha + |x'||y''| \beta \,, \ \ \ \ \ &
	D = |x''||y''|\alpha + |x'||y'|  \beta \,.
\end{array}
$$
Then,
\begin{enumerate}
\item It holds
$$
\begin{cases}
|A| \geq |B|,\ |A| \geq|C|, \ |A| \geq|D|\,, \\
|A| + |D| \geq |B| + |C|\,.
\end{cases}
$$
\item If either
$$ |A| = |B| \ \ \ \ \textrm{ and} \ \ \ \ |C| = |D|, $$
or
$$ |A| = |C| \ \ \ \ \textrm{ and} \ \ \ \ |B| = |D|, $$
then necessarily $\alpha = \beta = 0$.
\end{enumerate}

\end{lemma}
\begin{proof} The proof is elementary but requires to check some cases. In all of them we will use the following inequalities. Since $\alpha \geq |\beta |$,
$$
\alpha\geq 0 \quad \textrm{ and } \quad  -\alpha \leq \beta \leq \alpha\,.
$$
Moreover, since $x,y\in\ocal$, it holds
$$
|x'|>|x''| \quad \textrm{ and } \quad |y'|>|y''|\,.
$$


We start establishing the first statement. We show next that $A\geq 0$ and that
$$
A \geq |B|, \ A \geq |C| ,\ A \geq |D|\,.
$$


$\bullet$ $A \geq 0$:
$$
 A =  |x'||y'|  \alpha + |x''||y''|\beta \geq (|x'||y'|  - |x''||y''|)\alpha \geq 0\,.
$$

$\bullet$ $A \geq |B|$:
$$
A\pm B = (|x'|\alpha-|x''|\beta)(|y'|\pm |y''|) \geq 0\,.
$$

$\bullet$ $A \geq |C|$:
$$
A\pm C = (|y'|\alpha-|y''|\beta)(|x'|\pm |x''|)  \geq 0\,.
$$

$\bullet$ $A \geq |D|$:
$$
A\pm D = (|x'||y'| \pm |x''||y''|)(\alpha \pm \beta) \geq 0\,.
$$


It remains to show
$$
A + |D| \geq |B| + |C|\,.
$$
The proof of this fact is just a computation considering all the eight possible configurations of
the signs of $B$, $C$ and $D$. Since the roles of $B$ and $C$ are completely interchangeable, we
may assume that $B \geq C$ and we only need to check six cases. To do it, note first that
\begin{equation}
\label{Eq:LemmaABCDProof1}
A + D - B - C = (|x'|-|x''|)(|y'|-|y''|)(\alpha + \beta) \geq 0 \,,
\end{equation}
\begin{equation}
\label{Eq:LemmaABCDProof2}
A - D - B + C = (|x'|+|x''|)(|y'|-|y''|)(\alpha - \beta) \geq 0 \,,
\end{equation}
and
\begin{equation}
\label{Eq:LemmaABCDProof3}
A + D + B + C = (|x'|+|x''|)(|y'|+|y''|)(\alpha + \beta) \geq 0 \,,
\end{equation}
With these three relations at hand we check the six cases.

$\bullet$ If $B \geq 0$, $C \geq 0$ and $D \geq 0$, then by \eqref{Eq:LemmaABCDProof1} we have
$$
A + |D| - |B| - |C| = A + D - B - C \geq 0\,.
$$

$\bullet$ If $B \geq 0$, $C \geq 0$ and $D \leq 0$, we use the sign of $D$ and \eqref{Eq:LemmaABCDProof1} to
see that
$$
A + |D| - |B| - |C| = A - D - B - C =  (A + D - B - C) + (-2D) \geq 0\,.
$$

$\bullet$ If $B \geq 0$, $C \leq 0$ and $D \geq 0$, we use the sign of $D$ and \eqref{Eq:LemmaABCDProof2} to
see that
$$
A + |D| - |B| - |C| = A + D - B + C =  (A - D - B + C) + 2D \geq 0\,.
$$

$\bullet$ If $B \geq 0$, $C \leq 0$ and $D \leq 0$, then by \eqref{Eq:LemmaABCDProof2} we have
$$
A + |D| - |B| - |C| = A - D - B + C \geq 0\,.
$$

$\bullet$ If $B \leq 0$, $C \leq 0$ and $D \geq 0$, then by \eqref{Eq:LemmaABCDProof3} we have
$$
A + |D| - |B| - |C| = A + D + B + C \geq 0\,.
$$

$\bullet$ If $B \leq 0$, $C \leq 0$ and $D \leq 0$, we use the sign of $D$ and \eqref{Eq:LemmaABCDProof3} to see that
$$
A + |D| - |B| - |C| = A - D + B + C =  (A + D + B + C) + (-2D) \geq 0\,.
$$

This concludes the proof of the first statement.

We prove now the second point of the lemma. Since the roles of $B$ and $C$ are completely
interchangeable, we only need to show the result in the case $|A| = |B|$ and $|C| = |D|$.

Recall that $A \geq 0$. Hence, since $A = |B|$ and $|C| = |D|$, a simple computation shows that
$$
\alpha = \sign (B) \dfrac{|x''|}{|x'|}\beta \quad \textrm{ and } \quad
\beta = \sign (C) \sign(D) \dfrac{|x''|}{|x'|} \alpha \,.
$$
Hence, combining both equalities we obtain
$$
\alpha = \sign (B) \sign (C) \sign(D) \dfrac{|x''|^2}{|x'|^2} \alpha.
$$
Finally, if we assume $\alpha \neq 0$, then necessarily $\sign (B) \sign (C) \sign(D)=1$ and $|x'|
= |x''|$, but this is a contradiction with $x\in \ocal$. Therefore, $\alpha = 0$ and thus $\beta =
0$.
\end{proof}








%%%%%%%%%%%%%%%%%%%%%%%%%%%%%%%%%%%%%%%%%%%%%%%%%%%%%%%%%%%%%%%%%%%%%%%%%%%%
%%%%%%%%%%%%%%%%%%%%%%%%%%%%%%%%%%%%%%%%%%%%%%%%%%%%%%%%%%%%%%%%%%%%%%%%%%%%
\section{The integro-differential operator $L_K$ in the $(s,t)$ variables}
\label{Sec:stcomputations}
%%%%%%%%%%%%%%%%%%%%%%%%%%%%%%%%%%%%%%%%%%%%%%%%%%%%%%%%%%%%%%%%%%%%%%%%%%%%
%%%%%%%%%%%%%%%%%%%%%%%%%%%%%%%%%%%%%%%%%%%%%%%%%%%%%%%%%%%%%%%%%%%%%%%%%%%%

The goal of this appendix is to take advantage of the doubly radial symmetry of the functions we
are dealing with to write equation \eqref{Eq:NonlocalAllenCahn} in $(s,t)$ variables, passing from
an equation in $\R^{2m}$ to an equation in $(0,+\infty)\times (0,+\infty)\subset \R^2$.

\begin{lemma}
\label{Lemma:OperatorInSTVariables} Let $m \geq 1$, $\s\in(0,1)$ and let $w\in
C^\alpha(\R^{2m})$, with $\alpha > 2\s$, be a doubly radial function, i.e., depending only on the variables $s$ and $t$. Let $L_K$ be a rotation invariant operator, that is, $K(y) = K(|y|)$, of the form \eqref{Eq:DefOfLu}. Then, if we define $\tilde{w}:(0,+\infty)\times (0,+\infty) \to \R$ by $\tilde{w}(s,t) = w(s,0,...,0,t,0,...,0)$, it holds
$$ L_Kw(x) = \tilde{L}_K \tilde{w} (|x'|,|x''|), $$
with
%Then, for any $x = (s x_s, t x_t)$ with $x_s$, $x_t$ $\in \Sph^{m-1}$ ($x_s$, $x_t = \pm 1$ in the
%case $m=1$), $Lu(x)$ can be written in the following way:\todo{Is singular in $s=0$ or $t=0$??}
\begin{equation*}
\label{Eq:OperatorInSTVariables}
\widetilde{L}_K \tilde{w} (s,t) := \int_0^{+\infty}  \int_0^{+\infty} \sigma^{m-1} \tau^{m-1} \big(u(s,t) - u(\sigma, \tau)\big) J(s,t,\sigma, \tau)  \d \sigma\d \tau\,,
\end{equation*}
where:
\begin{enumerate}
	\item If $m= 1$,
	\begin{equation}
		\label{Eq:KernelInSTVariablesR2}
	J(s,t,\sigma, \tau) := \sum_{i=0}^1  \sum_{j =0}^1  K\Big(\sqrt{s^2 + t^2 + \sigma^2 + \tau^2 -2 s \sigma (-1)^i -2 t \tau (-1)^j}\Big)\,.
	\end{equation}
	
	\item If $m\geq 2$,
	\begin{align}
	J(s,t,\sigma, \tau) &:= c_m ^2  \int_{-1}^1  \int_{-1}^1  (1-\theta^2)^{\frac{m-2}{2}} (1-\overline{\theta}^2)^{\frac{m-2}{2}} \nonumber\\
	& \quad \quad \quad \quad \quad
	K\Big(\sqrt{s^2 + t^2 + \sigma^2 + \tau^2 -2 s \sigma \theta -2 t \tau \overline{\theta}}\Big) \d \theta \d \overline{\theta}\,, \label{Eq:KernelSTVariables2}
	\end{align}
	with
	$$
	c_m = \dfrac{2 \pi^{\frac{m-1}{2}}}{\Gamma (\frac{m-1}{2})}.
	$$
\end{enumerate}
%To compute it one should split three cases:
%$
%c_m = \begin{cases}  \dfrac{1}{\pi^2} & m= 2 \,,\\
%1 &  m= 3\,,\\
%\ds 4\pi^2 \prod_{k=1}^{m-3} \bpar{\int_0^\pi \sin^k \theta \d \theta }^2 & m \geq 4\,.
%\end{cases}
%$}
\end{lemma}


\begin{proof}
%We start with the case $m=1$. In this case, we will use explicitly that since $u$ is a function of
%$s$ and $t$, then $u$ is even with respect to the coordinate axis. Using this symmetry and the
%change $y = -\tilde{y}$, we have
%$$
%Lu(x) = \int_{\{y_2 > - y_1\}} \big( u(x) - u(y)\big) \{K(|x - y|) + K(|x + y|)\} \d y\,.
%$$
%If we call
%$$
%I(\Omega, x) := \int_{\Omega} \big( u(x) - u(y)\big) \{K(|x - y|) + K(|x + y|)\} \d y\,,
%$$
%then
%$$
%Lu(x) = I(\{y_2 > |y_1|\},x) + I(\{y_1 > |y_2|\},x)\,.
%$$
%We will check that $I(\{y_2 > |y_1|\},x)$ can be written in the form
%\eqref{Eq:OperatorInSTVariables} (integrated in the set $\tau > \sigma$). The computations for
%$I(\{y_1 > |y_2|\},x)$ are completely analogous.
%
%First, note that the set $\{y_2 > |y_1|\}$ can be written as
%$$
%\{y_2 > y_1 > 0\} \cup \phi(\{y_2 > y_1 > 0\}) \cup \{y_2 > 0, y_1 =0\}
%$$
%where $\phi$ is the reflection with respect to the $y_2$-axis. Therefore,
%\begin{align*}
%I(\{y_2 > |y_1|\}, x) & = \int_{\{y_2 > y_1 > 0\}} \big( u(x) - u(y)\big) \{K(|x - y|) + K(|x + y|)\} \d y  \\
%& \quad \quad + \int_{\phi(\{y_2 > y_1 > 0\})} \big( u(x) - u(y)\big) \{K(|x - y|) + K(|x + y|)\} \d y\,.
%\end{align*}
%By performing the change $\phi = \phi^{-1}$ in the second integral and using the symmetry of $u$,
%we end up with
%\begin{align*}
%I(\{y_2 > |y_1|\}, x) &=
%\int_{\{y_2 > y_1 > 0\}} \big( u(x) - u(y)\big) \cdot\\
%& \quad \quad \left \{K(|x - y|) + K(|x + y|) + K(|x - \phi_1(y)|) + K(|x + \phi_1(y)|) \right \} \d y\,.
%\end{align*}
%Then, if in the previous expression we write
%$$
%x = (s \sign(x_1), t \sign(x_2)) \quad \textrm{ and } \quad = (\sigma \sign(y_1), \tau \sign(y_2))\,,
%$$
%we find that
%$$
%I(\{y_2 > |y_1|\}, x) =
%\int_0^{\infty} \! \! \d \sigma \int_\sigma^{\infty} \! \! \d \tau \ \   \big( u(s,t) - u(\sigma, \tau)\big) J(s,t,\sigma, \tau)\,,
%$$
%with $J$ as in \eqref{Eq:KernelInSTVariablesR2}. Indeed, in $\{y_2 > y_1 > 0\}$ we have that
%$y=(\sigma, \tau)$, and it is not difficult to check that the expression
%$$
%K(|x - y|) + K(|x + y|) + K(|x - \phi_1(y)|) + K(|x + \phi_1(y)|)
%$$
%does not depend on $\sign(x_1)$ nor $\sign(x_1)$, so we can assume that $x=(s,t)$ and then
%\begin{align*}
%K(|x - y|) + K(|x + y|) + K(|x - \phi_1(y)|) + K(|x + \phi_1(y)|) = \\
%K\Big(\sqrt{s^2 + t^2 + \sigma^2 + \tau^2 + 2 s \sigma + 2 t \tau }\Big) +  K\Big(\sqrt{s^2 + t^2 + \sigma^2 + \tau^2 + 2 s \sigma  -2 t \tau }\Big) \\
%\quad + K\Big(\sqrt{s^2 + t^2 + \sigma^2 + \tau^2 -2 s \sigma + 2 t \tau }\Big) + K\Big(\sqrt{s^2 + t^2 + \sigma^2 + \tau^2 -2 s \sigma  -2 t \tau }\Big)\,.
%\end{align*}
%
%In a completely analogous way, we find that
%$$
%I(\{y_1 > |y_2|\},x) = \int_0^{\infty} \! \! \d \sigma \int_0^\sigma \! \! \d \tau \ \   \big( u(s,t) - u(\sigma, \tau)\big) J(s,t,\sigma, \tau)\,,
%$$
%and hence
%$$
%Lu(x) = \int_0^{\infty} \! \! \d \sigma \int_0^\infty \! \! \d \tau \ \   \big( u(s,t) - u(\sigma, \tau)\big) J(s,t,\sigma, \tau) =: \widetilde{L}u(s,t)\,.
%$$
%This concludes the proof when $m=1$.
%	
%We deal now with the case $m=2$.
Let $x = (s x_s, t x_t)$ with $x_s$, $x_t$ $\in \Sph^{m-1}$ and $y = (\sigma y_\sigma, \tau
y_\tau)$ with $y_\sigma$, $y_\tau$ $\in \Sph^{m-1}$. Then, decomposing $\R^{2m} = \R^m \times \R^m$
and using spherical coordinates in each $\R^m$ we obtain
\begin{align*}
L_Ku(x) &= \int_{\R^{2m}} \big( u(x) - u(y)\big) K( |x-y|) \d y &\\
&= \int_0^{+\infty}  \int_0^{+\infty} \sigma^{m-1} \tau^{m-1} \big(u(s,t) - u(\sigma, \tau)\big)  \\
&\quad \quad \quad \quad  \bpar{\int_{\Sph^{m-1}}  \int_{\Sph^{m-1}} K \Big( \sqrt{|sx_s - \sigma y_\sigma|^2 + |t x_t - \tau y_\tau|^2 } \Big) \d y_\sigma \d y_\tau } \d \sigma \d \tau
\end{align*}
Now, we define the kernel
\begin{equation}
\label{Eq:KernelSTVariablesProof1}
J(x_s, x_t, s,t,\sigma, \tau) := \int_{\Sph^{m-1}}  \int_{\Sph^{m-1}} K \Big( \sqrt{|sx_s - \sigma y_\sigma|^2 + |t x_t - \tau y_\tau|^2 }\Big ) \d y_\sigma \d y_\tau \,.
\end{equation}

First of all, it is easy to see that $J$ does not depend on $x_s$ nor $x_t$. Indeed, consider a
different point $(z_s, z_t)\in \Sph^{m-1} \times \Sph^{m-1}$ and let $M_s$ and $M_t$ be two
orthogonal transformations such that $M_s(x_s) = z_s$ and $M_t(x_t) = z_t$. Then, making the change
of variables $y_\sigma = M_s(\tilde{y}_\sigma)$ and $y_\tau = M_t(\tilde{y}_\tau)$, and using that
$M_s( \Sph^{m-1}) = M_t(\Sph^{m-1}) = \Sph^{m-1}$, we find out that
\begin{align*}
& \hspace{-1cm} J(z_s, z_t, s,t,\sigma, \tau) = \\
&= \int_{\Sph^{m-1}}  \int_{\Sph^{m-1}} K \Big( \sqrt{|s M_s(x_s) - \sigma y_\sigma|^2 + |t M_t(x_t) - \tau y_\tau|^2 }\Big) \d y_\sigma \d y_\tau \\
&= \int_{\Sph^{m-1}}  \int_{\Sph^{m-1}} K \Big( \sqrt{|s M_s(x_s) - \sigma M_s(\tilde{y}_\sigma)|^2 + |t M_t(x_t) - \tau M_t(\tilde{y}_\tau)|^2 }\Big) \d \tilde{y}_\sigma \d \tilde{y}_\tau \\
&= \int_{\Sph^{m-1}}  \int_{\Sph^{m-1}} K\Big ( \sqrt{|M_s(sx_s - \sigma \tilde{y}_\sigma)|^2 + |M_t(t x_t - \tau \tilde{y}_\tau)|^2 }\Big) \d \tilde{y}_\sigma \d \tilde{y}_\tau \\
&= \int_{\Sph^{m-1}}  \int_{\Sph^{m-1}} K\Big ( \sqrt{|sx_s - \sigma \tilde{y}_\sigma|^2 + |t x_t - \tau \tilde{y}_\tau|^2 }\Big) \d \tilde{y}_\sigma \d \tilde{y}_\tau \\
&= J(x_s, x_t, s,t,\sigma, \tau) \,.
\end{align*}

Therefore, we can replace $x_s$ and $x_t$ in \eqref{Eq:KernelSTVariablesProof1} by $e =(1,0,\ldots,
0) \in \Sph^{m-1}$. Thus, we have
\begin{equation*}
%\label{Eq:KernelSTVariablesProof2}
J(s,t,\sigma, \tau) := \int_{\Sph^{m-1}}  \int_{\Sph^{m-1}} K\Big( \sqrt{|s e - \sigma y_\sigma|^2 + |t e - \tau y_\tau|^2 }\Big) \d y_\sigma \d y_\tau \,.
\end{equation*}
For an easier notation, we rename $\omega = y_\sigma$ and $\tilde\omega = y_\tau$, and thus we have
\begin{align*}
|s e - \sigma y_\sigma|^2 + |t e - \tau y_\tau|^2 &= |s e - \sigma \omega|^2 + |t e - \tau \tilde\omega|^2\\
&= s^2 +\sigma^2 - 2 s \sigma e \cdot \omega + t^2 + \tau^2 - 2 t \tau e\cdot \tilde\omega \\
&= s^2 +\sigma^2 - 2 s \sigma \omega_1 + t^2 + \tau^2 - 2t \tau\tilde\omega_1\,.
\end{align*}
Then, we can rewrite $J$ as
\begin{equation*}
\label{Eq:KernelSTVariablesProof3}
J(s,t,\sigma, \tau) := \int_{\Sph^{m-1}}  \int_{\Sph^{m-1}} K\Big( \sqrt{s^2+\sigma^2- 2 s \sigma \omega_1 + t^2 + \tau^2 - 2t \tau\tilde\omega_1}\Big) \d \omega \d \tilde\omega \,.
\end{equation*}
At this point we have to distinguish the cases $m=1$ and $m\geq 2$. For the fist one, since
$\Sph^{0} = \{-1,1\}$ we directly obtain \eqref{Eq:KernelInSTVariablesR2}. For the second one,
since the integrand only depends on $\omega_1$ and $\tilde\omega_1$, we proceed as follows
\begin{align*}
\label{Eq:KernelSTVariablesProof4}
J(s,t,\sigma, \tau) &= \int_{\Sph^{m-1}}  \int_{\Sph^{m-1}} K\Big( \sqrt{s^2+\sigma^2- 2 s \sigma \omega_1 + t^2 + \tau^2 - 2t \tau\tilde\omega_1}\Big) \d \omega \d \tilde\omega \,\\
&= \int_{-1}^1 \d \omega_1 \int_{\partial B_{\rho(\omega_1)}} \d \omega_2\cdot\cdot\cdot\d \omega_m \int_{-1}^1 \d \tilde\omega_1 \int_{\partial B_{\rho(\tilde\omega_1)}} \d \tilde\omega_2\cdot\cdot\cdot\d \tilde\omega_m  \\
& \quad \quad \quad \quad \quad K\Big( \sqrt{s^2+\sigma^2- 2 s \sigma \omega_1 + t^2 + \tau^2 - 2t \tau\tilde\omega_1}\Big) \, \\
&= \int_{-1}^1 \int_{-1}^1  |\partial B_{\rho(\omega_1)}| |\partial B_{\rho(\tilde\omega_1)}|\,\\
& \quad \quad \quad \quad \quad K\Big( \sqrt{s^2+\sigma^2- 2 s \sigma \omega_1 + t^2 + \tau^2 - 2t \tau\tilde\omega_1}\Big) \d \omega_1 \d \tilde\omega_1.
\end{align*}
where $\rho(r) = \sqrt{1-r^2}$. Finally, we obtain \eqref{Eq:KernelSTVariables2} once we replace
$|\partial B_{r}|=c_m\,r^{m-2}$, where $c_m$ is the measure of the boundary of the ball of radius one in
$\R^{m-1}$.
\end{proof}

In the case the operator is the fractional Laplacian we can obtain an alternative expression of the kernel $J$ in terms of some hypergeometric functions. Although we are not using the next result in this work, we think that it could be very useful in future works, since we are writing the kernel in terms of functions that have already been studied and have interesting well-known properties.
\begin{lemma}
\label{Lemma:Appell} If $L_K = (-\Delta)^\s$ and $m\geq 2$, then
\begin{equation}
\label{Eq:Appell}
J(s,t,\sigma,\tau) = \frac{\pi^m\Gamma\left(\frac{m}{2}\right)^2}{\Gamma\left(\frac{m-1}{2}\right)^2\Gamma\left(\frac{m+1}{2}\right)^2} \frac{F_2\left( m+\s;\frac{m}{2},m;\frac{m}{2},m;\frac{4s\sigma}{(s+\sigma)^2+(t+\tau)^2},\frac{4t\tau}{(s+\sigma)^2+(t+\tau)^2} \right)}{[(s+\sigma)^2+(t+\tau)^2]^{m+\s}},
\end{equation}
where $F_2$ is the so-called Appell hypergeometric function (see \cite{Appell}).
\end{lemma}



\begin{proof}
If we take $K(z) = |z|^{-2m-2\s}$ in \eqref{Eq:KernelSTVariables2} we get
\begin{align*}
J(s,t,\sigma, \tau) = c_m ^2  \int_{-1}^1  \int_{-1}^1  \frac{(1-\theta^2)^{\frac{m-2}{2}} (1-\overline{\theta}^2)^{\frac{m-2}{2}}}{(s^2 + t^2 + \sigma^2 + \tau^2 -2 s \sigma \theta -2 t \tau \overline{\theta})^{m+\s}} \d \theta \d \overline{\theta}\,.
\end{align*}
Then, if we make the change of variables $\theta = 2\varpi_1-1$ and $\overline{\theta}=2\varpi_2-1$
we arrive at
\begin{align*}
J(s,t,\sigma, \tau) &= \frac{2^{2m-4} c_m^2}{[(s+\sigma)^2+(t+\tau)^2]^{m+\s}} \cdot \\
 & \quad \quad \quad \quad  \int_0^1 \int_0^1
\frac{\varpi_1^\frac{m-2}{2} (1-\varpi_1)^\frac{m-2}{2} \varpi_2^\frac{m-2}{2}
(1-\varpi_2)^\frac{m-2}{2}}{\left(1-\frac{4s\sigma}{(s+\sigma)^2+(t+\tau)^2}\,\varpi_1-\frac{4t\tau}{(s+\sigma)^2+(t+\tau)^2}\,\varpi_2
\right)^{m+\s}} \d \varpi_1 \d \varpi_2 \\
&= \frac{2^{2m-4} c_m^2}{[(s+\sigma)^2+(t+\tau)^2]^{m+\s}} \frac{\Gamma\left(\frac{m}{2} \right)^4}{\Gamma(m)^2} \cdot \\
& \quad \quad \quad \quad F_2\left( m+\s;\frac{m}{2},m;\frac{m}{2},m;\frac{4s\sigma}{(s+\sigma)^2+(t+\tau)^2},\frac{4t\tau}{(s+\sigma)^2+(t+\tau)^2} \right).
\end{align*}
We finally obtain \eqref{Eq:Appell} by using the duplication formula for the $\Gamma$-function.
\end{proof}

Now we rewrite the kernel inequality \eqref{Eq:KernelInequality} in $(s,t)$ variables. We
do not present a proof of this result since it is identical to the one of
Proposition~\ref{Prop:KernelInequalitySufficientCondition} but changing the notation.

\begin{lemma}
\label{Lemma:KernelInequalityCone} Let $m\geq 1$ and let $J$ the kernel defined in
\eqref{Eq:KernelSTVariables2} with $K(\sqrt{\cdot})$ strictly convex. Then, if $s>t$ and $\sigma > \tau$, we have
\begin{equation*}
%\label{Eq:KernelInequalityCone}
J(s,t,\sigma, \tau) > J(s,t,\tau, \sigma)\,.
\end{equation*}
\end{lemma}

%\begin{proof}
%We will just consider some simplifications of \eqref{Eq:KernelInequalityCone}. Eventually, we will
%use Lemma~\ref{Lemma:InequalitConvexFunctions} to deduce the desired inequality.
%
%First of all, note that it is enough to show that
%\begin{equation}
%\label{Eq:KernelInequalitySimplified1}
%\int_{-1}^1  \int_{-1}^1  (1-\alpha^2)^{\frac{m-3}{2}} (1-\beta^2)^{\frac{m-3}{2}}  \left \{\tilde{K}\Big(\sqrt{1 -2 s \sigma \alpha -2 t \tau \beta}\Big) - \tilde{K}\Big(\sqrt{1 -2 s \tau \alpha -2 t \sigma \beta}\Big)  \right \}\d \alpha \d \beta \geq 0\,,
%\end{equation}
%Where $\tilde{K}(z) = K((s^2 + t^2 + \sigma^2 + \tau^2)z)$. To see that this is equivalent to
%\eqref{Eq:KernelInequalityCone}, we just normalize the variables in the following way:
%$$
%\tilde{s} = \dfrac{s}{\sqrt{s^2 + t^2 + \sigma^2 + \tau^2}}\,, \quad \tilde{t} = \dfrac{t}{\sqrt{s^2 + t^2 + \sigma^2 + \tau^2}}\,,
%$$
%$$
%\tilde{\sigma} = \dfrac{\sigma}{\sqrt{s^2 + t^2 + \sigma^2 + \tau^2}}\, \quad \textrm{ and } \quad \tilde{\tau} = \dfrac{\tau}{\sqrt{s^2 + t^2 + \sigma^2 + \tau^2}}\,.
%$$
%
%The second simplification is the following. Since $s>t>0$ and $\sigma > \tau>0$, we may write
%$$
%s = (1+\varepsilon) t \quad \textrm{ and} \quad \sigma = (1 + \delta) \tau
%$$
%with $\varepsilon$, $\delta > 0$. Then, \eqref{Eq:KernelInequalitySimplified1} is equivalent to
%\begin{equation}
%\label{Eq:KernelInequalitySimplified2}
%\int_{-1}^1  \int_{-1}^1  (1-\alpha^2)^{\frac{m-3}{2}} (1-\beta^2)^{\frac{m-3}{2}}  \left \{\tilde{K}\Big(\sqrt{1 -2 t \tau \{(1+\varepsilon)(1+\delta) \alpha + \beta\}}\Big) - \tilde{K}\Big(\sqrt{1 -2 t \tau \{(1+\varepsilon) \alpha + (1+\delta)\beta\}}\Big)  \right \}\d \alpha \d \beta \geq 0\,.
%\end{equation}
%
%Now, we make some changes of variables to reduce the domain of integration. First, we divide
%$(-1,1)^2 \setminus \{|\alpha| = |\beta|\}$ into four sectors:
%$$
%Q_1 = \{\alpha > |\beta| \}\,, \quad Q_2 = \{\beta > |\alpha| \}\,, \quad Q_3 = \{\alpha < -|\beta|\}\,, \quad \textrm{ and } \quad Q_4 = \{ \beta < -|\alpha|\}\,.
%$$
%Consider the changes
%\begin{align*}
%  \psi_2 \colon Q_2 \  &\to \ Q_1 \\
%  \ (\alpha,\beta) &\mapsto (\beta,\alpha)
%\end{align*}
%\begin{align*}
%  \psi_3 \colon Q_3 \  &\to \ Q_1 \\
%  \ (\alpha,\beta) &\mapsto (-\alpha,-\beta)
%\end{align*}
%\begin{align*}
%  \psi_4 \colon Q_4 \  &\to \ Q_1 \\
%  \ (\alpha,\beta) &\mapsto (-\beta,-\alpha)
%\end{align*}
%Then, \eqref{Eq:KernelInequalitySimplified2} is now equivalent to show
%\begin{align*}
%\int \int_{Q_1} \left\{
%  \tilde{K}\Big(\sqrt{1 -2 t \tau \{(1+\varepsilon)(1+\delta) \alpha + \beta\}}\Big)
%+ \tilde{K}\Big(\sqrt{1 +2 t \tau \{(1+\varepsilon)(1+\delta) \alpha + \beta\}}\Big) \right. \\
%+ \tilde{K}\Big(\sqrt{1 -2 t \tau \{(1+\varepsilon)(1+\delta) \beta + \alpha\}}\Big)
%+ \tilde{K}\Big(\sqrt{1 +2 t \tau \{(1+\varepsilon)(1+\delta) \beta + \alpha\}}\Big) \\
%- \tilde{K}\Big(\sqrt{1 -2 t \tau \{(1+\varepsilon) \alpha + (1+\delta)\beta\}}\Big)
%- \tilde{K}\Big(\sqrt{1 +2 t \tau \{(1+\varepsilon) \alpha + (1+\delta)\beta\}}\Big) \\
%\left.- \tilde{K}\Big(\sqrt{1 -2 t \tau \{(1+\varepsilon) \beta + (1+\delta)\alpha\}}\Big)
%- \tilde{K}\Big(\sqrt{1 +2 t \tau \{(1+\varepsilon) \beta + (1+\delta)\alpha\}}\Big)
%\right\} \cdot \\
%\cdot (1-\alpha^2)^{\frac{m-3}{2}} (1-\beta^2)^{\frac{m-3}{2}} \d \alpha \d \beta \geq 0\,.
%\end{align*}
%Note that we are not considering anymore the set $\{|\alpha| = |\beta|\}$, that has measure zero.
%
%Now, if we call
%$$
%g(z) := \tilde{K}\Big(\sqrt{1 -2 t \tau z}\Big)
%+ \tilde{K}\Big(\sqrt{1 +2 t \tau z }\Big)\,,
%$$
%the previous inequality reads
%\begin{equation}
%\label{Eq:KernelInequalitySimplified3}
%\begin{split}
%	\int \int_{Q_1} \left\{
%	g\Big((1+\varepsilon)(1+\delta) \alpha + \beta\Big)
%	+ g\Big((1+\varepsilon)(1+\delta) \beta + \alpha\Big)
%	\right.\\
%	\left.
%	- g\Big((1+\varepsilon) \alpha + (1+\delta)\beta\Big)
%	- g\Big((1+\varepsilon) \beta + (1+\delta)\alpha\Big)
%	\right\} \cdot \\
%	\cdot (1-\alpha^2)^{\frac{m-3}{2}} (1-\beta^2)^{\frac{m-3}{2}} \d \alpha \d \beta \geq 0\,.
%\end{split}
%\end{equation}
%We claim that
%\begin{equation}
%\label{Eq:KernelInequalityLastSimplification}
%g\Big((1+\varepsilon)(1+\delta) \alpha + \beta\Big)
%+ g\Big((1+\varepsilon)(1+\delta) \beta + \alpha\Big)
% \geq
%g\Big((1+\varepsilon) \alpha + (1+\delta)\beta\Big)
%+g\Big((1+\varepsilon) \beta + (1+\delta)\alpha\Big) \,.
%\end{equation}
%This will conclude the proof.
%
%To prove the claim, we want to use Lemma~\ref{Lemma:InequalitConvexFunctions} with
%$$
%\begin{array}{cc}
%A = (1+\varepsilon)(1+\delta) \alpha + \beta\,, &
%B = (1+\varepsilon) \alpha + (1+\delta)\beta\,, \\
%C = (1+\delta)\alpha + (1+\varepsilon) \beta\,, &
%D = \alpha + (1+\varepsilon)(1+\delta) \beta\,.
%\end{array}
%$$
%Note that, since $\tilde{K}$ and $\sqrt{1 - z}$ are nonincreasing, $g$ is nondecreasing $[0,1)$.
%Moreover, since $\tilde{K}$ is convex and nonincreasing and $\sqrt{1 - z}$ is concave, $g$ is
%convex in $[0,1)$. But since $A$, $B$, $C$ and $D$ $\in (-1, 1)$ ---by the normalizations we have
%made--- and $g$ is even, we cannot apply directly Lemma~\ref{Lemma:InequalitConvexFunctions} ($g$
%is nonincreasing in $(-1,0]$). Instead, if we do it for  $|A|$, $|B|$, $|C|$ and $|D|$, then we can
%use the lemma in $[0,1)$. Hence, we should check that
%$$
%\begin{cases}
%|A| \geq |B|,\ |C|, \ |D|\,, \\
%|A| + |D| \geq |B| + |C|\,.
%\end{cases}
%$$
%The verification of these inequalities is a simple but tedious computation and it is presented in
%the appendix (see Lemma~\ref{Lemma:ComputationABCD}). Once we have these inequalities, we use
%Lemma~\ref{Lemma:InequalitConvexFunctions} to deduce
%$$
%g(|A|) + g(|D|) \geq g(|B|) + g(|C|)\,,
%$$
%which is equivalent to \eqref{Eq:KernelInequalityLastSimplification} since $g$ is even. This
%concludes the proof of \eqref{Eq:KernelInequalityCone}.
%
%
%
%Finally, to justify that the inequality in \eqref{Eq:KernelInequalityCone} is strict up to a set of
%measure zero, we consider the points where $J(s,t,\sigma, \tau) = J(s,t,\tau,\sigma)$. Following
%the previous arguments, this is equivalent to say that we have an equality in
%\eqref{Eq:KernelInequalitySimplified2}, and since the integrand of that equality is nonnegative, it
%is equivalent to say that we have an equality in \eqref{Eq:KernelInequalityLastSimplification} up
%to a set of measure zero. Then, we take into account the following: since the function $\sqrt{1-z}$
%is increasing and strictly convex and the kernel $K$ is decreasing, then $g''>0$ in $(0,1)$.
%Therefore, in view of Remark~\eqref{Remark:StrictInequalitConvexFunctions}, the equality $g(A) +
%g(D) = g(B) + g(C)$ is only possible in the case $A=B=C=D$, and it is easy to verify that this
%cannot happen unless $A=B=C=D=0$\todo{Check again}. This is equivalent to $s=t$ and $\sigma=\tau$,
%something that is impossible in $\ocal$.
%
%%, under these conditions, $\varepsilon = \delta = (\sqrt{5} - 1)/2$. If we fix such values of $\varepsilon$ and $\delta$ we have a set of measure zero, $ \{2 \tilde{s} = (1 + \sqrt{5}) \tilde{t} \} \cap \{2 \tilde{\sigma} = (1 + \sqrt{5}) \tilde{\tau} \}$, in the space of the normalized parameters $\tilde{s}$, $\tilde{t}$, $\tilde{\sigma}$, $\tilde{\tau}$, as well as in the space of original ones.
%\end{proof}


%%%%%%%%%%%%%%%%%%%%%%%%%%%%%%%%%%%%%%%%%%%%%%%%%%%%%%%%%%%%%%%%%%%%%%%%%%%%
%%%%%%%%%%%%%%%%%%%%%%%%%%%%%%%%%%%%%%%%%%%%%%%%%%%%%%%%%%%%%%%%%%%%%%%%%%%%
\section*{Acknowledgements}

The authors thank Xavier Cabré for his guidance and useful discussions on the topic of this paper.

%%%%%%%%%%%%%%%%%%%%%%%%%%%%%%%%%%%%%%%%%%%%%%%%%%%%%%%%%%%%%%%%%%%%%%%%%%%%
%%%%%%%%%%%%%%%%%%%%%%%%%%%%%%%%%%%%%%%%%%%%%%%%%%%%%%%%%%%%%%%%%%%%%%%%%%%%
\bibliographystyle{amsplain}
\bibliography{biblio}

\end{document}

