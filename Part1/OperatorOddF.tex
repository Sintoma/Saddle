%%%%%%%%%%%%%%%%%%%%%%%%
\section{Rotation invariant operators acting on doubly radial odd functions}
%%%%%%%%%%%%%%%%%%%%%%%%
\label{Sec:OperatorOddF}

This section is devoted to study rotation invariant operators of the class $\lcal_0$ when they act on doubly radial odd functions. First, we deduce an alternative expression for the operator in terms of a doubly radial kernel $\overline{K}$. Then, we present necessary and sufficient conditions on the kernel $K$ in order to $L_K$ belong to the class $\lcal_\star$ (we establish Theorem~\ref{Th:SufficientNecessaryConditions}). Finally, we show two maximum principles for odd functions.


%%%%%%%%%%%%%%%%%%%%%%%%%%%%%%%%%%%%%%%%%%%%%%%%%%%%%%%%%%%%%%%%%%%%%%%%%%%%%%%%%%%%%%%%%%%%%%%%%%%
%%%%%%%%%%%%%%%%%%%%%%%%%%%%%%%%%%%%%%%%%%%%%%%%%%%%%%%%%%%%%%%%%%%%%%%%%%%%%%%%%%%%%%%%%%%%%%%%%%%
\subsection{Alternative expressions for the operator $L_K$}
%%%%%%%%%%%%%%%%%%%%%%%%%%%%%%%%%%%%%%%%%%%%%%%%%%%%%%%%%%%%%%%%%%%%%%%%%%%%%%%%%%%%%%%%%%%%%%%%%%%
%%%%%%%%%%%%%%%%%%%%%%%%%%%%%%%%%%%%%%%%%%%%%%%%%%%%%%%%%%%%%%%%%%%%%%%%%%%%%%%%%%%%%%%%%%%%%%%%%%%

The main purpose of this subsection is to deduce an alternative expression for a rotation invariant operator $L_K \in \lcal_0$ acting on doubly radial functions. This expression is more suitable to work with and it will be used throughout the paper. Our first remark is that if $w$ is invariant by $O(m)^2$, the same holds for $L_Kw$. Indeed, for every $R \in O(m)^2$,
\begin{align*}
L_K w (Rx)
& = \int_{\R^{2m}} \{w(Rx) - w(y)\} K(|Rx - y|)  \d y\\
& = \int_{\R^{2m}} \{w(Rx) - w(R\tilde{y})\} K(|Rx - R\tilde{y}|) \d \tilde{y}\\
& = \int_{\R^{2m}} \{w(x) - w(\tilde{y})\} K(|x-\tilde{y}|) \d \tilde{y}\\
& = L_K w (x)\,.
\end{align*}
Here we have used the change $y = R\tilde{y}$ and the fact that $w(R \cdot) = w(\cdot)$ for every $R\in O(m)^2$.

Next, we present an alternative expression for the operator $L_K $ acting on doubly radial functions. This expression involves a new kernel which is also doubly radial.

\begin{lemma} \label{Lemma:AlternativeOperatorExpression}
Let $L_K \in \lcal_0(2m,\s)$ with $K$ radially symmetric, and let $w$ be a doubly radial function for which $L_K w$ is well-defined. Then, $L_K w$ can be expressed as
$$
L_K w(x) = \int_{\R^{2m}} \{w(x) - w(y)\} \overline{K}(x,y) \d y
$$
where $\overline{K}$ is symmetric, invariant by $O(m)^2$ in both arguments, and it is defined by
\begin{equation*}
%\label{Eq:KbarDef}
\overline{K}(x,y) := \average_{O(m)^2} K(|Rx - y|)\d R\,.
\end{equation*}
Here, $\d R$ denotes integration with respect to the Haar measure on $O(m)^2$.
\end{lemma}

Recall (see for instance \cite{Nachbin}) that the Haar measure on $O(m)^2$ exists and it is unique up to a
multiplicative constant. Let us state next the properties of this measure that will be used in the rest of the
paper. In the following, the Haar measure is denoted by $\mu$. First, since $O(m)^2$ is a compact
group, it is unimodular (see Chapter~II, Proposition~ 13 of \cite{Nachbin}). As a consequence, the
measure $\mu$ is left and right invariant, that is, $\mu(R\Sigma) = \mu(\Sigma) = \mu(\Sigma R) $
for every subset $\Sigma \subset O(m)^2$ and every $R\in O(m)^2$. Moreover, it holds
\begin{equation}
\label{Eq:Unimodular}
\average_{O(m)^2} g(R^{-1}) \d R = \average_{O(m)^2} g(R) \d R
\end{equation}	
for every $g\in L^1(O(m)^2)$ ---see \cite{Nachbin} for the details.

\begin{proof}[Proof of Lemma~\ref{Lemma:AlternativeOperatorExpression}]
Since $L_K w (x) = L_K w (Rx)$ for every $R\in O(m)^2$, by taking the mean over all the transformations in $O(m)^2$, we get
\begin{align*}
L_K w(x) &= \average_{O(m)^2} L_K w(Rx)\d R =  \average_{O(m)^2} \int_{\R^{2m}} \{w(x) - w(y)\}K(|Rx - y|) \d y \d R\\
&= \int_{\R^{2m}} \{w(x) - w(y)\}  \average_{O(m)^2} K(|Rx - y|) \d R  \d y = \int_{\R^{2m}} \{w(x) - w(y)\}  \overline{K}(x,y) \d y\,.
\end{align*}
Now, we show that $\overline{K}$ is symmetric. Using property \eqref{Eq:Unimodular}, we get
\begin{align*}
\overline{K}(y,x) &= \average_{O(m)^2} K(|R y - x|)\d R = \average_{O(m)^2} K(|R^{-1} (R y - x)|)\d R \\
&= \average_{O(m)^2} K(|R^{-1}x-y)|)\d R = \overline{K}(x,y)\,.
\end{align*}
It remains to show that
$\overline{K}$ is invariant by $O(m)^2$ in its two arguments. By the symmetry, it is enough to
check it for the first one. Let $\tilde{R} \in O(m)^2$. Then,
$$
\overline{K} (\tilde{R}x, y) = \average_{O(m)^2} K(|R \tilde{R} x - y|)\d R  = \average_{O(m)^2} K(|R x - y|)\d R = \overline{K} (x, y)\,,
$$
where we have used the right invariance of the Haar measure.
\end{proof}



In the following lemma we present some properties of the involution $(\cdot)^\star$ defined by \eqref{Eq:DefStar} and its relation with the doubly radial kernel $\overline{K}$ and the transformations of $O(m)^2$. In particular, in the proof of Theorem~\ref{Th:SufficientNecessaryConditions} it will be useful to consider the following transformation. For every $R\in O(m)^2$, we define  $R_\star\in O(m)^2$ by 
\begin{equation}
	\label{Eq:DefRStar}
	R_\star := (R(\cdot)^\star)^\star\,.
\end{equation}
Equivalently, if $R = (R_1, R_2)$ with $R_1$, $R_2 \in O(m)$, then $R_\star = (R_2, R_1)$.

\begin{lemma}
\label{Lemma:PropertiesStar}
Let $(\cdot)^\star: \R^{2m} \to \R^{2m}$ be the involution defined by $x^\star = (x',x'')^\star = (x'', x')$
---see \eqref{Eq:DefStar}.
Then,
\begin{enumerate}
\item
The Haar integral on $O(m)^2$ has the following invariance:
\begin{equation}
\label{Eq:InvarianceByStar}
\int_{O(m)^2} g(R_\star) \d R = \int_{O(m)^2} g(R) \d R \,,
\end{equation}
for every $g \in L^1(O(m)^2)$.
\item $\overline{K}(x^\star,y) = \overline{K} (x,y^\star)$.
\item $\overline{K}(x^\star,y^\star) = \overline{K} (x,y)$.
\end{enumerate}
\end{lemma}

\begin{proof}
The first statement is easy to check by using Fubini:
\begin{align*}
\int_{O(m)^2} g(R_\star) \d R & = \int_{O(m)} \!\! \d R_1 \int_{O(m)} \!\! \d R_2 \ \ g(R_2, R_1)  =  \int_{O(m)} \!\! \d R_2 \int_{O(m)} \!\! \d R_1 \ \ g(R_2, R_1) \\
& =  \int_{O(m)} \!\! \d R_1 \int_{O(m)} \!\! \d R_2 \ \ g(R_1, R_2)  =  \int_{O(m)^2} g(R) \d R\,.
\end{align*}

To show the second statement, we use the definition of $R_\star$ and \eqref{Eq:InvarianceByStar}
to see that
\begin{align*}
\overline{K}(x^\star,y) &= \average_{O(m)^2} K(|Rx^\star - y|) \d R = \average_{O(m)^2} K(|(Rx^\star - y)^\star|) \d R \\
&= \average_{O(m)^2} K(|(R x^\star)^\star - y^\star|) \d R = \average_{O(m)^2} K(|R_\star x - y^\star|) \d R \\
&= \average_{O(m)^2} K(|Rx - y^\star|) \d R = \overline{K}(x,y^\star)\,.
\end{align*}
As a consequence, we have that
$$\overline{K}(x^\star,y^\star) = \overline{K}(x,(y^\star)^\star) = \overline{K}(x,y)\,.$$
\end{proof}

To conclude this subsection, we present two alternative expressions for the operator $L_K$ when it acts on doubly radial odd functions. These expressions are suitable in the rest of the paper and also in the forthcoming one \cite{FelipeSanz-Perela:IntegroDifferentialII}, since the integrals appearing in the expression are computed only in $\ocal$, and this is important to prove maximum principle and other properties.

\begin{lemma}
	\label{Lemma:OperatorOddF}
	Let $w$ be a doubly radial function which is odd with respect to the Simons cone. Let $L_K \in \lcal_0(2m,\s,\lambda, \Lambda)$ be a rotation invariant operator and let $L_K^\ocal$ be defined by \eqref{Eq:OperatorOddF}. 
	
	Then, for every $x\in\ocal$,
	$$ 	L_K w (x) = L_K^\ocal w(x).   $$
	Indeed,
	\begin{align*}
	L_K w (x) &= \int_{\ocal} \{w(x) - w(y) \} \overline{K}(x, y) \d y +  \int_{\ocal} \{w(x) + w(y) \} \overline{K}(x, y^\star) \d y \\
	&= \int_{\ocal} \{w(x) - w(y) \} \{\overline{K}(x, y) - \overline{K}(x, y^\star)  \} \d y +  2 w(x) \int_{\ocal} \overline{K}(x, y^\star) \d y \,.
	\end{align*}
	Moreover,
	\begin{equation}
	\label{Eq:ZeroOrderTerm}
		\dfrac{1}{C} \dist(x,\ccal)^{-2\s} \leq \int_{\ocal} \overline{K}(x, y^\star) \d y \leq C \dist(x,\ccal)^{-2\s},
	\end{equation}
	with $C>0$ depending only on $m, \s, \lambda$ and $\Lambda$.
\end{lemma}

\begin{proof}
	The first statement is just a computation. Indeed,  using the change of variables  $\bar{y} = y^\star$ and the odd symmetry of $w$, we see that
	\begin{align*}
	\int_{\ical}  \{w(x) - w(y) \} \overline{K}(x, y)\d y &= \int_{\ocal} \{w(x) - w(y^\star) \}\overline{K}(x, y^\star)\d y \\
	&= \int_{\ocal} \{w(x) + w(y) \}\overline{K}(x, y^\star)\d y\,.
	\end{align*}
	Hence,
	\begin{align*}
	L_K w (x) &= \int_{\R^{2m}}  \{w(x) - w(y) \} \overline{K}(x, y)\d y \\
	&= \int_{\ocal}  \{w(x) - w(y) \} \overline{K}(x, y)\d y +\int_{\ical}  \{w(x) - w(y) \} \overline{K}(x, y)\d y \\
	&= \int_{\ocal} \{w(x) - w(y) \} \overline{K}(x, y) \d y +  \int_{\ocal} \{w(x) + w(y) \} \overline{K}(x, y^\star) \d y \,.
	\end{align*}
	By adding and subtracting $w(x)\overline{K}(x, y^\star)$ in the last integrand, we immediately deduce
	$$
	L_K w (x) =  \int_{\ocal} \{w(x) - w(y) \} \{\overline{K}(x, y) - \overline{K}(x, y^\star)  \} \d y +  2 w(x) \int_{\ocal} \overline{K}(x, y^\star) \d y\,.
	$$
	Note that we can add and subtract the term $w(x)\overline{K}(x, y^\star)$  since it is integrable with respect to $y$ in $\ocal$. This is a consequence of \eqref{Eq:ZeroOrderTerm}.
	
	Let us show now \eqref{Eq:ZeroOrderTerm}. In the following arguments we will use the letters $C$ and $c$ to denote positive constants, depending only on $m, \s, \lambda$ and $\Lambda$, that may change its value in each inequality. The upper bound is the simplest one since we only need to use the ellipticity of the kernel and the inclusion $\ical \subset \{y\in\R^{2m}:|x-y|\geq \dist(x,\ccal)\}$ for every $x\in \ocal$. Indeed,
	\begin{align*}
	\int_{\ocal} \overline{K}(x, y^\star) \d y &=  \int_{\ocal} K(|x-y^\star|) \d y = \int_{\ical} K(|x-y|) \d y \leq \int_{|x-y|\geq \dist(x,\ccal)} K(|x-y|) \d y \\
	&\leq C \int_{|x-y|\geq \dist(x,\ccal)} |x-y|^{-2m-2\s} \d y = C \int_{\dist(x,\ccal)}^\infty \rho^{-1-2s} \d \rho \\
	&= C \dist(x,\ccal)^{-2s}\,.
	\end{align*}

	In order to prove the lower bound, let us define $x_0 = x/\dist(x,\ccal)$. Note that $\dist (x_0, \ccal) = 1$. Then, we have
	\begin{align*}
	\int_{\ocal} \overline{K}(x, y^\star) \d y &=  \int_{\ical} K(|x-y|) \d y \\
	&\geq c \int_{\ical} |x-y|^{-2m-2\s} \d y \\
	&= c \int_{\ical} |x_0\dist(x,\ccal)-y|^{-2m-2\s} \d y \\
	&= c \,\dist(x,\ccal)^{-2s}\, \int_{\ical} |x_0-\tilde{y}|^{-2m-2\s} \d \tilde{y}\,.
	\end{align*}
	
	To conclude the proof, we claim that
	$$ \int_{\ical} |x_0-y|^{-2m-2\s} \d y \geq c>0\quad \textrm{for every } x_0 \textrm{ such that } \dist(x_0,\ccal) = 1,$$
	with a constant $c$ independent of $x_0$. To establish this claim we will use three facts. The first one is that, since $\dist(x_0,\ccal) = 1$, there exists a point $\overline{x_0}\in \ccal$ realizing this distance. Thus, we can easily deduce that $B_{3k+3}(\overline{x_0}) \setminus B_{3k+2}(\overline{x_0}) \subset  B_{3k+4}(x_0)\setminus B_{3k+1}(x_0)$ for every $k\geq 0$. The second fact is the identity $\ical = \ical \cap B_1^c(x_0)$, that also follows from $\dist(x_0,\ccal) = 1$. Finally, the last fact is a property of the Simons cone: $|B_R(z) \cap \ical| = 1/2 |B_R|$ for every $z\in \ccal$ (see Lemma 2.5 in \cite{Felipe-Sanz-Perela:SaddleFractional} for the proof). Combining these three facts, we get
	\begin{align*}
	\int_{\ical} |x_0-y|^{-2m-2\s} \d y &= \sum_{k=0}^\infty \int_{I\cap \left( B_{3k+4}(x_0)\setminus B_{3k+1}(x_0)\right)} |x_0-y|^{-2m-2\s} \d y\\
	&\geq \sum_{k=0}^\infty \int_{I\cap \left( B_{3k+4}(x_0)\setminus B_{3k+1}(x_0)\right)} (3k+4)^{-2m-2\s} \d y \\
	&= \sum_{k=0}^\infty (3k+4)^{-2m-2\s} |I\cap \left( B_{3k+4}(x_0)\setminus B_{3k+1}(x_0)\right)|\\
	&\geq \sum_{k=0}^\infty (3k+4)^{-2m-2\s} |I\cap \left( B_{3k+3}(\overline{x_0}) \setminus B_{3k+2}(\overline{x_0}) \right)|\\
	&= c \sum_{k=0}^\infty (3k+4)^{-2m-2\s} \left\{(3k+3)^{2m} -(3k+2)^{2m}\right\}\\
	&\geq c \sum_{k=0}^\infty (3k+4)^{-1-2\s} = c.
	\end{align*}
\end{proof}


%%%%%%%%%%%%%%%%%%%%%%%%%%%%%%%%%%%%%%%%%%%%%%%%%%%%%%%%%%%%%%%%%%%%%%%%
\subsection{Necessary and sufficient conditions for ellipticity}
%%%%%%%%%%%%%%%%%%%%%%%%%%%%%%%%%%%%%%%%%%%%%%%%%%%%%%%%%%%%%%%%%%%%%%%%




In this subsection, we establish Theorem~\ref{Th:SufficientNecessaryConditions}. As we have mentioned in the introduction, property \eqref{Eq:KernelInequality} is crucial in the rest of the results (including the ones in \cite{FelipeSanz-Perela:IntegroDifferentialII}), since it guarantees for instance that the operator $L_K$ has a maximum principle for odd functions (see Proposition~\ref{Prop:WeakMaximumPrincipleForOddFunctions}).

First, we give a sufficient condition on the kernel $K$ so that the operator $L_K$ belongs to the the class $\lcal_\star$. It is the following result.

\begin{proposition}
\label{Prop:KernelInequalitySufficientCondition} 
Let $K:(0,+\infty) \to \R$ define a positive radially symmetric kernel $K(|x-y|)$ in $\R^{2m}$. Define $\overline{K} : \R^{2m}\times \R^{2m} \to \R$ by \eqref{Eq:KbarDef'}. Assume that $K(\sqrt{\cdot})$ strictly convex in $(0,+\infty)$. Then, the associated kernel $\overline{K}$ satisfies
	\begin{equation}
	\label{Eq:KernelInequalityBis}
	\overline{K}(x,y) > \overline{K}(x, y^\star) \quad \text{ for every }x,y \in \ocal\,.
	\end{equation}
\end{proposition}

\begin{proof}
Since $\overline{K}$ is invariant by $O(m)^2$, it is enough to show \eqref{Eq:KernelInequalityBis} for points $x, y\in \ocal$ of the form $x = (|x'|e, |x''|e)$ and $y = (|y'|e, |y''|e)$, with $e \in \Sph^{m-1}$ an arbitrary unitary vector.

Now, define
\begin{equation}
\label{Eq:DefQ}
	\begin{split}
	Q_1 &:= \setcond{R = (R_1,R_2) \in O(m)^2}{e\cdot R_1 e > |e\cdot R_2 e|},\\
	Q_2 &:= \setcond{R = (R_1,R_2) \in O(m)^2}{e\cdot R_2 e > |e\cdot R_1 e|} = (Q_1)_\star,\\
	Q_3 &:= \setcond{R = (R_1,R_2) \in O(m)^2}{e\cdot R_1 e < -|e\cdot R_2 e|} = -Q_1,\\
	Q_4 &:= \setcond{R = (R_1,R_2) \in O(m)^2}{e\cdot R_2 e < - |e\cdot R_1 e|} = -(Q_1)_\star.
	\end{split}
\end{equation}
Recall that given $R=(R_1,R_2)\in O(m)^2$, then $R_\star=(R_2,R_1)\in O(m)^2$. Moreover, note that the sets $Q_i$ are disjoint, have the same measure and cover all $O(m)^2$ up to a set of measure zero. 

%Moreover, they satisfy the following:
%\begin{itemize}
%\item If $R = (R_1, R_2)\in Q_2$, then $R_\star = (R_2, R_1) \in Q_1$.
%\item If $R = (R_1, R_2)\in Q_3$, then $-R = (-R_1, -R_2) \in Q_1$.
%\item If $R = (R_1, R_2)\in Q_4$, then $-R_\star = (-R_2, -R_1) \in Q_1$.
%\end{itemize}
Therefore,
\begin{align*}
4\overline{K} (x, y) &= 4\average_{O(m)^2} K(|x - R y|)\d R \\
& = \average_{Q_1} K(|x - R y|)\d R + \average_{Q_2} K(|x - R y|)\d R \\
& \quad \quad
+ \average_{Q_3} K(|x - R y|)\d R +
\average_{Q_4} K(|x - R y|)\d R \\
&= \average_{Q_1} \{K(|x - R y|) + K(|x + R y|) \\
&\quad \quad + K(|x - R_\star y|) + K(|x + R_\star y|)\}\d R
\end{align*}
and
\begin{align*}
4\overline{K} (x, y^\star) &= 4\average_{O(m)^2} K(|x - R y^\star|)\d R \\
& = \average_{Q_1} \{K(|x - R y^\star|) + K(|x + R y^\star|) \\
&\quad \quad + K(|x - R_\star y^\star|) + K(|x + R_\star y^\star|)\}\d R.
\end{align*}
Thus, if we prove
\begin{equation}
\label{Eq:InequalityIntegrandKernelInequalityProof}
\begin{split}
K(|x - R y|) + K(|x + R y|) + K(|x - R_\star y|) + K(|x + R_\star y|)
\quad \quad \quad \quad \quad \quad \quad \quad
\\
\geq
K(|x - R y^\star|) + K(|x + R y^\star|)+K(|x - R_\star y^\star|) + K(|x + R_\star y^\star|)\,,
\end{split}
\end{equation}
for every $R\in Q_1$, we immediately deduce \eqref{Eq:KernelInequalityBis} with a non strict inequality. To see that it is indeed a strict one, we will show that the inequality in \eqref{Eq:InequalityIntegrandKernelInequalityProof} is strict for a.e. $R \in Q_1$.


For a short notation, we call
\begin{equation}
	\label{Eq:DefAlphaBeta}
	\alpha := e \cdot R_1 e  \quad \text{ and } \quad \beta := e \cdot R_2 e\,.
\end{equation}
Now, note that since  $x = (|x'|e, |x''|e)$ and $y = (|y'|e, |y''|e)$, we have
\begin{align*}
|x \pm Ry|^2&= |x' \pm R_1y'|^2 + |x'' \pm R_2y''|^2 \\
&= |x'|^2 + |y'|^2 \pm 2 x'\cdot R_1 y' +  |x''|^2 + |y''|^2 \pm 2 x''\cdot R_2 y''\\
&= |x|^2 + |y|^2 \pm 2 |x'||y'| \alpha \pm 2 |x''||y''| \beta.
\end{align*}
Similarly,
$$
|x \pm R_\star y|^2 =  |x|^2 + |y|^2 \pm 2 |x'||y'| \beta \pm 2 |x''||y''|\alpha,
$$
$$
|x \pm R y^\star|^2 =  |x|^2 + |y|^2 \pm 2 |x'||y''| \alpha \pm 2 |x''||y'|\beta,
$$
and
$$
|x \pm R_\star y^\star|^2 = |x|^2 + |y|^2 \pm 2 |x'||y''| \beta \pm 2 |x''||y'| \alpha.
$$

We define now
$$
g(\tau) := K \bpar{\sqrt{|x|^2 + |y|^2 + 2 \tau }} + K \bpar{\sqrt{|x|^2 + |y|^2 - 2 \tau}}.
$$
Thus, proving \eqref{Eq:InequalityIntegrandKernelInequalityProof} is equivalent to show that, for every $\alpha$, $\beta \in [-1,1]$ such that $\alpha > |\beta|$, it holds
\begin{equation}
\label{Eq:InequalityIntegrandKernelInequalityProof2}
\begin{split}
g\Big(|x'||y'| \alpha + |x''||y''| \beta \Big)
+ g\Big(|x'||y'| \beta + |x''||y''| \alpha \Big) \hspace{2cm}
\\ \geq
g\Big(|x'||y''| \alpha + |x''||y'|\beta \Big)
+ g\Big(|x'||y''| \beta + |x''||y'| \alpha \Big)\,.
\end{split}
\end{equation}

Let
$$
\begin{array}{cc}
A_{\alpha,\beta} := |x'||y'|  \alpha + |x''||y''|\beta \,, &
B_{\alpha,\beta} := |x'||y''| \alpha + |x''||y'| \beta \,, \\
C_{\alpha,\beta} := |x''||y'| \alpha + |x'||y''| \beta \,, &
D_{\alpha,\beta} := |x''||y''|\alpha + |x'||y'|  \beta \,.
\end{array}
$$
With this notation and taking into account that $g$ is even,
\eqref{Eq:InequalityIntegrandKernelInequalityProof2} is equivalent to
\begin{equation}
\label{Eq:InequalityIntegrandKernelInequalityProof3}
g(|A_{\alpha,\beta}|) + g(|D_{\alpha,\beta}|) \geq g(|C_{\alpha,\beta}|) + g(|B_{\alpha,\beta}|)\,,
\end{equation}
for every $\alpha$, $\beta \in [-1,1]$ such that $\alpha > |\beta|$. Note that $g$ is defined in the open interval $I = (-(|x|^2 + |y|^2)/2,\ (|x|^2 + |y|^2)/2)$ and that $A_{\alpha,\beta}$, $B_{\alpha,\beta}$, $C_{\alpha,\beta}$, $D_{\alpha,\beta} \in I$.

To show \eqref{Eq:InequalityIntegrandKernelInequalityProof3}, we use Proposition~\ref{Prop:EquivalenceK(sqrt)Convex<->Inequality} of the Appendix~\ref{Sec:AuxiliaryResults}. There, it is stated that, thanks to the convexity properties of $g$, in order to establish \eqref{Eq:InequalityIntegrandKernelInequalityProof3} it is enough to check that
$$
\begin{cases}
|A_{\alpha,\beta}| \geq |B_{\alpha,\beta}|,\ \ |A_{\alpha,\beta}| \geq |C_{\alpha,\beta}|, \ \ |A_{\alpha,\beta}| \geq |D_{\alpha,\beta}|\,, \\
|A_{\alpha,\beta}| + |D_{\alpha,\beta}| \geq |B_{\alpha,\beta}| + |C_{\alpha,\beta}|\,.
\end{cases}
$$
The verification of these inequalities is a simple but tedious computation and it is presented in Appendix~\ref{Sec:AuxiliaryResults2} ---see point (1) of Lemma~\ref{Lemma:ComputationABCD}. Once we have these inequalities, by Proposition~\ref{Prop:EquivalenceK(sqrt)Convex<->Inequality} we deduce \eqref{Eq:InequalityIntegrandKernelInequalityProof3}.

To finish, we must see that the equality in \eqref{Eq:InequalityIntegrandKernelInequalityProof3} is never attained. By Proposition~\ref{Prop:EquivalenceK(sqrt)Convex<->Inequality}, we know that a necessary condition for the equality to hold is that either $|A_{\alpha,\beta}| = |B_{\alpha,\beta}|$ and $|C_{\alpha,\beta}| = |D_{\alpha,\beta}|$, or $|A_{\alpha,\beta}| = |C_{\alpha,\beta}|$ and $|B_{\alpha,\beta}| = |D_{\alpha,\beta}|$. Nevertheless, by point (2) of Lemma~\ref{Lemma:ComputationABCD}, this yields $\alpha = \beta = 0$, and this cannot happen. Thus, the inequality in \eqref{Eq:InequalityIntegrandKernelInequalityProof} is strict, and this leads to \eqref{Eq:KernelInequalityBis}.
\end{proof}


Now, we give a necessary condition on the kernel $K$ so that inequality \eqref{Eq:KernelInequality} holds.

\begin{proposition}
\label{Prop:KernelInequalityNecessaryCondition} Let $K:(0,+\infty) \to \R$ define a positive radially symmetric kernel $K(|x-y|)$ in $\R^{2m}$. Define $\overline{K} : \R^{2m}\times \R^{2m} \to \R$ by \eqref{Eq:KbarDef'}. 

If
\begin{equation}
\label{Eq:KernelInequalityAE}
\overline{K}(x,y) > \overline{K}(x, y^\star) \quad \text{ for almost every }x,y \in \mathcal{O}\,,
\end{equation}
then $K(\sqrt{\cdot})$ cannot be concave in any interval $I\subset [0,+\infty)$.
\end{proposition}

\begin{proof}
We prove it by contraposition. In fact, we will show that if there exists an interval where $K(\sqrt{\cdot})$ is concave, then we can find an open set in $\ocal \times \ocal$ with positive measure where \eqref{Eq:KernelInequalityAE} is not satisfied.

Let $\ell_2>\ell_1>0$ be such that $K(\sqrt{\cdot})$ is concave in $(\ell_1,\ell_2)$ and define the set $\Omega_{\ell_1,\ell_2}\subset \R^{4m}$ as the points $(x,y)\in \ocal\times \ocal$ satisfying
\begin{equation}
\label{Eq:OmegaSetDefinition}
\beqc{\PDEsystem}
(|x'|-|y'|)^2+(|x''|-|y''|)^2&>&\ell_1,\\
(|x'|+|y'|)^2+(|x''|+|y''|)^2&<&\ell_2.
\eeqc
\end{equation}

First, it is easy to see that $\Omega_{\ell_1,\ell_2}$ is a nonempty open set. In fact, points of the form $(x',0,y',0)\in (\R^m)^4$ such that $(|x'|-|y'|)^2>\ell_1$ and $(|x'|+|y'|)^2 <\ell_2$ belong to $\Omega_{\ell_1,\ell_2}$. Then, if we prove that $\overline{K}(x,y) \leq \overline{K}(x, y^\star)$ in $\Omega_{\ell_1,\ell_2}$ we are done.

Given $x,y\in \Omega_{\ell_1,\ell_2}$, we are going to show, as in the previous proof, that
\begin{equation}
\label{Eq:InequalityIntegrandKernelInequalityProof4}
\begin{split}
K(|x - R y|) + K(|x + R y|) + K(|x - R_\star y|) + K(|x + R_\star y|)
\quad \quad \quad \quad \quad \quad
\\
\leq
K(|x - R y^\star|) + K(|x + R y^\star|)+K(|x - R_\star y^\star|) + K(|x + R_\star y^\star|)\,, 
\end{split}
\end{equation}
for any $R\in Q_1$, where $Q_1$ is defined in \eqref{Eq:DefQ} (see the proof of Proposition~\ref{Prop:KernelInequalitySufficientCondition}). As before, we can assume that $x$ and $y$ are of the form $x = (|x'|e, |x''|e)$ and $y = (|y'|e, |y''|e)$, with $e \in \Sph^{m-1}$ an arbitrary unitary vector. Then, by defining $\alpha$ and $\beta$ as in \eqref{Eq:DefAlphaBeta}, we see that proving \eqref{Eq:InequalityIntegrandKernelInequalityProof4} is equivalent to establish that
\begin{equation}
\label{Eq:InequalityIntegrandKernelInequalityProof5}
g(A_{\alpha,\beta}) + g(D_{\alpha,\beta}) \leq g(B_{\alpha,\beta}) + g(C_{\alpha,\beta})\,,
\end{equation}
for every $\alpha, \beta \in [-1,1]$ such that $\alpha>|\beta|$, where
$$
\begin{array}{cc}
A_{\alpha,\beta} = |x'||y'|  \alpha + |x''||y''|\beta \,, &
B_{\alpha,\beta} = |x'||y''| \alpha + |x''||y'| \beta \,, \\
C_{\alpha,\beta} = |x''||y'| \alpha + |x'||y''| \beta \,, &
D_{\alpha,\beta} = |x''||y''|\alpha + |x'||y'|  \beta \,.
\end{array}
$$
and
\begin{align*}
g(\tau) &= K\left( \sqrt{|x|^2+|y|^2+2\tau} \right) + K\left( \sqrt{|x|^2+|y|^2-2\tau} \right).
\end{align*}




Now, by \eqref{Eq:OmegaSetDefinition}, we have $\ell_1 < |x|^2+|y|^2 <\ell_2$. As a consequence of this and the concavity of $K(\sqrt{\cdot})$ in $(\ell_1,\ell_2)$, it is easy to see (by using Lemma~\ref{Lemma:ConvexFunctions} in the Appendix~\ref{Sec:AuxiliaryResults}) that $g$ is concave in $ \left( -\overline{\ell}, \overline{\ell}\right) $, and decreasing in $(0,\overline{\ell})$, where 
$$
\overline{\ell} := \min{\left\{\frac{\ell_2-|x|^2-|y|^2}{2},\frac{|x|^2+|y|^2-\ell_1}{2}\right\}}.$$
Note that, since $\ell_1 < |x|^2+|y|^2 <\ell_2$, we have $\overline{\ell}>0$.


We claim that $A_{\alpha,\beta}, B_{\alpha,\beta}, C_{\alpha,\beta}$ and $D_{\alpha,\beta}$ belong to $(-\overline{\ell},\overline{\ell})$ for every $\alpha, \beta \in [-1,1]$ such that $\alpha>|\beta|$. Indeed, it is easy to check that for every $\alpha, \beta \in [-1,1]$ such that $\alpha>|\beta|$, the numbers $A_{\alpha,\beta}, B_{\alpha,\beta}, C_{\alpha,\beta}$ and $D_{\alpha,\beta}$ belong to the open interval $(-|x'||y'|-|x''||y''|,|x'||y'|+|x''||y''|)$. Furthermore, since $x,y \in \Omega_{\ell_1,\ell_2}$, we obtain from \eqref{Eq:OmegaSetDefinition} that
$$
\beqc{\PDEsystem}
|x'||y'|+|x''||y''|&<&\dfrac{\ell_2-|x|^2-|y|^2}{2}\\
|x'||y'|+|x''||y''|&<&\dfrac{|x|^2+|y|^2-\ell_1}{2}
\eeqc 
$$
and thus $ |x'||y'|+|x''||y''|<\overline{\ell}$ and the claim is proved.

Finally, by applying Lemma~\ref{Lemma:ConvexFunctions} to the function $-g$, we obtain that inequality \eqref{Eq:InequalityIntegrandKernelInequalityProof5} is satisfied, which yields \eqref{Eq:InequalityIntegrandKernelInequalityProof4}. Finally, by integrating \eqref{Eq:InequalityIntegrandKernelInequalityProof4} with respect to all the rotations $R\in Q_1$ we get $$ \overline{K}(x,y) \leq \overline{K}(x, y^\star),$$ for every $(x,y)\in \Omega_{\ell_1,\ell_2}$, contradicting \eqref{Eq:KernelInequalityAE}.
\end{proof}

From the two previous results, Theorem~\ref{Th:SufficientNecessaryConditions} follows immediately.

\begin{proof}[Proof of Theorem~\ref{Th:SufficientNecessaryConditions}]
	The first statement is exactly the same as Proposition~\ref{Prop:KernelInequalitySufficientCondition}. Assume now that $K$ is a $C^1$ function and that \eqref{Eq:KernelInequality} holds. Then, by Proposition~\ref{Prop:KernelInequalityNecessaryCondition}, $K(\sqrt{\cdot})$ is not concave in any interval of $[0,+\infty)$. By the regularity of $K$, this automatically yields that $K(\sqrt{\cdot})$ is strictly convex.
\end{proof}

\begin{remark}
	Note that there are radially symmetric kernels $K$ that are not $C^1$ for which we do not know if $L_K$ belongs to the class $\lcal_\star$. For instance, given $0<\s<1$, if we consider the kernel
	$$ K(\tau) = \frac{1}{\tau^{2m+2\s}} \chi_{(0,1)}(\tau)+\frac{1}{10\tau^{2m+2\s}-9} \chi_{[1,+\infty)}(\tau), $$
	it is easy to check that $K$ is continuous and decreasing but $K(\sqrt{\tau})$ is not convex in $(0,+\infty)$ even though it does not have any interval of concavity (see Figure~\ref{Fig:Grafica}).
	\begin{figure}
	\centering
	\begin{tikzpicture}
	\begin{axis}[
	axis x line=middle, axis y line=left,
	every axis x label/.style={at={(current axis.right of origin)},anchor=west},
	every axis y label/.style={at={(current axis.north west)},above=2mm},
	ymin=0, ymax=3, ylabel=$K(\sqrt{\tau})$,
	xmin=0.4, xmax=2, xlabel=$\tau$
	]
	\addplot[domain=0.4:1, samples=100] {1/(x^1.5)};
	\addplot[domain=1:2, samples=100] {0.1/(x^1.5-0.9)};
	\end{axis}
	\end{tikzpicture}
	\caption{$K(\sqrt{\tau})$ for $m=1$ and $\s=1/2$.}
	\label{Fig:Grafica}
	\end{figure}
\end{remark}

%%%%%%%%%%%%%%%%%%%%%%%%%%%%%%%%%%%%%%%%%%%%%%%%%%%%%%%%%%%%%%%%%%%%%%
%%%%%%%%%%%%%%%%%%%%%%%%%%%%%%%%%%%%%%%%%%%%%%%%%%%%%%%%%%%%%%%%%%%%%%
\subsection{Maximum principles for doubly radial odd functions}
%%%%%%%%%%%%%%%%%%%%%%%%%%%%%%%%%%%%%%%%%%%%%%%%%%%%%%%%%%%%%%%%%%%%%%
%%%%%%%%%%%%%%%%%%%%%%%%%%%%%%%%%%%%%%%%%%%%%%%%%%%%%%%%%%%%%%%%%%%%%%

In this subsection we prove a weak and a strong maximum principles for doubly radial functions that are odd with respect to the Simons cone. The formulation of this maximum principle is very suitable since it only assumes hypotheses on $\ocal$. The key ingredient in the proofs is the kernel inequality \eqref{Eq:KernelInequality}.


We first establish a weak maximum principle.

\begin{proposition}[Weak maximum principle for odd functions with respect to $\ccal$]
\label{Prop:WeakMaximumPrincipleForOddFunctions} Let $\Omega \subset \ocal$ an open set and let $L_K  \in \lcal_\star (2m,  \s)$.  Let $u\in C^{\alpha}(\Omega)\cap L^\infty(\R^{2m})$, with $\alpha > 2\s$, be a doubly radial function which is odd with respect to the Simons cone. Assume that
$$
\beqc{\PDEsystem}
L_K u + c(x) u & \geq & 0 & \text{ in } \Omega\,,\\
u & \geq & 0 & \text{ in } \ocal \setminus \Omega\,,
\eeqc
$$
with $c \geq 0$, and that either
$$
\Omega \text{ is bounded} \quad \text{ or } \liminf_{x \in \ocal,\,|x|\to +\infty} u(x) \geq 0\,.
$$
Then, $u \geq 0$ in $\Omega$.
\end{proposition}

\begin{proof}
By contradiction, assume that $u$ takes negative values in $\Omega$. Under the hypotheses we are assuming, a negative minimum must be achieved. Thus, there exists $x_0\in \Omega$ such that
$$
u(x_0) = \min_{\Omega} u =: m < 0\,.
$$
Then, using the expression of $L_K$ for odd functions (see Lemma~\ref{Lemma:OperatorOddF}), we have
$$
L_K u (x_0) = \int_{\ocal} \{m - u(y) \} \{\overline{K}(x_0, y) - \overline{K}(x_0, y^\star)  \} \d y +  2 m \int_{\ocal} \overline{K}(x_0, y^\star) \d y\,.
$$
Now, since $m - u(y) \leq 0$ in $\ocal$, $m<0$, $c\geq 0$ and $\overline{K}(x_0, y) \geq \overline{K}(x_0, y^\star)>0$ (since $L_K\in \lcal_\star$), we get
$$
0 \leq L_K  u(x_0) + c(x_0) u(x_0) \leq m \left(2\int_{\ocal} \overline{K}(x_0, y^\star) \d y + c(x_0)\right)  < 0\,,
$$
a contradiction.
\end{proof}

\begin{remark}
Note that since the operator $L_K$ includes itself a positive zero order term in addition to the integro-differential part, the condition $c\geq 0$ in the previous proposition can be lightly relaxed. Indeed, if we follow the proof of the result, we can deduce that the hypothesis on $c$ that we can assume is 
$$ c(x) > -2\int_{\ocal} \overline{K}(x, y^\star) \d y. $$
This hypothesis seems hard to be checked for applications apart from the case $c\geq 0$. Nevertheless, recall that by Lemma~\ref{Lemma:OperatorOddF} we have an explicit lower bound of $ \int_{\ocal} \overline{K}(x, y^\star) \d y $ in terms of the function $\dist(x,\ccal)$ that could be used to check the previous condition.
\end{remark}

The following is a strong maximum principle for odd functions.

\begin{proposition}[Strong maximum principle for odd functions with respect to $\ccal$]
\label{Prop:StrongMaximumPrincipleForOddFunctions} Let $\Omega \subset \ocal$ an open set and let $L_K  \in \lcal_\star (2m,  \s)$.  Let $u\in C^{\alpha}(\Omega)\cap L^\infty(\R^{2m})$, with $\alpha > 2\s$, be a doubly radial function which is odd with respect to the Simons cone. Assume that $L_K u + c(x) u\geq 0$ in $\Omega$, with $c(x)$ any function, and that $u\geq 0$ in $\ocal$. Then, either $u\equiv 0$ or $u > 0$ in $\Omega$.
\end{proposition}

\begin{proof}
Assume that $u \not \equiv 0$. We shall prove that $u > 0$ in $\Omega$. By contradiction, assume that there exists a point $x_0\in \Omega$ such that $u(x_0)= 0$. Then, using the expression of $L_K $ for odd functions given in Lemma~\ref{Lemma:OperatorOddF}, the kernel inequality \eqref{Eq:KernelInequality} and the fact that $u\geq 0$ in $\ocal$, we obtain
$$
0 \leq L_K u(x_0) + c(x_0) u(x_0) = - \int_{\ocal} u(y)\big \{\overline{K}(x_0, y) - \overline{K}(x_0, y^\star) \big \}\d y < 0\,,
$$
a contradiction.
\end{proof}


%%%%%%%%%%%%%%%%%%%%%%%%%%%%%%%%%%%%%%%%%%%%%%%%%%%%%%%%%%%%%%%%%%%%%%
%%%%%%%%%%%%%%%%%%%%%%%%%%%%%%%%%%%%%%%%%%%%%%%%%%%%%%%%%%%%%%%%%%%%%%
