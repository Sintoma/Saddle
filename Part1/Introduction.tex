%%%%%%%%%%%%%%%%%%%%%%%
\section{Introduction}
%%%%%%%%%%%%%%%%%%%%%%%
\label{Sec:Introduction}


In this paper we study solutions to the equation
\begin{equation}
\label{Eq:NonlocalAllenCahn}
L_K u = f(u) \quad \textrm{ in } \R^{2m},
\end{equation}
which are odd with respect to the Simons cone. The interest on these solutions is motivated by the nonlocal version of a conjecture by De Giorgi with the aim of finding a counterexample in high dimensions through what we call saddle-shaped solutions (see details below).

The equation is driven by an integro-differential operator $L_K$ of the form
\begin{equation}
\label{Eq:DefOfLu}
L_Ku(x) = \PV \int_{\R^n} \{u(x) - u(y)\} K(x-y)\d y,
\end{equation}
where the kernel $K$ satisfies
\begin{equation}
\label{Eq:Symmetry&IntegrabilityOfK}
K\geq 0\,, \quad K(y) = K(-y) \quad \textrm{ and } \quad \int_{\R^n} \min \left\{ |y|^2, 1 \right\} K(y) \d y < + \infty\,.
\end{equation}
The most canonical example of such operators is the fractional Laplacian
$$
\fraclaplacian u = \PV c_{n, \s}\int_{\R^n} \dfrac{u(x) - u(y)}{|x-y|^{n + 2\s}}\d y\,,
$$
where $c_{n, \s}$ is a normalizing constant (for its exact value see for instance \cite{HitchhikerGuide}).

Throughout the paper, we assume that the operators of our study are uniformly elliptic. That is, their kernels are bounded from above and below by the one of the fractional Laplacian:
\begin{equation}
\label{Eq:Ellipticity}
c_{n,\s}\dfrac{\lambda}{|y|^{n+2\s}} \leq K(y) \leq c_{n,\s}\dfrac{\Lambda}{|y|^{n+2\s}}\,, \quad 0< \lambda \leq \Lambda \,,
\end{equation}
where $c_{n,\s}$ is the constant appearing in the definition of the fractional Laplacian. This condition is one of the most frequently adopted when dealing with nonlocal operators of the form (99) since it is known to yield Hölder regularity of solutions (see [99] and [99]). The family of linear operators satisfying conditions (K1) and (K3') are the so-called $\lcal_0(n,\s,\lambda, \Lambda)$ ellipticity class.
The bounds of (K3') allow the kernels to be very oscillating and irregular, and that is why they are sometimes called rough kernels.

Moreover, for some purposes we will need the operators to be invariant under rotations (with kernel being radially symmetric). When the operator $L_K$ belongs to the ellipticity class $\lcal_0(n,\s,\lambda, \Lambda)$ and is invariant under rotations we will say that it is in the ellipticity class $\Lr(n,\s,\lambda, \Lambda)$.


For short we will usually write $\lcal_0$ or $\Lr$, and we will make explicit the parameters only when needed.

On the other hand, given $f$ a $C^1$ nonlinearity, we define
$$
G(u)= \int_u^1 f(t) \d t\,.
$$
Then, we have that $G$ is a $C^2$ function satisfying $G' = -f$. In this paper, we assume the following conditions on $G$:
\begin{equation}
\label{Eq:HipothesisfOdd}
G \textrm{ is even,}
\end{equation}
and
\begin{equation}
\label{Eq:HipothesisGWells}
G\geq G(\pm 1 )=0 \textrm{ in } \R\,
\end{equation}

Note that the previous conditions on $G$ yields that $f$ is an odd function with $f(0)=f(\pm 1)=0$.


\bigskip
\bigskip
\bigskip
-------------
\bigskip
\bigskip
\bigskip

The Simons cone will be a central object along this paper. It is defined in $\R^{2m}$ by
\begin{equation}
\label{Eq:SimonsCone}
\mathscr{C} = \setcond{x = (x', x'') \in \R^{2m}}{|x'| = |x''|}\,.
\end{equation}

This cone is of importance in the theory of minimal surfaces. It has zero mean curvature at every point $x\in \ccal \setminus \{0\}$, in all even dimensions, and it is a minimizer of the perimeter functional when $2m\geq 8$. Concerning the nonlocal setting, $\ccal$ has also zero nonlocal mean curvature in all even dimensions, although it is not known if it is a minimizer of the nonlocal perimeter (see the introduction of 99 and references therein).

We should also mention \cite{DipierroSerraValdinoci}\todo{CHECK} where they study nonlocal minimal surfaces with general kernels and blablabla.mal surfaces with general kernels and blablabla.

Through the paper we will also use the letters $\ocal$ and $\ical$ to denote both sides of the cone:
\begin{equation}
\label{Eq:DefOandI}
\ocal:= \setcond{x = (x', x'') \in \R^{2m}}{|x'| > |x''|} \ \textrm{ and } \
\ical:= \setcond{x = (x', x'') \in \R^{2m}}{|x'| < |x''|}.
\end{equation}

Both domains $\ocal$ and $\ical$ belong to a family of sets in $\R^{2m}$ which are called of \emph{double revolution}. They are sets that are invariant under orthogonal transformations in the first $m$ variables and also under orthogonal transformations in the last $m$ variables. That is, $\Omega\subset \R^{2m}$ is a set of double revolution if $R(\Omega) = \Omega$ for any given transformation $R\in O(m)^2 = O(m) \times O(m)$, where  $O(m)$ is the orthogonal group of $\R^m$

In this paper we deal with functions that are \emph{doubly radial}. These are functions $w:\R^{2m}  \to \R$ that only depend on the modulus of the first $m$ variables and on the modulus of the last $m$ ones, i.e., $w(x) = w(|x'|,|x''|)$. Equivalently, $w(Rx) = w(x)$ for every $R \in O(m)^2$.

In order to define certain symmetries of a function with respect to the Simons cone, we consider the following isometry, that will play a significant role in this article:
\begin{equation}
\label{Eq:DefStar}
\begin{matrix}
(\cdot)^\star \colon & \R^{2m}= \R^{m}\times \R^{m}  &\to&  \R^{2m}= \R^{m}\times \R^{m}  \\
& x = (x',x'') &\mapsto & x^\star = (x'',x')\,.
\end{matrix}
\end{equation}
Note that this isometry is actually an involution that maps $\ocal$ into $\ical$ (and vice versa) and leaves the cone $\ccal$ invariant. Taking into account this transformation, we say that a doubly radial function $w$ is \emph{odd with respect to the Simons cone} if $w(x) = -w(x^\star)$. Similarly, we say that a doubly radial function $w$ is \emph{even with respect to the Simons cone} if $w(x) = w(x^\star)$.


With these definitions in hand, we can define now properly saddle-shaped solutions:
\begin{definition}
	\label{Def:SaddleShapedSol}
	We say that $u$ is a \emph{saddle-shaped solution} (or simply \emph{saddle solution}) of \eqref{Eq:NonlocalAllenCahn} if
	\begin{enumerate}
		\item $u$ is doubly radial.
		\item $u$ is odd with respect to the Simons cone.
		\item $u > 0$ in $\ocal$.
	\end{enumerate}
\end{definition}


Note that these solutions are even with respect to the coordinate axis and that their zero level set is the Simons cone $\mathscr{C} = \{|x'|=|x''|\}$. 

%
%Therefore, saddle-shaped solutions are candidates to build a counterexample of the De Giorgi conjecture in high dimensions, since if one could prove that they are global minimizers in $\R^8$, by the result in \cite{JerisonMonneau} one would have a counterexample of the De Giorgi conjecture in $\R^9$ (as an alternative to that of \cite{delPinoKowalczykWei}).

Saddle-shaped solutions for the classical Allen-Cahn equation involving the Laplacian were first studied by Dang, Fife, and Peletier in \cite{DangFifePeletier} in dimension $2m=2$. They established the existence and uniqueness of this type of solutions, as well as some monotonicity properties and asymptotic behavior. In \cite{Schatzman}, Schatzman studied the instability property of saddle solutions in $\R^2$. Later, Cabré and Terra  proved the existence of a saddle solution in every dimension $2m\geq 2$, and they established some qualitative properties such as asymptotic behavior, monotonicity properties, as well as instability in dimensions $2m = 4$ and $2m = 6$ (see \cite{CabreTerraI,CabreTerraII}). The uniqueness in dimensions higher than $2$ was established by Cabré in \cite{Cabre-Saddle}, where he also proved that the saddle solution is stable in dimensions $2m \geq 14$.

In the fractional framework, there are only three works concerning saddle-shaped solutions to \eqref{Eq:AllenCahn}. In  \cite{Cinti-Saddle,Cinti-Saddle2}, Cinti proved the existence of a saddle-shaped solution as well as some qualitative properties such as asymptotic behavior, monotonicity properties, and instability in low dimensions. In [99], we proved further properties of these solutions, the main ones being uniqueness and, when $2m\geq 14$, stability. In the three previous papers, the main tool used is the extension problem for the fractional Laplacian (see 99). Nevertheless, this technique has the limitation that it cannot be carried out for general integro-differential operators different from the fractional Laplacian. Therefeore, some purely nonlocal techniques are developed through this paper.



%The first one is \cite{CozziPassalacqua}, where Cozzi and Passalacqua study layer solutions to the equation \eqref{Eq:NonlocalAllenCahn}.


The interest on this problem originates from a famous conjecture of De Giorgi for the classical Allen-Cahn equation. It reads as follows. Let $u$ be a bounded solution of $-\Delta  u = u - u^3$ in $\R^n$ which is monotone in one direction, say $\partial_{x_n} u > 0$. Then, if $n\leq 8$, $u$ is one dimensional, i.e., $u$ depends only on one Euclidean variable. This conjecture was proved true in dimension $n=2$ by Ghoussoub and Gui in \cite{GhoussoubGui}, and in dimension $n=3$ by Ambrosio and Cabré in \cite{AmbrosioCabre}. For dimensions $4\leq n \leq 8$, it was established by Savin in \cite{Savin-DeGiorgi} but with the extra assumption of
\begin{equation}
\label{Eq:SavinCondition}
	\lim_{x_n \to \pm \infty} u(x',x_n) = \pm 1 \quad \text{ for all } x'\in \R^{n-1}\,.
\end{equation}
A counterexample to the conjecture was given by del~Pino, Kowalczyk and Wei in \cite{delPinoKowalczykWei}. 


The corresponding conjecture in the nonlocal setting, where one replaces the operator $-\Delta$ by $\fraclaplacian$, has been widely studied in the last years. In this framework, the conjecture has been proven to be true in dimension $n=2$ by Cabré and Solà-Morales in \cite{CabreSolaMorales} for $\s=1/2$, and extended to every power $0<\s<1$ by Cabré and Sire in \cite{CabreSireI} and also by Sire and Valdinoci in \cite{SireValdinoci}. In dimension $n=3$, the conjecture has been proved by Cabré and Cinti for $1/2 \leq \s < 1$ in \cite{CabreCinti-EnergyHalfL, CabreCinti-SharpEnergy}. Recently, in \cite{Savin-Fractional,Savin-Fractional2} Savin has established the validity of the conjecture in dimensions $4\leq n \leq 8$ and for $1/2 \leq \s < 1$, but assuming the additional hypothesis \eqref{Eq:SavinCondition}. The case $0<\s<1/2$ has been treated by Dipierro, Serra, and Valdinoci in \cite{DipierroSerraValdinoci}. They prove that the conjecture is true in dimension $n$ if all minimizing nonlocal minimal cones in dimension $n-1$ are flat. As a consequence of this fact and some results classifying minimizing nonlocal minimal cones (see \cite{SavinValdinoci-Cones} and \cite{CaffarelliValdinoci}), the conjecture is proved to be true in dimension $n=3$ for all $0<\s<1/2$, and in dimensions $4\leq n \leq 8$ if $\s\in(0,1/2)$ is close to $1/2$. The most recent result concerning the proof of the conjecture is the one of Figalli and Serra in \cite{FigalliSerra}, where they have established the conjecture in dimension $n=4$ and $\s=1/2$ without requiring the additional limiting assumption \eqref{Eq:SavinCondition}. Note that, without \eqref{Eq:SavinCondition}, the analogous result for the Laplacian in dimension $n=4$ is not known. A counterexample to the De Giorgi conjecture for fractional Allen-Cahn equation in dimensions $n \geq 9$ for $\s \in ( 1/2 , 1)$ has been very recently announced in \cite{ChanLiuWei}.



While studying the conjecture raised by De Giorgi, another natural question has appeared: do global minimizers in $\R^n$ of the Allen-Cahn energy have one-dimensional symmetry? A deep result from Savin \cite{Savin-DeGiorgi} states that in dimension $n \leq 7$ this is indeed true. On the other hand, it is conjectured that this is false for $n\geq 8$ and that the saddle-shaped solution is a counterexample (since the Simons cone is a global minimizer of the perimeter functional in these dimensions). The answer to this question would provide some insights to the original conjecture of De Giorgi. This is due to a result by Jerison and Monneau \cite{JerisonMonneau}, where they show that a counterexample to the original conjecture of De Giorgi in $\R^{n+1}$ can be constructed with a rather natural procedure if there exists a global minimizer of $-\Delta u = f(u)$ in $\R^n$ which is bounded and even with respect to each coordinate. This would be an alternative method to the one of \cite{delPinoKowalczykWei} to construct a counterexample to the conjecture. The saddle-shaped solution is of special interest in the search for this counterexample, since it is even with respect to all the coordinate axis and it is canonically associated to the Simons cone, which in turn is the simplest nonplanar minimizing minimal surface.