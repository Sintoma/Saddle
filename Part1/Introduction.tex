%%%%%%%%%%%%%%%%%%%%%%%
\section{Introduction}
%%%%%%%%%%%%%%%%%%%%%%%
\label{Sec:Introduction}


In this paper we study solutions to the semilinear integro-differential equation
\begin{equation}
\label{Eq:NonlocalAllenCahn}
L_K u = f(u) \quad \textrm{ in } \R^{2m}
\end{equation}
which are odd with respect to the Simons cone ---see \eqref{Eq:SimonsCone}. The interest on these solutions is motivated by the nonlocal version of a conjecture by De Giorgi (see the details below) with the aim of finding a counterexample in high dimensions through what we call saddle-shaped solutions (see the definition at the end of this introduction). Moreover, this problem is related to the regularity theory of nonlocal minimal surfaces.

This is the first of two articles concerning equation \eqref{Eq:NonlocalAllenCahn}. In the present paper we address the problem of studying odd solutions with respect to the Simons cone (this terminology is  defined more precisely below). In particular, we find a suitable expression for the operator $L_K$  acting on this type of functions, given by \eqref{Eq:OperatorOddF}, as well as for its associated energy (see Lemma~\ref{Lemma:ShortExpressionEnergy}). We deduce necessary and sufficient conditions for the operator to have a maximum principle in this setting (see Theorem~\ref{Th:CharacterizationLstar}) and, under these assumptions, we establish an energy estimate for doubly radial odd minimizers (Theorem~\ref{Th:EnergyEstimate} below). Finally, as an application of these results we prove, by using variational arguments, the existence of saddle-shaped solutions to problem \eqref{Eq:NonlocalAllenCahn} when $f$ is of Allen-Cahn type ---see the comment right after \eqref{Eq:HipothesesG}. This is Theorem~\ref{Th:Existence}.

In the forthcoming paper \cite{FelipeSanz-Perela:IntegroDifferentialII} we focus our study on saddle-shaped solutions to \eqref{Eq:NonlocalAllenCahn} (assuming $f$ of Allen-Cahn type). Using the setting for odd functions established in the present paper, we give an alternative proof of the existence based on the monotone iteration scheme. In \cite{FelipeSanz-Perela:IntegroDifferentialII}, we also show the uniqueness of the saddle-shaped solution by establishing its asymptotic behavior, as well as a maximum principle for the linearized operator.


Equation \eqref{Eq:NonlocalAllenCahn} is driven by an integro-differential operator $L_K$ of the form
\begin{equation}
\label{Eq:DefOfLu}
L_Ku(x) = \int_{\R^n} \{u(x) - u(y)\} K(x-y)\d y,
\end{equation}
where the kernel $K$ satisfies
\begin{equation}
\label{Eq:Symmetry&IntegrabilityOfK}
K\geq 0\,, \quad K(y) = K(-y) \quad \textrm{ and } \quad \int_{\R^n} \min \left\{ |y|^2, 1 \right\} K(y) \d y < + \infty\,.
\end{equation}
The integral in \eqref{Eq:DefOfLu} has to be understood in the principal value sense, as well as all the integrals involving nonlocal operators in the rest of the paper.
The most canonical example of such operators is the fractional Laplacian
$$
\fraclaplacian u = c_{n, \s} \int_{\R^n} \dfrac{u(x) - u(y)}{|x-y|^{n + 2\s}}\d y\,,
$$
where $c_{n, \s}$ is a normalizing constant (for its exact value see for instance \cite{HitchhikerGuide}).

Recall that the fractional Laplacian has an associated extension problem that allows the use of local arguments to deal with equations such as \eqref{Eq:NonlocalAllenCahn}. This is not the case for general operators such as $L_K$, and therefore some purely nonlocal techniques are developed through this paper. 

Throughout the paper, we assume that the operators of our study are uniformly elliptic. That is, their kernels are bounded from above and below by a positive multiple of the one of the fractional Laplacian:
\begin{equation}
\label{Eq:Ellipticity}
\lambda \dfrac{c_{n,\s}}{|y|^{n+2\s}} \leq K(y) \leq \Lambda\dfrac{c_{n,\s}}{|y|^{n+2\s}}\,, \quad \text{ with }  0< \lambda \leq \Lambda \,,
\end{equation}
where $c_{n,\s}$ is the constant appearing in the definition of the fractional Laplacian. This condition is one of the most frequently adopted when dealing with nonlocal operators of the form \eqref{Eq:DefOfLu}, since it is known to yield Hölder regularity of solutions (see \cite{RosOton-Survey,SerraC2s+alphaRegularity}). The family of linear operators satisfying conditions \eqref{Eq:Symmetry&IntegrabilityOfK} and \eqref{Eq:Ellipticity} is the so-called $\lcal_0(n,\s,\lambda, \Lambda)$ ellipticity class. For short we will usually write $\lcal_0$ and we will make explicit the parameters only when needed. Note that, under the assumption \eqref{Eq:Ellipticity}, $L_K u$ is well-defined if $u\in C^\alpha(\R^{2m})\cap L^\infty(\R^{2m})$, for some $\alpha > 2 \s$.

Moreover, for some purposes we will need the operators to be invariant under rotations. This is equivalent to saying that their kernel is radially symmetric, $K(y) = K(|y|)$. 

Concerning the nonlinearity of the equation, given $f$ a $C^1$ function, we define
$$
G(u)= \int_u^1 f(\tau) \d \tau\,.
$$
Then, we have that $G$ is a $C^2$ function satisfying $G' = -f$. In this paper, we assume the following conditions on $G$:
\begin{equation}
\label{Eq:HipothesesG}
G \textrm{ is even, and } G\geq G(\pm 1 )=0 \textrm{ in } \R\,.
\end{equation}
Note that the previous conditions on $G$ yield that $f$ is an odd function with $f(0)=f(\pm 1)=0$. In some cases, as in Theorem~\ref{Th:Existence} below, we will further assume that $G(0)>0$. In such situation, equation \eqref{Eq:NonlocalAllenCahn} can be seen as a model for phase transitions, and we will say that $f$ is of Allen-Cahn type.




The Simons cone will be a central object along this paper. It is defined in $\R^{2m}$ by
\begin{equation}
\label{Eq:SimonsCone}
\mathscr{C} := \setcond{x = (x', x'') \in \R^m \times \R^m=\R^{2m}}{|x'| = |x''|}.
\end{equation}
This cone is of special importance in the theory of minimal surfaces. It has zero mean curvature at every point $x\in \ccal \setminus \{0\}$, in all even dimensions, and it is a minimizer of the perimeter functional when $2m\geq 8$. Concerning the nonlocal setting, $\ccal$ has also zero nonlocal mean curvature in all even dimensions, although it is not known if it is a minimizer of the nonlocal perimeter in dimensions $2m\geq 4$ (in \cite{SavinValdinoci-Cones} it is proven that all nonlocal minimal cones in $\R^2$ are flat). See also the introduction of \cite{Felipe-Sanz-Perela:SaddleFractional} and the references therein for more details.

Through the whole article we will use $\ocal$ and $\ical$ to denote each of the parts in which $\R^{2m}$ is divided by the cone $\ccal$:
$$
\ocal:= \setcond{x = (x', x'') \in \R^{2m}}{|x'| > |x''|} \textrm{ and } \,
\ical:= \setcond{x = (x', x'') \in \R^{2m}}{|x'| < |x''|}\!.
$$

Both $\ocal$ and $\ical$ belong to a family of sets in $\R^{2m}$ which are called of \emph{double revolution}. These are sets that are invariant under orthogonal transformations in the first $m$ variables and also under orthogonal transformations in the last $m$ variables. That is, $\Omega\subset \R^{2m}$ is a set of double revolution if $R\Omega = \Omega$ for every given transformation $R\in O(m)^2 = O(m) \times O(m)$, where  $O(m)$ is the orthogonal group of $\R^m$.

In this paper we deal with functions that are \emph{doubly radial}. These are functions $w:\R^{2m}  \to \R$ that only depend on the modulus of the first $m$ variables and on the modulus of the last $m$ ones, i.e., $w(x) = w(|x'|,|x''|)$. Equivalently, $w(Rx) = w(x)$ for every $R \in O(m)^2$.

In order to define certain symmetries of functions with respect to the Simons cone, we consider the following isometry, that will play a significant role in this article:
\begin{equation}
\label{Eq:DefStar}
\begin{matrix}
(\cdot)^\star \colon & \R^{2m}= \R^{m}\times \R^{m}  &\to&  \R^{2m}= \R^{m}\times \R^{m}  \\
& x = (x',x'') &\mapsto & x^\star = (x'',x')\,.
\end{matrix}
\end{equation}
Note that this isometry is actually an involution that maps $\ocal$ into $\ical$ (and vice versa) and leaves the cone $\ccal$ invariant. Taking into account this transformation, we say that a doubly radial function $w$ is \emph{odd with respect to the Simons cone} if $w(x) = -w(x^\star)$. Similarly, we say that a doubly radial function $w$ is \emph{even with respect to the Simons cone} if $w(x) = w(x^\star)$.

The usual strategy ---see \cite{CabreTerraI, CabreTerraII,Cabre-Saddle, CabreRosOton-DoubleRev, Cinti-Saddle, Cinti-Saddle2}--- to work with doubly radial solutions (and in particular saddle-shaped solutions) to a semilinear equation like \eqref{Eq:NonlocalAllenCahn} or its local version, is to use the radial variables 
$$
s := |x'| \quad \text{ and } \quad t:=|x''|\,.
$$
This is specially useful when dealing with the Laplacian, since the operator can be written very easily in these variables and the resulting PDE in $(0,+\infty)\times (0,+\infty)$ is suitable to work with. The same happens in the case of the fractional Laplacian thanks to the local extension problem (see \cite{CaffarelliSilvestre}). When we try to follow the same strategy by writing a general operator such as $L_K$ in $(s,t)$ variables, the expression of the new operator is more complex. Indeed, if $w:\R^{2m} \to \R$ is doubly radial and we define $\widetilde{w}(s,t) := w(s,0,...,0,t,0,...,0)$, it holds
$$ L_Kw(x) = \widetilde{L}_K \widetilde{w} (|x'|,|x''|), $$
with
$$
\widetilde{L}_K \widetilde{w} (s,t) := \int_0^{+\infty}  \int_0^{+\infty} \sigma^{m-1} \tau^{m-1} \big(w(s,t) - w(\sigma, \tau)\big) J(s,t,\sigma, \tau)  \d \sigma\d \tau\,,
$$
where
\begin{align*}
J(s,t,\sigma, \tau) &:= \int_{\Sph^{m-1}}  \int_{\Sph^{m-1}} K\Big( \sqrt{s^2+\sigma^2- 2 s \sigma \omega_1 + t^2 + \tau^2 - 2t \tau\tilde\omega_1}\Big) \d \omega \d \tilde\omega\,.
\end{align*}
For more details on this expression, see Appendix~\ref{Sec:stcomputations}. 

Note that $\widetilde{L}_K$ is an integro-differential operator in $(0,+\infty)\times(0,+\infty)$, but the expression of its kernel is quite involved. Despite the fact that all the arguments in this paper can be done working with the radial variables  $(s,t)$, the notation becomes rather cumbersome. For this reason, we follow a different approach that we find more clear and concise. It consists in rewriting the operator $L_K$ without making any change of variables, but with a different kernel that is doubly radial. As it is explained with more details in Section~\ref{Sec:OperatorOddF}, if $K$ is a radially symmetric kernel, then we can write $L_K$ acting on a doubly radial function $w$ as
$$
L_K w(x) = \int_{\R^{2m}} \{w(x) - w(y)\} \overline{K}(x,y) \d y\,,
$$
where $\overline{K}$ is defined by
\begin{equation*}
%\label{Eq:KbarDef'}
\overline{K}(x,y) := \average_{O(m)^2} K(|Rx - y|)\d R\,.
\end{equation*}
Here, $\d R$ denotes integration with respect to the Haar measure on $O(m)^2$ (see Section~\ref{Sec:OperatorOddF} for the details).

The new kernel $\overline{K}$ is symmetric (as $K$), that is, $\overline{K}(x,y) = \overline{K}(y,x)$, but is not translation invariant. Nevertheless, it is doubly radial in its both variables, and this property is crucial to establish some maximum principles for odd functions.

Some of our arguments to prove theses results are inspired by the techniques developed in \cite{ChenLiLi, JarohsWeth}, where they establish some maximum principles for odd functions with respect to a hyperplane. Such maximum principles differ from the usual ones by the fact that an assumption on the sign of a function is needed only at one side of the hyperplane. Let us clarify this. In the usual maximum principle for nonlocal operators, one assumes that a function has a constant (and appropriate) sign in the whole complementary of the set where an equation is satisfied. When there is odd symmetry with respect to a hyperplane, such assumption cannot be done. However, by only assuming a constant (and appropriate) sign in one side of the hyperplane, a maximum principle can be deduced. In particular, if the kernel is decreasing, it is easy to prove a maximum principle of this type. 

In our case, we find analogous maximum principles but with a more complex symmetry, and therefore the kernel is required to satisfy further assumptions. More precisely, we find that the assumption \eqref{Eq:SqrtConvex} below is a sufficient condition for the maximum principle for odd functions with respect to the Simons cone to hold. In order to see this, we  first need to use the symmetry of the functions to rewrite $L_K$ as a integro-differential operator that only takes into account the values of the function in $\ocal$ (since the operator itself incorporates the odd symmetry). The new operator is the following:

\begin{equation}
\label{Eq:OperatorOddF}
	L_K^\ocal w (x) := \int_{\ocal} \{w(x) - w(y) \} \{\overline{K}(x, y) - \overline{K}(x, y^\star)  \} \d y +  2 w(x) \int_{\ocal} \overline{K}(x, y^\star) \d y \,.
\end{equation}

Note that this operator (in contrast with $L_K$) acts on functions that only need to be defined in $\ocal$. Moreover, we show that $L_K^\ocal$ acting on a doubly radial function $w:\ocal \to \R$ corresponds to consider the odd extension of $w$ with respect to the Simons cone and apply the operator $L_K$ to this extended function. This can be easily seen by using the change of variables given $(\cdot)^\star$ --- defined in \eqref{Eq:DefStar}.

Expression \eqref{Eq:OperatorOddF} has an integro-differential part plus a zero order term with a nonnegative coefficient, which is comparable to $\dist(x,\ccal)^{-2\s}$ (see Lemma~\ref{Lemma:OperatorOddF}). Therefore, the natural assumption to make for that operator to have a maximum principle is that the kernel of the integro-differential term is positive. That is, $\overline{K}(x, y) - \overline{K}(x, y^\star)>0$ for every $x,y\in \ocal$. Indeed, we show in Section~\ref{Sec:OperatorOddF} that this assumption guarantees that $L_K$ has a maximum principle for odd functions. 

\todo{Este parrafo quizas se quita}
Since in this article we will always consider odd functions, we will do an abuse of notation and we will denote both operators by $L_K$, since they are the same.

The previous positivity assumption motivates the following definition.

\begin{definition}[Ellipticity class $\lcal_\star$]
	Let $L_K \in L_0(2m,\s,\lambda, \Lambda)$ with kernel $K$ radially symmetric. We say that $L_K\in \lcal_\star (2m,\s,\lambda, \Lambda)$ whenever the associated kernel $\overline{K}$ satisfies
	\begin{equation}
		\label{Eq:KernelInequality}
		\overline{K}(x,y) > \overline{K}(x, y^\star) \quad \text{ for every }x,y \in \ocal\,.
	\end{equation}
\end{definition}

Our first main result concerns necessary and sufficient conditions on the original kernel $K$ in order to $L_K$ belong to $\lcal_\star$.

\begin{theorem}
	\label{Th:CharacterizationLstar}
	Let $L_K \in L_0(2m,\s,\lambda, \Lambda)$ with kernel $K$ radially symmetric, and assume that 
	\begin{equation}
		\label{Eq:SqrtConvex}	
		K(\sqrt{\tau}) \text{ is a strictly convex function of }\tau\,.
	\end{equation}
	Then, $L_K\in \lcal_\star(2m,\s,\lambda, \Lambda)$. Moreover, if $K\in C^1((0,+\infty))$, then \eqref{Eq:SqrtConvex} is a necessary condition for $L_K$ to belong to $\lcal_\star$.
\end{theorem}

This theorem is proved in Section~\ref{Sec:OperatorOddF} (see Propositions~\ref{Prop:KernelInequalitySufficientCondition} and \ref{Prop:KernelInequalityNecessaryCondition}). It is based on a suitable division of the group $O(m)^2$ and a result on convex functions proved in Appendix~\ref{Sec:AuxiliaryResults} (Proposition~\ref{Prop:EquivalenceK(sqrt)Convex<->Inequality}).

Equation \eqref{Eq:NonlocalAllenCahn} is the Euler-Lagrange equation associated to the energy functional
\begin{equation*}
%\label{Eq:Energy}
\ecal(w, \Omega) := \dfrac{1}{4}\int\int_{\R^{2n} \setminus (\R^n\setminus\Omega)^2} |w(x) - w(y)|^2 K(x-y) \d x \d y + \int_{\Omega} G(w) \d x \,.
\end{equation*}
Using the same type of arguments as for the operator $L_K$, we can rewrite the energy of doubly radial and odd functions with a suitable new expression that involves the kernel $\overline{K}$ (see Lemma~\ref{Lemma:ShortExpressionEnergy} in  Section~\ref{Sec:EnergyForOddF}). Thanks to this, we are able to establish the second main result of this paper. It is the following energy estimate for doubly radial and odd minimizers of $\ecal$. In the next statement, $\widetilde{\H}^K_{0, \mathrm{odd}}(B_R)$ denotes the space of doubly radial and odd functions that vanish outside $B_R$ and for which the energy $\ecal$ is well defined (see Section~\ref{Sec:EnergyForOddF} for the precise definition).

\begin{theorem}
	\label{Th:EnergyEstimate} 
	Let $K$ be a kernel such that $L_K\in \lcal_\star(2m, \s, \lambda, \Lambda)$ and assume that $G$ is a potential satisfying \eqref{Eq:HipothesesG}. Let $S>2$ and let $u$ be a minimizer of the energy $\ecal$ in $B_{R}$, with $R>S+2$, among functions that are in $\widetilde{\H}^K_{0, \mathrm{odd}}(B_R)$. Then
	%$$ \lim_{R\to +\infty} \frac{1}{S^n} \ecal (u,B_S) = 0. $$
	%More precisely,
	$$ \ecal (u,B_S) \leq \begin{cases}
	C \ S^{2m-2\s}\ \ \ \ &\textrm{if } \ \ \s\in(0,1/2),\\
	C\ \log(S)\,S^{2m-1}\ \ \ \ &\textrm{if } \ \ \s=1/2,\\
	C \ S^{2m-1}\ \ \ \ &\textrm{if } \ \ \s\in(1/2,1),\\
	\end{cases} $$
	with $C$ a positive constant depending only on $m$, $\s$, $\Lambda$ and $G$.
\end{theorem}



This result has been proved in the case of the fractional Laplacian by Cinti \cite{Cinti-Saddle,Cinti-Saddle2}, but using the local extension problem (see also 99 and 99 for non-doubly radial minimizers). In our case, since this technique is not available, we follow the arguments of Savin and Valdinoci in \cite{SavinValdinoci-EnergyEstimate}, where they prove a similar energy estimate for minimizers without any symmetry. The strategy they follow to establish the result is to compare the energy of $u$ with the one of a suitable competitor which is constructed combining $u$ with a radially symmetric auxiliary function. Such competitor is not permitted in our case, since it is not odd with respect to the Simons cone. Nevertheless, we show in Section~\ref{Sec:EnergyForOddF} how to adapt the auxiliary functions of \cite{SavinValdinoci-EnergyEstimate} to our setting in order to establish Theorem~\ref{Th:EnergyEstimate}. In the arguments, the assumption \eqref{Eq:KernelInequality} is crucial.


As an application of the previous results, we prove, by using standard variational methods, the existence of saddle-shaped solutions to  \eqref{Eq:NonlocalAllenCahn} when $f$ is of Allen-Cahn type. We say that a bounded solution $u$ to \eqref{Eq:NonlocalAllenCahn} is a \emph{saddle-shaped} solution if $u$ is doubly radial, odd with respect to the Simons cone and positive in $\ocal$. 

\begin{theorem}[Existence of saddle-shaped solutions]
	\label{Th:Existence}
    Let $f=-G'$ satisfy \eqref{Eq:HipothesesG}, and such that $G(0)>0$. Assume $L_K\in \lcal_\star$. Then, for every even dimension $2m \geq 2$, there exists a saddle-shaped solution to \eqref{Eq:NonlocalAllenCahn}. In addition, $u$ satisfies $|u|<1$ in $\R^{2m}$.
\end{theorem}

We are interested in the study of this type of solutions since they are relevant in connection with a famous conjecture for the (classical) Allen-Cahn equation raised by De Giorgi, that reads as follows. Let $u$ be a bounded monotone (in some direction) solution to $-\Delta u = u - u^3$ in $\R^n$, then, if $n \leq 8$, $u$ depends on only one Euclidean variable, that is, all its level sets are hyperplanes. This conjecture is not completely closed (see \cite{FarinaValdinoci-DeGiorgi} and references therein) but a counterexample in dimension $n=9$ was build in \cite{delPinoKowalczykWei} using the so-called gluing method. Saddle-shaped solutions are natural objects to build a counterexample in a simpler way, as explained next. On the one hand, Jerison and Monneau \cite{JerisonMonneau} showed that a counterexample to the conjecture of De Giorgi in $\R^{n+1}$ can be constructed with a rather natural procedure if there exists a global minimizer of $-\Delta u = f(u)$ in $\R^n$ which is bounded and even with respect to each coordinate, but is not one-dimensional. On the other hand, by the $\Gamma$-converge results from Modica and Mortola (see \cite{Modica,ModicaMortola}) and the fact that the Simons cone is the simplest nonplanar minimizing minimal surface, saddle-shaped solutions are expected to be global minimizer of the Allen-Cahn equation (this is still an open problem).

Similar facts happen in the nonlocal setting (see the introduction of \cite{Felipe-Sanz-Perela:SaddleFractional} for further details). For this reason, saddle-shaped solutions are of great interest in the study of the nonlocal version of the conjecture of De Giorgi for equation \eqref{Eq:NonlocalAllenCahn}.

Saddle-shaped solutions to the classical Allen-Cahn equation involving the Laplacian were studied in \cite{DangFifePeletier, Schatzman, CabreTerraI,CabreTerraII, Cabre-Saddle}. In these works, they established the existence, uniqueness and some qualitative properties of this type of solutions, such as instability when $2m\leq 6$ and stability if $2m\geq 14$. Stability in dimensions $8, 10$ and $12$ is still an open problem, as well as minimality in dimensions $2m\geq 8$.

In the fractional framework, there are only three works concerning saddle-shaped solutions to the equation $\fraclaplacian u = f(u)$. In  \cite{Cinti-Saddle,Cinti-Saddle2}, Cinti proved the existence of a saddle-shaped solution as well as some qualitative properties such as asymptotic behavior, monotonicity properties, and instability in low dimensions. In a previous paper by the authors \cite{Felipe-Sanz-Perela:SaddleFractional}, further properties of these solutions have been proved, the main ones being uniqueness and, when $2m\geq 14$, stability. To our knowledge, the present paper together with its second part \cite{FelipeSanz-Perela:IntegroDifferentialII} are the first ones studying saddle-shaped solutions for general integro-differential equations of the form \eqref{Eq:NonlocalAllenCahn}. In the three previous papers \cite{Cinti-Saddle, Cinti-Saddle2, Felipe-Sanz-Perela:SaddleFractional}, the main tool used is the extension problem for the fractional Laplacian (see \cite{CaffarelliSilvestre}). Nevertheless, this technique has the limitation that it cannot be carried out for general integro-differential operators different from the fractional Laplacian. Therefore, some purely nonlocal techniques are developed through this paper.

In the forthcoming paper \cite{FelipeSanz-Perela:IntegroDifferentialII} we establish an alternative proof for the existence by using other techniques (monotone iteration and maximum principle). In both proofs, the assumtion \eqref{Eq:KernelInequality} is crucial. In such paper we also show the uniqueness of the saddle-shaped solution through its asymptotic behavior and a maximum principle for the linearized operator.

The paper is organized as follows. Section~\ref{Sec:OperatorOddF} is devoted to study the operator $L_K$ acting on doubly radial odd functions. We deduce the expression of the doubly radial kernel $\overline{K}$ and we prove some properties. We also show Theorem~\ref{Th:CharacterizationLstar} and some maximum principles. In Section~\ref{Sec:EnergyForOddF} we study the energy functional associated to \eqref{Eq:NonlocalAllenCahn} and in Section~\ref{Sec:EnergyEstimate} we establish the energy estimate stated in Theorem~\ref{Th:EnergyEstimate}. Finally, in Section~\ref{Sec:Existence} we prove the existence of the saddle-shaped solution to the integro-differential Allen-Cahn equation. At the end of the paper there are three appendices. Appendix~\ref{Sec:AuxiliaryResults} is devoted to some results on convex functions, and Appendix~\ref{Sec:AuxiliaryResults2} contains some auxiliary computations. Both are used in the proof of Theorem~\ref{Th:CharacterizationLstar}. In Appendix~\ref{Sec:stcomputations} we include some computations in $(s,t)$ variables for future reference.