%%%%%%%%%%%%%%%%%%%%%%%
\section{Introduction}
%%%%%%%%%%%%%%%%%%%%%%%
\label{Sec:Introduction}


We study the equation
\begin{equation}
\label{Eq:NonlocalAllenCahn}
Lu = f(u) \quad \textrm{ in } \R^{2m}
\end{equation}
where $f$ is of bistable type and $L$ is an integro-differential operator of the form
\begin{equation}
\label{Eq:DefOfLu}
Lu(x) = \PV \int_{\R^n} \{u(x) - u(y)\} K(x-y)\d y
\end{equation}
where $K$ is a  kernel satisfying
\begin{equation}
\label{Eq:Symmetry&IntegrabilityOfK}
K\geq 0\,, \quad K(y) = K(-y) \quad \textrm{ and } \quad \int_{\R^n} \min \left\{ |y|^2, 1 \right\} K(y) \d y < + \infty\,.
\end{equation}
The most canonical example of such operators is the fractional Laplacian
$$
\fraclaplacian u = \PV c_{n, \s}\int_{\R^n} \dfrac{u(x) - u(y)}{|x-y|^{n + 2\s}}\d y\,,
$$
where $c_{n, \s}$ is a normalizing constant (for its exact value see for instance \cite{HitchhikerGuide}).\todo{Mirar a ver si esta es una buena referencia o se pone otra}

Throughout the paper, we assume that the operators of our study are uniformly elliptic in the sense of the following definition.
\begin{definition}
\label{Def:L_0Class}
We say that an operator $L$ of the form \eqref{Eq:DefOfLu} belongs to the class $\lcal_0(n,\s,\lambda, \Lambda)$ if its kernel $K$ satisfies \eqref{Eq:Symmetry&IntegrabilityOfK} and \begin{equation}
\label{Eq:Ellipticity}
c_{n,\s}\dfrac{\lambda}{|y|^{n+2\s}} \leq K(y) \leq c_{n,\s}\dfrac{\Lambda}{|y|^{n+2\s}}\,, \quad 0< \lambda \leq \Lambda \,,
\end{equation}
where $c_{n,\s}$ is the constant appearing in the definition of the fractional Laplacian. We will say that $L$ belongs to the ellipticity class $\Lr(n,\s,\lambda, \Lambda)$ if $L\in \lcal_0(n,\s,\lambda, \Lambda)$ and its kernel is radially symmetric, i.e., $K(y)=K(|y|)$.
\end{definition}
For short we will usually write $\lcal_0$ or $\Lr$, and we will make explicit the parameters only when needed.



\bigskip
[...]
\bigskip

Given $f$ a $C^1$ nonlinearity, we define
$$
G(u)= \int_u^1 f(t) \d t\,.
$$
We have that $G$ is a $C^2$ function satisfying $G' = -f$. In this paper, we assume some, or all, of the following conditions on $f$.
\begin{equation}
\label{Eq:HipothesisfOdd}
f \textrm{ is odd;}
\end{equation}
\begin{equation}
\label{Eq:HipothesisGWells}
G\geq 0 \quad \textrm{ in } \R, \quad G > 0 \ \textrm{ in }  (-1,1), \quad \textrm{ and }\quad G(\pm 1 )=0\,;
\end{equation}
\begin{equation}
\label{Eq:HipothesisfConcave}
f \textrm{ is concave in }  (0,1).
\end{equation}



Note that \eqref{Eq:HipothesisfOdd} and \eqref{Eq:HipothesisGWells} yield that $f(0)=f(\pm 1)=0$. 

Note that \eqref{Eq:HipothesisfOdd} is equivalent to say that $G$ is even.

Note that \eqref{Eq:HipothesisfOdd}, \eqref{Eq:HipothesisGWells}, and \eqref{Eq:HipothesisfConcave} and the fact that $f(1)=0$ yield $f'(0)>0$ and $f'(\pm 1) < 0$. As a consequence, $f > 0$ in $(0,1)$.

Comentario: $f'(\pm 1) < 0$ equivale a $G''(\pm 1) > 0\,,$ que es la hipotesis junto con las otras para que exista el Layer

Note that, since $f$ is concave in $(0,1)$ and $f(0)=0$, then 
\begin{equation}
\label{Eq:PropertyConcavityf}
f'(t)t \leq f(t) \quad \textrm{ for all } t\in (0,1)\,.
\end{equation}
The inequality is strict if we have strict concavity.


\bigskip
\bigskip
\bigskip
-------------
\bigskip
\bigskip
\bigskip

[...]

The interest on this problem originates from the famous De Giorgi conjecture for the classical Allen-Cahn equation.

\begin{conjecture}[De Giorgi, 1978]
	Let $u$ be a bounded solution of the Allen-Cahn equation
	\begin{equation}
	\label{Eq:LocalAllenCahn}
	-\Delta  u = u - u^3 \quad \text{ in } \R^n
	\end{equation}
	such that it is monotone in one direction, say $\partial_{x_n} u > 0$. Then, if $n\leq 8$, $u$ is one dimensional, i.e., $u$ depends only on one Euclidean variable.
\end{conjecture}

This conjecture was proved true in dimension $n=2$ by Ghoussoub and Gui in \cite{GhoussoubGui}, and in dimension $n=3$ by Ambrosio and Cabré in \cite{AmbrosioCabre}. For dimensions $4\leq n \leq 8$, it was established by Savin in \cite{Savin-DeGiorgi} but with the extra assumption of
\begin{equation}
\label{Eq:SavinCondition}
	\lim_{x_n \to \pm \infty} u(x',x_n) = \pm 1 \quad \text{ for all } x'\in \R^{n-1}\,.
\end{equation}
A counterexample to the conjecture was given by del~Pino, Kowalczyk and Wei in \cite{delPinoKowalczykWei}. 


One can also formulate the same conjecture in the nonlocal setting by changing the Laplacian by a nonlocal operator, the most canonical one being $\fraclaplacian$. In this framework, for the equation $\fraclaplacian u = f(u)$ in $\R^n$ and $\s\in (0,1)$, the conjecture has been proven to be true in dimension $n=2$ by Cabré and Solà-Morales in \cite{CabreSolaMorales} for $\s=1/2$ and extended to every power $0<\s<1$ by Cabré and Sire in \cite{CabreSireI} and also by Sire and Valdinoci in \cite{SireValdinoci}. In dimension $3$ the conjecture has been proved by Cabré and Cinti for $1/2 \leq \s < 1$ in \cite{CabreCinti-EnergyHalfL, CabreCinti-SharpEnergy}. Recently, in \cite{Savin-Fractional} Savin has established the validity of the conjecture in dimensions $4\leq n \leq 8$ and for $1/2 < \s < 1$, but assuming the condition \eqref{Eq:SavinCondition}. In that paper he has also announced that the same holds for $\s=1/2$. The case $0<\s<1/2$ has been also treated by Dipierro, Serra and Valdinoci in \cite{DipierroSerraValdinoci} \todo{Revisar, dependeria de la clasificación de los conos}. The most recent result related to the conjecture is the one of Figalli and Serra in \cite{FigalliSerra}, where they have proven the conjecture in dimension $n=4$ and $\s=1/2$. Note that this is the only result that is available (by now) exclusively in the nonlocal setting and not for the Laplacian.

After all the years of study of the conjecture raised by De Giorgi, another question appeared naturally: do global minimizers of the energy associated to the equation  $-\Delta u = f(u)$ in $\R^n$ have one-dimensional symmetry? A deep result from Savin \cite{Savin-DeGiorgi} is that in dimension $n \leq 7$ this is indeed true, and the conjecture is that for $n\geq 8$ is false. The answer to this question would provide some insights of the original conjecture of De Giorgi. This is due to a result by Jerison and Monneau in \cite{JerisonMonneau}, where they show that a counterexample of the original conjecture in $\R^{n+1}$ can be constructed from a bounded, even with respect to each coordinate, global minimizer of $-\Delta u = f(u)$ in $\R^n$. Hence, finding a global minimizer that is not one-dimensional would give a natural counterexample to the original conjecture.

Saddle-shaped solutions are of special interest in the search for this counterexample. To define these solutions properly, we need to introduce some notation and definition.

First of all, recall that the Simons cone is defined in $\R^{2m}$ with $2m = n$ by
\begin{equation}
\label{Eq:SimonsCone}
	\mathscr{C} = \setcond{x = (x', x'') \in \R^{2m}}{|x'| = |x''|}\,.
\end{equation}
The Simons cone is proven to be a (classical) stationary minimal surface. Moreover, if $n\geq 8$, is also minimizing (see 99 and the coments... \todo{comentar algo despues con aquello del límite?}). 
Through the paper we will also use the letters $\ocal$ and $\ical$ to denote the outside and inside of the cone:
\begin{equation}
\label{Eq:DefOandI}
\ocal:= \setcond{x = (x', x'') \in \R^{2m}}{|x'| > |x''|} \ \textrm{ and } \
\ical:= \setcond{x = (x', x'') \in \R^{2m}}{|x'| < |x''|}.
\end{equation}


Let $SO(m)$ be the special orthogonal group of $\R^m$, that is, the group of rotations of $\R^m$. We will work with the group $SO(m)^2 = SO(m) \times SO(m)$. Note that $SO(m)^2 \subset SO(2m)$ and therefore, for any $R\in SO(m)^2$, $|Rx| = |x|$. Moreover, the sets $\ocal$ and $\ical$ are invariant under the action of the group and belong to a more general class of domains defined next.

\begin{definition}
\label{Def:DoubleRevolutionSet}
We say that a set $\Omega\subset \R^{2m}$ is of \emph{double revolution} if it is invariant under $SO(m)^2$, i.e., if it is invariant under orthogonal transformations in the first $m$ variables and also under orthogonal transformations in the last $m$ variables.
\end{definition}

More defs (introducir mejor)

\begin{definition}
\label{Def:DoublyRadial}
We say that a function $w:\R^{2m} \to \R$ is \emph{doubly radial} if it only depends on the modulus of the first $m$ variables and on the modulus of the last $m$ ones, i.e., $w(x) = w(|x'|,|x''|)$. Equivalently, if $w(Rx) = w(x)$ for every $R \in SO(m)^2$.
\end{definition}

Through the paper we will consider the following isometry:
\begin{equation}
\label{Eq:DefStar}
\begin{matrix}
(\cdot)^\star \colon & \R^{2m}= \R^{m}\times \R^{m}  &\to&  \R^{2m}= \R^{m}\times \R^{m}  \\
	& x = (x',x'') &\mapsto & x^\star = (x'',x')\,.
\end{matrix}
\end{equation}
Note that this isometry satisfies
\begin{enumerate}
\item $((\cdot)^\star)^{-1} = (\cdot)^\star$.
\item $\ocal^\star= \ical$ and $\ical^\star = \ocal$.
\end{enumerate}


\begin{definition}
\label{Def:OddwrtSimonsCone}
We say that a doubly radial function $w$ is \emph{odd with respect to the Simons cone} if $w(|x'|,|x''|) = -w(|x''|,|x'|)$ for every $x = (x', x'') \in \R^{2m}$, or equivalently, if $w(x) = -w(x^\star)$. Similarly, we say that a doubly radial function $w$ is \emph{even with respect to the Simons cone} if $w(|x'|,|x''|) = w(|x''|,|x'|)$ for every $x = (x', x'') \in \R^{2m}$, or equivalently, if $w(x) = w(x^\star)$.
\end{definition}

With these definitions in hand, we can define now properly saddle-shaped solutions:
\begin{definition}
\label{Def:SaddleShapedSol}
We say that $u$ is a \emph{saddle-shaped solution} (or simply \emph{saddle solution}) of \eqref{Eq:NonlocalAllenCahn} if
\begin{enumerate}
\item $u$ is doubly radial.
\item $u$ is odd with respect to the Simons cone.
\item $u > 0$ in $\ocal$.
\end{enumerate}
\end{definition}


Note that these solutions are even with respect to the coordinate axis and that their zero level set is the Simons cone $\mathscr{C} = \{|x'|=|x''|\}$. Therefore, saddle-shaped solutions are candidates to build a counterexample of the De Giorgi conjecture in high dimensions, since if one could prove that they are global minimizers in $\R^8$, by the result in \cite{JerisonMonneau} one would have a counterexample of the De Giorgi conjecture in $\R^9$ (as an alternative to that of \cite{delPinoKowalczykWei}).

Saddle-shaped solutions for the classical equation with the Laplacian were first studied by Dang, Fife, and Peletier in \cite{DangFifePeletier} in dimension $2m=2$. They established the existence and uniqueness of this type of solutions, as well as some monotonicity properties and asymptotic behavior. In \cite{Schatzman}, Schatzman studied the instability property of saddle solutions in $\R^2$. In higher even dimensions, Cabré and Terra  proved the existence of a saddle solution in every dimension $2m\geq 2$ and they established also some qualitative properties such as monotonicity properties, asymptotic behavior, as well as instability in dimensions $2m = 4$ and $2m = 6$ (see \cite{CabreTerraI,CabreTerraII}). The uniqueness in dimensions higher than $2$ was established by Cabré in \cite{Cabre-Saddle}, where he also proved that saddle solutions are stable (see the definition below\todo{ver si lo quitamos o como lo ponemos}) in dimensions $2m \geq 14$.

In the nonlocal framework, there are some works concerning saddle-shaped solutions to \eqref{Eq:NonlocalAllenCahn} with $L= \fraclaplacian$. In  \cite{Cinti-Saddle}, Cinti established the existence of saddle-shaped solutions to $(-\Delta)^{1/2}u = f(u)$ in $\R^{2m}$, as well as some qualitative properties such as asymptotic behavior, monotonicity properties, and instability in dimensions $2m = 4$ and $2m = 6$ (instability in dimension $2m=2$ follows by a result of Cabré and Solà-Morales in \cite{CabreSolaMorales}). More recently, she has extended the same results to all $\s \in (0,1)$ in 99\todo{Citar paper de Cinti}.


To the best of our knowledge, there are no more works studying the saddle-shaped solutions in the nonlocal setting. Moreover, the problem has not been studied for general operators $L\in \lcal_0$. Regarding the nonlocal Allen-Cahn equation with general kernels, we have to mention two works.

The first one is \cite{CozziPassalacqua}, where Cozzi and Passalacqua study layer solutions to the equation \eqref{Eq:NonlocalAllenCahn}.

We should also mention \cite{DipierroSerraValdinoci}\todo{CHECK} where they study nonlocal minimal surfaces with general kernels and blablabla.mal surfaces with general kernels and blablabla.



\bigskip
\bigskip
\bigskip
-------------
\bigskip
\bigskip
\bigskip









The second variation of the energy is
\begin{equation}
 \label{Eq:SecondVariation}	
 Q_u(\xi) := \dfrac{1}{2} \int_{\R^{2m}} \int_{\R^{2m}} |\xi (x) - \xi(y)|^2 K(x - y) \d x \d y - \int_{\R^{2m}} f'(u) \xi^2 \d x\,.
\end{equation}

\begin{definition}
	\label{Def:Stability}
	We say that a solution $u$ of \eqref{Eq:NonlocalAllenCahn} is \emph{stable} in a set $\Omega \subset \R^{2m}$ if 
	\begin{equation}
	\label{Eq:StablityCondition}	
	Q_u(\xi) = \dfrac{1}{2} \int_{\R^{2m}} \int_{\R^{2m}} |\xi (x) - \xi(y)|^2 K(x - y) \d x \d y - \int_{\Omega} f'(u) \xi^2 \d x \geq 0
	\end{equation}
	for every $\xi \in C^\infty_c (\Omega)$.
\end{definition}

\begin{lemma}[see \cite{HamelRosOtonSireValdinoci}]
	\label{Lemma:EquivalenceStability}
	A solution $u$ is stable in $\Omega$ (in the sense of Definition~\ref{Def:Stability}) if and only if there exists a continuous function $\varphi$ such that $\varphi > 0$ in $\Omega$ and $L\varphi \geq f'(u) \varphi$ in $\Omega$.
\end{lemma}

The proof of this result can be found in \cite{HamelRosOtonSireValdinoci}. Nevertheless, one of the implications is quite simple and since is an argument that will be repeated in the paper, we show it here. 

Assume that $\varphi > 0$ is a supersolution of the linearized operator in $\Omega$. Then, let $\xi \in C^\infty_c (\Omega)$ and we compute
\begin{align*}
	\int_\Omega f'(u) \xi^2 \d x &= \int_\Omega f'(u) \varphi \dfrac{\xi^2}{\varphi} \d x \\
	&\leq \int_\Omega L\varphi \dfrac{\xi^2}{\varphi}  \d x \\ 
	&= \dfrac{1}{2} \int_{\R^{2m}} \int_{\R^{2m}} \big ( \varphi(x) - \varphi(y) \big) \bpar{\dfrac{\xi^2(x)}{\varphi(x)} - \dfrac{\xi^2(y)}{\varphi(y)} } K(x - y) \d x \d y
	 \\ 
	&\leq \dfrac{1}{2} \int_{\R^{2m}} \int_{\R^{2m}} |\xi (x) - \xi(y)|^2 K(x - y) \d x \d y\,.
\end{align*}
Here we have used that the kernel is positive and that
$$
\big ( \varphi(x) - \varphi(y) \big) \bpar{\dfrac{\xi^2(x)}{\varphi(x)} - \dfrac{\xi^2(y)}{\varphi(y)} } \leq |\xi (x) - \xi(y)|^2\,.
$$
Indeed, developing the squares and the products, this last inequality is equivalent to
$$
2 \xi(x) \xi(y) \leq \dfrac{\varphi(x)}{\varphi(y)} \xi^2(y) +  \dfrac{\varphi(y)}{\varphi(x)} \xi^2 (x)\,,
$$
which is equivalent to
$$
\bpar{\xi (x)\sqrt{\dfrac{\varphi(y)}{\varphi(x)}} - \xi(y) \sqrt{\dfrac{\varphi(x)}{\varphi(y)} } }^2 \geq 0\,.
$$

\todo[inline]{Esto lo quitaremos que no hacemos nada de estabilidad}

\bigskip
\bigskip
\bigskip
-------------
\bigskip
\bigskip
\bigskip


\bigskip
\bigskip
\bigskip
-------------
\bigskip
\bigskip
\bigskip

\begin{theorem}[Existence of the saddle-shaped solution]
	\label{Th:Existence}
    Let $f$ satisfy .... and let $L\in \lcal_\star$. Then, for every dimension $2m \geq 2$, there exists a saddle-shaped solution to \eqref{Eq:NonlocalAllenCahn}. In addition, $u$ satisfies $|u|<1$ in $\R^{2m}$ and \todo{Poner algo más?}.
\end{theorem}



\todo[inline]{juntar teoremas?}