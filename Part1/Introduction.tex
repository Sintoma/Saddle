%%%%%%%%%%%%%%%%%%%%%%%
\section{Introduction}
%%%%%%%%%%%%%%%%%%%%%%%
\label{Sec:Introduction}


In this paper we study solutions to the semilinear integro-differential equation
\begin{equation}
\label{Eq:NonlocalAllenCahn}
L_K u = f(u) \quad \textrm{ in } \R^{2m}
\end{equation}
which are odd with respect to the Simons cone --- defined in \eqref{Eq:SimonsCone}. The interest on these solutions, often called saddle-shaped solutions, is motivated by the nonlocal version of a conjecture by De Giorgi on the Allen-Cahn equation (see details below) with the aim of finding a counterexample in high dimensions. Moreover, this problem is related to the regularity theory of nonlocal minimal surfaces.

There are only three papers in the literature concerning saddle-shaped solutions to \eqref{Eq:NonlocalAllenCahn} with $L_K$ being the fractional Laplacian: \cite{Cinti-Saddle, Cinti-Saddle2} by Cinti and \cite{Felipe-Sanz-Perela:SaddleFractional} by the authors. In all of them the main tool is the extension problem. This paper, together with its second part  \cite{FelipeSanz-Perela:IntegroDifferentialII}, is the first one to study \eqref{Eq:NonlocalAllenCahn} without the extension. For this reason our arguments are purely nonlocal and hold for a more general class of kernels.

%This is the first of two articles concerning equation \eqref{Eq:NonlocalAllenCahn}. In the present paper we address the problem of studying doubly radial solutions which are odd with respect to the Simons cone (this terminology is  defined more precisely below). In particular, we find a suitable expression for the operator $L_K$  acting on this type of functions, given by \eqref{Eq:OperatorOddF}, as well as for its associated energy ---see \eqref{Eq:ShortExpressionEnergyIntro}. We deduce necessary and sufficient conditions for the operator to have a maximum principle in this setting (see Theorem~\ref{Th:SufficientNecessaryConditions}) and, under these assumptions, we establish an energy estimate for doubly radial odd minimizers (Theorem~\ref{Th:EnergyEstimate} below). Finally, as an application of these results we prove, by using variational arguments, the existence of saddle-shaped solutions to problem \eqref{Eq:NonlocalAllenCahn} when $f$ is of Allen-Cahn type ---see the comment right after \eqref{Eq:HipothesesG}. This is Theorem~\ref{Th:Existence}.

%In the forthcoming paper \cite{FelipeSanz-Perela:IntegroDifferentialII} we focus our study on saddle-shaped solutions to \eqref{Eq:NonlocalAllenCahn} (assuming $f$ of Allen-Cahn type). Using the setting for odd functions established in the present article, we give an alternative proof of the existence of saddle solutions based on the monotone iteration scheme. In \cite{FelipeSanz-Perela:IntegroDifferentialII}, we also show the uniqueness of this type of solution by establishing its asymptotic behavior, as well as a maximum principle for the linearized operator.


Equation \eqref{Eq:NonlocalAllenCahn} is driven by an integro-differential operator $L_K$ of the form
\begin{equation}
\label{Eq:DefOfLu}
L_Ku(x) = \int_{\R^n} \{u(x) - u(y)\} K(x-y)\d y,
\end{equation}
where the kernel $K$ satisfies
\begin{equation}
\label{Eq:Symmetry&IntegrabilityOfK}
K\geq 0\,, \quad K(z) = K(-z) \quad \textrm{ and } \quad \int_{\R^n} \min \left\{ |z|^2, 1 \right\} K(z) \d z < + \infty\,.
\end{equation}
The integral in \eqref{Eq:DefOfLu} has to be understood in the principal value sense. The most canonical example of such operators is the fractional Laplacian, defined for $\s\in(0,1)$ as
$$
\fraclaplacian u = c_{n, \s} \int_{\R^n} \dfrac{u(x) - u(y)}{|x-y|^{n + 2\s}}\d y\,,
$$
where $c_{n, \s}$ is a normalizing constant.

Recall that the fractional Laplacian has an associated extension problem (see \cite{CaffarelliSilvestre}) that allows the use of local arguments to deal with equations such as \eqref{Eq:NonlocalAllenCahn}. This is not the case for general operators $L_K$, and therefore some purely nonlocal techniques are developed along this work. 

Throughout the paper, we assume that $L_K$ is uniformly elliptic, that is,
\begin{equation}
\label{Eq:Ellipticity}
\lambda \dfrac{c_{n,\s}}{|z|^{n+2\s}} \leq K(z) \leq \Lambda \dfrac{c_{n,\s}}{|z|^{n+2\s}}\,, 
\end{equation}
where $\lambda$ and $\Lambda$ are two positive constants. This condition is frequently adopted since it yields Hölder regularity of solutions (see \cite{RosOton-Survey,SerraC2s+alphaRegularity}). The family of linear operators satisfying conditions \eqref{Eq:Symmetry&IntegrabilityOfK} and \eqref{Eq:Ellipticity} is the so-called $\lcal_0(n,\s,\lambda, \Lambda)$ ellipticity class. For short we will usually write $\lcal_0$ and we will make explicit the parameters only when needed. %Note that, under the assumption \eqref{Eq:Ellipticity}, $L_K u$ is well-defined if $u\in C_{\rm loc}^\alpha(\R^{2m})\cap L^\infty(\R^{2m})$, for some $\alpha > 2 \s$.

Moreover, for many  purposes we will need the operators to be invariant under rotations. This is equivalent to saying that the kernel is radially symmetric, $K(z) = K(|z|)$. 




The Simons cone will be a central object along this paper. It is defined in $\R^{2m}$ by
\begin{equation}
\label{Eq:SimonsCone}
\mathscr{C} := \setcond{x = (x', x'') \in \R^m \times \R^m=\R^{2m}}{|x'| = |x''|}.
\end{equation}
This cone is of special importance in the theory of local and nonlocal minimal surfaces, and its variational properties are related to the conjecture of De Giorgi (see the end of this introduction for more details). Through the whole article we will use $\ocal$ and $\ical$ to denote each of the parts in which $\R^{2m}$ is divided by the cone $\ccal$:
$$
\ocal:= \setcond{x = (x', x'') \in \R^{2m}}{|x'| > |x''|} \textrm{ and } \,
\ical:= \setcond{x = (x', x'') \in \R^{2m}}{|x'| < |x''|}\!.
$$

Both $\ocal$ and $\ical$ belong to a family of sets in $\R^{2m}$ which are called of \emph{double revolution}. These are sets that are invariant under orthogonal transformations in the first $m$ variables, as well as under orthogonal transformations in the last $m$ variables. That is, $\Omega\subset \R^{2m}$ is a set of double revolution if $R\Omega = \Omega$ for every given transformation $R\in O(m)^2 = O(m) \times O(m)$, where  $O(m)$ is the orthogonal group of $\R^m$.

In this paper we deal with functions that are \emph{doubly radial}. These are functions $w:\R^{2m}  \to \R$ that only depend on the modulus of the first $m$ variables and on the modulus of the last $m$ ones, i.e., $w(x) = w(|x'|,|x''|)$. Equivalently, $w(Rx) = w(x)$ for every $R \in O(m)^2$.

In order to define oddness and evenness of functions with respect to the Simons cone, we consider the following isometry, which will play a significant role in this article:
\begin{equation}
\label{Eq:DefStar}
\begin{matrix}
(\cdot)^\star \colon & \R^{2m}= \R^{m}\times \R^{m}  &\to&  \R^{2m}= \R^{m}\times \R^{m}  \\
& x = (x',x'') &\mapsto & x^\star = (x'',x')\,.
\end{matrix}
\end{equation}
Note that this isometry is actually an involution that maps $\ocal$ into $\ical$ (and vice versa) and leaves the cone $\ccal$ invariant ---although not all points in $\ccal$ are fixed points of $(\cdot)^\star$. Taking into account this transformation, we say that a doubly radial function $w$ is \emph{odd with respect to the Simons cone} if $w(x) = -w(x^\star)$. Similarly, we say that a doubly radial function $w$ is \emph{even with respect to the Simons cone} if $w(x) = w(x^\star)$.

Regarding the doubly radial symmetry we define the following variables
$$
s := |x'| \quad \text{ and } \quad t:=|x''|\,.
$$
They are specially useful when dealing with the Laplacian in these coordinates, since
\begin{equation}
\label{Eq:Laplacian-st}
\Delta w = w_{ss} + w_{tt} + \frac{m-1}{s}w_s + \frac{m-1}{t}w_t
\end{equation}
becomes an expression suitable to work with. A similar formula appears in the case of the fractional Laplacian thanks to the local extension problem. Having a PDE in the two variables $(s,t)\in \R^2$ is useful to perform certain computations (see \cite{CabreTerraI, CabreTerraII,Cabre-Saddle, CabreRosOton-DoubleRev} for the local case and \cite{Cinti-Saddle, Cinti-Saddle2, Felipe-Sanz-Perela:SaddleFractional} for the fractional framework).

%------------------------------------------------------------------------------
%A usual strategy ---see \cite{CabreTerraI, CabreTerraII,Cabre-Saddle, CabreRosOton-DoubleRev, Cinti-Saddle, Cinti-Saddle2, Felipe-Sanz-Perela:SaddleFractional}--- to work with doubly radial solutions (and in particular saddle-shaped solutions) to a semilinear equation like \eqref{Eq:NonlocalAllenCahn} or its local version, is to use the radial variables 
%
%$$
%s := |x'| \quad \text{ and } \quad t:=|x''|\,.
%$$
%This is specially useful when dealing with the Laplacian, since the operator can be written very easily in these variables and the resulting PDE in $(0,+\infty)\times (0,+\infty)$ is suitable to work with. Recall that the Laplacian of a doubly radial function $w$ can be written as
%\begin{equation}
%	\label{Eq:Laplacian-st}
%	\Delta w = w_{ss} + w_{tt} + \frac{m-1}{s}w_s + \frac{m-1}{t}w_t\,.
%\end{equation}
% The same happens in the case of the fractional Laplacian thanks to the local extension problem, as done in \cite{Cinti-Saddle, Cinti-Saddle2, Felipe-Sanz-Perela:SaddleFractional}. Having a PDE in the two variables $(s,t)\in \R^2$ is useful to perform certain computations (see 99 for the local case and 99 for the nonlocal framework). 
 
 %For instance, the operator defined by \eqref{Eq:Laplacian-st} is not uniformly elliptic in $(0,+\infty)^2$. In addition, the set $\{st=0\}$ requires special treatment.
 
 If we try to follow the same strategy by writing a rotation invariant operator $L_K$ in $(s,t)$ variables, the expression of the new operator is quite complex. Indeed, if $w:\R^{2m} \to \R$ is doubly radial and we define $\widetilde{w}(s,t) := w(s,0,...,0,t,0,...,0)$, it holds
$$ L_Kw(x) = \widetilde{L}_K \widetilde{w} (|x'|,|x''|)$$
with
\begin{equation}
	\label{Eq:L_K-st}
	\widetilde{L}_K \widetilde{w} (s,t) := \int_0^{+\infty}  \int_0^{+\infty} \sigma^{m-1} \tau^{m-1} \big(\widetilde{w}(s,t) - \widetilde{w}(\sigma, \tau)\big) J(s,t,\sigma, \tau)  \d \sigma\d \tau
\end{equation}
and
\begin{align*}
J(s,t,\sigma, \tau) &:= \int_{\Sph^{m-1}}  \int_{\Sph^{m-1}} K\Big( \sqrt{s^2+\sigma^2- 2 s \sigma \omega_1 + t^2 + \tau^2 - 2t \tau\tilde\omega_1}\Big) \d \omega \d \tilde\omega\,.
\end{align*}

Note that $\widetilde{L}_K$ is an integro-differential operator in $(0,+\infty)\times(0,+\infty)$, but the expression of its kernel is quite involved. Indeed, such an expression does not become simpler even when $L_K$ is the fractional Laplacian. In this case, the kernel $J$ involves hypergeometric functions of two variables, the so-called Appell functions (see Appendix~\ref{Sec:stcomputations} for more details on it), but this does not simplify computations. 



Instead of working with the $(s,t)$ variables, we follow another approach that we find more clear and concise. It consists on rewriting the operator $L_K$ with a different kernel $\overline{K} : \R^{2m}\times \R^{2m} \to \R$ that is doubly radial with respect to its both arguments, but in such a way that it still acts on functions defined in $\R^{2m}$ ---and not in $(0,+\infty)^2$. As it is explained in detail in Section~\ref{Sec:OperatorOddF}, if $K$ is a radially symmetric kernel, then we can write $L_K$ acting on a doubly radial function $w$ as
\begin{equation}
\label{Eq:L_KWithKbar}
L_K w(x) = \int_{\R^{2m}} \{w(x) - w(y)\} \overline{K}(x,y) \d y\,,
\end{equation}
where $\overline{K} : \R^{2m}\times \R^{2m} \to \R$ is doubly radial in both arguments and is defined by
\begin{equation}
\label{Eq:KbarDef'}
\overline{K}(x,y) := \average_{O(m)^2} K(|Rx - y|)\d R\,.
\end{equation}
Here, $\d R$ denotes integration with respect to the Haar measure on $O(m)^2$ (see Section~\ref{Sec:OperatorOddF} for the details).

This new expression \eqref{Eq:L_KWithKbar} has some advantages compared with \eqref{Eq:L_K-st}. First, the computations in this new setting are shorter and more transparent than the analogous ones using $(s,t)$ variables. This also makes the notation more concise. Furthermore we avoid some issues of the $(s,t)$ variables such as the special treatment of the set $\{st=0\}$. Although in this paper we do not work in $(s,t)$ variables, we include an appendix at the end of the article with some computations using them (see Appendix~\ref{Sec:stcomputations}). We think that this could be useful in future works.  

%we are still dealing with functions defined in $\R^{2m}$. Hence, we avoid the previously mentioned issues that the $(s,t)$ variables have. Furthermore, the computations with this new expression are analogous to the ones that one would do using  $(s,t)$ variables. Last, the notation is more clear and concise. Although in this paper we do not work in $(s,t)$ variables, we include an appendix at the end of the article with some computations in these variables (see Appendix~\ref{Sec:stcomputations}). We think that this could be useful in future works.  
%The new kernel $\overline{K}$ is symmetric (as $K$), that is, $\overline{K}(x,y) = \overline{K}(y,x)$, but is not translation invariant. Nevertheless, it is doubly radial in its both variables, and this property is crucial to establish some maximum principles for odd functions.

Once we have rewritten $L_K$ with a doubly radial kernel $\overline{K}$, as in \eqref{Eq:L_KWithKbar}, we shall find a suitable expression of the operator when acting on odd functions with respect to the Simons cone. Note that such functions are defined by their values in $\ocal$ and therefore we want to rewrite $L_K$ taking this into account. To this purpose, we define the new operator
\begin{equation}
\label{Eq:OperatorOddF}
\begin{split}
L_K^\ocal w (x)  &:= \int_{\ocal} \{w(x) - w(y) \} \overline{K}(x, y) \d y +  \int_{\ocal} \{w(x) + w(y) \} \overline{K}(x, y^\star) \d y \\
&= \int_{\ocal} \{w(x) - w(y) \} \{\overline{K}(x, y) - \overline{K}(x, y^\star)  \} \d y +  2 w(x) \int_{\ocal} \overline{K}(x, y^\star) \d y \,,
\end{split}
\end{equation}
where $(\cdot)^\star$ is defined in \eqref{Eq:DefStar}. As we show in Section~\ref{Sec:OperatorOddF}, $L_K^\ocal$ acting on a doubly radial function $w:\ocal \to \R$ coincides with $L_K$ acting on the odd extension of $w$ with respect to the Simons cone.

Our first main result concerns necessary and sufficient conditions on the original kernel $K$ for this operator to have a positive kernel.  As we will stress through this paper, and also in the forthcoming work \cite{FelipeSanz-Perela:IntegroDifferentialII}, the positivity of the kernel in \eqref{Eq:OperatorOddF} is crucial in order to develop a theory on the saddle-shaped solution. In particular, under this assumption a maximum principle for doubly radial odd functions will hold (see Proposition~\ref{Prop:MaximumPrincipleForOddFunctions} below).

\begin{theorem}
	\label{Th:SufficientNecessaryConditions}
	Let $K:(0,+\infty) \to (0,+\infty)$ and consider the radially symmetric kernel $K(|x-y|)$ in $\R^{2m}$. Define $\overline{K} : \R^{2m}\times \R^{2m} \to \R$ by \eqref{Eq:KbarDef'}.
	
	If 
	\begin{equation}
	\label{Eq:SqrtConvex}	
	K(\sqrt{\tau}) \text{ is a strictly convex function of }\tau\,,
	\end{equation}
	then $L_K$ has a positive kernel in $\ocal$ when acting on doubly radial functions which are odd with respect to the Simons cone $\ccal$. More precisely, it holds
	\begin{equation}
	\label{Eq:KernelInequality}
	\overline{K}(x,y) > \overline{K}(x, y^\star) \quad \text{ for every }x,y \in \ocal\,.
	\end{equation}
	
	In addition, if $K\in C^2((0,+\infty))$, then \eqref{Eq:SqrtConvex} is not only a sufficient condition for \eqref{Eq:KernelInequality} to hold, but also a necessary one.
\end{theorem}

This theorem is proved in Section~\ref{Sec:OperatorOddF} (see Propositions~\ref{Prop:KernelInequalitySufficientCondition} and \ref{Prop:KernelInequalityNecessaryCondition}). Its proof is based on breaking the integral defining $\overline{K}$ in four clever regions ---see \eqref{Eq:DefQ}--- that allow to compare the integrands for $y\in \ocal$ and for its reflected $y^*\in \ical$. We will use a result on convex functions proved in Appendix~\ref{Sec:AuxiliaryResults} (Proposition~\ref{Prop:EquivalenceK(sqrt)Convex<->Inequality}). In the previous statement, by strict convexity in \eqref{Eq:SqrtConvex} we mean that
$$
K(\sqrt{\tau_1}) + K(\sqrt{\tau_2}) > 2 K(\sqrt{(\tau_1 + \tau_2)/2})
$$
for every $\tau_1$, $\tau_2 \in (0,+\infty)$.

In \cite{JarohsWeth}, Jarohs and Weth study solutions to general integro-differential equations which are odd with respect to a hyperplane. Here the natural sufficient condition on $K$ to have a positive kernel when acting on odd functions is that $K$ is decreasing in the orthogonal direction to the hyperplane. That this suffices is readily deduced after making a change of variables given by the symmetry with respect to such hyperplane. In our case, since we deal with a more complex symmetry, the kernel $K$ is required to satisfy further assumptions than just monotonicity. Moreover, the proof of Theorem~\ref{Th:SufficientNecessaryConditions} is quite involved and requires a finer argument. Indeed, if we simply make the change $y \mapsto y^\star$ in $\eqref{Eq:DefOfLu}$, following \cite{JarohsWeth}, we should prove that $K(|x-y|) > K(|x-y^\star|)$ for every $x$ and $y$ in $\ocal$, but this is false even in the easiest case $L_K = \fraclaplacian$ and $2m=2$. Instead, if we write $L_K$ in the form \eqref{Eq:L_KWithKbar} with the kernel $\overline{K}$, the analogous positivity condition \eqref{Eq:KernelInequality} holds if we assume $K(\sqrt{\cdot})$ to be convex. Here the use of the $(s,t)$ variables would not simplify the proof of Theorem~\ref{Th:SufficientNecessaryConditions}. As mentioned in Appendix~\ref{Sec:stcomputations}, an analogous result can be established for the kernel $J$ in \eqref{Eq:L_K-st}, but its proof presents exactly the same difficulties as the one for $\overline{K}$.

%Regarding the sufficient conditions on the kernel $K$, since we are dealing with a more complex symmetry than the one in \cite{JarohsWeth}, the kernel is required to satisfy further assumptions than just monotonicity. First, we assume that $K$ is radially symmetric to be able to write $L_K$ in the form \eqref{Eq:L_KWithKbar}. Moreover, we require the convexity condition \eqref{Eq:SqrtConvex}, which is stronger than the decreasing assumption in \cite{JarohsWeth} (note that if $K$ satisfies \eqref{Eq:Symmetry&IntegrabilityOfK}, an assumption that we make through the paper, then \eqref{Eq:SqrtConvex} yields that $K$ is radially decreasing). 

The first direct consequence of the positivity condition \eqref{Eq:KernelInequality} is the following maximum principle.

\begin{proposition}[Maximum principle for odd functions with respect to $\ccal$]
	\label{Prop:MaximumPrincipleForOddFunctions} Let $\Omega \subset \ocal$ be an open set and let $L_K$ be an integro-differential operator with a radially symmetric kernel $K$ satisfying the positivity condition \eqref{Eq:KernelInequality}.  Let $u\in C^{\alpha}(\Omega)\cap L^\infty(\R^{2m})$, with $\alpha > 2\s$, be a doubly radial function which is odd with respect to the Simons cone. 
	
	\begin{enumerate}[label=(\roman{*})]
		\item  (Weak maximum principle)
		Assume that
		$$
		\beqc{\PDEsystem}
		L_K u + c(x) u & \geq & 0 & \text{ in } \Omega\,,\\
		u & \geq & 0 & \text{ in } \ocal \setminus \Omega\,,
		\eeqc
		$$
		with $c \geq 0$, and that either
		$$
		\Omega \text{ is bounded} \quad \text{ or } \liminf_{x \in \ocal,\,|x|\to +\infty} u(x) \geq 0\,.
		$$
		Then, $u \geq 0$ in $\Omega$.
		
		\item (Strong maximum principle)  
		Assume that $L_K u + c(x) u\geq 0$ in $\Omega$, with $c$ any continuous function, and that $u\geq 0$ in $\ocal$. Then, either $u\equiv 0$ in $\ocal$ or $u > 0$ in $\Omega$.
	\end{enumerate} 
\end{proposition}

This statement differs from the usual maximum principle for $L_K$ in the fact that we only assume that $u$ is nonpositive in $\ocal\setminus \Omega$, instead of in $\R^{2m}\setminus \Omega$ (an assumption that makes no sense for odd functions). This form of maximum principle is analogous to the ones in \cite{ChenLiLi, JarohsWeth}, where similar statements are considered for functions that are odd with respect to a hyperplane.

Since in this paper we will always consider doubly radial functions $u$ which are odd with respect to the Simons cone, $L_K u =L_K^\ocal u$ in $\ocal$. Thus, to simplify the notation we will always write $L_K$ for $L_K^\ocal$. To mean that Proposition~\ref{Prop:MaximumPrincipleForOddFunctions} holds, we will say that $L_K$ has a maximum principle in $\ocal$ when acting on doubly radial odd functions.
%Some of our arguments to prove the results of this paper are inspired by the techniques developed in \cite{ChenLiLi, JarohsWeth}, where they establish some maximum principles for odd functions with respect to a hyperplane. Such maximum principles differ from the usual ones by the fact that an assumption on the sign of a function is needed only at one side of the hyperplane. Let us clarify this. In the usual maximum principle for nonlocal operators, one assumes that a function has a constant (and appropriate) sign in the whole complementary of the set where an equation is satisfied. When there is odd symmetry with respect to a hyperplane, such assumption cannot be done. However, by only assuming a constant (and appropriate) sign in one side of the hyperplane, a maximum principle can be deduced. In particular, if the kernel is decreasing, it is easy to prove a maximum principle of this type. 

%In our case, we find analogous maximum principles but with a more complex symmetry, and therefore the kernel is required to satisfy further assumptions. More precisely, we find that the assumption \eqref{Eq:SqrtConvex} below is a sufficient condition for the maximum principle for odd functions with respect to the Simons cone to hold. In order to see this, we  first need to use the symmetry of the functions to rewrite $L_K$ as an operator that only takes into account the values of the function in $\ocal$ (since the operator itself incorporates the odd symmetry). The new operator is the following:



%To simplify the notation, we introduce the following definition. It is the class of all operators which are radially symmetric, uniformly elliptic and have a positive kernel in $\ocal$ when acting on doubly radial functions which are odd with respect to the Simons cone.
%
%\begin{definition}[Ellipticity class $\lcal_\star$]
%	Let $L_K \in \lcal_0(2m,\s,\lambda, \Lambda)$ with kernel $K$ radially symmetric. We say that $L_K\in \lcal_\star (2m,\s,\lambda, \Lambda)$ whenever the associated kernel $\overline{K}$ satisfies \eqref{Eq:KernelInequality}.
%\end{definition}

Let us now turn to the variational problem from which equation \eqref{Eq:NonlocalAllenCahn} arises. As it is well known, \eqref{Eq:NonlocalAllenCahn} is the Euler-Lagrange equation associated to the energy functional
\begin{equation}
\label{Eq:Energy}
\begin{split}
\ecal(w, \Omega) &:= 
 \dfrac{1}{4} \left \{ \int_\Omega \int_\Omega |w(x) - w(y)|^2 K(x-y) \d x \d y \right. \qquad \qquad \\
& \quad \quad \quad +\left. 2 \int_\Omega \int_{\R^{2m} \setminus \Omega} |w(x) - w(y)|^2 K(x-y) \d x \d y \right \} + \int_{\Omega} G(w) \d x \,,
\end{split}
\end{equation}
where $G$ a $C^2$ function satisfying $G' = -f$. In this paper, we assume the following conditions on $G$:
\begin{equation}
\label{Eq:HipothesesG}
G \textrm{ is even and } G\geq G(\pm 1 )=0 \textrm{ in } \R\,.
\end{equation}
Note that the previous conditions on $G$ yield that $f$ is a $C^1$ odd function with $f(0)=f(\pm 1)=0$. In some cases, as in Theorem~\ref{Th:Existence} below, we will further assume that $G(0)>0$. In such situation, equation \eqref{Eq:NonlocalAllenCahn} can be seen as a model for phase transitions. The Allen-Cahn nonlinearity, $f(u) = u-u^3$, is the most typical example.


Using the same type of arguments as for the operator $L_K$, we can rewrite the energy of doubly radial odd functions with a suitable new expression that involves the kernel $\overline{K}$ and that only takes into account the values of the functions in $\ocal$. This will be extremely useful in many computations and estimates involving the nonlocal energy $\ecal$ (see Sections~\ref{Sec:EnergyForOddF} and \ref{Sec:EnergyEstimate}). To write this new expression, we introduce the following notation.  For $A$, $B\subset \ocal$, two sets of double revolution, we define
\begin{equation*}
%\label{Eq:DefIw}
\begin{split}
I_w(A,B) := 2\int_A  \int_B  \ |w(x)-w(y)|^2 \left\{ \overline{K}(x,y) - \overline{K}(x,y^\star) \right\} \d x \d y  \\
+\, 4 \int_A  \int_B  \left\{w^2(x)+w^2(y)\right\} \overline{K}(x,y^\star) \d x \d y\,.
\end{split}
\end{equation*}
Then, as proved in Section~\ref{Sec:EnergyForOddF} (see Lemma~\ref{Lemma:ShortExpressionEnergy}), we can rewrite the energy of a doubly radial odd function $w$ as
\begin{equation}
\label{Eq:ShortExpressionEnergyIntro}
\ecal(w, \Omega) = \frac{1}{4} \big \{I_w(\Omega\cap\ocal,\Omega\cap\ocal) +  2I_w(\Omega\cap\ocal,\ocal\setminus\Omega) \big \} + 2\int_{\Omega\cap \ocal} G(w) \d x \,.
\end{equation}



Thanks to this new expression for the energy, we are able to establish the second main result of this paper. It is the following energy estimate for doubly radial odd minimizers of $\ecal$. To define such minimizers properly, we denote by $\widetilde{\H}^K_{0, \mathrm{odd}}(B_R)$ the space of doubly radial odd functions that vanish outside $B_R$ and for which the energy $\ecal$ is well defined (see Section~\ref{Sec:EnergyForOddF} for the precise definition). Then, we say that $u\in \widetilde{\H}^K_{0, \mathrm{odd}}(B_R)$ is a doubly radial odd minimizers of $\ecal$ in $B_R$ if
$$
\ecal(u,B_R) \leq \ecal (w,B_R)
$$
for every $w\in \widetilde{\H}^K_{0, \mathrm{odd}}(B_R)$. 

\begin{theorem}
	\label{Th:EnergyEstimate} 
	Let $K$ be a radially symmetric kernel satisfying the positivity condition \eqref{Eq:KernelInequality} and such that $L_K\in \lcal_0(2m, \s, \lambda, \Lambda)$. Assume that $G$ is a potential satisfying \eqref{Eq:HipothesesG}. Let $S>2$ and let $u\in \widetilde{\H}^K_{0, \mathrm{odd}}(B_R)$ be a doubly radial odd minimizer of $\ecal$ in $B_R$, with $R>S+5$. Then
	%$$ \lim_{R\to +\infty} \frac{1}{S^n} \ecal (u,B_S) = 0. $$
	%More precisely,
	$$ \ecal (u,B_S) \leq \begin{cases}
	C \ S^{2m-2\s}\ \ \ \ &\textrm{if } \ \ \s\in(0,1/2),\\
	C\ \log(S)\,S^{2m-1}\ \ \ \ &\textrm{if } \ \ \s=1/2,\\
	C \ S^{2m-1}\ \ \ \ &\textrm{if } \ \ \s\in(1/2,1),\\
	\end{cases} $$
	where $C$ is a positive constant depending only on $m$, $\s$, $\lambda$, $\Lambda$, and $\norm{G}_\infty$.
\end{theorem}

Note that this result does not follow from the energy estimate for general minimizers 
%in $H^\s_0(B_R)$ 
stated in \cite{SavinValdinoci-EnergyEstimate} by Savin and Valdinoci. The minimizers that they consider do not have any type of symmetry. In our case, the function $u$ in the previous statement minimizes the energy in a smaller class of functions and the result in \cite{SavinValdinoci-EnergyEstimate} cannot be applied. Nevertheless, we are able to adapt the arguments of Savin and Valdinoci to our setting.  The strategy they follow is to compare the energy of $u$ with the one of a suitable competitor which is constructed by taking the minimum between $u$ and a radially symmetric auxiliary function ---see \eqref{Eq:DefPhiS} below. Such competitor is not permitted in our case, since it is not odd with respect to the Simons cone. Nevertheless, we show in Section~\ref{Sec:EnergyForOddF} how to modify the auxiliary functions of \cite{SavinValdinoci-EnergyEstimate} to carry out the same type of arguments. In the proof, the assumption \eqref{Eq:KernelInequality} is crucial.

The particular result of Theorem~\ref{Th:EnergyEstimate} for the fractional Laplacian has been proved by Cabré and Cinti \cite{CabreCinti-EnergyHalfL} in the case of the half-Laplacian, and extended to all the powers $0<\s<1$ by Cinti \cite{Cinti-Saddle2} (see \cite{CabreCinti-SharpEnergy} for an extension to non-doubly radial minimizers). These papers use the local extension problem and therefore their proofs cannot be extended to general operators like $L_K$. Our proof, following \cite{SavinValdinoci-EnergyEstimate}, overcomes this issue.


As an application of the previous results, we prove, by using standard variational methods, the existence of saddle-shaped solution to \eqref{Eq:NonlocalAllenCahn} when $f$ is of Allen-Cahn type. We say that a bounded solution $u$ to \eqref{Eq:NonlocalAllenCahn} is a \emph{saddle-shaped} solution if $u$ is doubly radial, odd with respect to the Simons cone, and positive in $\ocal$. 

\begin{theorem}[Existence of saddle-shaped solution]
	\label{Th:Existence}
    Let $G$ satisfy \eqref{Eq:HipothesesG}, $G(0)>0$, and let $f=-G'$. Let $K$ be a radially symmetric kernel satisfying the positivity condition \eqref{Eq:KernelInequality} and such that $L_K\in \lcal_0(2m, \s, \lambda, \Lambda)$. 
    
    Then, for every even dimension $2m \geq 2$, there exists a saddle-shaped solution $u$ to \eqref{Eq:NonlocalAllenCahn}. In addition, $u$ satisfies $|u|<1$ in $\R^{2m}$.
\end{theorem}

We are interested in the study of this type of solutions since they are relevant in connection with a famous conjecture for the (classical) Allen-Cahn equation raised by De Giorgi, that reads as follows. Let $u$ be a bounded monotone (in some direction) solution to $-\Delta u = u - u^3$ in $\R^n$, then, if $n \leq 8$, $u$ depends only on one Euclidean variable, that is, all its level sets are hyperplanes. This conjecture is not completely closed (see \cite{FarinaValdinoci-DeGiorgi} and references therein) but a counterexample in dimension $n=9$ was build in \cite{delPinoKowalczykWei} by using the so-called gluing method. Saddle-shaped solutions are natural objects to build a counterexample in a simpler way, as explained next. On the one hand, Jerison and Monneau \cite{JerisonMonneau} showed that a counterexample to the conjecture of De Giorgi in $\R^{n+1}$ can be constructed with a rather natural procedure if there exists a global minimizer of $-\Delta u = f(u)$ in $\R^n$ which is bounded and even with respect to each coordinate, but is not one-dimensional. On the other hand, by the $\Gamma$-converge results from Modica and Mortola (see \cite{Modica,ModicaMortola}) and the fact that the Simons cone is the simplest nonplanar minimizing minimal surface, saddle-shaped solutions are expected to be global minimizers of the Allen-Cahn equation in dimensions $2m\geq 8$ (this is still an open problem).

Similar facts happen in the nonlocal setting (see the introduction of \cite{Felipe-Sanz-Perela:SaddleFractional} for further details). For this reason, saddle-shaped solutions are of interest in the study of the nonlocal version of the conjecture of De Giorgi for equation \eqref{Eq:NonlocalAllenCahn}.

Saddle-shaped solutions to the local Allen-Cahn equation involving the Laplacian were studied in \cite{DangFifePeletier, Schatzman, CabreTerraI,CabreTerraII, Cabre-Saddle}. In these works, it is established the existence, uniqueness, and some qualitative properties of this type of solutions, such as their instability when $2m\leq 6$ and their stability if $2m\geq 14$. Stability in dimensions $8, 10$, and $12$ is still an open problem, as well as minimality in dimensions $2m\geq 8$.

In the fractional framework, there are only three works concerning saddle-shaped solutions to the equation $\fraclaplacian u = f(u)$. In \cite{Cinti-Saddle,Cinti-Saddle2}, Cinti proved the existence of saddle-shaped solution as well as some qualitative properties such as their asymptotic behavior, some monotonicity properties, and their instability in low dimensions. In a previous paper by the authors \cite{Felipe-Sanz-Perela:SaddleFractional}, further properties of these solutions have been established, the main ones being uniqueness and, when $2m\geq 14$, stability. The present paper together with its second part \cite{FelipeSanz-Perela:IntegroDifferentialII} are the first ones studying saddle-shaped solutions for general integro-differential equations of the form \eqref{Eq:NonlocalAllenCahn}. In the three previous papers \cite{Cinti-Saddle, Cinti-Saddle2, Felipe-Sanz-Perela:SaddleFractional}, the main tool used is the extension problem for the fractional Laplacian (see \cite{CaffarelliSilvestre}). As mentioned, this technique cannot be carried out for general integro-differential operators different from the fractional Laplacian. Therefore, some purely nonlocal techniques are developed through both papers.

In the forthcoming paper \cite{FelipeSanz-Perela:IntegroDifferentialII}, we study saddle-shaped solutions to \eqref{Eq:NonlocalAllenCahn} in more detail taking advantage of the setting for odd functions built in the present article. We give an alternative proof for the existence of a saddle-shaped solution by using monotone iteration and maximum principle techniques. As in the proof of Theorem~\ref{Th:Existence}, the assumtion \eqref{Eq:KernelInequality} is crucial. Furthermore, we prove the asymptotic behaviour of this type of solutions by using some symmetry and Liouville type results for general integro-differential operators that we establish in the same paper. Finally, we also show in \cite{FelipeSanz-Perela:IntegroDifferentialII} the uniqueness of the saddle-shaped solution through a maximum principle for the linearized operator, which we also prove in that article.

Let us make some final remarks on the minimality and stability properties of the Simons cone. Recall that, in the classical theory of minimal surfaces, it is well known that the Simons cone has zero mean curvature at every point $x\in \ccal \setminus \{0\}$, in all even dimensions, and it is a minimizer of the perimeter functional when $2m\geq 8$. Concerning the nonlocal setting, $\ccal$ has also zero nonlocal mean curvature in all even dimensions, although it is not known if it is a minimizer of the nonlocal perimeter in any dimension. If $2m=2$ it cannot be a minimizer since in \cite{SavinValdinoci-Cones} it is proven that all minimizing nonlocal minimal cones in $\R^2$ are flat. In higher dimensions, the only available results appear in \cite{DaviladelPinoWei, Felipe-Sanz-Perela:SaddleFractional} but concern stability, a weaker property than minimality. In \cite{DaviladelPinoWei},  Dávila, del Pino, and Wei characterize the stability of Lawson cones ---a more general class of cones that includes $\ccal$--- through an inequality involving only two hypergeometric constants which depend only on $\s$ and the dimension $n$. This inequality is checked numerically in \cite{DaviladelPinoWei}, finding that, in dimensions $n \leq 6$ and for $\s$ close to zero, no Lawson cone with zero nonlocal mean curvature is stable. Numerics also shows that all Lawson cones in dimension $7$ are stable if $\s$ is close to zero. These results for small $\s$ fit with the general belief that, in the fractional setting, the Simons cone should be stable (and even a minimizer) in dimensions $2m \geq 8$ (as in the local case), probably for all $\s\in(0,1/2)$, though this is still an open problem. In \cite{Felipe-Sanz-Perela:SaddleFractional}, we proved, by using the saddle-shaped solution to the fractional Allen-Cahn equation and a $\Gamma$-convergence result of \cite{CabreCintiSerra-Stable}, that the Simons cone is a stable $(2\s)$-minimal cone in dimensions $2m\geq 14$. To the best of our knowledge, this is the first analytical proof of a stability result for the Simons cone in any dimension.



This paper is organized as follows. Section~\ref{Sec:OperatorOddF} is devoted to study the operator $L_K$ acting on doubly radial odd functions. We deduce the expression of the kernel $\overline{K}$ and rewrite the operator acting on doubly radial odd functions, finding the expression \eqref{Eq:OperatorOddF}. We also show Theorem~\ref{Th:SufficientNecessaryConditions} and Proposition~\ref{Prop:MaximumPrincipleForOddFunctions}. In Section~\ref{Sec:EnergyForOddF} we study the energy functional associated to \eqref{Eq:NonlocalAllenCahn} and in Section~\ref{Sec:EnergyEstimate} we establish the energy estimate stated in Theorem~\ref{Th:EnergyEstimate}. Finally, in Section~\ref{Sec:Existence} we prove the existence of a saddle-shaped solution to the integro-differential Allen-Cahn equation. At the end of the paper there are three appendices. Appendix~\ref{Sec:AuxiliaryResults} is devoted to some results on convex functions, and Appendix~\ref{Sec:AuxiliaryResults2} contains some auxiliary computations. Both are used in the proof of Theorem~\ref{Th:SufficientNecessaryConditions}. In Appendix~\ref{Sec:stcomputations} we include some results and expressions in $(s,t)$ variables for future reference.