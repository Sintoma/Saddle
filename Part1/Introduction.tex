%%%%%%%%%%%%%%%%%%%%%%%
\section{Introduction}
%%%%%%%%%%%%%%%%%%%%%%%
\label{Sec:Introduction}


We study the equation
\begin{equation}
\label{Eq:NonlocalAllenCahn}
Lu = f(u) \quad \textrm{ in } \R^{2m}
\end{equation}
where $f$ is of bistable type and $L$ is an integro-differential operator of the form
\begin{equation}
\label{Eq:DefOfLu}
Lu(x) = \PV \int_{\R^n} \{u(x) - u(y)\} K(x-y)\d y
\end{equation}
where $K$ is a  kernel satisfying
\begin{equation}
\label{Eq:Symmetry&IntegrabilityOfK}
K\geq 0\,, \quad K(y) = K(-y) \quad \textrm{ and } \quad \int_{\R^n} \min \left\{ |y|^2, 1 \right\} K(y) \d y < + \infty\,.
\end{equation}
The most canonical example of such operators is the fractional Laplacian
$$
\fraclaplacian u = \PV c_{n, \s}\int_{\R^n} \dfrac{u(x) - u(y)}{|x-y|^{n + 2\s}}\d y\,,
$$
where $c_{n, \s}$ is a normalizing constant (for its exact value see for instance \cite{HitchhikerGuide}).\todo{Mirar a ver si esta es una buena referencia o se pone otra}

Throughout the paper, we assume that the operators of our study are uniformly elliptic in the sense of the following definition.
\begin{definition}
\label{Def:L_0Class}
We say that an operator $L$ of the form \eqref{Eq:DefOfLu} belongs to the class $\lcal_0(n,\s,\lambda, \Lambda)$ if its kernel $K$ satisfies \eqref{Eq:Symmetry&IntegrabilityOfK} and \begin{equation}
\label{Eq:Ellipticity}
c_{n,\s}\dfrac{\lambda}{|y|^{n+2\s}} \leq K(y) \leq c_{n,\s}\dfrac{\Lambda}{|y|^{n+2\s}}\,, \quad 0< \lambda \leq \Lambda \,,
\end{equation}
where $c_{n,\s}$ is the constant appearing in the definition of the fractional Laplacian. We will say that $L$ belongs to the ellipticity class $\Lr(n,\s,\lambda, \Lambda)$ if $L\in \lcal_0(n,\s,\lambda, \Lambda)$ and its kernel is radially symmetric, i.e., $K(y)=K(|y|)$.
\end{definition}
For short we will usually write $\lcal_0$ or $\Lr$, and we will make explicit the parameters only when needed.



\bigskip
[...]
\bigskip

Given $f$ a $C^1$ nonlinearity, we define
$$
G(u)= \int_u^1 f(t) \d t\,.
$$
We have that $G$ is a $C^2$ function satisfying $G' = -f$. In this paper, we assume some, or all, of the following conditions on $f$.
\begin{equation}
\label{Eq:HipothesisfOdd}
f \textrm{ is odd;}
\end{equation}
\begin{equation}
\label{Eq:HipothesisGWells}
G\geq 0 \quad \textrm{ in } \R, \quad G > 0 \ \textrm{ in }  (-1,1), \quad \textrm{ and }\quad G(\pm 1 )=0\,;
\end{equation}
\begin{equation}
\label{Eq:HipothesisfConcave}
f \textrm{ is concave in }  (0,1).
\end{equation}



Note that \eqref{Eq:HipothesisfOdd} and \eqref{Eq:HipothesisGWells} yield that $f(0)=f(\pm 1)=0$. 

Note that \eqref{Eq:HipothesisfOdd} is equivalent to say that $G$ is even.

Note that \eqref{Eq:HipothesisfOdd}, \eqref{Eq:HipothesisGWells}, and \eqref{Eq:HipothesisfConcave} and the fact that $f(1)=0$ yield $f'(0)>0$ and $f'(\pm 1) < 0$. As a consequence, $f > 0$ in $(0,1)$.

Comentario: $f'(\pm 1) < 0$ equivale a $G''(\pm 1) > 0\,,$ que es la hipotesis junto con las otras para que exista el Layer

Note that, since $f$ is concave in $(0,1)$ and $f(0)=0$, then 
\begin{equation}
\label{Eq:PropertyConcavityf}
f'(t)t \leq f(t) \quad \textrm{ for all } t\in (0,1)\,.
\end{equation}
The inequality is strict if we have strict concavity.


\bigskip
\bigskip
\bigskip
-------------
\bigskip
\bigskip
\bigskip

[...]

The interest on this problem originates from the famous De Giorgi conjecture for the classical Allen-Cahn equation.

\begin{conjecture}[De Giorgi, 1978]
	Let $u$ be a bounded solution of the Allen-Cahn equation
	\begin{equation}
	\label{Eq:LocalAllenCahn}
	-\Delta  u = u - u^3 \quad \text{ in } \R^n
	\end{equation}
	such that it is monotone in one direction, say $\partial_{x_n} u > 0$. Then, if $n\leq 8$, $u$ is one dimensional, i.e., $u$ depends only on one Euclidean variable.
\end{conjecture}

This conjecture was proved true in dimension $n=2$ by Ghoussoub and Gui in \cite{GhoussoubGui}, and in dimension $n=3$ by Ambrosio and Cabré in \cite{AmbrosioCabre}. For dimensions $4\leq n \leq 8$, it was established by Savin in \cite{Savin-DeGiorgi} but with the extra assumption of
\begin{equation}
\label{Eq:SavinCondition}
	\lim_{x_n \to \pm \infty} u(x',x_n) = \pm 1 \quad \text{ for all } x'\in \R^{n-1}\,.
\end{equation}
A counterexample to the conjecture was given by del~Pino, Kowalczyk and Wei in \cite{delPinoKowalczykWei}. 


One can also formulate the same conjecture in the nonlocal setting by changing the Laplacian by a nonlocal operator, the most canonical one being $\fraclaplacian$. In this framework, for the equation $\fraclaplacian u = f(u)$ in $\R^n$ and $\s\in (0,1)$, the conjecture has been proven to be true in dimension $n=2$ by Cabré and Solà-Morales in \cite{CabreSolaMorales} for $\s=1/2$ and extended to every power $0<\s<1$ by Cabré and Sire in \cite{CabreSireI} and also by Sire and Valdinoci in \cite{SireValdinoci}. In dimension $3$ the conjecture has been proved by Cabré and Cinti for $1/2 \leq \s < 1$ in \cite{CabreCinti-EnergyHalfL, CabreCinti-SharpEnergy}. Recently, in \cite{Savin-Fractional} Savin has established the validity of the conjecture in dimensions $4\leq n \leq 8$ and for $1/2 < \s < 1$, but assuming the condition \eqref{Eq:SavinCondition}. In that paper he has also announced that the same holds for $\s=1/2$. The case $0<\s<1/2$ has been also treated by Dipierro, Serra and Valdinoci in \cite{DipierroSerraValdinoci} \todo{Revisar, dependeria de la clasificación de los conos}. The most recent result related to the conjecture is the one of Figalli and Serra in \cite{FigalliSerra}, where they have proven the conjecture in dimension $n=4$ and $\s=1/2$. Note that this is the only result that is available (by now) exclusively in the nonlocal setting and not for the Laplacian.

After all the years of study of the conjecture raised by De Giorgi, another question appeared naturally: do global minimizers of the energy associated to the equation  $-\Delta u = f(u)$ in $\R^n$ have one-dimensional symmetry? A deep result from Savin \cite{Savin-DeGiorgi} is that in dimension $n \leq 7$ this is indeed true, and the conjecture is that for $n\geq 8$ is false. The answer to this question would provide some insights of the original conjecture of De Giorgi. This is due to a result by Jerison and Monneau in \cite{JerisonMonneau}, where they show that a counterexample of the original conjecture in $\R^{n+1}$ can be constructed from a bounded, even with respect to each coordinate, global minimizer of $-\Delta u = f(u)$ in $\R^n$. Hence, finding a global minimizer that is not one-dimensional would give a natural counterexample to the original conjecture.

Saddle-shaped solutions are of special interest in the search for this counterexample. To define these solutions properly, we need to introduce some notation and definition.

First of all, recall that the Simons cone is defined in $\R^{2m}$ with $2m = n$ by
\begin{equation}
\label{Eq:SimonsCone}
	\mathscr{C} = \setcond{x = (x', x'') \in \R^{2m}}{|x'| = |x''|}\,.
\end{equation}
The Simons cone is proven to be a (classical) stationary minimal surface. Moreover, if $n\geq 8$, is also minimizing (see 99 and the coments... \todo{comentar algo despues con aquello del límite?}). 
Through the paper we will also use the letters $\ocal$ and $\ical$ to denote the outside and inside of the cone:
\begin{equation}
\label{Eq:DefOandI}
\ocal:= \setcond{x = (x', x'') \in \R^{2m}}{|x'| > |x''|} \ \textrm{ and } \
\ical:= \setcond{x = (x', x'') \in \R^{2m}}{|x'| < |x''|}.
\end{equation}


Let $SO(m)$ be the special orthogonal group of $\R^m$, that is, the group of rotations of $\R^m$. We will work with the group $SO(m)^2 = SO(m) \times SO(m)$. Note that $SO(m)^2 \subset SO(2m)$ and therefore, for any $R\in SO(m)^2$, $|Rx| = |x|$. Moreover, the sets $\ocal$ and $\ical$ are invariant under the action of the group and belong to a more general class of domains defined next.

\begin{definition}
\label{Def:DoubleRevolutionSet}
We say that a set $\Omega\subset \R^{2m}$ is of \emph{double revolution} if it is invariant under $SO(m)^2$, i.e., if it is invariant under orthogonal transformations in the first $m$ variables and also under orthogonal transformations in the last $m$ variables.
\end{definition}

More defs (introducir mejor)

\begin{definition}
\label{Def:DoublyRadial}
We say that a function $w:\R^{2m} \to \R$ is \emph{doubly radial} if it only depends on the modulus of the first $m$ variables and on the modulus of the last $m$ ones, i.e., $w(x) = w(|x'|,|x''|)$. Equivalently, if $w(Rx) = w(x)$ for every $R \in SO(m)^2$.
\end{definition}

Through the paper we will consider the following isometry:
\begin{equation}
\label{Eq:DefStar}
\begin{matrix}
(\cdot)^\star \colon & \R^{2m}= \R^{m}\times \R^{m}  &\to&  \R^{2m}= \R^{m}\times \R^{m}  \\
	& x = (x',x'') &\mapsto & x^\star = (x'',x')\,.
\end{matrix}
\end{equation}
Note that this isometry satisfies
\begin{enumerate}
\item $((\cdot)^\star)^{-1} = (\cdot)^\star$.
\item $\ocal^\star= \ical$ and $\ical^\star = \ocal$.
\end{enumerate}


\begin{definition}
\label{Def:OddwrtSimonsCone}
We say that a doubly radial function $w$ is \emph{odd with respect to the Simons cone} if $w(|x'|,|x''|) = -w(|x''|,|x'|)$ for every $x = (x', x'') \in \R^{2m}$, or equivalently, if $w(x) = -w(x^\star)$. Similarly, we say that a doubly radial function $w$ is \emph{even with respect to the Simons cone} if $w(|x'|,|x''|) = w(|x''|,|x'|)$ for every $x = (x', x'') \in \R^{2m}$, or equivalently, if $w(x) = w(x^\star)$.
\end{definition}

With these definitions in hand, we can define now properly saddle-shaped solutions:
\begin{definition}
\label{Def:SaddleShapedSol}
We say that $u$ is a \emph{saddle-shaped solution} (or simply \emph{saddle solution}) of \eqref{Eq:NonlocalAllenCahn} if
\begin{enumerate}
\item $u$ is doubly radial.
\item $u$ is odd with respect to the Simons cone.
\item $u > 0$ in $\ocal$.
\end{enumerate}
\end{definition}


Note that these solutions are even with respect to the coordinate axis and that their zero level set is the Simons cone $\mathscr{C} = \{|x'|=|x''|\}$. Therefore, saddle-shaped solutions are candidates to build a counterexample of the De Giorgi conjecture in high dimensions, since if one could prove that they are global minimizers in $\R^8$, by the result in \cite{JerisonMonneau} one would have a counterexample of the De Giorgi conjecture in $\R^9$ (as an alternative to that of \cite{delPinoKowalczykWei}).

Saddle-shaped solutions for the classical equation with the Laplacian were first studied by Dang, Fife, and Peletier in \cite{DangFifePeletier} in dimension $2m=2$. They established the existence and uniqueness of this type of solutions, as well as some monotonicity properties and asymptotic behavior. In \cite{Schatzman}, Schatzman studied the instability property of saddle solutions in $\R^2$. In higher even dimensions, Cabré and Terra  proved the existence of a saddle solution in every dimension $2m\geq 2$ and they established also some qualitative properties such as monotonicity properties, asymptotic behavior, as well as instability in dimensions $2m = 4$ and $2m = 6$ (see \cite{CabreTerraI,CabreTerraII}). The uniqueness in dimensions higher than $2$ was established by Cabré in \cite{Cabre-Saddle}, where he also proved that saddle solutions are stable (see the definition below\todo{ver si lo quitamos o como lo ponemos}) in dimensions $2m \geq 14$.

In the nonlocal framework, there are some works concerning saddle-shaped solutions to \eqref{Eq:NonlocalAllenCahn} with $L= \fraclaplacian$. In  \cite{Cinti-Saddle}, Cinti established the existence of saddle-shaped solutions to $(-\Delta)^{1/2}u = f(u)$ in $\R^{2m}$, as well as some qualitative properties such as asymptotic behavior, monotonicity properties, and instability in dimensions $2m = 4$ and $2m = 6$ (instability in dimension $2m=2$ follows by a result of Cabré and Solà-Morales in \cite{CabreSolaMorales}). More recently, she has extended the same results to all $\s \in (0,1)$ in 99\todo{Citar paper de Cinti}.


To the best of our knowledge, there are no more works studying the saddle-shaped solutions in the nonlocal setting. Moreover, the problem has not been studied for general operators $L\in \lcal_0$. Regarding the nonlocal Allen-Cahn equation with general kernels, we have to mention two works.

The first one is \cite{CozziPassalacqua}, where Cozzi and Passalacqua study layer solutions to the equation \eqref{Eq:NonlocalAllenCahn}.

We should also mention \cite{DipierroSerraValdinoci}\todo{CHECK} where they study nonlocal minimal surfaces with general kernels and blablabla.mal surfaces with general kernels and blablabla.



\bigskip
\bigskip
\bigskip
-------------
\bigskip
\bigskip
\bigskip

The usual strategy to deal with doubly radial solutions (and in particular saddle-shaped solutions) to a semilinear equation like \eqref{Eq:NonlocalAllenCahn} is to work with the radial variables 
$$
s = |x'| \quad \text{ and } \quad t=|x''|\,.
$$
This is specially useful when dealing with the Laplacian, since the operator can be written very easily in these coordinates and then the resulting PDE in $(0,+\infty)\times (0,+\infty)$ is suitable to work with. The same happens in the case of the fractional Laplacian thanks to the local extension problem. When we try to follow the same strategy by writing a general operator such as $L_K$ in $(s,t)$ variables, the expression of the new operator is more complex (see Appendix~\ref{Sec:stcomputations}). Despite the fact that all computations can be done in these radial variables, the notation becomes cumbersome. For this reason, we follow a different approach that consists of rewriting the operator $L_K$ without any change of coordinates but with a different kernel that is doubly radial. As it is explained with more detail in Section~\ref{Sec:Preliminaries}, if $K$ is a radially symmetric kernel, then we find the following expression:
$$
L_K w(x) = \int_{\R^{2m}} \{w(x) - w(y)\} \overline{K}(x,y) \d y
$$
where $\overline{K}$ is doubly radial in both variables and defined by
\begin{equation}
\label{Eq:KbarDef'}
\overline{K}(x,y) := \average_{O(m)^2} K(|Rx - y|)\d R\,.
\end{equation}
Here, $\d R$ denotes integration with respect to the Haar measure on $O(m)^2$ (see Section~\ref{Sec:Preliminaries} for the details).


One of them main ingredients needed in our proofs (mostly in \cite{FelipeSanz-Perela:IntegroDifferentialII}) is a maximum principle for this operator, but for odd functions with respect to the cone $\ccal$. The classical maximum principle holds for the operator $L_K$ thanks to the positivity of the kernel and reads as follows. For $\Omega \subset \R^n$, if $L_K w \geq 0$ in $\Omega$ and $w \geq 0$ in $\R^n \setminus \Omega$, then $w\geq 0$ in $\Omega$. Such a statement is not suitable for odd functions in general (note that if $w$ is odd, so is $L_K w$ and therefore it makes more sense to assume $L_K w \geq 0$ in a subset  at one side of the cone \todo{Mejorar}). For this reason, we need to use the symmetry of the functions to rewrite the operator taking only into account ``what happens'' in $\ocal$. Using the change of variables given by $(\cdot)^\star$ ---defined in \eqref{Eq:DefStar}---, we find that $L_K$ acting on an odd function $w$ corresponds to apply the following operator to $w$. 
\begin{equation}
	L_K' w (x) := \int_{\ocal} \{w(x) - w(y) \} \{\overline{K}(x, y) - \overline{K}(x, y^\star)  \} \d y +  2 w(x) \int_{\ocal} \overline{K}(x, y^\star) \d y \,.
\end{equation}


Another point of view is considering the previous expression as an operator acting on doubly radial functions $w$ defined only on $\ocal$. Then, $L_K'$ corresponds to consider the odd extension of  $w$ with respect to the Simons cone and apply the operator $L_K$ to this extended function. Since in this article we will always consider odd functions, we will do an abuse of notation and we will denote both operators by $L_K$, since they are the same.

Note that this last expression has an integro-differental term plus a zero order term with the good sign. Thus, the natural assumption to make for that operator to have a maximum principle is that its ``kernel'' is positive. That is, $\overline{K}(x, y) - \overline{K}(x, y^\star)>0$. Indeed, we show in Section~\ref{Sec:Preliminaries} that this assumption guarantees that $L_K$ has a maximum principle for odd functions. The previous positivity assumption motivates the following definition.

\begin{definition}
	Let $L_K \in \Lr(2m,\s,\lambda, \Lambda)$ be an integro-differential operator with a radially symmetric kernel $K$. We say that $L_K\in \lcal_\star$ whenever the associated kernel $\overline{K}$ satisfies
	\begin{equation}
		\label{Eq:KernelInequality}
		\overline{K}(x,y) > \overline{K}(x, y^\star) \quad \text{ for every }x,y \in \ocal\,.
	\end{equation}
\end{definition}

Our first main result is a partial characterization of the kernels corresponding to the operators in the class $\lcal_\star$.

\begin{theorem}
	\label{Th:CharacterizationLstar}
	Let $L_K \in \Lr(2m,\s,\lambda, \Lambda)$ and assume that 
	\begin{equation}
		\label{Eq:SqrtConvex}	
		K(\sqrt{\tau}) \text{ is a convex function of }\tau\,.
	\end{equation}
	Then, $L_K\in \lcal_\star$. Moreover, if $K\in C^1((0,+\infty))$, then \eqref{Eq:SqrtConvex} is a necessary condition for $L_K$ to belong to $\lcal_\star$.
\end{theorem}

This theorem is proved in Section~\ref{Sec:Preliminaries} (see Propositions~\ref{Prop:KernelInequalityReflexion} and \ref{Prop:ContraryKernelInequalityReflexion}). It is based on a suitable division of the space $O(m)^2$ and a result on convex functions proved in the Appendix~\ref{Sec:AuxiliaryResults} (Proposition~\ref{Prop:EquivalenceK(sqrt)Convex<->Inequality}).



\todo[inline]{conectar}

The energy functional associated to \eqref{Eq:NonlocalAllenCahn} is the following.
\begin{equation}
\label{Eq:Energy}
\ecal(w, \Omega) := \dfrac{1}{4}\int\int_{\R^{2n} \setminus (\R^n\setminus\Omega)^2} |w(x) - w(y)|^2 K(x-y) \d x \d y + \int_{\Omega} G(w)\,.
\end{equation}
Using the same type of arguments as for the operator $L_K$, we can rewrite the energy of doubly radial and odd functions with a suitable expression with the kernel $\overline{K}$ (see Section~\ref{Sec:Nonlocal_AllenCahn_Energy}). Thanks to this new expression we are able to establish the second main result of this paper. It is the following energy estimate for doubly radial and odd minimizers of $\ecal$. In the next statement, $\widetilde{\H}^K_{0, \mathrm{odd}}(B_R)$ denotes the space of doubly radial and odd functions that vanish outside $B_R$ and for which the energy $\ecal$ is well defined (see Section~\ref{Sec:Nonlocal_AllenCahn_Energy} for the precise definition).

\begin{theorem}
	\label{Th:EnergyEstimate} 
	Let $K$ be a kernel such that $L_K\in \lcal_\star(2m, \s, \lambda, \Lambda)$. Let $S>0$ and let $u$ be a minimizer of the energy $\ecal$ in $B_{R}$, with $R>S+2$, among functions that are in $\widetilde{\H}^K_{0, \mathrm{odd}}(B_R)$. \todo{Cuando se define esto..}Then
	%$$ \lim_{R\to +\infty} \frac{1}{S^n} \ecal (u,B_S) = 0. $$
	%More precisely,
	$$ \ecal (u,B_S) \leq \begin{cases}
	C \ S^{2m-2\s}\ \ \ \ &\textrm{if } \ \ \s\in(0,1/2),\\
	C\ \log(S)\,S^{2m-2\s}\ \ \ \ &\textrm{if } \ \ \s=1/2,\\
	C \ S^{2m-1}\ \ \ \ &\textrm{if } \ \ \s\in(1/2,1),\\
	\end{cases} $$
	with $C$ a positive constant depending only on $m$, $\s$, $\Lambda$ and $G$.
\end{theorem}



This result has been proved in the case of the fractional Laplacian by Cinti \cite{Cinti-Saddle,Cinti-Saddle2}, but using the local extension problem. In our case, since this technique is not available, we follow the proof of Savin and Valdinoci in \cite{SavinValdinoci-EnergyEstimate}, where they prove a similar energy estimate for minimizers without any symmetry. The strategy to establish the result is to compare the energy of $u$ with the energy of a suitable competitor built after ``cutting'' $u$ with some radial functions. Such competitor is not permitted in our case since it is not odd with respect to the Simons cone. In Section~\ref{Sec:Nonlocal_AllenCahn_Energy} we show how to adapt the ideas of \cite{SavinValdinoci-EnergyEstimate} to our setting in order to establish Theorem~\ref{Th:EnergyEstimate}. In the arguments, the assumption \eqref{Eq:KernelInequality} is crucial.



As an application of the previous results, we are able to prove, by using standard variational methods, the existence of saddle-shaped solutions to the Allen-Cahn equation \eqref{Eq:NonlocalAllenCahn}.

\begin{theorem}[Existence of saddle-shaped solutions]
	\label{Th:Existence}
    Let $f$ satisfy \eqref{Eq:HipothesisfOdd} and \eqref{Eq:HipothesisGWells}\todo{Quitar positiva}, and let $L_K\in \lcal_\star$. Then, for every dimension $2m \geq 2$, there exists a saddle-shaped solution to \eqref{Eq:NonlocalAllenCahn}. In addition, $u$ satisfies $|u|<1$ in $\R^{2m}$.
\end{theorem}

In the forthcoming paper \cite{FelipeSanz-Perela:IntegroDifferentialII} we establish the same result with other techniques (monotone iteration and maximum principle). In both proofs, the assumtion \eqref{Eq:KernelInequality} is crucial.


The paper is organized as follows. Section~\ref{Sec:Preliminaries} is devoted to study the operator $L_K$ acting on doubly radial and odd functions. We deduce the expression of the doubly radial kernel $\overline{K}$ and we prove some properties. We also show Theorem~\ref{Th:CharacterizationLstar} and some maximum principles. In Section~\ref{Sec:Nonlocal_AllenCahn_Energy} we study the energy functional associated to \eqref{Eq:NonlocalAllenCahn} and we establish the energy estimate stated in Theorem~\ref{Th:EnergyEstimate}. Finally, in Section~\ref{Sec:Existence} we prove the existence of the saddle-shaped solution to the Allen-Cahn equation. At the end of the paper there are three appendices. Appendix~\ref{Sec:AuxiliaryResults} is devoted to some results on convex functions, and Appendix~\ref{Sec:AuxiliaryResults2} contains some auxiliary computations. Both are used in the proof of Theorem~\ref{Th:CharacterizationLstar}. In Appendix~\ref{Sec:stcomputations} we include some computations in $(s,t)$ variables for future reference.