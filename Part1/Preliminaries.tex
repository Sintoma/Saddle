%%%%%%%%%%%%%%%%%%%%%%%%
\section{Preliminaries}
%%%%%%%%%%%%%%%%%%%%%%%%
\label{Sec:Preliminaries}




%%%%%%%%%%%%%%%%%%%%%%%%%%%%%%%%%%%%%%%%%%%%%%%%%%%%%%%%%%%%%%%%%%%%%%%%%%%%%%%%%%%%%%%%%%%%%%%%%%%
%%%%%%%%%%%%%%%%%%%%%%%%%%%%%%%%%%%%%%%%%%%%%%%%%%%%%%%%%%%%%%%%%%%%%%%%%%%%%%%%%%%%%%%%%%%%%%%%%%%
\subsection{The operators of $\Lr$ acting on doubly radial functions}
%%%%%%%%%%%%%%%%%%%%%%%%%%%%%%%%%%%%%%%%%%%%%%%%%%%%%%%%%%%%%%%%%%%%%%%%%%%%%%%%%%%%%%%%%%%%%%%%%%%
%%%%%%%%%%%%%%%%%%%%%%%%%%%%%%%%%%%%%%%%%%%%%%%%%%%%%%%%%%%%%%%%%%%%%%%%%%%%%%%%%%%%%%%%%%%%%%%%%%%

The main purpose of this subsection is to deduce an alternative expression for an operator $L_K \in \Lr$ acting on doubly radial functions. This expression is more suitable to work with and it will be used throughout the paper. Recall that the operators in the class $\Lr$ are always uniformly elliptic ($L_K \in \lcal_0$) and have a radially symmetric kernel, i.e., $K(y) = K(|y|)$.

Our first remark is that if $w$ is invariant by $O(m)^2$, so is $Lw$. Indeed, for every $R \in
O(m)^2$,
\begin{align*}
L_K w (Rx)
& = \int_{\R^{2m}} \{w(Rx) - w(y)\} K(|Rx - y|)  \d y\\
& = \int_{\R^{2m}} \{w(Rx) - w(R\tilde{y})\} K(|Rx - R\tilde{y}|) \d \tilde{y}\\
& = \int_{\R^{2m}} \{w(x) - w(\tilde{y})\} K(|x-\tilde{y}|) \d \tilde{y}\\
& = L_K w (x)\,.
\end{align*}
Here we have used that $w(R \cdot) = w(\cdot)$ for every $R\in O(m)^2$ and the change $y =
R\tilde{y}$.


Next, we present an alternative expression for the operator $L_K $ acting on doubly radial functions.

\begin{lemma} \label{Lemma:AlternativeOperatorExpression}
Let $L_K \in \Lr$ and let $w$ be a doubly radial function for which $L_K w$ is well-defined. Then, $L_K w$ can be expressed as
$$
L_K w(x) = \int_{\R^{2m}} \{w(x) - w(y)\} \overline{K}(x,y) \d y
$$
where $\overline{K}$ is symmetric, invariant by $O(m)^2$ in both arguments and is defined by
\begin{equation}
\label{Eq:KbarDef}
\overline{K}(x,y) := \average_{O(m)^2} K(|Rx - y|)\d R\,.
\end{equation}
Here, $\d R$ denotes integration with respect to the Haar measure on $O(m)^2$.
\end{lemma}

Recall (see for instance 99) that the Haar measure on $O(m)^2$ exists and it is unique up to a
multiplicative constant. Let us state next the properties that will be used in the rest of the
paper. In the following, the Haar measure is denoted by $\mu$. First, since $O(m)^2$ is a compact
group, it is unimodular (see Chapter~II, Proposition~ 13 of \cite{Nachbin}). As a consequence, the
measure $\mu$ is left and right invariant, that is, $\mu(R\Sigma) = \mu(\Sigma) = \mu(\Sigma R) $
for every subset $\Sigma \subset O(m)^2$ and every $R\in O(m)^2$. Moreover, it holds
\begin{equation}
\label{Eq:Unimodular}
\average_{O(m)^2} g(R^{-1}) \d R = \average_{O(m)^2} g(R) \d R
\end{equation}	
for every $g\in L^1(O(m)^2)$ ---see \cite{Nachbin} for the details.

\begin{proof}[Proof of Lemma~\ref{Lemma:AlternativeOperatorExpression}]
Since $L_K w (x) = L_K w (Rx)$ for every $R\in O(m)^2$, by taking the mean over all the transformations in $O(m)^2$, we get
\begin{align*}
L_K w(x) &= \average_{O(m)^2} L_K w(Rx)\d R =  \average_{O(m)^2} \int_{\R^{2m}} \{w(x) - w(y)\}K(|Rx - y|) \d y \d R\\
&= \int_{\R^{2m}} \{w(x) - w(y)\}  \average_{O(m)^2} K(|Rx - y|) \d R  \d y = \int_{\R^{2m}} \{w(x) - w(y)\}  \overline{K}(x,y) \d y\,.
\end{align*}
Now, we show that $\overline{K}$ is symmetric:
\begin{align*}
\overline{K}(y,x) &= \average_{O(m)^2} K(|R y - x|)\d R = \average_{O(m)^2} K(|R^{-1} (R y - x)|)\d R \\
&= \average_{O(m)^2} K(|R^{-1}x-y)|)\d R = \overline{K}(x,y)\,.
\end{align*}
In this last step we have used property \eqref{Eq:Unimodular}. It remains to show that
$\overline{K}$ is invariant by $O(m)^2$ in its two arguments. By the symmetry, it is enough to
check it for the first one. Let $\tilde{R} \in O^2(m)^2$. Then,
$$
\overline{K} (\tilde{R}x, y) = \average_{O(m)^2} K(|R \tilde{R} x - y|)\d R  = \average_{O(m)^2} K(|R x - y|)\d R = \overline{K} (x, y)\,
$$
where we have used the right invariance of the Haar measure.
\end{proof}



In the following lemma we present some properties of the involution $(\cdot)^\star$ defined by \eqref{Eq:DefStar} and its relation with the doubly radial kernel $\overline{K}$ and the transformations of $O(m)^2$. In particular, in the proof of Theorem~\ref{Th:CharacterizationLstar} it will be useful to consider the following transformation. For every $R\in O(m)^2$, we define  $R_\star\in O(m)^2$ by 
\begin{equation}
	\label{Eq:DefRStar}
	R_\star := (R(\cdot)^\star)^\star\,.
\end{equation}
Equivalently, if $R = (R_1, R_2)$ with $R_1$, $R_2 \in O(m)$, then $R_\star = (R_2, R_1)$.

\begin{lemma}
\label{Lemma:PropertiesStar}
Let $(\cdot)^\star$ be the isometry defined by $x^\star = (x',x'')^\star = (x'', x')$
---see \eqref{Eq:DefStar}.
Then,
\begin{enumerate}
\item
The Haar integral on $O(m)^2$ has the following invariance:
\begin{equation}
\label{Eq:InvarianceByStar}
\int_{O(m)^2} g(R_\star) \d R = \int_{O(m)^2} g(R) \d R \,,
\end{equation}
for every $g \in L^1(O(m)^2)$.
\item $\overline{K}(x^\star,y) = \overline{K} (x,y^\star)$.
\item $\overline{K}(x^\star,y^\star) = \overline{K} (x,y)$.
\item Let $w$ is a doubly radial function which is odd with respect to the Simons cone, and let $L_K \in \Lr$. Then,
\begin{align*}
L_K w (x) &= \int_{\ocal} \{w(x) - w(y) \} \overline{K}(x, y) \d y +  \int_{\ocal} \{w(x) + w(y) \} \overline{K}(x, y^\star) \d y \\
&= \int_{\ocal} \{w(x) - w(y) \} \{\overline{K}(x, y) - \overline{K}(x, y^\star)  \} \d y +  2 w(x) \int_{\ocal} \overline{K}(x, y^\star) \d y \,.
\end{align*}
\end{enumerate}
\end{lemma}

\begin{proof}
The first statement is easy to check \eqref{Eq:InvarianceByStar} by using Fubini:
\begin{align*}
\int_{O(m)^2} g(R_\star) \d R & = \int_{O(m)} \!\! \d R_1 \int_{O(m)} \!\! \d R_2 \ \ g(R_2, R_1)  =  \int_{O(m)} \!\! \d R_2 \int_{O(m)} \!\! \d R_1 \ \ g(R_2, R_1) \\
& =  \int_{O(m)} \!\! \d R_1 \int_{O(m)} \!\! \d R_2 \ \ g(R_1, R_2)  =  \int_{O(m)^2} g(R) \d R\,.
\end{align*}

To show the second statement, we use the definition of $R_\star$ and \eqref{Eq:InvarianceByStar}
to see that
\begin{align*}
\overline{K}(x^\star,y) &= \average_{O(m)^2} K(|Rx^\star - y|) \d R = \average_{O(m)^2} K(|(Rx^\star - y)^\star|) \d R \\
&= \average_{O(m)^2} K(|(R x^\star)^\star - y^\star|) \d R = \average_{O(m)^2} K(|R_\star x - y^\star|) \d R \\
&= \average_{O(m)^2} K(|Rx - y^\star|) \d R = \overline{K}(x,y^\star)\,.
\end{align*}
As a consequence, we have that
$$\overline{K}(x^\star,y^\star) = \overline{K}(x,(y^\star)^\star) = \overline{K}(x,y)\,.$$



Finally, the last statement is just a computation with a change of variables. First, using the
change $\bar{y} = y^\star$ and the odd symmetry of $u$, we see that
\begin{align*}
\int_{\ical}  \{w(x) - w(y) \} \overline{K}(x, y)\d y &= \int_{\ocal^\star} \{w(x) - w(y) \}\overline{K}(x, y)\d y \\
&= \int_{\ocal} \{w(x) - w(y^\star) \}\overline{K}(x, y^\star)\d y \\
&= \int_{\ocal} \{w(x) + w(y) \}\overline{K}(x, y^\star)\d y\,.
\end{align*}
Hence,
\begin{align*}
L_K w (x) &= \int_{\R^{2m}}  \{w(x) - w(y) \} \overline{K}(x, y)\d y \\
&= \int_{\ocal}  \{w(x) - w(y) \} \overline{K}(x, y)\d y +\int_{\ical}  \{w(x) - w(y) \} \overline{K}(x, y)\d y \\
&= \int_{\ocal} \{w(x) - w(y) \} \overline{K}(x, y) \d y +  \int_{\ocal} \{w(x) + w(y) \} \overline{K}(x, y^\star) \d y \,.
\end{align*}
By adding and subtracting $w(x)\overline{K}(x, y^\star)$ in the last integrand we immediately deduce
$$
L_K w (x) \int_{\ocal} \{w(x) - w(y) \} \{\overline{K}(x, y) - \overline{K}(x, y^\star)  \} \d y +  2 w(x) \int_{\ocal} \overline{K}(x, y^\star) \d y\,.
$$
\end{proof}


\subsection{Characterization of $\lcal_\star$}

In this subsection, we establish Theorem~\ref{Th:CharacterizationLstar}.

...

\begin{proposition}
\label{Prop:KernelInequalityReflexion} Assume that the kernel $K$ is radially symmetric, satisfies
hypothesis \eqref{Eq:Symmetry&IntegrabilityOfK} and \eqref{Eq:Ellipticity}, and that
$K(\sqrt{\cdot})$ is strictly convex in $(0,+\infty)$. Then,
\begin{equation}
\label{Eq:KernelInequalityReflexion}
\overline{K}(x,y) > \overline{K}(x, y^\star) \quad \text{ for every }x,y \in \ocal\,,
\end{equation}
where $\overline{K}$ is defined by \eqref{Eq:KbarDef}.
\end{proposition}

\todo[inline]{La condicion de convexidad da que es decreciente y luego ya te da que la g es creciente, mirar si eso hay que decirlo o va incluido en los lemas del final}
\begin{proof}
Let $x$, $y \in \ocal$ and let $x_0 = (|x'|e, |x''|e)$ and $y_0 = (|y'|e, |y''|e)$, for an
arbitrary unitary vector $e \in \Sph^{m-1} \subset \R^m$. Hence, since $\overline{K}$ is invariant
by $O(m)^2$,
$$
\overline{K}(x,y) = \overline{K}(x_0, y_0) \quad \text{ and } \quad \overline{K}(x,y^\star) = \overline{K}(x_0, y_0^\star)\,.
$$
To see this, let $R_x$, $R_y \in O(m)^2$ satisfying $x = R_x x_0$ and $y = R_y y_0$. Then,
$$
\overline{K}(x,y) = \overline{K}(R_x x_0,R_y y_0)  = \overline{K}(x_0, y_0)
$$
and, using that $(R_y y_0)^\star = (R_y)_\star y_0^\star$ ---see \eqref{Eq:DefRStar}---,
$$
\overline{K}(x,y^\star) = \overline{K}(R_x x_0,(R_y y_0)^\star) = \overline{K}(R_x x_0,(R_y)_\star y_0^\star)  = \overline{K}(x_0, y_0^\star)\,.
$$
Therefore, it is enough to show \eqref{Eq:KernelInequalityReflexion} for points $x$ and $y$ of the
form $x = (|x'|e, |x''|e)$ and $y = (|y'|e, |y''|e)$, with $e \in \Sph^{m-1}$ an arbitrary unitary
vector.

Now, define
\begin{align*}
Q_1 &:= \setcond{R = (R_1,R_2) \in O(m)^2}{e\cdot R_1 e > |e\cdot R_2 e|},\\
Q_2 &:= \setcond{R = (R_1,R_2) \in O(m)^2}{e\cdot R_2 e > |e\cdot R_1 e|},\\
Q_3 &:= \setcond{R = (R_1,R_2) \in O(m)^2}{e\cdot R_1 e < -|e\cdot R_2 e|},\\
Q_4 &:= \setcond{R = (R_1,R_2) \in O(m)^2}{e\cdot R_2 e < - |e\cdot R_1 e|}.
\end{align*}
Note that $Q_i \cap Q_j = \emptyset$ for $i\neq j$. Moreover, the set
$$
O(m)^2 \setminus \{Q_1 \cup Q_2 \cup Q_3 \cup Q_4\} = \setcond{R\in O(m)^2}{|e\cdot R_1 e| = |e\cdot R_2 e|}
$$
has zero measure. Note also that
\begin{itemize}
\item If $R = (R_1, R_2)\in Q_2$, $R_\star = (R_2, R_1) \in Q_1$.
\item If $R = (R_1, R_2)\in Q_3$, $-R = (-R_1, -R_2) \in Q_1$.
\item If $R = (R_1, R_2)\in Q_4$, $-R_\star = (-R_2, -R_1) \in Q_1$.
\end{itemize}
Therefore,
\begin{align*}
\overline{K} (x, y) &= \average_{O(m)^2} K(|x - R y|)\d R \\
& = \average_{Q_1} K(|x - R y|)\d R + \average_{Q_2} K(|x - R y|)\d R \\
& \quad \quad
+ \average_{Q_3} K(|x - R y|)\d R +
\average_{Q_4} K(|x - R y|)\d R \\
&= \average_{Q_1} \{K(|x - R y|) + K(|x + R y|) \\
&\quad \quad + K(|x - R_\star y|) + K(|x + R_\star y|)\}\d R
\end{align*}
and
\begin{align*}
\overline{K} (x, y^\star) &= \average_{O(m)^2} K(|x - R y^\star|)\d R \\
& = \average_{Q_1} \{K(|x - R y^\star|) + K(|x + R y^\star|) \\
&\quad \quad + K(|x - R_\star y^\star|) + K(|x + R_\star y^\star|)\}\d R
\end{align*}
Thus, if we prove
\begin{equation}
\label{Eq:InequalityIntegrandKernelInequalityProof}
\begin{split}
K(|x - R y|) + K(|x + R y|) + K(|x - R_\star y|) + K(|x + R_\star y|)
\quad \quad \quad \quad \quad \quad \quad \quad
\\
\geq
K(|x - R y^\star|) + K(|x + R y^\star|)+K(|x - R_\star y^\star|) + K(|x + R_\star y^\star|)\,,
\end{split}
\end{equation}
for every $R\in Q_1$, we immediately deduce \eqref{Eq:KernelInequalityReflexion} with a non strict
inequality. \todo{Decir que todos los términos son finitos, los del rhs pq cada punto esta en un
lado del cono y los otros pq sino implicaria que $e \cdot R_2 e =1$ y esto no puede pasar en
$Q_1$.}To see that it is indeed a strict one, we must show that the inequality in
\eqref{Eq:InequalityIntegrandKernelInequalityProof} is strict for a.e. $R \in Q_1$.


For a short notation, we call
$$
\alpha := e \cdot R_1 e  \quad \text{ and } \quad \beta := e \cdot R_2 e\,.
$$
Note that
\begin{align*}
|x \pm Ry|^2&= |x' \pm R_1y'|^2 + |x'' \pm R_2y''|^2 \\
&= |x'|^2 + |y'|^2 \pm 2 x'\cdot R_1 y' +  |x''|^2 + |y''|^2 \pm 2 x''\cdot R_2 y''\\
&= \bpar{ |x|^2 + |y|^2 \pm 2 |x'||y'| \alpha \pm 2 |x''||y''| \beta}\,.
\end{align*}
Similarly,
$$
|x \pm R_\star y| = \bpar{ |x|^2 + |y|^2 \pm 2 |x'||y'| \beta \pm 2 |x''||y''|\alpha}\,,
$$
$$
|x \pm R y^\star| = \bpar{ |x|^2 + |y|^2 \pm 2 |x'||y''| \alpha \pm 2 |x''||y'|\beta}\,,
$$
and
$$
|x \pm R_\star y^\star| = \bpar{ |x|^2 + |y|^2 \pm 2 |x'||y''| \beta \pm 2 |x''||y'| \alpha}\,.
$$

We define now
$$
g(t) := K \bpar{\sqrt{|x|^2 + |y|^2 + 2 t}} + K \bpar{\sqrt{|x|^2 + |y|^2 - 2 t}}
$$
and we see that \eqref{Eq:InequalityIntegrandKernelInequalityProof} is equivalent to
\begin{equation}
\label{Eq:InequalityIntegrandKernelInequalityProof2}
\begin{split}
g\Big(|x'||y'| \alpha + |x''||y''| \beta \Big)
+ g\Big(|x'||y'| \beta + |x''||y''| \alpha \Big) \hspace{2cm}
\\ \geq
g\Big(|x'||y''| \alpha + |x''||y'|\beta \Big)
+ g\Big(|x'||y''| \beta + |x''||y'| \alpha \Big)\,,
\end{split}
\end{equation}
for every $\alpha$, $\beta \in [-1,1]$ such that $\alpha > |\beta|$. Let
$$
\begin{array}{cc}
A_{\alpha,\beta} = |x'||y'|  \alpha + |x''||y''|\beta \,, &
B_{\alpha,\beta} = |x'||y''| \alpha + |x''||y'| \beta \,, \\
C_{\alpha,\beta} = |x''||y'| \alpha + |x'||y''| \beta \,, &
D_{\alpha,\beta} = |x''||y''|\alpha + |x'||y'|  \beta \,.
\end{array}
$$
With this notation and taking into account that $g$ is even,
\eqref{Eq:InequalityIntegrandKernelInequalityProof2} is equivalent to
\begin{equation}
\label{Eq:InequalityIntegrandKernelInequalityProof3}
g(|A_{\alpha,\beta}|) + g(|D_{\alpha,\beta}|) \geq g(|C_{\alpha,\beta}|) + g(|B_{\alpha,\beta}|)\,,
\end{equation}
for every $\alpha$, $\beta \in [-1,1]$ such that $\alpha > |\beta|$. Note that $g$ is defined in
the open interval $I = (-(|x|^2 + |y|^2)/2,\ (|x|^2 + |y|^2)/2)$ and that $A_{\alpha,\beta}$,
$B_{\alpha,\beta}$, $C_{\alpha,\beta}$, $D_{\alpha,\beta} \in I$.

To show \eqref{Eq:InequalityIntegrandKernelInequalityProof3}, we use
Proposition~\ref{Prop:EquivalenceK(sqrt)Convex<->Inequality} of the Appendix~\ref{Sec:AuxiliaryResults}. Hence, we should
check that
$$
\begin{cases}
|A_{\alpha,\beta}| \geq |B_{\alpha,\beta}|,\ |A_{\alpha,\beta}| \geq |C_{\alpha,\beta}|, \ |A_{\alpha,\beta}| \geq |D_{\alpha,\beta}|\,, \\
|A_{\alpha,\beta}| + |D_{\alpha,\beta}| \geq |B_{\alpha,\beta}| + |C_{\alpha,\beta}|\,.
\end{cases}
$$
The verification of these inequalities is a simple but tedious computation and it is presented in
Appendix~\ref{Sec:AuxiliaryResults2} (see point (1) of Lemma~\ref{Lemma:ComputationABCD}). Once we have these inequalities, by Proposition~\ref{Prop:EquivalenceK(sqrt)Convex<->Inequality} we deduce \eqref{Eq:InequalityIntegrandKernelInequalityProof3}.

To finish, we must see that the equality in \eqref{Eq:InequalityIntegrandKernelInequalityProof3} is
never attained. By Proposition~\ref{Prop:EquivalenceK(sqrt)Convex<->Inequality}, we know that a
necessary condition for the equality to hold is that either $|A_{\alpha,\beta}| =
|B_{\alpha,\beta}|$ and $|C_{\alpha,\beta}| = |D_{\alpha,\beta}|$, or $|A_{\alpha,\beta}| =
|C_{\alpha,\beta}|$ and $|B_{\alpha,\beta}| = |D_{\alpha,\beta}|$. Nevertheless, by point (2) of
Lemma~\eqref{Lemma:ComputationABCD}, this yields $\alpha = \beta = 0$, but this cannot happen.
\end{proof}

Once we have presented a sufficient condition of the kernel to be positive, we can also obtain a
necessary condition.


\begin{proposition}
\label{Prop:ContraryKernelInequalityReflexion} Assume that the kernel $K$ is radially symmetric,
nonincreasing and satisfies the hypotheses \eqref{Eq:Symmetry&IntegrabilityOfK} and
\eqref{Eq:Ellipticity}. If
\begin{equation}
\label{Eq:KernelInequalityReflexion2}
\overline{K}(x,y) > \overline{K}(x, y^\star) \quad \text{ for almost every }x,y \in \mathcal{O}\,,
\end{equation}
then $K(\sqrt{\cdot})$ cannot be concave in any interval $I\subset [0,+\infty)$.
\end{proposition}

\begin{proof}
We prove it by contraposition. In fact, we will show that the existence of an interval where
$K(\sqrt{\cdot})$ is concave means the existence of a open set in $\ocal \times \ocal$ with
positive measure where \eqref{Eq:KernelInequalityReflexion2} is not satisfied.

Let $t_2>t_1>0$ be such that $K(\sqrt{t})$ is concave in $(t_1,t_2)$ and define the set
$\Omega_{t_1,t_2}\subset \R^{4m}$ as the points $(x,y)\in \ocal\times \ocal$ satisfying
\begin{equation}
\label{eq:OmegaSetDefinition}
\beqc{\PDEsystem}
|x|^2+|y|^2&>&t_1,\\
|x|^2+|y|^2&<&t_2,\\
(|x'|-|x''|)^2+(|y'|-|y''|)^2&>&t_1,\\
(|x'|+|x''|)^2+(|y'|+|y''|)^2&<&t_2.
\eeqc
\end{equation}

First, it is easy to see that $\Omega_{t_1,t_2}$ is a nonempty open set. In fact, points of the
form $(x',0,y',0)\in (\R^m)^4$ such that $t_1\leq |x'|^2+|y'|^2<t_2$ belong to $\Omega_{t_1,t_2}$.
Then, if we prove that $\overline{K}(x,y) \leq \overline{K}(x, y^\star)$ in $\Omega_{t_1,t_2}$ we
are done.

Let $x,y\in \Omega_{t_1,t_2}$, we are going to show that
\begin{equation}
\label{Eq:InequalityIntegrandKernelInequalityProof4}
\begin{split}
K(|x - R y|) + K(|x + R y|) + K(|x - R_\star y|) + K(|x + R_\star y|)
\quad \quad \quad \quad \quad \quad \quad \quad
\\
\leq
K(|x - R y^\star|) + K(|x + R y^\star|)+K(|x - R_\star y^\star|) + K(|x + R_\star y^\star|)\,,
\end{split}
\end{equation}
for any $R$ in $Q_1$, where $Q_1$ is defined in the proof of
Proposition~\ref{Prop:KernelInequalityReflexion}. As before, we can assume that $x$ and $y$ are of
the form $x = (|x'|e, |x''|e)$ and $y = (|y'|e, |y''|e)$, with $e \in \Sph^{m-1}$ an arbitrary
unitary vector. Then, defining $\alpha$ and $\beta$ as in the previous proof we see that proving
\eqref{Eq:InequalityIntegrandKernelInequalityProof4} is equivalent to prove that
\begin{equation}
\label{Eq:InequalityIntegrandKernelInequalityProof5}
g(A_{\alpha,\beta}) + g(D_{\alpha,\beta}) \leq g(B_{\alpha,\beta}) + g(C_{\alpha,\beta})\,,
\end{equation}
for every $\alpha, \beta \in [-1,1]$ such that $\alpha>|\beta|$,
$$
\begin{array}{cc}
A_{\alpha,\beta} = |x'||y'|  \alpha + |x''||y''|\beta \,, &
B_{\alpha,\beta} = |x'||y''| \alpha + |x''||y'| \beta \,, \\
C_{\alpha,\beta} = |x''||y'| \alpha + |x'||y''| \beta \,, &
D_{\alpha,\beta} = |x''||y''|\alpha + |x'||y'|  \beta \,.
\end{array}
$$
and
\begin{align*}
g(t) &= K\left( \sqrt{|x|^2+|y|^2+2t} \right) + K\left( \sqrt{|x|^2+|y|^2-2t} \right).
\end{align*}

Now, from being $K(\sqrt{t})$ concave in $(t_1,t_2)$ and $t_1<|x|^2+|y|^2<t_2$, we obtain by Remark
99 that $g$ is concave in $ \left( -\overline{t}, \overline{t}\right) $, and decreasing in
$(0,\overline{t})$ with $\overline{t} =
\min{\left(\frac{t_2-|x|^2-|y|^2}{2},\frac{|x|^2+|y|^2-t_1}{2}\right)}$.

On the other hand it is easy to check that $A_{\alpha,\beta}, B_{\alpha,\beta}, C_{\alpha,\beta}$
and $D_{\alpha,\beta}$ belong to the open interval $(-|x'||y'|-|x''||y''|,|x'||y'|+|x''||y''|)$ for
every $\alpha, \beta \in [-1,1]$ such that $\alpha>|\beta|$.

Therefore, since $x,y \in \Omega_{t_1,t_2}$, we obtain from the last inequalities in
\eqref{eq:OmegaSetDefinition} that
$$
\beqc{\PDEsystem}
|x'||y'|+|x''||y''|&<&\dfrac{t_2-|x|^2-|y|^2}{2}\\
|x'||y'|+|x''||y''|&<&\dfrac{|x|^2+|y|^2-t_1}{2}
\eeqc \Longrightarrow  |x'||y'|+|x''||y''|<\overline{t},
$$
which means that $A_{\alpha,\beta}, B_{\alpha,\beta}, C_{\alpha,\beta}$ and $D_{\alpha,\beta}$
belong to $(-\overline{t},\overline{t})$ for every $\alpha, \beta \in [-1,1]$ such that
$\alpha>|\beta|$. Hence, by applying Lemma~\ref{Lemma:InequalitConvexFunctions} to the function
$-g$ (taking into account Remark~\ref{Remark:InequalitConvexFunctions}) we obtain that inequality
\eqref{Eq:InequalityIntegrandKernelInequalityProof4} is satisfied. Finally, from integrating
\eqref{Eq:InequalityIntegrandKernelInequalityProof3} with respect to all the rotations $R\in Q_1$
we get
$$ \overline{K}(x,y) \leq \overline{K}(x, y^\star),$$
for every $(x,y)\in \Omega_{t_1,t_2}$, contradicting \eqref{Eq:KernelInequalityReflexion2}.
\end{proof}

The inequality given in Proposition~\ref{Prop:KernelInequalityReflexion} is crucial to have a maximum principle, a key tool in the developement of the rest of the paper. Therefore, while studying saddle-shaped solutions we will consider operators in the class $\Lr$ whose kernel satisfies \eqref{Eq:KernelInequalityReflexion}. The class of such operators will be denoted by $\lcal_\star$.

\todo[inline]{Supongo que esto irá en la introducción}

%%%%%%%%%%%%%%%%%%%%%%%%%%%%%%%%%%%%%%%%%%%%%%%%%%%%%%%%%%%%%%%%%%%%%%
%%%%%%%%%%%%%%%%%%%%%%%%%%%%%%%%%%%%%%%%%%%%%%%%%%%%%%%%%%%%%%%%%%%%%%
\subsection{Maximum principles for doubly radial odd functions}
%%%%%%%%%%%%%%%%%%%%%%%%%%%%%%%%%%%%%%%%%%%%%%%%%%%%%%%%%%%%%%%%%%%%%%
%%%%%%%%%%%%%%%%%%%%%%%%%%%%%%%%%%%%%%%%%%%%%%%%%%%%%%%%%%%%%%%%%%%%%%

In this subsection we prove a weak and a strong maximum principles for doubly radial functions that
are odd with respect to the Simons cone. The key ingredient in these proofs is the kernel
inequality of Proposition~\ref{Prop:KernelInequalityReflexion}.

\begin{proposition}[Weak maximum principle for odd functions with respect to $\ccal$]
\label{Prop:WeakMaximumPrincipleForOddFunctions} Let $u\in C^{\alpha}(\R^{2m})$ with
$\alpha > 2\s$ be a doubly radial function which is odd with respect to the Simons cone. Let
$\Omega \subset \ocal$ and let $L_K  \in \lcal_\star$. Assume that
$$
\beqc{\PDEsystem}
L_K u & \geq & 0 & \text{ in } \Omega\,,\\
u & \geq & 0 & \text{ in } \ocal \setminus \Omega\,,
\eeqc
$$
and that either $\Omega$ is bounded or \todo{Pensar si sabemos hacerlo para no acotados. En el
paper no se usa pero quedaría mejor un resultado un poco más general}
$$
\liminf_{x \in \ocal,\,|x|\to \infty} u(x) \geq 0\,.
$$
Then, $u \geq 0$ in $\Omega$.
\end{proposition}

\begin{proof}
By contradiction, assume that $u$ takes negative values in $\Omega$. Under the hypotheses we are
assuming, a negative minimum must be achieved. Thus, there exists $x_0\in \Omega$ such that
$$
u(x_0) = \min_{\Omega} u =: m < 0\,.
$$
Then, using the expression of $L$ for odd functions (see Lemma~\ref{Lemma:PropertiesStar}), we have
$$
L_K u (x_0) = \int_{\ocal} \{m - u(y) \} \overline{K}(x_0, y) \d y +  \int_{\ocal} \{m + u(y) \} \overline{K}(x_0, y^\star) \d y\,.
$$
Now, since $m - u(y) \leq 0$ in $\ocal$ and $\overline{K}(x_0, y) \geq \overline{K}(x_0, y^\star)$ (see Proposition~\ref{Prop:KernelInequalityReflexion}), we have
$$
\{m - u(y) \} \overline{K}(x_0, y) \leq \{m - u(y) \} \overline{K}(x_0, y^\star)
$$
and therefore, since $m<0$, we get
$$
0 \leq L_K  u(x_0) \leq 2m \int_{\ocal} \overline{K}(x_0, y^\star) \d y < 0\,,
$$
a contradiction.
\end{proof}

The following is a strong maximum principle for odd functions.

\begin{proposition}[Strong maximum principle for odd functions with respect to $\ccal$]
\label{Prop:StrongMaximumPrincipleForOddFunctions} Let $u\in C^{\alpha}(\R^{2m})$ with
$\alpha > 2\s$ be a doubly radial function which is odd with respect to the Simons cone.  Let
$\Omega \subset \ocal$ and assume that $L_K u \geq 0$ in $\Omega$, where $L_K  \in \lcal_\star$. Assume also that $u\geq 0$ in $\ocal$.
Then, either $u\equiv 0$ or $u > 0$ in $\Omega$.
\end{proposition}

\begin{proof}
Assume that $u \not \equiv 0$. We shall prove that $u > 0$ in $\Omega$. By contradiction, assume
that there exists a point $x_0\in \Omega$ such that $u(x_0)= 0$. Then, using the expression of $L_K $
for odd functions given in Lemma~\ref{Lemma:PropertiesStar}, the kernel inequality of	
Proposition~\ref{Prop:KernelInequalityReflexion} and the fact that $u\geq 0$ in $\ocal$, we obtain
$$
0 \leq L_K u(x_0) = \int_{\ocal} u(y)\big \{\overline{K}(x_0, y^\star) - \overline{K}(x_0, y) \big \}\d y < 0\,,
$$
a contradiction.
\end{proof}


%%%%%%%%%%%%%%%%%%%%%%%%%%%%%%%%%%%%%%%%%%%%%%%%%%%%%%%%%%%%%%%%%%%%%%
%%%%%%%%%%%%%%%%%%%%%%%%%%%%%%%%%%%%%%%%%%%%%%%%%%%%%%%%%%%%%%%%%%%%%%
