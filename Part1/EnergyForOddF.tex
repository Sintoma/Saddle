%%%%%%%%%%%%%%%%%%%%%%%%
\section{The energy functional for doubly radial odd functions}
%%%%%%%%%%%%%%%%%%%%%%%%
\label{Sec:EnergyForOddF}


This section is devoted to the energy functional associated to the semilinear equation \eqref{Eq:NonlocalAllenCahn}. We first define appropriately the functional spaces where we are going to apply classic techniques of calculus of variations. Next we rewrite the energy in terms of the new kernel $\overline{K}$ and we give an alternative expression for the energy of doubly radial odd functions. Finally, we establish some results that are useful when using variational techniques, and that will be exploited in the next section.


Let us start by defining the functional spaces that we are going to consider in the rest of the paper. Given a set $\Omega \subset \R^n$ and a translation invariant and positive kernel $K$ satisfying \eqref{Eq:Symmetry&IntegrabilityOfK}, we define the space
$$
\H^K(\Omega) := \setcond{w \in L^2(\Omega)}{[w]^2_{\H^K(\Omega)} < + \infty},
$$
where
$$
[w]^2_{\H^K(\Omega)} := \dfrac{1}{2}\int\int_{\R^{2n} \setminus (\R^n\setminus\Omega)^2} |w(x) - w(y)|^2 K(x-y) \d x \d y\,.
$$
We also define
\begin{align*}
	\H^K_0(\Omega) &:= \setcond{w \in \H^K(\Omega)}{ w = 0 \quad \textrm{a.e. in } \R^n \setminus \Omega} \\
	&\ = \setcond{w \in \H^K(\R^n)}{ w = 0 \quad \textrm{a.e. in } \R^n \setminus \Omega}.
\end{align*}

Assume that $\Omega \subset \R^{2m}$ is a domain of double revolution. Then, we define
$$
\widetilde{\H}^K(\Omega) := \setcond{w \in \H^K(\Omega)}{w \textrm{ is doubly radial a.e.}}.
$$
and
$$
\widetilde{\H}^K_0(\Omega) := \setcond{w \in \H^K_0(\Omega)}{w \textrm{ is doubly radial a.e.}}.
$$
We will add the subscript `odd' and `even' to these spaces to consider only functions that are odd (respectively even) with respect to the Simons cone.


\begin{remark}
\label{Remark:DecompositionHK}
If $\widetilde{\H}^K_0(\Omega)$ is equipped with the scalar product
$$
\langle v,w \rangle_{\widetilde{\H}^K_0(\Omega)} := \dfrac{1}{2}\int_{\R^{2m}} \int_{\R^{2m}}  \big(v(x) - v(y)\big)\big(w(x) - w(y)\big) K(x-y) \d x \d y\,,
$$
then, it is easy to check that $\widetilde{\H}^K_0(\Omega)$ can be decomposed as the orthogonal
direct sum of $\widetilde{\H}^K_{0,\, \mathrm{even}}(\Omega)$ and $\widetilde{\H}^K_{0,\,
\mathrm{odd}}(\Omega)$.
\end{remark}

Note that when $K$ satisfies \eqref{Eq:Ellipticity}, then $\H^K_0 (\Omega) = \H^\s_0 (\Omega)$,
which is the space associated to the kernel of the fractional Laplacian, $K(y) = |y|^{-n-2\s}$.
Furthermore, $\H^\s(\Omega) \subset H^\s(\Omega)$, the usual fractional Sobolev space (see
\cite{HitchhikerGuide}).  For more comments on this, see~\cite{CozziPassalacqua}, and the references therein.

Once presented the functional setting of our problem, we define now  the energy functional associated to equation \eqref{Eq:NonlocalAllenCahn}. 


Given a kernel $K$ satisfying \eqref{Eq:Symmetry&IntegrabilityOfK}, a potential $G$ and a function $w\in \H^K(\Omega)$, with $\Omega\subset \R^{n}$, we define the following energy:
$$
\ecal(w, \Omega) := \ecal_\mathrm{K}(w,\Omega) + \ecal_\mathrm{P}(w,\Omega)\,,
$$
where
$$
\ecal_\mathrm{K}(w, \Omega) := \dfrac{1}{2} [w]^2_{\H^K(\Omega)} \quad \text{ and } \quad  \ecal_\mathrm{P}(w, \Omega) := \int_{\Omega} G(w)
\,.
$$
We will call $\ecal_\mathrm{K}$ and $\ecal_\mathrm{P}$ the \emph{kinetic} and \emph{potential} energy respectively.


Note that, for functions $w\in \H^K_0(\Omega)$, it holds $\ecal_\mathrm{K}(w,\Omega) = \ecal_\mathrm{K}(w,\R^n)$. Moreover, if $G\geq 0$, the energy satisfies $\ecal(w, \Omega) \leq \ecal(w, \Omega')$  whenever $ \Omega \subset \Omega'$.

Sometimes is it useful to rewrite the kinetic energy as
\begin{equation}
	\label{Eq:KineticEnergyInteractions}
	\begin{split}
	\ecal_\mathrm{K}(w, \Omega) := \dfrac{1}{4} \int_\Omega \int_\Omega |w(x) - w(y)|^2 K(x-y) \d x \d y \qquad \qquad \\
	+ \dfrac{1}{2} \int_\Omega \int_{\R^n \setminus \Omega} |w(x) - w(y)|^2 K(x-y) \d x \d y\,.
	\end{split}	
\end{equation}
Roughly speaking, we have split the kinetic energy into two parts: ``interactions inside-inside'' and ``interactions inside-outside''. Our goal is, as in the previous section, to rewrite the kinetic energy in terms of the doubly radial kernel $\overline{K}$ and with integrals computed only in $\ocal$. In particular, we are interested in finding an expression similar to \eqref{Eq:KineticEnergyInteractions} for the kinetic energy. To do this, we introduce the following notation for the interaction. For $A$, $B\subset \ocal$ of double revolution, we define
\begin{equation}
	\label{Eq:DefIw}
	\begin{split}
	I_w(A,B) := \int_A  \int_B  \ |w(x)-w(y)|^2 \left\{ \overline{K}(x,y) - \overline{K}(x,y^\star) \right\} \d x \d y  \\
	+2 \int_A  \int_B  \left\{w^2(x)+w^2(y)\right\} \overline{K}(x,y^\star) \d x \d y\,.
	\end{split}
\end{equation}
Thanks to this notation, we rewrite the kinetic energy as follows.


\begin{lemma}
	\label{Lemma:ShortExpressionEnergy}
Let $\Omega\subset \R^{2m}$ be a domain of double revolution and let $w\in \widetilde{\H}^K_{0,\, \mathrm{odd}}(\Omega)$. Let $K$ be a radially symmetric kernel satisfying \eqref{Eq:Symmetry&IntegrabilityOfK}. Then, 
\begin{align}
\label{Eq:ShortExpressionEnergy}
\ecal_\mathrm{K}(w, \Omega) = \frac{1}{2} I_w(\Omega\cap\ocal,\Omega\cap\ocal) + I_w(\Omega\cap\ocal,\ocal\setminus\Omega),
\end{align}
where $I_w(\cdot, \cdot)$ is the interaction defined in \eqref{Eq:DefIw}.
\end{lemma}

\begin{proof}
	
First, in \eqref{Eq:KineticEnergyInteractions} we consider the change $ y = R\tilde{y}$. Then, since $w$ is doubly radial and $\Omega$ is of double revolution, by taking the average among all $R\in O(m)^2$ as in Lemma~\ref{Lemma:AlternativeOperatorExpression}, we obtain 
$$
\ecal_\mathrm{K}(w, \Omega) = \frac{1}{4} \int_{\Omega} \int_{\Omega} |w(x)-w(y)|^2 \overline{K}(x,y) \d x \d y + \frac{1}{2} \int_{\Omega} \int_{\R^n \setminus \Omega} |w(x)-w(y)|^2 \overline{K}(x,y) \d x \d y\,.
$$
Now we split $\Omega$ into $\Omega \cap \ocal$ and $\Omega \setminus \ocal$. By using the change of variables given by $(\cdot)^\star$ and the odd symmetry of $w$, we get
\begin{align*}
\ecal_\mathrm{K}(w, \Omega) =  & \frac{1}{2} \int_{\Omega\cap \ocal} \int_{\Omega \cap \ocal} |w(x)-w(y)|^2 \overline{K}(x,y) + |w(x)+w(y)|^2 \overline{K}(x,y^\star) \d x \d y  \\
& + \int_{\Omega\cap \ocal} \int_{\ocal \setminus \Omega} |w(x)-w(y)|^2 \overline{K}(x,y) + |w(x)+w(y)|^2 \overline{K}(x,y^\star) \d x \d y  \\
= & \frac{1}{2} \int_{\Omega\cap \ocal} \int_{\Omega \cap \ocal}  \ |w(x)-w(y)|^2 \left\{ \overline{K}(x,y) - \overline{K}(x,y^\star) \right\}  +2 \left\{w^2(x)+w^2(y)\right\} \overline{K}(x,y^\star) \d x \d y\,\\
&+ \int_{\Omega\cap \ocal} \int_{\ocal \setminus \Omega}  \ |w(x)-w(y)|^2 \left\{ \overline{K}(x,y) - \overline{K}(x,y^\star) \right\}  +2 \left\{w^2(x)+w^2(y)\right\} \overline{K}(x,y^\star) \d x \d y\,\\
= & \frac{1}{2} I_w(\Omega\cap\ocal,\Omega\cap\ocal) + I_w(\Omega\cap\ocal,\ocal\setminus\Omega).
\end{align*}
\end{proof}

Using the previous expression for the energy, we can establish now the following lemma regarding the decrease of the energy under some operations.
\begin{lemma}
\label{Lemma:DecreaseEnergy} 
Let $K$ be a kernel such that $L_K \in \lcal_\star$. Given $u\in\widetilde{\H}^K_{\mathrm{odd}}(\Omega)$, we define
\begin{equation*}
v(x) = \begin{cases}
\hspace{3.2mm}|u(x)| \,\,\, &\text{if } \,\,\, x\in\ocal,\\
-|u(x)| \,\,\, &\text{if } \,\,\, x\in\ical\, ,
\end{cases}
\quad 
\text{ and }
\quad
w(x) = \begin{cases}
\hspace{3.6mm}\min\{1,u(x)\} \,\,\, &\text{if } \,\,\, x\in\ocal,\\
\,\,\,\max\{-1,u(x)\} \,\,\, &\text{if } \,\,\, x\in\ical\,.
\end{cases}
\end{equation*}
If $G$ satisfies \eqref{Eq:HipothesesG}, then
$$ \ecal(v,\Omega) \leq \ecal(u,\Omega) \quad 
\text{ and }
\quad \ecal(w,\Omega) \leq \ecal(u,\Omega) \,.  $$
\end{lemma}

\begin{proof}
We first establish the result for $v$. Let us show that  $\ecal_\mathrm{K}(v) \leq \ecal_\mathrm{K}(u)$. Note that $v\in \widetilde{\H}^K_{\mathrm{odd}}(\Omega)$. Thus, by using the expression of the kinetic energy given in \eqref{Eq:ShortExpressionEnergy} and the fact that $\overline{K}(x,y) > \overline{K}(x,y^\star)> 0$ if $x,y\in \ocal$ ---see \eqref{Eq:KernelInequality}---, we only need to check that $|v(x)-v(y)|^2\leq |u(x)-u(y)|^2$ and $v^2(x)\leq u^2(x)$ whenever $x,y\in\ocal$. The first condition follows from the equivalence
$$ \big||u(x)|-|u(y)|\big|^2\leq |u(x)-u(y)|^2 \Longleftrightarrow w(x)w(y) \leq |w(x)w(y)|,  
$$
while the second one is trivial and it is in fact an equality. Concerning the potential energy, since $G$ is an even function we have that $\ecal_\mathrm{P}(v) = \ecal_\mathrm{P}(u)$, and therefore we get the desired result for $v$ by adding the kinetic and potential energies.

We show now the result for $w$. Let us show that  $\ecal_\mathrm{K}(w) \leq \ecal_\mathrm{K}(u)$. As before, $w\in \widetilde{\H}^K_{\mathrm{odd}}(\Omega)$ and thus, in view of \eqref{Eq:ShortExpressionEnergy} and the kernel inequality \eqref{Eq:KernelInequality}, we only need to check that $|w(x)-w(y)|^2\leq |u(x)-u(y)|^2$ and $w^2(x) \leq u^2(x)$ whenever $x,y\in\ocal$. The first inequality is trivial whenever $u(x)\leq 1$ and $u(y)\leq 1$, or $u(x)\geq 1$ and $u(y)\geq 1$. If $u(x)\geq 1$ and $u(y)\leq 1$, then $ |u(x)-u(y)|^2-|w(x)-w(y)|^2 = |u(x)-u(y)|^2-|1-u(y)|^2 = (u(x)-1))^2+2(u(x)-1)(1-u(y)) \geq 0$. The second inequality follows from the fact that $w^2(x) = u^2(x)$ when $u(x)\leq 1$, while $w^2(x) = 1 \leq u^2(x)$ if $u(x)\geq 1$. Concerning the potential energy, since $G$ is such that $G(x)\geq G(1) = G(-1) = 0$ if $|x|\leq 1$, then clearly $\ecal_\mathrm{P}(w) \leq \ecal_\mathrm{P}(u)$, and therefore we get the desired result by adding the kinetic and potential energies.
\end{proof}



Next we present a result that will be used later to show that a function $u\in \widetilde{\H}^K_{0}(\Omega)$ that minimizes the energy $\ecal$, only among doubly radial functions, is actually a weak solution to a semilinear Dirichlet problem in $\Omega$. In its proof we do not need to use the kernel $\overline{K}$.

\begin{proposition}
	\label{Prop:WeakSolutionForAllTestFunctions}
	Let $\Omega \subset \R^{2m}$ be a bounded set of double revolution and let $L_K \in \lcal_0$ with kernel $K$ radially symmetric. Let $u\in \widetilde{\H}^K_{0}(\Omega)$ such that
	$$
	\int_{\R^{2m}}\int_{\R^{2m}} \{u(x)-u(y)\}\{\xi(x)-\xi(y)\} K(|x-y|) \d x \d y = \int_{\R^{2m}} f(u(x)) \xi(x) \d x
	$$
	for every $\xi \in C^\infty_0(\Omega)$ that is doubly radial. Then, $u$ is a weak solution to
	$$
	\beqc{\PDEsystem}
	L_K u &=& f(u) & \text{in } \Omega\,,\\
	u &=& 0 & \text{in } \R^{2m}\setminus \Omega\,,
	\eeqc
	$$
	i.e.,
	$$
	\int_{\R^{2m}}\int_{\R^{2m}} \{u(x)-u(y)\}\{\eta(x)-\eta(y)\} K(|x-y|) \d x \d y = \int_{\R^{2m}} f(u(x)) \eta(x) \d x
	$$
	for every $\eta \in C^\infty_0(\Omega)$ (not necessarily symmetric).
\end{proposition}

\begin{proof}
	Let $\eta \in C^\infty_0(\Omega)$. We define its associated doubly radial function as
	$$
	\overline{\eta}(x) := \average_{O(m)^2}\eta(R x)\d R\,.
	$$
	
	Now, on the one hand, given $R\in O(m)^2$ and using the change $x = R\tilde{x}$, $y = R \tilde{y}$ and the fact that $u$ is doubly radial, we get
	\begin{align*}
	&\int_{\R^{2m}}\int_{\R^{2m}} \{u(x)-u(y)\}\{\eta(x)-\eta(y)\} K(|x-y|) \d x \d y = \\
	&\quad \quad \quad = \int_{\R^{2m}}\int_{\R^{2m}} \{u(x)-u(y)\}\{\eta(R x)-\eta(R y)\} K(|x-y|) \d x \d y\,.
	\end{align*}
	Integrating the previous expression with respect to $R$ and taking the average, we get
	\begin{align*}
	& \int_{\R^{2m}}\int_{\R^{2m}} \{u(x)-u(y)\}\{\eta(x)-\eta(y)\} K(|x-y|) \d x \d y = \\
	&\quad \quad \quad =\average_{O(m)^2} \int_{\R^{2m}}\int_{\R^{2m}} \{u(x)-u(y)\}\{\eta(R x)-\eta(R y)\} K(|x-y|) \d x \d y \d R \\
	%&\quad \quad \quad= \int_{\R^{2m}}\int_{\R^{2m}} \{u(x)-u(y)\}\left \{\average_{O(m)^2}\eta(R x)\d R-\average_{O(m)^2} \eta(Ry) \d R \right \} K(|x-y|) \d x \d y \\
	&\quad \quad \quad= \int_{\R^{2m}}\int_{\R^{2m}} \{u(x)-u(y)\}\left \{\overline{\eta}(x) -\overline{\eta}(y)  \right \} K(|x-y|) \d x \d y \,.
	\end{align*}
	
	On the other hand, using also the change $x = R\tilde{x}$, we have
	$$
	\int_{\Omega} f(u(x)) \eta(x) \d x = \int_{\Omega} f(u(R^{-1}x)) \eta(x) \d x = \int_{\Omega} f(u(x)) \eta(Rx) \d x\,.
	$$
	Integrating this expression with respect to $R$ and taking the average, we get
	$$
	\int_{\Omega} f(u(x)) \eta(x) \d x = \average_{O(m)^2} \int_{\Omega} f(u(x)) \eta(Rx) \d x \d R = \int_{\Omega} f(u(x))\overline{\eta}(x) \d x\,.
	$$
	
	Hence, since $\overline{\eta} \in C^\infty_0(\Omega)$ is doubly radial, we have
	\begin{align*}
		&\int_{\R^{2m}}\int_{\R^{2m}} \{u(x)-u(y)\}\{\eta(x)-\eta(y)\} K(|x-y|) \d x \d y - \int_{\Omega} f(u(x)) \eta(x) \d x \\
		&\quad \quad= \int_{\R^{2m}}\int_{\R^{2m}} \{u(x)-u(y)\}\left \{\overline{\eta}(x) -\overline{\eta}(y)  \right \} K(|x-y|) \d x \d y - \int_{\Omega} f(u(x))\overline{\eta}(x) \d x \\
		&\quad \quad= 0\,,
	\end{align*}
	and thus the result is proved.
\end{proof}



\begin{remark}
	\label{Remark:InteriorRegularity}
	This proposition combined with some regularity results for operators in the class $\lcal_0(n,\s,\lambda, \Lambda)$ yield that bounded minimizers among doubly radial functions of the energy $\ecal(\cdot,\Omega)$ are classical solutions to $L_K u = f(u)$ in $\Omega$. Indeed, if $w\in L^\infty (\R^n)$ is a weak solution to $L_K w = h$ in $B_1\subset \R^n$, then
	\begin{equation}
	\label{Eq:C2sEstimate}
	\norm{w}_{C^{2\s} (\overline{B_{1/2}})} \leq C\bpar{\norm{h}_{L^\infty (B_1)} + \norm{w}_{L^\infty  (\R^n)}}.
	\end{equation} 
	If, in addition, $w \in C^\alpha (\R^n)$ with $\alpha + 2\s$ not an integer, then
	\begin{equation}
	\label{Eq:Calpha->Calpha+2sEstimate}
	\norm{w}_{C^{\alpha + 2\s} (\overline{B_{1/2}})} \leq C\bpar{\norm{h}_{C^{\alpha} (\overline{B_1})} + \norm{w}_{C^\alpha (\R^n)} },
	\end{equation}
	where the previous two constants $C$ depend only on $n$, $\s$, $\lambda$ and $\Lambda$ (see \cite{RosOton-Survey,SerraC2s+alphaRegularity} and the references therein).
	
	From the previous estimates and using the same argument as in Corollaries 2.4 and 2.5 of \cite{RosOtonSerra-Regularity}, \eqref{Eq:C2sEstimate} and \eqref{Eq:Calpha->Calpha+2sEstimate} can be rewriten, respectively, as
	\begin{equation}
	\label{Eq:C2sEstimateBalls}
	\norm{w}_{C^{2\s} (\overline{B_{1/4}})} \leq C\bpar{\norm{h}_{L^\infty (B_1)} + \norm{w}_{L^\infty  (B_1)} + \norm{\dfrac{w(x)}{(1+|x|)^{n+2\s}}}_{L^1(\R^n)} },
	\end{equation}
	and 
	\begin{equation}
	\label{Eq:Calpha->Calpha+2sEstimateBalls}
	\norm{w}_{C^{\alpha + 2\s} (\overline{B_{1/4}})} \leq C\bpar{\norm{h}_{C^{\alpha} (\overline{B_1})} + \norm{w}_{C^\alpha (\overline{B_1})} + \norm{\dfrac{w(x)}{(1+|x|)^{n+2\s}}}_{L^1(\R^n)} }.
	\end{equation}
	Therefore, by applying these estimates (maybe a translated and rescaled version of them) to a weak solution $u\in L^\infty(\R^{2m})$ of $L_K u = f(u)$ in $\Omega$, with $f$ a $C^1$ nonlinearity, we easily conclude that $u$ is a classical solution, that is, the equation makes sense pointwise.
\end{remark}




