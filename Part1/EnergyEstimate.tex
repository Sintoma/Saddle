%%%%%%%%%%%%%%%%%%%%%%%%%%%%%%%%%%%%%%%%%%%%%%%%%%%%%%%%%%%%%%%%%%%%%%
%%%%%%%%%%%%%%%%%%%%%%%%%%%%%%%%%%%%%%%%%%%%%%%%%%%%%%%%%%%%%%%%%%%%%%
\section{An energy estimate for doubly radial odd minimizers}
\label{Sec:EnergyEstimate}

In this section we present an estimate for the energy in $B_S$ of minimizers in the space $\widetilde{\H}^K_{0, \mathrm{odd}}(B_R)$. That is, we prove Theorem~\ref{Th:EnergyEstimate}. In order to establish this result, we follow the ideas of Savin and Valdinoci in \cite{SavinValdinoci-EnergyEstimate}, where they show the same estimate but for minimizers without any symmetry. The strategy is to compare $u$ with a suitable competitor which is constructed combining $u$ with an auxiliary function. In our case, since $u$ is a minimizer among functions in $\widetilde{\H}^K_{0, \mathrm{odd}}(B_R)$, we need to adapt the auxiliary function used in \cite{SavinValdinoci-EnergyEstimate} so that the resulting competitor is admissible, i.e., is a doubly radial function which is odd with  respect to the Simons cone $\ccal$.

The auxiliary function needed to build the competitor is defined as follows. For points $x\in \ocal$, we set
$$ \Psi_S(x) := \max\left\{-1+2\min\{(|x|-S-1)_+,1\},-\dist(x,\ccal) \right\},  $$
and we define it in $\ical$ by considering its odd reflection. It is clear that it is a bounded function with $||\Psi_S||_{L^\infty(\R^{2m})}=1$. In our arguments we will also use the following function and set:
$$ d_S(x) := \max\left\{1,\min\{S+1-|x|,\dist(x,\ccal)\} \right\},  $$
and
\begin{equation}
\label{Eq:DefOmegaS}
\Omega_S := \left( B_{S+2}\setminus \overline{B_s} \right) \cup \left( B_{S+2} \cap \{\dist(x,\ccal)< 1\}\right).
\end{equation} 

\begin{figure}
	\centering
	\begin{subfigure}{0.49\textwidth}
		\centering
		\definecolor{lila_custom}{RGB}{201,69,254}
\definecolor{naranja_custom}{RGB}{255,148,0}


\begin{tikzpicture}[y=0.80pt, x=0.80pt, yscale=-1.000000, xscale=1.000000, inner sep=0pt, outer sep=0pt]


%region-down
\path[scale=0.938,fill=lila_custom, opacity = 0.7, line width=0.400pt] (248.0459,242.6193) ..
controls (275.7060,284.1286) and (276.2013,310.5581) .. (276.7985,330.5197) ..
controls (233.8448,330.5197) and (216.3370,330.5469) .. (160.2334,330.5469) ..
controls (186.2822,304.6380) and (219.4410,271.2569) .. (248.0459,242.6193) --
cycle;



%big circle
\path[draw=black,line join=miter,line cap=butt,line width=1.4pt]
(243.0000,193.8560) .. controls (275.7522,226.7370) and (289.9164,269.7115) ..
(289.9164,309.8622);

%big circle2
\path[draw=naranja_custom,line join=miter,line cap=butt,line width=1.8pt]
(243.0000+11,193.8560+12) .. controls (275.7522-9,226.7370-6) and (289.9164,258.7115) ..
(289.9164,309.8622);

%x-axis    
\path[draw=black,line join=miter,line cap=butt,line width=1.5pt]
(126.5055,309.8063) -- (330.4872,309.8063);

%x-axis cap    
\path[draw=black,fill=black,even odd rule,line width=0.497pt]
(330.4872,309.8063) -- (328.1493,312.1289) -- (336.3320,309.8063) --
(328.1493,307.4838) -- cycle;

%y-axis
\path[draw=black,line join=miter,line cap=butt,line width=1.5pt]
(127.0562,310.75) -- (127.0562,99.5223);

%y-axis cap    
\path[draw=black,fill=black,even odd rule,line width=0.497pt] (127.0562,99.5223)
-- (129.3788,101.8602) -- (127.0562,93.6775) -- (124.7336,101.8602) -- cycle;



%medium circle
\path[draw=black,line join=miter,line cap=butt,line width=1.4pt]
(221.4938,215.3997) .. controls (250.7785,244.9692) and (259.4986,284.9193) ..
(259.4986,309.8622);

%small circle    
\path[draw=black,line join=miter,line cap=butt,line width=1.4pt]
(199.9500,236.9435) .. controls (222.0483,259.0975) and (229.1002,286.5606) ..
(229.1002,309.8622);

%cone   
\path[draw=black,line join=miter,line cap=butt,miter limit=4.00,even odd
rule,line width=1.57pt] (126.7629,310.0643) -- (326.1869,110.8844);

%cone+1    
\path[draw=black,line join=miter,line cap=butt,even odd rule,line width=0.704pt]
(150.2563,309.8481) -- (318.0555,142.0518);

%cone-1 
%\path[draw=black,line join=miter,line cap=butt,even odd rule,line width=0.704pt]
%(127.1847,287.2416) -- (299.4946,114.9526); 


%Linea Cota
\path[draw=black,line join=miter,line cap=butt,line width=0.5pt]
(298.3343,142.4919) -- (306.3936,150.7321);


%Flecha Izquierda Cota
\path[draw=black,fill=black,even odd rule,line width=0.200pt]
(298.9657,143.1375) -- (300.2914,143.1522) -- (297.3269,141.4619) --
(298.9510,144.4632) -- cycle;

%Flecha Derecha Cota
\path[draw=black,fill=black,even odd rule,line width=0.200pt]
(305.7622,150.0865) -- (304.4364,150.0718) -- (307.4010,151.7621) --
(305.7769,148.7608) -- cycle;

%Cota naranja
\path[draw=black,line join=miter,line cap=butt,line width=0.5pt]
(300, 230) .. controls (300, 235) and (295, 250) ..
(285, 250);

%Flecha cota naranja
\path[draw=black,fill=black,even odd rule,line width=0.200pt]
(285, 250) -- (285+0.9375, 250+0.9375) -- (285-2.3437, 250) --
(285+0.9375, 250-0.9375) -- cycle;

\node at (315,138) {\normalsize $\mu^{-1}$};
\node at (127,85) {\normalsize $|x''|$};
\node at (350, 311) {\normalsize $|x'|$};
\node at (228,320) {\normalsize $S$};
\node at (255, 320) {\normalsize $S\!+\!1$};
\node at (292, 320) {\normalsize $S\!+\!2$};
\node at (334, 101) {\normalsize $\ccal$};
\node at (298, 220) {\normalsize $\Psi_S\! = \!1$};
\node at (220, 286) {\normalsize $\Psi_S\! = \!-1$};


\end{tikzpicture}


	\end{subfigure}
	\begin{subfigure}{0.49\textwidth}
		\centering
		\definecolor{azul_custom}{RGB}{66,240,209}
\definecolor{lila_custom}{RGB}{201,69,254}
\definecolor{naranja_custom}{RGB}{255,148,0}


\begin{tikzpicture}[y=0.80pt, x=0.80pt, yscale=-1.000000, xscale=1.000000, inner sep=0pt, outer sep=0pt]


%region
\path[fill=azul_custom,line cap=round,miter limit=4.00,line width=1.216pt]
(210.3958,249.7906) .. controls (188.0101,272.1763) and (172.1936,287.9233) ..
(150.1544,309.9625) .. controls (148.2221,309.9625) and (137.7690,309.7049) ..
(127.0089,309.7371) .. controls (127.0089,300.7477) and (127.0826,301.1491) ..
(127.0826,287.3606) .. controls (136.2973,278.1061) and (177.3049,237.1011) ..
(187.7910,226.5786) .. controls (170.0299,214.0930) and (149.7638,207.7938) ..
(127.0312,207.7938) .. controls (127.0312,190.9391) and (126.9993,161.0751) ..
(126.9993,146.8284) .. controls (170.9394,146.8284) and (208.6039,162.4681) ..
(243.0000,193.8560) .. controls (271.9385,225.9716) and (289.9164,263.3638) ..
(289.9164,309.8622) .. controls (260.7310,309.8622) and (257.9656,309.8622) ..
(229.1006,309.8622) .. controls (229.1006,279.3951) and (221.7362,269.4328) ..
(210.3958,249.7907) -- cycle;


%x-axis    
\path[draw=black,line join=miter,line cap=butt,line width=1.5pt]
(126.5055,309.8063) -- (330.4872,309.8063);

%x-axis cap    
\path[draw=black,fill=black,even odd rule,line width=0.497pt]
(330.4872,309.8063) -- (328.1493,312.1289) -- (336.3320,309.8063) --
(328.1493,307.4838) -- cycle;

%y-axis
\path[draw=black,line join=miter,line cap=butt,line width=1.5pt]
(127.0562,310.75) -- (127.0562,99.5223);

%y-axis cap    
\path[draw=black,fill=black,even odd rule,line width=0.497pt] (127.0562,99.5223)
-- (129.3788,101.8602) -- (127.0562,93.6775) -- (124.7336,101.8602) -- cycle;


%big circle
\path[draw=black,line join=miter,line cap=butt,line width=1.4pt]
(127.0312,146.9770) .. controls (166.9982,146.9770) and (210.2478,160.9750) ..
(243.0000,193.8560) .. controls (275.7522,226.7370) and (289.9164,269.7115) ..
(289.9164,309.8622);

%medium circle
\path[draw=black,line join=miter,line cap=butt,line width=1.4pt]
(127.0312,177.3949) .. controls (151.7088,177.3949) and (192.2090,185.8302) ..
(221.4938,215.3997) .. controls (250.7785,244.9692) and (259.4986,284.9193) ..
(259.4986,309.8622);

%small circle    
\path[draw=black,line join=miter,line cap=butt,line width=1.4pt]
(127.0312,207.7933) .. controls (150.2506,207.7933) and (177.3815,214.3180) ..
(199.9500,236.9435) .. controls (222.0483,259.0975) and (229.1002,286.5606) ..
(229.1002,309.8622);

%cone   
\path[draw=black,line join=miter,line cap=butt,miter limit=4.00,even odd
rule,line width=1.57pt] (126.7629,310.0643) -- (326.1869,110.8844);

%cone+1    
\path[draw=black,line join=miter,line cap=butt,even odd rule,line width=0.704pt]
(150.2563,309.8481) -- (318.0555,142.0518);

%cone-1 
\path[draw=black,line join=miter,line cap=butt,even odd rule,line width=0.704pt]
(127.1847,287.2416) -- (299.4946,114.9526); 

%Linea Cota
\path[draw=black,line join=miter,line cap=butt,line width=0.5pt]
(298.3343,142.4919) -- (306.3936,150.7321);


%Flecha Izquierda Cota
\path[draw=black,fill=black,even odd rule,line width=0.200pt]
(298.9657,143.1375) -- (300.2914,143.1522) -- (297.3269,141.4619) --
(298.9510,144.4632) -- cycle;

%Flecha Derecha Cota
\path[draw=black,fill=black,even odd rule,line width=0.200pt]
(305.7622,150.0865) -- (304.4364,150.0718) -- (307.4010,151.7621) --
(305.7769,148.7608) -- cycle;

\node at (315,138) {\normalsize $\mu^{-1}$};
\node at (127,85) {\normalsize $|x''|$};
\node at (350, 311) {\normalsize $|x'|$};
\node at (228,320) {\normalsize $S$};
\node at (255, 320) {\normalsize $S\!+\!1$};
\node at (292, 320) {\normalsize $S\!+\!2$};
\node at (334, 101) {\normalsize $\ccal$};
\node at (264, 250) {\normalsize $\Omega_S$};

\end{tikzpicture}











	\end{subfigure}
	\caption{(a) The $1$ and $-1$ level sets of $\Psi_S$. (b) The set $\Omega_S$.}
	\label{Fig:PsiSandOmegaS}
\end{figure}

Note that both $\Psi_S$ and $d_S$ are Lipschitz functions, with Lipschitz norm independent of $S$. Moreover $\Psi_S$ is odd and $d_S$ even with respect to the Simons cone. Regarding the set $\overline{\Omega_S}$, we can see it as the preimage of $1$ through $d_S$ inside $\overline{B_{S+2}}$ (see Figure~\ref{Fig:PsiSandOmegaS}). Furthermore, its measure is well known to be of order $2m-1$ (see the proof of the energy estimate in \cite{CabreTerraI}). That is,
\begin{equation}
\label{Eq:MeasureOmegaS}
|\Omega_S| \leq C\,S^{2m-1}.
\end{equation}


Now we show some auxiliary results concerning the previous definitions, needed in the proof of Theorem~\ref{Th:EnergyEstimate}.

\begin{lemma}
\label{Lemma: AdaptedLipschitzConditionWith_dFunction}
Given $S>0$, if either $(x,y) \in \left(\Omega_S\cap \ocal\right) \times \ical$ or $(x,y)\in \left(B_{S+2}\cap \ocal\right) \times \ocal$, then
$$ |\Psi_S(x) - \Psi_S(y)| \leq C \frac{|x-y|}{d_S(x)} \ \ \ \ \ \textrm{whenever} \ \ |x-y|\leq d_S(x), $$
with $C>0$ independent of $S$.
\end{lemma}

\begin{proof}
Note first that if $x\in \Omega_S \cap \ocal$, then $d_S(x)=1$ and the result is trivial by the Lipschitz continuity of $\Psi_S$. Hence, we only need to establish the result for the case $x\in B_S\cap \{\dist(x,\ccal)\geq 1\} \cap \ocal$ and $y\in \ocal$. Under these hypotheses, we have that $\Psi_S(x)=-1$ (see Figure~\ref{Fig:PsiSandOmegaS}) and $d_S(x) = \min\{S+1-|x|,\dist(x,\ccal)\}$. Moreover, since $x\in B_S$ and $\dist(x,\ccal)\geq 1$ we get $d_S(x) \leq S+1-|x|$. Therefore, if $|x-y|\leq d_S(x)$ we obtain
$$ |y|\leq |x-y| + |x| \leq d_S(x)+|x| \leq S+1. $$

Now we distinguish two cases, either $\{\dist(\cdot,\ccal)\geq 1\}$ or $\{\dist(\cdot,\ccal)\leq 1\}$. Assume first that $y\in B_{S+1} \cap \{\dist(\cdot,\ccal)\geq 1\}\cap \ocal$. Then, $\Psi_S(y)=-1$ and the result is trivial from being also $\Psi_S(x)=-1$. Thus, it only remains to show the result in the case $x\in B_S \cap \{\dist(\cdot,\ccal)\geq 1\}\cap \ocal$ and $y\in B_{S+1} \cap \{\dist(\cdot,\ccal)\leq 1\}\cap \ocal$. Note that under these assumptions, $\Psi_S(x)=-1$ and $\Psi_S(y)=-\dist(y,\ccal)$.


Given $x,y \in \R^{2m}$ it is easy to prove by using the triangular inequality and the definition of distance to the cone that
\begin{equation} \label{Eq:TriangularCone}
\dist(x,\ccal) \leq |x-y| + \dist(y,\ccal).
\end{equation}
Therefore we have
\begin{equation} \label{Eq:TriangularCone2}
1-|x-y|-\dist(y,\ccal) \leq 1-\dist(x,\ccal) \leq 0
\end{equation}
Now, multiplying \eqref{Eq:TriangularCone} by $|1-\dist(y,\ccal)|$ and using \eqref{Eq:TriangularCone2} we obtain
\begin{align*}
|1-\dist(y,\ccal)|\,\dist(x,\ccal) &\leq |1-\dist(y,\ccal)| \left(|x-y| + \dist(y,\ccal)\right) \\
%&= \left(1-\dist(y,\ccal)\right) \left(|x-y| + \dist(y,\ccal)\right) \\
&= |x-y|+\dist(y,\ccal) \left\{ -|x-y|+ 1- \dist(y,\ccal) \right\} \\
&\leq |x-y|.
\end{align*}

Hence,
$$ |\Psi_S(x)-\Psi_S(y)| = |1-\dist(y,\ccal)| \leq \frac{|x-y|}{\dist(x,\ccal)} \leq  \frac{|x-y|}{d_S(x)},$$
completing the proof.
\end{proof}

Another auxiliary result that we will need in the proof of Theorem~\ref{Th:EnergyEstimate} is the following estimate for the function $d_S$. 

\begin{lemma}
\label{Lemma:Integrability_dFunction}
Given $S>0$ we have
$$ \int_{B_{S+2}} d_S(x)^{-2\s} \d x \leq \begin{cases}
C \ S^{2m-2\s}\ \ \ \ &\textrm{if } \ \ \s\in(0,1/2),\\
C\ \log(S)\,S^{2m-2\s}\ \ \ \ &\textrm{if } \ \ \s=1/2,\\
C \ S^{2m-1}\ \ \ \ &\textrm{if } \ \ \s\in(1/2,1),\\
\end{cases} $$
with $C>0$ independent of $S$ and only depending on $m$ and $\s$.
\end{lemma}


\begin{proof}
In order to prove this result we first note that $d_S(x)=1$ in $\Omega_S$. Thus, the contribution to the integral of this part is just the measure of the set $\Omega_S$ (see equation \eqref{Eq:MeasureOmegaS}). That is,
$$\int_{\Omega_S} d_S(x)^{-2\s} \d x = |\Omega_S| \leq C\,S^{2m-1}.$$

For the other part of the integral we can write
\begin{align*}
\int_{B_{S+2}\setminus \Omega_S} d_S(x)^{-2\s} \d x &= \int_{B_{S}\cap \dist\{x,\ccal\}>1} d_S(x)^{-2\s} \d x \\
& \leq \int_{B_{S}\cap \dist\{x,\ccal\}>1} \left( S+1-|x| \right)^{-2\s} \d x + \int_{B_{S}\cap \dist\{x,\ccal\}>1} \dist(x,\ccal)^{-2\s} \d x.
\end{align*}
The desired estimate for the first integral can be found in \cite{SavinValdinoci-EnergyEstimate}. Therefore, in order to complete the proof it only remains to estimate the second integral. It can be estimated by writing it in $(y,z)$ variables, where
$$
y = \dfrac{|x'|+|x''|}{\sqrt{2}} \, \quad \text{ and } z = \dfrac{|x'|-|x''|}{\sqrt{2}}\,.
$$
In this case, $z$ is the signed distance to the cone. Thus,
\begin{align*}
\int_{B_{S}\cap \dist\{x,\ccal\}>1} \dist(x,\ccal)^{-2\s} \d x &\leq C \int \int_{B_{S}\cap \{y\geq|z|>1\}} |z|^{-2\s} \, (y^2-z^2)^{m-1} \d y\d z \\
& \leq C \int \int_{B_{S}\cap \{y\geq|z|>1\}} |z|^{-2\s} \, y^{2m-2} \d y\d z \\
& \leq C\, \int_1^S \d z \int_0^S \d y\ z^{-2\s} \, y^{2m-2} \\
& \leq C\, \left(\int_1^S z^{-2\s} \d z \right)  \left(  \int_0^S \d y \, y^{2m-2} \right) \\
& \leq \begin{cases}
C \ S^{2m-2\s}\ \ \ \ &\textrm{if } \ \ \s\in(0,1/2),\\
C\ \log(S)\,S^{2m-2\s}\ \ \ \ &\textrm{if } \ \ \s=1/2,\\
C \ S^{2m-1}\ \ \ \ &\textrm{if } \ \ \s\in(1/2,1).\\
\end{cases}
\end{align*}
\end{proof}

Note that $(y,z)$ variables are very useful when dealing with doubly radial odd functions ---see \cite{CabreTerraI, CabreTerraII, Cabre-Saddle, Cinti-Saddle,Cinti-Saddle2, Felipe-Sanz-Perela:SaddleFractional}.

The last auxiliary result we need in order to establish the energy estimate is the following inequality.

\begin{lemma}
\label{Lemma: InteractionInequalityMinimumFunction}
Let $A\subset B_R \subset \R^{2m}$ be a set of double revolution such that $A^\star = A$ and let $\omega, \phi, \varphi \in \widetilde{\H}^K(B_R)$ be such that
$$\begin{cases}
\omega = \phi \leq \varphi \ \ \ \ \textrm{in } \ \ \ \ocal \setminus A\,,\\
\omega = \varphi \leq \phi \ \ \ \ \textrm{in } \ \ \ \ocal \cap A\,.
\end{cases}$$
Then, if $K$ satisfies \eqref{Eq:KernelInequality}, it holds
\begin{align*}
I_\omega(\ocal\cap A, \ocal \setminus A) \leq I_\phi(\ocal\cap A, \ocal \setminus A) + I_\varphi(\ocal\cap A, \ocal \setminus A)\,,
\end{align*}
where $I_w(\cdot, \cdot)$ is the interaction defined in \eqref{Eq:DefIw}.
\end{lemma}

\begin{proof}
A simple computation shows that if $x\in \ocal \cap A$ and $y\in \ocal \setminus A$ we have that
$$ |\phi(x)-\phi(y)|^2+|\varphi(x)-\varphi(y)|^2\geq |\omega(x)-\omega(y)|^2. $$
Indeed,
\begin{align*}
|\phi(x)-\phi(y)|^2+|\varphi(x)&-\varphi(y)|^2 - |\omega(x)-\omega(y)|^2 \\
&= |\phi(x)-\phi(y)|^2+|\varphi(x)-\varphi(y)|^2 - |\varphi(x)-\phi(y)|^2 \\
&= \phi^2(x)-2\phi(x)\phi(y)+\varphi^2(y)-2\varphi(x)\varphi(y)+2\varphi(x)\phi(y) \\
&= \left( \phi(x) - \varphi(y)\right) ^2+2\left( \phi(x)-\varphi(x) \right) \left( \varphi(y)-\phi(y) \right) \\
&\geq 0.
\end{align*}
Therefore, by using this inequality and the reflexion property of the kernel, \eqref{Eq:KernelInequality}, we obtain
\begin{align*}
I_\phi(\ocal\cap A, \ocal \setminus A) &+ I_\varphi(\ocal\cap A, \ocal \setminus A) - I_\omega(\ocal\cap A, \ocal \setminus A) =\\
&\hspace{-26mm}= \int_{\ocal\cap A} \d x \int_{\ocal\setminus A} \d y \Big( 2\left\{\phi^2(x)+\phi^2(y)+\varphi^2(x)+\varphi^2(y)-\omega^2(x)-\omega^2(y) \right\} \overline{K}(x,y^\star) \Big.\\
&\hspace{-18mm}\Big.+  \left\{|\phi(x)-\phi(y)|^2+|\varphi(x)-\varphi(y)|^2-|\omega(x)-\omega(y)|^2 \right\} \left\{\overline{K}(x,y)-\overline{K}(x,y^\star)\right\}\Big)\\
&\hspace{-26mm}\geq 2\int_{\ocal\cap A} \d x \int_{\ocal\setminus A} \d y \left\{
\phi^2(x)+\varphi^2(y)\right\} \overline{K}(x,y^\star) \geq 0.
\end{align*}
\end{proof}



With all these ingredients we can establish now the sharp energy estimate.

\begin{proof}[Proof of Theorem~\ref{Th:EnergyEstimate}]

Note that, by Lemma~\ref{Lemma:DecreaseEnergy}  we can assume without loss of generality that if $u$ is a minimizer of $\ecal$ in $B_R$, then $-1 \leq u \leq 1$, $u \geq 0$ in $\ocal$, and $u \leq 0$ in $\ical$. 

\textbf{Step 1. We show that $0\leq u < 1$ in $\ocal$.} 

In order to prove it we first need to show that $u$ is a weak solution of
\begin{equation}
\label{Eq:ProofEnergyEstimateProblemBR}
	\beqc{\PDEsystem}
	L_K  u &=& f(u) & \textrm{ in } B_R\,,\\
	u &=& 0 & \textrm{ in }\R^{2m} \setminus B_R.
	\eeqc
\end{equation}
To see this, we consider on the one hand perturbations $u +  \varepsilon \xi$, with $\xi \in \widetilde{\H}^K_{0, \,\mathrm{odd}}(B_R)$ and such that $\xi$ has compact support in $B_R$. Then, since $u$ is a minimizer among functions in $\widetilde{\H}^K_{0, \,\mathrm{odd}}(B_R)$, we get
$$
0 = \dfrac{\d}{\d \varepsilon}\evalat{\varepsilon = 0} \ecal(u +  \varepsilon \xi, B_R) = \langle u,\xi \rangle_{\widetilde{\H}^K_0(B_R)} - \langle f(u),\xi \rangle_{L^2(B_R)}\,.
$$
On the other hand, take $\xi \in \widetilde{\H}^K_{0, \,\mathrm{even}}(B_R)$. Since $u$ is odd with respect to the Simons cone, so is $f(u)$. Then, by Remark~\ref{Remark:DecompositionHK} and the same decomposition in $L^2(B_R)$, we find that
$$
\langle v_R,\xi \rangle_{\widetilde{\H}^K_0(B_R)} = 0 \quad \textrm{ and } \quad  \langle f(v_R),\xi \rangle_{L^2(B_R)} = 0\,.
$$
Therefore, we have that
$$
\langle u,\xi \rangle_{\widetilde{\H}^K_0(B_R)} = \langle f(u),\xi \rangle_{L^2(B_R)}
$$
for every $\xi \in\widetilde{\H}^K_0(B_R)$ with compact support in  $B_R$. Thus,
$$
\int_{\R^{2m}}\int_{\R^{2m}} \{u(x)-u(y)\}\{\xi(x)-\xi(y)\} K(|x-y|) \d x \d y = \int_{\R^{2m}} f(u(x)) \xi(x) \d x
$$
for every $\xi \in C^\infty_0(\Omega)$ that is doubly radial. 

By Proposition~\ref{Prop:WeakSolutionForAllTestFunctions}, $u$ is a weak solution of \eqref{Eq:ProofEnergyEstimateProblemBR}, and in view of the regularity theory for operators in $\lcal_0$ (see Remark~\ref{Remark:InteriorRegularity}), since $u$ is bounded, it is a classical solution.

From being $u$ a classical solution it is easy to show that it cannot be $1$ or $-1$ and therefore that it satisfies $0\leq u < 1$ in $\ocal$. In order to see this, let us suppose that there exists $x_0\in\R^{2m}$ such that $|u(x_0)|=1$. Without loss of generality we can assume that $x_0\in\ocal\cap B_R$. Then, from equation \eqref{Eq:ProofEnergyEstimateProblemBR} and the fact of being $x_0$ an absolute maximum, we can arrive at a contradiction:
\begin{align*}
0 &= f(1) = f(u(x_0)) = L_K u(x_0) = \int_\ocal (1-u(y)) \overline{K}(x,y) + (1+u(y)) \overline{K}(x,y^\star)  \d y \\
&\geq \int_\ocal (1-u(y)) \overline{K}(x,y^\star) + (1+u(y)) \overline{K}(x,y^\star)  \d y = 2\int_\ocal \overline{K}(x,y^\star) \d y\\
&>0.
\end{align*}

\textbf{Step 2. We build a suitable competitor for $u$ and compare their energies.}

Now, for $x\in \ocal$ we define
$$ 
v(x) := \min\{u(x),\Psi_S(x)\}, 
$$
and we define it in $\ical$ by considering its odd reflection with respect to the Simons cone. Let us also define
$$
A = \{v=\Psi_S\} = \{\Psi_S \leq u\}. 
$$
Then, it is easy to check that we have the inclusions
\begin{equation}
\label{Eq:EnergyEstimateProofInclusionsA}
	B_{S+1} \subset A \subset B_{S+2}\,.
\end{equation}
Indeed, note first that we only need to prove it inside $\ocal$, by the symmetry of $A$ with respect to the Simons cone. On the one hand, if $ x\in B_{S+1}\cap \ocal$, then $\Psi_S(x) = \max\{-1,-\dist(x,\ccal)\} \leq 0 \leq u(x)$, which yields $v(x) = \Psi_S(x)$ Thus, $x\in A\cap \ocal$. On the other hand, if $ x\in A\cap \ocal$ then $\Psi_S(x) \leq u(x) < 1$. This can only happen if $x\in B_{S+2}$.

Note that both $u$ and $v$ are equal outside $B_{S+2} \subset B_R$, and therefore $v$ is an admissible competitor. By comparing the energies of $u$ and $v$ we will obtain the desired estimate. 

Let us decompose the energy of $u$ in $B_R$ in terms of interactions between sets that involve $A$. That is, using expression \eqref{Eq:ShortExpressionEnergy}, we get
\begin{align*}
\ecal(u,B_R) &= \frac{1}{2}I_u(\ocal \cap A, \ocal \cap A) + I_u(\ocal \cap A, \ocal \setminus A) \\
& \hspace{5mm} + \frac{1}{2}I_u\big((\ocal \setminus A) \cap B_R, (\ocal \setminus A) \cap B_R\big) + I_u\big((\ocal \setminus A) \cap B_R, \ocal \setminus B_R\big) \\
& \hspace{5mm} + \int_A G(u) + \int_{B_R\setminus A} G(u)
\end{align*}
Since $u$ is a minimizer, $v=\Psi_S$ in $A$ and $u=v$ outside of $A$,  from the previous expression we obtain
\begin{align*}
0 &\leq \ecal(v,B_R)-\ecal(u,B_R) = \frac{1}{2}I_{\Psi_S}(\ocal \cap A, \ocal \cap A) - \frac{1}{2}I_u(\ocal \cap A, \ocal \cap A)\\
& \hspace{5mm} + I_v(\ocal \cap A, \ocal \setminus A) - I_u(\ocal \cap A, \ocal \setminus A) + \int_A G(\Psi_S) - \int_{A} G(u).
\end{align*}
Since $v = \min\{u,\Psi_S\}$ in $\ocal$ we can apply Lemma \ref{Lemma: InteractionInequalityMinimumFunction} with $\omega = v$, $\Psi_S = \varphi$, and $u= \phi$, to get $I_v(\ocal \cap A, \ocal\setminus A) \leq I_u(\ocal \cap A, \ocal\setminus A) + I_{\Psi_S}(\ocal \cap A, \ocal\setminus A)$. Therefore,
\begin{align*}
\frac{1}{2}I_u(\ocal \cap A, \ocal \cap A) + \int_{A} G(u) &\leq \frac{1}{2}I_{\Psi_S}(\ocal \cap A, \ocal \cap A) + I_{\Psi_S}(\ocal \cap A, \ocal \setminus A) + \int_A G(\Psi_S)  \\
&= \ecal(\Psi_S, A) \leq \ecal(\Psi_S,B_{S+2})
\end{align*}
From this and using \eqref{Eq:EnergyEstimateProofInclusionsA}, we deduce an estimate for the energy of $u$ in $B_S$ as follows.
\begin{align*}
\ecal(u,B_S) &\leq \frac{1}{2}I_u(\ocal \cap A, \ocal \cap A) + \int_{A} G(u) + I_u(\ocal \setminus B_{S+1}, \ocal \cap B_S) \\
& \leq  \ecal(\Psi_S,B_{S+2}) + I_u(\ocal \setminus B_{S+1}, \ocal \cap B_S).
\end{align*}
Thus, to obtain the desired energy estimate we only have to bound the right-hand side of the last inequality.


\textbf{Step 3. We estimate the remaining terms.}

In the following arguments, we use the definition of the energy that involves the original kernel $K$ and not $\overline{K}$.

\textbf{3.1. Estimate for $\ecal(\Psi_S,B_{S+2})$.}
First, by using the change of variables given by $(\cdot)^\star$ and the ellipticity of $K$, condition \eqref{Eq:Ellipticity}, we obtain
\begin{align*}
\ecal(\Psi_S,B_{S+2}) &= \frac{1}{4} \int_{B_{S+2}} \d x \int_{B_{S+2}} \d y\ |\Psi_S(x)-\Psi_S(y)|^2K(|x-y|)  \\
&\hspace{5mm} +\frac{1}{2} \int_{B_{S+2}} \d x \int_{\R^{2m} \setminus B_{S+2}} \d y \ |\Psi_S(x)-\Psi_S(y)|^2K(|x-y|) + \int_{B_{S+2}} G(\Psi_S) \\
&\leq \frac{1}{2} \int_{B_{S+2}} \d x \int_{\R^{2m}} \d y\ |\Psi_S(x)-\Psi_S(y)|^2K(|x-y|) + \int_{B_{S+2}} G(\Psi_S) \\
&= \int_{B_{S+2} \cap \ocal} \d x \int_{\R^{2m}} \d y\ |\Psi_S(x)-\Psi_S(y)|^2K(|x-y|) \d x\d y + \int_{B_{S+2}} G(\Psi_S) \\
&\leq \Lambda\, c_{n,\s} \int_{B_{S+2} \cap \ocal} \d x \int_{\R^{2m}} \d y \ \frac{|\Psi_S(x)-\Psi_S(y)|^2}{|x-y|^{n+2\s}} + \int_{B_{S+2}} G(\Psi_S).
\end{align*}
Now, we split the domain of integration of the kinetic energy into three parts.
\begin{align*}
\ecal(\Psi_S,B_{S+2}) &\leq \Lambda\, c_{n,\s} \int_{B_{S+2} \cap \ocal} \d x \int_{\ocal} \d y \frac{|\Psi_S(x)-\Psi_S(y)|^2}{|x-y|^{n+2\s}} \\
&\hspace{5mm} + \Lambda\, c_{n,\s} \int_{\Omega_S \cap \ocal} \d x \int_{\ical} \d y \frac{|\Psi_S(x)-\Psi_S(y)|^2}{|x-y|^{n+2\s}} \\
&\hspace{5mm} + \Lambda\, c_{n,\s} \int_{(B_{S+2}\setminus \Omega_S) \cap \ocal} \d x \int_{\ical} \d y \frac{|\Psi_S(x)-\Psi_S(y)|^2}{|x-y|^{n+2\s}} + \int_{B_{S+2}} G(\Psi_S) \\
&=:\Lambda\, c_{n,\s} (I_1+I_2+I_3)+I_G,
\end{align*}
where $\Omega_S$ is defined in \eqref{Eq:DefOmegaS}. 

Let us estimate this four integrals. To estimate $I_1$, we use Lemma~\eqref{Lemma: AdaptedLipschitzConditionWith_dFunction} and the fact that $\Psi_S$ is bounded by $1$.
\begin{align*}
I_1 &= \int_{B_{S+2} \cap \ocal} \int_{\ocal} \frac{|\Psi_S(x)-\Psi_S(y)|^2}{|x-y|^{n+2\s}} \d y\d x\\
&= \int_{B_{S+2} \cap \ocal} \int_{\ocal\cap\{|x-y|\leq d_S(x)\}} \frac{|\Psi_S(x)-\Psi_S(y)|^2}{|x-y|^{n+2\s}} \d y\d x\\
&\hspace{5mm} + \int_{B_{S+2} \cap \ocal} \int_{\ocal\cap\{|x-y|\geq d_S(x)\}} \frac{|\Psi_S(x)-\Psi_S(y)|^2}{|x-y|^{n+2\s}} \d y\d x\\
&\leq C \int_{B_{S+2} \cap \ocal} d_S(x)^{-2}\left(\int_{\ocal\cap\{|x-y|\leq d_S(x)\}} |x-y|^{2-n-2\s} \d y\right)\d x\\
&\hspace{5mm} + C \int_{B_{S+2} \cap \ocal} \left(\int_{\ocal\cap\{|x-y|\geq d_S(x)\}} |x-y|^{-n-2\s} \d y \right)\d x\\
&\leq C \int_{B_{S+2} \cap \ocal} d_S(x)^{-2}\left(\int_0^{d_S(x)} \rho^{1-2\s} \d \rho\right)\d x + C \int_{B_{S+2} \cap \ocal}  \left(\int_{d_S(x)}^\infty \rho^{-1-2\s} \d\rho\right) \d x\\
&\leq C \int_{B_{S+2} \cap \ocal} d_S(x)^{-2\s} \d x.
\end{align*}
The bound of $I_2$ is essentially the same using also Lemma~\ref{Lemma: AdaptedLipschitzConditionWith_dFunction} and the inclusion $\Omega_S \subset B_{S+2}$. That is,
\begin{align*}
I_2 &\leq C \int_{\Omega_S \cap \ocal} d_S(x)^{-2}\left(\int_0^{d_S(x)} \rho^{1-2\s} \d \rho\right)\d x + C \int_{\Omega_S \cap \ocal} \left( \int_{d_S(x)}^\infty \rho^{-1-2\s} \d\rho \right) \d x\\
&\leq C \int_{\Omega _S\cap \ocal} d_S(x)^{-2\s} \d x \leq C \int_{B_{S+2} \cap \ocal} d_S(x)^{-2\s} \d x.
\end{align*}
For the case of $I_3$, we use the fact that given $x\in (B_{S+2}\setminus \Omega_S)\cap \ocal$, then $\dist(x,\ccal)\geq d_S(x)$ and therefore $\ical \subset \R^{2m}\setminus B_{d_S(x)}(x)$. We obtain
\begin{align*}
I_3 &= \int_{(B_{S+2}\setminus \Omega_S) \cap \ocal} \d x \int_{\ical} \d y \frac{|\Psi_S(x)-\Psi_S(y)|^2}{|x-y|^{n+2\s}} \leq C \int_{(B_{S+2}\setminus \Omega_S) \cap \ocal} \d x \int_{\R^{2m}\setminus B_{d_S(x)}(x)} \d y \ |x-y|^{-n-2\s} \\
&\leq C \int_{B_{S+2} \cap \ocal} \left(\int_{d_S(x)}^\infty \rho^{-1-2\s} \d \rho \right)\d x \leq C \int_{B_{S+2} \cap \ocal} d_S(x)^{-2\s} \d x.
\end{align*}
Finally, we estimate $I_G$. Since $\Psi_S$ is either $1$ or $-1$ in $B_{S+2}\setminus \Omega_S$, and $G(-1)=G(1)=0$, we have
\begin{align*}
I_G = \int_{B_{S+2}} G(\Psi_S) = \int_{\Omega_S} G(\Psi_S) \leq C | \Omega_S| \leq C\,S^{2m-1}\,,
\end{align*}
where we have used \eqref{Eq:MeasureOmegaS}.Therefore, we obtain
\begin{align*}
\ecal(\Psi_S,B_{S+2}) &\leq C \left(\int_{B_{S+2} \cap \ocal} d_S(x)^{-2\s} \d x + S^{2m-1} \right)\leq C\left(\int_{B_{S} \cap \ocal} d_S(x)^{-2\s} \d x + S^{2m-1} \right).
\end{align*}

%%%%%%%%%

\textbf{3.2. Estimate for $I_u(\ocal \setminus B_{S+1}, \ocal \cap B_S)$.} First, we claim that $|x-y|\geq d_S(x)$ whenever $x\in B_S\cap \ocal$ and $y\in \R^{2m}\setminus B_{S+1}$. Indeed, if $x\in B_S$, then it is easy to see that $d_S(x) \leq S+1-|x|$ and therefore we have $|x-y|\geq |y|-|x|\geq |y|+d_S(x)-S-1 \geq  d_S(x)$, since $|y| \geq S+1$. Thus, using this inequality and the ellipticity of $K$, we get
\begin{align*}
I_u(\ocal \setminus B_{S+1}, \ocal \cap B_S) &\leq C \int_{B_S \cap \ocal} \d x \int_{\R^{2m}\setminus B_{S+1}} \d y \ \frac{|u(x)-u(y)|^2}{|x-y|^{2m+2\s}} \\
& \leq C \int_{B_S \cap \ocal} \d x \int_{|x-y|\geq d_S(x)} \d y \ |x-y|^{-2m-2\s} \\
&\leq C \int_{B_S \cap \ocal} d_S(x)^{-2\s} \d x.
\end{align*}

\textbf{Step 4. Conclusion.}

Finally, by adding up the estimates of Step~3 and applying Lemma~\ref{Lemma:Integrability_dFunction}, we obtain the desired result. That is,
\begin{align*}
\ecal(u,B_S) &\leq \ecal(\Psi_S,B_{S+2}) + I_u(\ocal \setminus B_{S+1}, \ocal \cap B_S) \leq C\left(\int_{B_S \cap \ocal} d_S(x)^{-2\s} \d x + S^{2m-1} \right)\\
&\leq \begin{cases}
C \ S^{2m-2\s}\ \ \ &\textrm{if } \ \ \s\in(0,1/2),\\
C\ \log(S)\,S^{2m-2\s}\ \ \ \ &\textrm{if } \ \ \s=1/2,\\
C \ S^{2m-1}\ \ \ \ &\textrm{if } \ \ \s\in(1/2,1).\\
\end{cases}
\end{align*}
\end{proof}
