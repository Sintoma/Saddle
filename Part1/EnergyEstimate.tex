%%%%%%%%%%%%%%%%%%%%%%%%%%%%%%%%%%%%%%%%%%%%%%%%%%%%%%%%%%%%%%%%%%%%%%
%%%%%%%%%%%%%%%%%%%%%%%%%%%%%%%%%%%%%%%%%%%%%%%%%%%%%%%%%%%%%%%%%%%%%%
\section{An energy estimate for doubly radial odd minimizers}
\label{Sec:EnergyEstimate}

In this section we present an estimate for the energy in the ball $B_S$ of minimizers in the space $\widetilde{\H}^K_{0, \mathrm{odd}}(B_R)$ with $R > S+ 4$. That is, we prove Theorem~\ref{Th:EnergyEstimate}. In order to establish this result, we follow the ideas of Savin and Valdinoci in \cite{SavinValdinoci-EnergyEstimate}, where they show the same type of estimate but for minimizers without any symmetry.


First of all, let us comment briefly the strategy used in \cite{SavinValdinoci-EnergyEstimate}. The argument is based in comparing the energy of the minimizer $u$ in $B_R\subset \R^n$ with the energy of a suitable competitor $v$. This function $v$ satisfies, in $B_{S+2}\subset B_R \subset \R^n$, the following assumptions:
\begin{enumerate}[label=(\textit{\roman*})]
	\item $-1 \leq v \leq 1$.
	\item $v=u$ in $\partial B_{S+2}$.
	\item The set $\{v\not \equiv -1\}\cap B_{S+2}$ has measure of order $n-1$ in $S$.
	\item $v\in \Lip(\overline{B_{S+2}})$ with a Lipschitz constant independent of $R$ and $S$.
\end{enumerate} 
By the second property, $v$ can be extended by $u$ outside $B_{S+2}$, becoming an admissible competitor. Then, the desired estimate follows by finding precise bounds of the energy of $v$ in $B_{S+2}$. The function $v$ is constructed in $B_{S+2}$ as
$$
v = \min \{u, \phi_S\}\,
$$
where
\begin{equation}
\label{Eq:DefPhiS}
\phi_S (x) =-1+2\min\{(|x|-S-1)_+,1\}\,,
\end{equation}
and then extended to be $u$ outside $B_{S+2}$. Note that $\phi_S \equiv -1$ in $B_{S+1}$ and $\phi_S = 1$ in $\partial B_{S+2}$ and thus, since $-1<u<1$, the first three assumptions on $v$ are satisfied. The fourth follows from interior regularity results and the Lipschitz regularity of $\phi_S$.

To estimate the energy of $v$ in $B_{S+2}$, it is suitable to divide the nonlocal terms with double integrals in sets where $|x-y|\geq d(x)$ and $|x-y|\leq d(x)$, for a suitable function $d(x)$. This allows to have the precise estimate analogous to \eqref{Eq:EnergyEstimate}. The function that is used in \cite{SavinValdinoci-EnergyEstimate} is 
\begin{equation}
\label{Eq:DefdSavinValdinoci}
d (x) = \max \{S+1-|x| , 1\},
\end{equation}
which in $B_{S+2}$ coincides with $\max\{ \dist (x, \partial B_{S+1}), 1\}$.

In our case, the strategy is the same but we will need to adapt some ingredients (such as the competitor $v$ and the function $d$). First, note that the previous construction for $v$ cannot be used in our setting, since it would not give us a doubly radial function which is odd with respect to the Simons cone $\ccal$, and therefore $v$ would not be an admissible competitor ---recall that we consider $u$ to be a minimizer among functions in $\widetilde{\H}^K_{0, \mathrm{odd}}(B_R)$. 

To overcome this problem, we will consider a function $w$ defined in $B_{S+2}\cap \ocal$ and satisfying the four previous assumptions on $v$. In addition, we will require $w$ to be doubly radial and to vanish on the Simons cone (then we will consider its odd extension through $\ccal$). Apart from this, we will need some technical details, described next. 

First, to simplify our proof and be more precise, we define the following set (see Figure~\ref{Fig:PsiSandOmegaS}~(b)):
\begin{equation}
\label{Eq:DefOmegaS}
\Omega_S := \left( \overline{B_{S+2}}\setminus B_S \right) \cup \left(  \overline{B_{S+2}} \cap \{\mu \dist(x,\ccal) \leq 1\}\right),
\end{equation} 
with $\mu>0$ a positive number that will be chosen later and will depend on the Lipschitz constant of $u$ in $B_{S+3}$. $\Omega_S$ will be exactly the set where $w\not \equiv -1$ inside $B_{S+2}$. It is easy to see that its measure is of order $2m-1$ in $S$. That is,
 \begin{equation}
 \label{Eq:MeasureOmegaS}
 |\Omega_S| \leq C\,S^{2m-1},
 \end{equation}
with a constant $C$ depending only on $m$ and $\mu$. This can be checked following the computations in the proof of the energy estimate for the local equation in Theorem~1.3 of \cite{CabreTerraI}.

\begin{figure}
	\centering
	\begin{subfigure}{0.21\textwidth}
		\centering
		\definecolor{lila_custom}{RGB}{201,69,254}
\definecolor{naranja_custom}{RGB}{255,148,0}


\begin{tikzpicture}[y=0.80pt, x=0.80pt, yscale=-1.000000, xscale=1.000000, inner sep=0pt, outer sep=0pt]


%region-down
\path[scale=0.938,fill=lila_custom, opacity = 0.7, line width=0.400pt] (248.0459,242.6193) ..
controls (275.7060,284.1286) and (276.2013,310.5581) .. (276.7985,330.5197) ..
controls (233.8448,330.5197) and (216.3370,330.5469) .. (160.2334,330.5469) ..
controls (186.2822,304.6380) and (219.4410,271.2569) .. (248.0459,242.6193) --
cycle;



%big circle
\path[draw=black,line join=miter,line cap=butt,line width=1.4pt]
(127.0312,146.9770) .. controls (166.9982,146.9770) and (210.2478,160.9750) ..
(243.0000,193.8560) .. controls (275.7522,226.7370) and (289.9164,269.7115) ..
(289.9164,309.8622);

%big circle2
\path[draw=naranja_custom,line join=miter,line cap=butt,line width=1.8pt]
(243.0000+11,193.8560+12) .. controls (275.7522-9,226.7370-6) and (289.9164,258.7115) ..
(289.9164,309.8622);

%x-axis    
\path[draw=black,line join=miter,line cap=butt,line width=1.5pt]
(126.5055,309.8063) -- (330.4872,309.8063);

%x-axis cap    
\path[draw=black,fill=black,even odd rule,line width=0.497pt]
(330.4872,309.8063) -- (328.1493,312.1289) -- (336.3320,309.8063) --
(328.1493,307.4838) -- cycle;

%y-axis
\path[draw=black,line join=miter,line cap=butt,line width=1.5pt]
(127.0562,310.9866) -- (127.0562,99.5223);

%y-axis cap    
\path[draw=black,fill=black,even odd rule,line width=0.497pt] (127.0562,99.5223)
-- (129.3788,101.8602) -- (127.0562,93.6775) -- (124.7336,101.8602) -- cycle;



%medium circle
\path[draw=black,line join=miter,line cap=butt,line width=1.4pt]
(127.0312,177.3949) .. controls (151.7088,177.3949) and (192.2090,185.8302) ..
(221.4938,215.3997) .. controls (250.7785,244.9692) and (259.4986,284.9193) ..
(259.4986,309.8622);

%small circle    
\path[draw=black,line join=miter,line cap=butt,line width=1.4pt]
(127.0312,207.7933) .. controls (150.2506,207.7933) and (177.3815,214.3180) ..
(199.9500,236.9435) .. controls (222.0483,259.0975) and (229.1002,286.5606) ..
(229.1002,309.8622);

%cone   
\path[draw=black,line join=miter,line cap=butt,miter limit=4.00,even odd
rule,line width=1.57pt] (126.7629,310.0643) -- (326.1869,110.8844);

%cone+1    
\path[draw=black,line join=miter,line cap=butt,even odd rule,line width=0.704pt]
(150.2563,309.8481) -- (318.0555,142.0518);


%Linea Cota
\path[draw=black,line join=miter,line cap=butt,line width=0.5pt]
(298.3343,142.4919) -- (306.3936,150.7321);


%Flecha Izquierda Cota
\path[draw=black,fill=black,even odd rule,line width=0.200pt]
(298.9657,143.1375) -- (300.2914,143.1522) -- (297.3269,141.4619) --
(298.9510,144.4632) -- cycle;

%Flecha Derecha Cota
\path[draw=black,fill=black,even odd rule,line width=0.200pt]
(305.7622,150.0865) -- (304.4364,150.0718) -- (307.4010,151.7621) --
(305.7769,148.7608) -- cycle;

%Cota naranja
\path[draw=black,line join=miter,line cap=butt,line width=0.5pt]
(300, 230) .. controls (300, 235) and (295, 250) ..
(285, 250);

%Flecha cota naranja
\path[draw=black,fill=black,even odd rule,line width=0.200pt]
(285, 250) -- (285+0.9375, 250+0.9375) -- (285-2.3437, 250) --
(285+0.9375, 250-0.9375) -- cycle;

\node at (315,138) {\normalsize $\mu^{-1}$};
\node at (127,85) {\normalsize $|x''|$};
\node at (350, 311) {\normalsize $|x'|$};
\node at (228,320) {\normalsize $S$};
\node at (255, 320) {\normalsize $S\!+\!1$};
\node at (292, 320) {\normalsize $S\!+\!2$};
\node at (334, 101) {\normalsize $\ccal$};
\node at (298, 220) {\normalsize $\Psi_S\! = \!1$};
\node at (220, 286) {\normalsize $\Psi_S\! = \!-1$};


\end{tikzpicture}


	\end{subfigure}
	\begin{subfigure}{0.21\textwidth}
		\centering
		\definecolor{azul_custom}{RGB}{66,240,209}
\definecolor{lila_custom}{RGB}{201,69,254}
\definecolor{naranja_custom}{RGB}{255,148,0}


\begin{tikzpicture}[y=0.80pt, x=0.80pt, yscale=-1.000000, xscale=1.000000, inner sep=0pt, outer sep=0pt]


%region
\path[fill=azul_custom,line cap=round,miter limit=4.00,line width=1.216pt]
(210.3958,249.7906) .. controls (188.0101,272.1763) and (172.1936,287.9233) ..
(150.1544,309.9625) .. controls (148.2221,309.9625) and (137.7690,309.7049) ..
(127.0089,309.7371) .. controls (127.0089,300.7477) and (127.0826,301.1491) ..
(127.0826,287.3606) .. controls (136.2973,278.1061) and (177.3049,237.1011) ..
(187.7910,226.5786) .. controls (170.0299,214.0930) and (149.7638,207.7938) ..
(127.0312,207.7938) .. controls (127.0312,190.9391) and (126.9993,161.0751) ..
(126.9993,146.8284) .. controls (170.9394,146.8284) and (208.6039,162.4681) ..
(243.0000,193.8560) .. controls (271.9385,225.9716) and (289.9164,263.3638) ..
(289.9164,309.8622) .. controls (260.7310,309.8622) and (257.9656,309.8622) ..
(229.1006,309.8622) .. controls (229.1006,279.3951) and (221.7362,269.4328) ..
(210.3958,249.7907) -- cycle;


%x-axis    
\path[draw=black,line join=miter,line cap=butt,line width=1.5pt]
(126.5055,309.8063) -- (330.4872,309.8063);

%x-axis cap    
\path[draw=black,fill=black,even odd rule,line width=0.497pt]
(330.4872,309.8063) -- (328.1493,312.1289) -- (336.3320,309.8063) --
(328.1493,307.4838) -- cycle;

%y-axis
\path[draw=black,line join=miter,line cap=butt,line width=1.5pt]
(127.0562,310.75) -- (127.0562,99.5223);

%y-axis cap    
\path[draw=black,fill=black,even odd rule,line width=0.497pt] (127.0562,99.5223)
-- (129.3788,101.8602) -- (127.0562,93.6775) -- (124.7336,101.8602) -- cycle;


%big circle
\path[draw=black,line join=miter,line cap=butt,line width=1.4pt]
(127.0312,146.9770) .. controls (166.9982,146.9770) and (210.2478,160.9750) ..
(243.0000,193.8560) .. controls (275.7522,226.7370) and (289.9164,269.7115) ..
(289.9164,309.8622);

%medium circle
\path[draw=black,line join=miter,line cap=butt,line width=1.4pt]
(127.0312,177.3949) .. controls (151.7088,177.3949) and (192.2090,185.8302) ..
(221.4938,215.3997) .. controls (250.7785,244.9692) and (259.4986,284.9193) ..
(259.4986,309.8622);

%small circle    
\path[draw=black,line join=miter,line cap=butt,line width=1.4pt]
(127.0312,207.7933) .. controls (150.2506,207.7933) and (177.3815,214.3180) ..
(199.9500,236.9435) .. controls (222.0483,259.0975) and (229.1002,286.5606) ..
(229.1002,309.8622);

%cone   
\path[draw=black,line join=miter,line cap=butt,miter limit=4.00,even odd
rule,line width=1.57pt] (126.7629,310.0643) -- (326.1869,110.8844);

%cone+1    
\path[draw=black,line join=miter,line cap=butt,even odd rule,line width=0.704pt]
(150.2563,309.8481) -- (318.0555,142.0518);

%cone-1 
\path[draw=black,line join=miter,line cap=butt,even odd rule,line width=0.704pt]
(127.1847,287.2416) -- (299.4946,114.9526); 

%Linea Cota
\path[draw=black,line join=miter,line cap=butt,line width=0.5pt]
(298.3343,142.4919) -- (306.3936,150.7321);


%Flecha Izquierda Cota
\path[draw=black,fill=black,even odd rule,line width=0.200pt]
(298.9657,143.1375) -- (300.2914,143.1522) -- (297.3269,141.4619) --
(298.9510,144.4632) -- cycle;

%Flecha Derecha Cota
\path[draw=black,fill=black,even odd rule,line width=0.200pt]
(305.7622,150.0865) -- (304.4364,150.0718) -- (307.4010,151.7621) --
(305.7769,148.7608) -- cycle;

\node at (315,138) {\normalsize $\mu^{-1}$};
\node at (127,85) {\normalsize $|x''|$};
\node at (350, 311) {\normalsize $|x'|$};
\node at (228,320) {\normalsize $S$};
\node at (255, 320) {\normalsize $S\!+\!1$};
\node at (292, 320) {\normalsize $S\!+\!2$};
\node at (334, 101) {\normalsize $\ccal$};
\node at (264, 250) {\normalsize $\Omega_S$};

\end{tikzpicture}











	\end{subfigure}
	\caption{(a) The $1$ and $-1$ level sets of $\Psi_S$. (b) The set $\Omega_S$.}
	\label{Fig:PsiSandOmegaS}
\end{figure}



Apart from the previous considerations, we also need to adapt the function $d(x)$ of \cite{SavinValdinoci-EnergyEstimate} to our setting of odd functions. We will consider the following one: 
$$ 
d_S(x) := \max\left\{1,\min\{S+1-|x|,\mu \dist(x,\ccal)\} \right\},  
$$
where $\mu$ is the same constant as in \eqref{Eq:DefOmegaS}. Note that $d_S(x) = 1$ if $x\in \Omega_S$. Moreover, if $x\in B_{S+2}\setminus \Omega_S = B_S \cap \{\mu\dist(\cdot,\ccal) \geq 1\}$,  the expression of $d_S(x)$ is simpler:
\begin{align*}
     \restr{d_S}{B_{S+2}\setminus \Omega_S} (x) 
	 = \min \{\dist(x, \partial B_{S+1}),\mu \dist(x,\ccal)\} .
\end{align*}
As we will see in the proof of Theorem~\ref{Th:EnergyEstimate}, by splitting some double integrals in sets where $|x-y|\geq d_S(x)$ and $|x-y|\leq d_S(x)$, we will obtain the precise bound \eqref{Eq:EnergyEstimate}. 

Last, let us mention a technicality that will be needed when performing our estimates. It is related with the function $d_S$. In one estimate appearing in the proof of Theorem~\ref{Th:EnergyEstimate}, we will need to assume that there exists a positive constant $C$, independent of $R$ and $S$, such that
$$ 
|w(x)-w(y)| \leq \frac{C}{\dist(x,\ccal)}|x-y| 
$$
whenever $x,y \in B_{S+1}\cap \ocal$, $\mu \dist(x,\ccal)\geq 1$, and $\mu \dist(y,\ccal)\leq 1$. For more details, see the proof of Theorem~\ref{Th:EnergyEstimate}.

In the following lemma we state the properties that suffice to establish the energy estimate.

\begin{lemma}
	\label{Lemma:ExistenceCompetitor}
	Let $S\geq 2$ and $R > S + 4$. Let $u\in \widetilde{\H}^K_{0, \mathrm{odd}}(B_R)$ be a doubly radial odd minimizer of the energy \eqref{Eq:Energy}. Then, there exists  a function $w:\overline{ B_{S+2}\cap \ocal} \to \R$, and a positive number $\mu$ depending only on $m$, $\s$, $\lambda$, $\Lambda$, and $\norm{G}_{C^2([-1,1])}$, satisfying the following:
	
	\begin{enumerate}[label={\normalfont (\textcolor{red}{H\arabic*})}, ref=H\arabic*]
	\item
	\label{Eq:wH1} $-1 \leq w \leq 1$.
	\item
	\label{Eq:wH2} $w$ doubly radial and $w=0$ in $\ccal$.
	\item
	\label{Eq:wH3} $w=u$ in $\partial B_{S+2} \cap \ocal$.
	\item
	\label{Eq:wH4} $w\equiv-1$ in $(B_{S+2}\cap \ocal)\setminus \Omega_S$.
	\item
	\label{Eq:wH5} $w\in \Lip(\overline{B_{S+2}})$ with a Lipschitz constant independent of $R$ and $S$, and
	$$ |w(x)-w(y)| \leq \frac{C}{\dist(x,\ccal)}|x-y| $$
	whenever $x,y \in B_{S+1}\cap \ocal$, $\mu \dist(x,\ccal)\geq 1$ and $\mu \dist(y,\ccal)\leq 1$, and with $C$ a constant independent of $R$ and $S$.
	\end{enumerate} 
	
	As a consequence, if we consider the odd extension of $w$ through $\ccal$ and then we set $w=u$ in $B_{S+2}^c$, it follows that $w$ is an admissible competitor for doubly radial minimizers of the energy \eqref{Eq:Energy}. Moreover, $w\in \Lip(\overline{B_{S+3}})$ with a Lipschitz constant independent of $R$ and $S$.
\end{lemma}


To construct this function we will define, in $B_{S+2}$,
$$
w:= \min \{u, \Psi_S\}\,
$$
where $\Psi_S:B_{S+2} \cap \ocal \to \R$ is essentially the function $\phi_S$ of \eqref{Eq:DefPhiS} but modified in order to vanish at $\ccal$. Its precise expression in $B_{S+2}\cap \ocal$ is
\begin{equation}
\label{Eq:DefPsiS}
\Psi_S(x) :=
\begin{cases}
\phi_S (x) \mu\dist(x,\ccal) &  \mu\dist(x,\ccal) \leq 1 \,, \\
\phi_S (x) &  \mu\dist(x,\ccal) \geq 1\,,
\end{cases}
\end{equation}
where $\mu$ is a positive number that depends on the Lipschitz constant of $u$ in $B_{S+3}$ and makes \eqref{Eq:wH3} to be satisfied.

We postpone the proof of this lemma and, assuming it true for the moment, let us establish the energy estimate of Theorem~\ref{Th:EnergyEstimate}. 

\begin{proof}[Proof of Theorem~\ref{Th:EnergyEstimate}]
	Take $w$ constructed in Lemma~\ref{Lemma:ExistenceCompetitor} as a competitor. Hence, since $u$ is a doubly radial odd minimizer, $ \ecal (u, B_R) \leq \ecal (w, B_R)$. Since $u\equiv w$ in $B_{S+2}^c$, it follows that
	$$ 
	\ecal (u, B_{S+2}) \leq \ecal (w, B_{S+2}), 
	$$
	and by the monotonicity of the energy $\ecal$ by inclusions we get
	$$ 
	\ecal (u, B_{S}) \leq \ecal (w, B_{S+2}). 
	$$
	Therefore, to obtain our result we only need to estimate $\ecal (w, B_{S+2})$. 
	
	
	In the following estimates, the letter $C$ will be a constant depending only on $m$, $\s$, $\lambda$, $\Lambda$, and $\norm{G}_{C^2([-1,1])}$. Recall that $\mu$ given by Lemma~\ref{Lemma:ExistenceCompetitor} depends only on the previous quantities.
	
	First, note that using the ellipticity of the kernel $K$ and the change of variables given by $(\cdot)^\star$ it follows that
	$$ 
	\ecal(w,B_{S+2}) \leq C \int_{B_{S+2}\cap \ocal} \d x \int_{\R^n} \d y \frac{|w(x)-w(y)|^2}{|x-y|^{2m+2\s}} + \int_{B_{S+2}} G(w) \d x. 
	$$
	Now we estimate separately the potential and kinetic energies.
	
	\medskip
	
	\textbf{Estimate for the potential energy.}
	Since, $w=\pm 1$ in $B_{S+2} \setminus \Omega_S$ ---by \eqref{Eq:wH4}--- and $G(1) = G(-1) = 0$, it is clear that
	$$ 
	\int_{B_{S+2}} G(w) = \int_{\Omega_S} G(w) \leq C |\Omega_S| \leq C S^{2m-1}.
	$$  
	Here we have used \eqref{Eq:MeasureOmegaS}.
	
	\medskip
	
	\textbf{Estimate for the kinetic energy.}
	We split the integral in three terms, as follows.
	\begin{align*}
	\int_{B_{S+2}\cap \ocal} \d x \int_{\R^n} \d y \frac{|w(x)-w(y)|^2}{|x-y|^{2m+2\s}} &= \int_{\Omega_{S}\cap \ocal} \d x \int_{\R^{2m}} \d y \ \frac{|w(x)-w(y)|^2}{|x-y|^{2m+2\s}} \\
	&\ \ + \int_{(B_S\setminus \Omega_S)\cap \ocal} \d x \int_{B_{S+1}\cap \ocal} \d y \ \frac{|w(x)-w(y)|^2}{|x-y|^{2m+2\s}} \\
	&\ \ + \int_{(B_S\setminus \Omega_S)\cap \ocal} \d x \int_{(B_{S+1}\cap \ocal)^c} \d y \ \frac{|w(x)-w(y)|^2}{|x-y|^{2m+2\s}} \\
	&=: I_1+I_2+I_3.
	\end{align*}
	Now we estimate each term separately.
	
	We estimate the first integral:
	\begin{align*}
	I_1 &= \int_{\Omega_{S}\cap \ocal} \d x \int_{\R^{2m}} \d y \  \frac{|w(x)-w(y)|^2}{|x-y|^{2m+2\s}} \\
	&= \int_{\Omega_{S}\cap \ocal} \d x \int_{|x-y|\leq 1} \d y \ \frac{|w(x)-w(y)|^2}{|x-y|^{2m+2\s}} + \int_{\Omega_{S}\cap \ocal} \d x \int_{|x-y|\geq 1} \d y \ \frac{|w(x)-w(y)|^2}{|x-y|^{2m+2\s}} \\
	&\leq C \int_{\Omega_{S}\cap \ocal} \d x \int_0^1 \d r \ r^{1-2\s} + C \int_{\Omega_{S}\cap \ocal} \d x \int_1^\infty \d r \ r^{-1-2\s} \\
	&= C |\Omega_S| \leq C \ S^{2m-1}.
	\end{align*}
	In the first addend of the previous estimate we have used that $w$ is Lipschitz in $\overline{B_{S+3}}$ ---see \eqref{Eq:wH5} and the last comment in Lemma~\ref{Lemma:ExistenceCompetitor}--- while in the second one we have  used only that $w$ is bounded \eqref{Eq:wH1}. 
	
	Next, we estimate $I_2$. To do it, we first claim that, if $|x-y|\leq d_S(x)$, then
	\begin{equation}
		\label{Eq:EnergyEstimateProofLipschitz}
		|w(x)-w(y)| \leq \dfrac{C}{d_S(x)} |x-y|	
	\end{equation}
	for every $x\in (B_S\setminus \Omega_S)\cap \ocal$ and $y \in B_{S+1}\cap \ocal$. To see this, note first that since $x\in B_S\setminus \Omega_S \subset \{\mu \dist (\cdot, \ccal) \geq 1\}$, then $d_S(x)=\min \{S+1-|x| , \mu \dist (x, \ccal)\}$, and therefore it suffices to show that
	$$
	|w(x)-w(y)| \leq \dfrac{C}{\dist (x, \ccal)} |x-y|
	$$
	for $x\in B_S \cap \ocal$ with $\mu \dist (x, \ccal) \geq 1$ and $y \in B_{S+1}\cap \ocal$ (recall that $C$ may depend on $\mu$). Now, if we assume that $\mu \dist (y, \ccal) \geq 1$, it follows that $w(x)=w(y)=-1$ and \eqref{Eq:EnergyEstimateProofLipschitz} is trivially true. On the other hand, if we assume that $\mu \dist (y, \ccal) \leq 1$, then \eqref{Eq:EnergyEstimateProofLipschitz} follows from \eqref{Eq:wH5}. Therefore, the claim is proved.
	
	Using \eqref{Eq:EnergyEstimateProofLipschitz}, we proceed as before splitting the integrals to obtain
	\begin{align*}
	I_2 &= \int_{(B_S\setminus \Omega_S)\cap \ocal} \d x \int_{B_{S+1}\cap \ocal} \d y \  \frac{|w(x)-w(y)|^2}{|x-y|^{2m+2\s}}   \\
	&\leq \int_{(B_S\setminus \Omega_S)\cap \ocal} \d x \int_{ \{|x-y|\leq d_S(x)\} \cap B_{S+1}\cap \ocal} \d y \  \frac{|w(x)-w(y)|^2}{|x-y|^{2m+2\s}} \\
	& \quad \quad + \int_{(B_S\setminus \Omega_S)\cap \ocal} \d x \int_{|x-y|\geq d_S(x)} \d y \ \frac{|w(x)-w(y)|^2}{|x-y|^{2m+2\s}} \\
	&\leq C \int_{(B_S\setminus \Omega_S)\cap \ocal} d_S(x)^{-2\s} \d x .
	\end{align*}
	Here we have used \eqref{Eq:EnergyEstimateProofLipschitz} to estimate the first term, while for the second one we have used only that $w$ is bounded \eqref{Eq:wH1}. 
	
	Last, we estimate $I_3$. To do it, we first claim that if $x\in (B_S\setminus \Omega_S) \cap \ocal$ and $y\in (B_{S+1}\cap \ocal)^c = I \cup B_{S+1}^c$, then $|x-y|\geq C d_S(x)$. Indeed, on the one hand it is clear that, if $y\in B_{S+1}^c$, then $|x-y|\geq \dist(x,\partial B_{S+1})\geq d_S(x)$. On the other hand, if $y\in \ical$, then $|x-y|\geq \dist(x,\ccal) \geq  d_S(x) / \mu$. Here we used that since $x\in B_S\setminus \Omega_S $, if holds $d_S(x)=\min \{\dist(x,\partial B_{S+1}), \mu \dist (x, \ccal)\}$. 
	
	Using the previous claim, we obtain
	\begin{align*}
	I_3 &= \int_{(B_S\setminus \Omega_S)\cap \ocal} \d x \int_{(B_{S+1}\cap \ocal)^c} \d y \  \frac{|w(x)-w(y)|^2}{|x-y|^{2m+2\s}} \\
	&\leq \int_{(B_S\setminus \Omega_S)\cap \ocal} \d x \int_{|x-y|\geq C d_S(x)} \d y \ \frac{|w(x)-w(y)|^2}{|x-y|^{2m+2\s}} \\
	&\leq C \int_{(B_S\setminus \Omega_S)\cap \ocal}  d_S(x)^{-2\s} \d x .
	\end{align*}
	
	Now, if we add up $I_1, I_2$, and $I_3$, we get
	\begin{equation}
		\label{Eq:EnergyEstimateProofLastEstimate}
	\ecal(w,B_{S+2}) \leq C \left( \int_{(B_S\setminus \Omega_S)\cap \ocal}  d_S(x)^{-2\s} \d x + S^{2m-1}\right).
	\end{equation}
	We conclude the proof by estimating the integral of $d_S(x)^{-2\s}$, as follows.
	\begin{align*}
	\int_{B_{S+2}\setminus \Omega_S} d_S(x)^{-2\s} \d x &= \int_{B_{S}\cap \{\mu \dist(x,\ccal)>1\}} d_S(x)^{-2\s} \d x \\
	& \leq \int_{B_S} \left( S+1-|x| \right)^{-2\s} \d x \\
	& \quad \quad + \int_{B_{S}\cap \{\mu \dist(x,\ccal)>1\}} \dist(x,\ccal)^{-2\s} \d x.
	\end{align*}
	We next estimate these two integrals.
	
	
	The first integral can be estimated by using spherical coordinates and the change $\tau = r/(S+1)$. Indeed,
	\begin{align*}
	\int_{B_S} \left( S+1-|x| \right)^{-2\s} \d x & = C \int_0^S \dfrac{r^{2m-1}}{(S+1-r)^{2\s}} \d r \\
	& \leq C (S+1)^{2m - 2\s}\int_0^{1 - \frac{1}{S+1}} \dfrac{\tau^{2m-1}}{(1-\tau)^{2\s}} \d \tau\\
	&\leq C (S+1)^{2m - 2\s}\int_0^{1 - \frac{1}{S+1}} (1-\tau)^{-2\s} \d \tau \\
	& \leq \begin{cases}
	 C \ S^{2m-2\s}\ \ \ \ &\textrm{if } \ \ \s\in(0,1/2),\\
	 C\ \log(S)\,S^{2m-1}\ \ \ \ &\textrm{if } \ \ \s=1/2,\\
	 C \ S^{2m-1}\ \ \ \ &\textrm{if } \ \ \s\in(1/2,1).\\
	 \end{cases}
	\end{align*}
	To estimate the second integral, we write it in $(\euscr{y},\euscr{z})$ variables, where
	%\todo{elegir fuente}
	$$
	\euscr{y} := \dfrac{|x'|+|x''|}{\sqrt{2}} \, \quad \text{ and } \euscr{z} := \dfrac{|x'|-|x''|}{\sqrt{2}}\,.
	$$
	Note that $\euscr{z}$ is the signed distance to the cone (see Lemma~4.2 in \cite{CabreTerraI}). Thus,
	\begin{align*}
	\int_{B_{S}\cap \{\mu \dist(x,\ccal)>1\}} \dist(x,\ccal)^{-2\s} \d x &\leq C \int \int_{B_{S}\cap \{\euscr{y}\geq|\euscr{z}|>1/\mu\}} |\euscr{z}|^{-2\s} \, (\euscr{y}^2-\euscr{z}^2)^{m-1} \d \euscr{y}\d \euscr{z} \\
	& \leq C \int \int_{B_{S}\cap \{\euscr{y}\geq|\euscr{z}|>1/\mu\}} |\euscr{z}|^{-2\s} \, \euscr{y}^{2m-2} \d \euscr{y}\d \euscr{z} \\
	& \leq C\, \int_{1/\mu}^S \d \euscr{z}   \ \euscr{z}^{-2\s}\int_0^S \d \euscr{y}\  \euscr{y}^{2m-2} \\
	& \leq \begin{cases}
	C \ S^{2m-2\s}\ \ \ \ &\textrm{if } \ \ \s\in(0,1/2),\\
	C\ \log(S)\,S^{2m-1}\ \ \ \ &\textrm{if } \ \ \s=1/2,\\
	C \ S^{2m-1}\ \ \ \ &\textrm{if } \ \ \s\in(1/2,1).\\
	\end{cases}
	\end{align*}
	
	Using these two estimates in \eqref{Eq:EnergyEstimateProofLastEstimate}, and absorbing the lower order term with $S^{2m-1}$ (recall that $S\geq 2$), the desired result follows.
\end{proof}

To conclude the section, we establish Lemma~\ref{Lemma:ExistenceCompetitor}, which completes the previous proof.

\begin{proof}[Proof of Lemma~\ref{Lemma:ExistenceCompetitor}] We divide it into two steps.
	\medskip
		
	\textbf{Step 1. We show that $0\leq u < 1$ in $\ocal$.}
	 
	First note that since $u$ is a minimizer of $\ecal$ in $B_R$, by Lemma~\ref{Lemma:DecreaseEnergy} we can assume without loss of generality that $-1 \leq u \leq 1$, $u \geq 0$ in $\ocal$, and $u \leq 0$ in $\ical$. 
	
	
	Now we want to show that $u < 1$ in $\ocal$. We first claim that $u$ is a classical solution to
	\begin{equation}
	\label{Eq:ProofEnergyEstimateProblemBR}
	\beqc{\PDEsystem}
	L_K  u &=& f(u) & \textrm{ in } B_R\,,\\
	u &=& 0 & \textrm{ in }\R^{2m} \setminus B_R.
	\eeqc
	\end{equation}
	To see this, we consider perturbations $u +  \varepsilon \xi$ with $\varepsilon>0$. On the one hand, take $\xi \in \widetilde{\H}^K_{0, \,\mathrm{odd}}(B_R)$. Then, since $u$ is a minimizer among functions in $\widetilde{\H}^K_{0, \,\mathrm{odd}}(B_R)$, we get
	$$
	0 = \dfrac{\d}{\d \varepsilon}\evalat{\varepsilon = 0} \ecal(u +  \varepsilon \xi, B_R) = \langle u,\xi \rangle_{\widetilde{\H}^K_0(B_R)} - \langle f(u),\xi \rangle_{L^2(B_R)}\,.
	$$
	On the other hand, take $\xi \in \widetilde{\H}^K_{0, \,\mathrm{even}}(B_R)$. Since $u$ is odd with respect to the Simons cone, the same holds for $f(u)$ ---recall that $f$ is odd. Then, by Remark~\ref{Remark:DecompositionHK} we find that
	$$
	\langle u,\xi \rangle_{\widetilde{\H}^K_0(B_R)} = 0 \quad \textrm{ and } \quad  \langle f(u),\xi \rangle_{L^2(B_R)} = 0\,.
	$$
	Therefore, 
	$$
	\langle u,\xi \rangle_{\widetilde{\H}^K_0(B_R)} = \langle f(u),\xi \rangle_{L^2(B_R)}
	$$
	for every $\xi \in\widetilde{\H}^K_0(B_R)$ with compact support in  $B_R$. By approximation, $\xi$ can be taken to be $C^\infty_c(B_R)$. Thus,
	$$
	\int_{\R^{2m}}\int_{\R^{2m}} \{u(x)-u(y)\}\{\xi(x)-\xi(y)\} K(|x-y|) \d x \d y = \int_{\R^{2m}} f(u(x)) \xi(x) \d x
	$$
	for every $\xi \in C^\infty_c(\Omega)$ that is doubly radial. Finally, by Proposition~\ref{Prop:WeakSolutionForAllTestFunctions}, $u$ is a weak solution to \eqref{Eq:ProofEnergyEstimateProblemBR}, and in view of the regularity theory for operators in $\lcal_0$ (see Remark~\ref{Remark:InteriorRegularity}), since $u$ is bounded, it is a classical solution and the claim is proved.
	
	Since $u$ is a classical solution that is odd and continuous, it follows that $u \not \equiv 1$ in $\ocal$. Therefore, by the usual strong maximum principle (recall that $u\leq 1$), we get $0\leq u < 1$ in $\ocal$. 
	
	\medskip
	
	\textbf{Step 2. We construct $w$ and verify \eqref{Eq:wH1}-\eqref{Eq:wH5}.} 
	
	Let $\mu>0$ be a positive number to be chosen later and consider the following function $\Psi_S:\overline{B_{S+2} \cap \ocal} \to \R$ defined by \eqref{Eq:DefPsiS}, that is, 
	$$
	\Psi_S(x) :=
	\begin{cases}
	\phi_S (x) \mu\dist(x,\ccal) &  \mu\dist(x,\ccal) \leq 1 \,, \\
	\phi_S (x) &  \mu\dist(x,\ccal) \geq 1\,,
	\end{cases}
	$$
	with 
	$$
	\phi_S (x) =-1+2\min\{(|x|-S-1)_+,1\}\,.
	$$
	
	Obviously, $\Psi_S$ is Lipschitz in $\overline{B_{S+2}}$, with a Lipschitz constant depending only on $\mu$.
	
	Now, we define $w:\overline{B_{S+2} \cap \ocal} \to \R$ as
	$$
	w:= \min \{u, \Psi_S\}.
	$$
	Note that $w=\Psi_S$ in $B_{S+1}$ since $u\geq 0$ and $\Psi_S\leq 0$ there.	We claim that if we choose $\mu$ appropriately, then $w$ satisfies \eqref{Eq:wH1}-\eqref{Eq:wH5}. 
	
	First, \eqref{Eq:wH1} holds trivially, since both functions $u$ and $\Psi_S$ are bounded by $1$ and $-1$ by above and below. Moreover, since both functions are doubly radial ---recall that the distance to the cone, in $\ocal$, is $(|x'|-|x''|)/\sqrt{2}$--- and $u\geq0$, then \eqref{Eq:wH2} follows.
	
	Now, we choose $\mu$ such that \eqref{Eq:wH3} holds. On the one hand, if $x\in \partial B_{S+2}\cap \ocal$ and $\mu\dist(x,\ccal) \geq 1$, we have $\Psi_S (x) = \phi_S(x) = 1 > u$, and therefore $w(x) = u(x)$. On the other hand, for $x\in \partial B_{S+2}\cap \ocal$ with $\mu\dist(x,\ccal) \leq 1$, we have $\Psi_S (x) = \mu \dist (x,\ccal)$. We take  
	\begin{equation} 
	\label{Eq:ChoiceMu} 
	\mu = \norm{u}_{\mathrm{Lip}(\overline{B_{S+3}})}, 
	\end{equation} 
	and therefore
	$$ 
	|u(y) - u(z)|\leq \mu |y-z| \quad \text{ for every } y, \ z \in \overline{B_{S+3}}. 
	$$ 
	Thus, by taking $y=x$ and  $z\in \ccal$ to be a point realizing $\dist(x,\ccal)$, we obtain that
	$$ 
	u(x) = |u(x)|\leq \mu |x-z|  = \mu \dist (x,\ccal) = \Psi_S (x).
	$$ 
	Thus, $w(x) = u(x)$ and \eqref{Eq:wH3} holds.
	
	The verification of \eqref{Eq:wH4} is easy, since $\Psi_S \equiv-1 < u$ in $(B_{S+2}\cap \ocal)\setminus \Omega_S$.
	
	Finally, we check \eqref{Eq:wH5} but with constants depending only on $\mu$. Obviously, $w$ is Lipschitz in $\overline{B_{S+2}}$ since it is the minimum of two Lipschitz functions. In addition, the Lipschitz constant of $w$ depends only on $\mu$, which was chosen in \eqref{Eq:ChoiceMu}. Next, we show the last statement. Let $x,y \in B_{S+1}\cap \ocal$, $\mu \dist(x,\ccal)\geq 1$ and $\mu \dist(y,\ccal)\leq 1$. Then, $w(x)= \Psi_S(x) = \phi_S(x) = -1$ and $w(y)=\Psi_S(y) = \mu \dist(y,\ccal)$. Therefore, we need to show that
	\begin{equation}
	\label{Eq:IneqDistanceCone}
	|1-\mu \dist(y,\ccal)|\leq \dfrac{|x-y|}{\dist(x,\ccal)}\,.
	\end{equation}
	
	We will prove that, indeed, \eqref{Eq:IneqDistanceCone} holds for every $x,y \in \R^{2m}$ with $\mu \dist(x,\ccal)\geq 1$ and $\mu \dist(y,\ccal)\leq 1$. First, by using the triangular inequality and the definition of distance to the Simons cone, it is easy to verify that
	\begin{equation} 
	\label{Eq:TriangularCone1}
	\dist(x,\ccal) \leq |x-y| + \dist(y,\ccal).
	\end{equation}
	Therefore, since $\mu \dist(x,\ccal)\geq 1$, we have
	\begin{equation} 
	\label{Eq:TriangularCone2}
	1-\mu |x-y|- \mu \dist(y,\ccal) \leq 1-\mu\dist(x,\ccal) \leq 0
	\end{equation}
	Last, multiplying \eqref{Eq:TriangularCone1} by $|1-\mu\dist(y,\ccal)|$, and using that $\mu \dist(y,\ccal)\leq 1$ and \eqref{Eq:TriangularCone2}, we obtain
	\begin{align*}
	|1-\mu\dist(y,\ccal)|\,\dist(x,\ccal) &\leq (1-\mu\dist(y,\ccal)) \left(|x-y| + \dist(y,\ccal)\right) \\
	&= |x-y|+\dist(y,\ccal) \left\{ 1 -\mu |x-y|- \mu \dist(y,\ccal) \right\} \\
	&\leq |x-y|,
	\end{align*}
	which establishes \eqref{Eq:IneqDistanceCone}.
	
	At this point we have shown \eqref{Eq:wH1}-\eqref{Eq:wH5} up to prove that $\mu$ does not depend on $R$ and $S$. We do it in next.	By Step~1, we know that $u$ solves $L_K u = f(u)$ in $B_R$ with $R> S+4$. Therefore, by applying repeatedly the estimates \eqref{Eq:C2sEstimateBalls} and \eqref{Eq:Calpha->Calpha+2sEstimateBalls} in balls centered at points of $B_{S+3}$, it is easy to see that
	$$
	\mu = \norm{u}_{\mathrm{Lip}(\overline{B_{S+3}})} \leq C,
	$$ 
	with a constant $C$ depending only on $m$, $\s$, $\lambda$, $\Lambda$, and $\norm{f}_{C^1([-1,1])}$ (and thus, independent of $R$ and $S$). Recall that $G' = -f$ and thus $\norm{f}_{C^1([-1,1])} \leq \norm{G}_{C^2([-1,1])}$. This concludes the proof.	
\end{proof}



