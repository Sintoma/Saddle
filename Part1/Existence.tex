%%%%%%%%%%%%%%%%%%%%%%%%%%%%%%%%%%%%%%%%%%%%%%%%%%
\section{Existence of the saddle-shaped solution}
%%%%%%%%%%%%%%%%%%%%%%%%%%%%%%%%%%%%%%%%%%%%%%%%%%
\label{Sec:Existence}


In this section we give two different proofs of the existence of saddle-shaped solutions. The first one is based on the direct method of the calculus of variations, and it uses all the results appearing in Section~\ref{Sec:Nonlocal_AllenCahn_Energy}.

\begin{proof}[Proof of Theorem~\ref{Th:Existence}]
Since $\ecal(w,B_R)$ is bounded below ---by $0$---, we can take a minimizing sequence $u_R^j\in \widetilde{\H}^K_{0, \,\mathrm{odd}}(B_R)$. Note that, by Lemma~\ref{Lemma:DecreaseEnergy} we can assume that $-1 \leq u_R^j \leq 1$ and that $u_R^j \geq 0$ in $\ocal$ and $u_R^j \leq 0$ in $\ical$. 

Now, using $\eqref{Eq:Ellipticity}$, $G\geq 0$ and the fact that $u_R^j$ is a minimizing sequence, we deduce that 
$$
[u_R^j]_{H^s(B_R)} \leq \dfrac{c_{n,s}}{\lambda}  [u_R^j]_{\H^K(B_R)}\leq \dfrac{2 c_{n,s}}{\lambda}\ecal(u_R^j,B_R) \leq C
$$
for a constant $C$ that does not depend on $j$. Therefore, $\{u_R^j\}$ is bounded in $H^s(B_R)$ and then, by the compact embedding $H^s(B_R) \subset \subset L^2(B_R)$ (see Theorem~7.1 of \cite{HitchhikerGuide}), there exists a subsequence, still denoted by $u_R^j$,  that converges to some $u_R \in L^2(B_R)$, and thus, a.e. in $B_R$. By Fatou's lemma, we have
$$
\ecal(u_R, B_R)
\leq \liminf_{j\to \infty} \ecal(u_R^j, B_R) = \inf \setcond{\ecal(w, B_R)}{w \in \widetilde{\H}^K_{0, \,\mathrm{odd}}(B_R)}.
$$
Therefore, $u_R \in \widetilde{\H}^K_0(B_R)$ is a minimizer of $\ecal(\cdot, B_R)$ in $\widetilde{\H}^K_{0, \,\mathrm{odd}}(B_R)$. Moreover, it satisfies $-1\leq u_R \leq 1$ in $B_R$, $u_R\geq 0$ in $\ocal$, $u_R(x) = - u_R(x^\star)$ for every $x\in \R^{2m}$ and $u_R \equiv 0 $ in $\R^{2m} \setminus B_R$.

Arguing exactly as in the proof of Theorem~\ref{Th:EnergyEstimate}, we deduce that $u_R$ is a classical solution of
\begin{equation}
\label{Eq:ProofExistenceProblemBR}
	\beqc{\PDEsystem}
	L u_R &=& f(u_R) & \textrm{ in } B_R\,,\\
	u_R &=& 0 & \textrm{ in }\R^{2m} \setminus B_R.
	\eeqc
\end{equation}


The next step is to pass to the limit in $R$ to obtain a solution in $\R^{2m}$. Let $S>0$ and $T =4\lceil 1/\s\rceil$ and consider the family $\{u_R\}$, for $R> S + T$, of solutions in $B_{S+T}$. By applying the estimate \eqref{Eq:UniformC2alphaEstimateBalls} in balls of radius $1$ and centered at points in $\overline{B_{S}}$, we obtain a uniform $C^{2,\alpha}(\overline{B_S})$ bound for $u_R$. By the Arzela-Ascoli theorem, a subsequence of $\{u_R\}$ converges in $C^2(\overline{B_S})$ to a solution in $B_S$. Taking now $S = 1,2,3,\ldots$ and using a diagonal argument, we obtain a sequence $u_{R_j}$ converging in $C^2_{\loc}(\R^{2m})$ to a solution $u \in C^2(\R^{2m})$ of \eqref{Eq:NonlocalAllenCahn}.

Therefore, we have $u$ a solution of $Lu = f(u)$ in $\R^{2m}$ which is doubly radial. Furthermore, $u$ is odd with respect to the Simons cone $\ccal$, i.e., $u(x) = -u(x^\star)$ for $x\in \R^{2m}$, and $0 \leq u\leq 1$ in $\ocal$.

Finally, we show that $0<u<1$ in $\ocal$. This will ensure that $u$ is a saddle-shaped solution. First, note that $|u| < 1$ by the strong maximum principle (since $u$ vanishes at $\ccal$, $u \not \equiv 1$  and $u\not\equiv -1$). Let us show that $u\not\equiv 0$. To do this, we use the energy estimate of Theorem~\ref{Th:EnergyEstimate}. That is, if we consider $u_R$ the minimizer of $\ecal(\cdot, B_R)$ with $R > 2$, we have
$$
\ecal (u_R,B_S) \leq \begin{cases}
C \ S^{2m-2\s}\ \ \ \ &\textrm{if } \ \ \s\in(0,1/2),\\
C\ \log(S)\,S^{2m-2\s}\ \ \ \ &\textrm{if } \ \ \s=1/2,\\
C \ S^{2m-1}\ \ \ \ &\textrm{if } \ \ \s\in(1/2,1),\\
\end{cases} $$
for every $0 < S < R-2$ and with a constant $C$ independent of $R$ and $S$. By letting $R \to \infty$ we obtain the same estimate for $u$. By contradiction, assume $u\equiv 0$. Then, the previous estimate leads to
$$
c_m G(0)S^{2m} = \ecal(0, B_S) \leq \begin{cases}
C \ S^{2m-2\s}\ \ \ \ &\textrm{if } \ \ \s\in(0,1/2),\\
C\ \log(S)\,S^{2m-2\s}\ \ \ \ &\textrm{if } \ \ \s=1/2,\\
C \ S^{2m-1}\ \ \ \ &\textrm{if } \ \ \s\in(1/2,1),\\
\end{cases} $$
and this is a contradiction for $S$ large enough. Therefore, since $u \not \equiv 0$, the strong maximum principle for odd functions (see Proposition~\ref{Prop:StrongMaximumPrincipleForOddFunctions}) yields that $u>0$ in $\ocal$. 
\end{proof}

